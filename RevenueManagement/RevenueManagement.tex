% !TEX encoding = UTF-8 Unicode
\chapter{Das Konzept des Revenue Managements}\label{KapitelDP}
\markboth{3 Revenue Management}{}
\setcounter{footnote}{36}  %um durchgehende Fußnotennummerierung zu haben, hier die Anzahl der bisherigen Fußnoten eintragen


\section{Herkunft des Revenue Managements}
Zur Entscheidungsunterstützung bei der Annahme von Kundenaufträgen wird in der aktuellen Forschung vermehrt auf das Konzept des Revenue Managements zurückgegriffen.\footnote{Vgl. \cite{klein2001revenue}, S. 246.} Da eine kurzfristige Anpassung der mittelfristig bereitgestellten Kapazitäten einer Dienstleistungsproduktion an eine unsichere und schwankende Nachfrage nicht möglich ist, wird mit dem Konzept eine effizientere Auslastung der bestehenden Kapazitäten ermöglicht.\footnote{Vgl. \cite{ing2005revenue}, S. 124.}

Der Begriff \textit{Revenue Management} wird im deutschsprachigen Raum meist mit \textit{Ertrags}\-\textit{management} oder \textit{Erlösmanagement} übersetzt.\footnote{Vgl. z. B. \cite{zehle1991yield}, S. 486.} Yield Management wird als Synonym benutzt.\footnote{Vgl. z. B. \cite{kolisch2006revenue}, S. 319.} Dabei greift der Begriff \textit{Yield} zu kurz, da damit in der Luftverkehrsbranche der Erlös je Passagier und geflogener Meile bezeichnet wird.\footnote{Vgl. z. B. \cite{weatherford1998tutorial}, S. 69.} Die Bezeichnung \textit{Revenue Management} hat sich jedoch gegenüber Yield Management durchgesetzt, da der Yield (Durchschnittsertrag) sich theoretisch nur durch einen Passagier maximieren lässt und somit die Maximierung als Zielsetzung nicht sinnvoll ist.\footnote{Vgl. \cite{Klein:2008aa}, S. 6; \cite{ing2005revenue}, S. 124-125.} Erste Ansätze des Konzepts sind in der Praxis entwickelt. Durch die Deregulierung des amerikanischen Luftverkehrsmarktes im Jahr 1978 mussten die traditionellen Fluggesellschaften ihre Wettbewerbsfähigkeit gegenüber Billiganbietern erhöhen und entwickelten das frühe Revenue Management (RM).\footnote{Vgl. \cite{Petrick:2009aa}, S. 1-3.}

%\cite{kimms2005revenue} versuchen durch eine umfangreiche Diskussion einige Erklärungsansätze aufzuzeigen (Warum? Wovon?).
%Das RM hat vor allem aus dem älteren, englischsprachigen Bereich einen engen Bezug zu konkreten Anwendungsgebieten. Die Autoren zeigen auf, dass viele Autoren versuchen das komplexe Konzept des Revenue Managements in einer kurzen Erklärung zu überführen. Dieses läuft letztlich darauf hinaus, dass diese Autoren einige situative Merkmale und Instrumente des Managements vermischen, gleichzeitig aber versuchen, die Zielsetzung festzulegen und das Anwendungsgebiet auf bestimmte Branchen zu beschränken. %\cite{kimms2005revenue} weisen darauf hin, dass eine differenzierte Betrachtung des Konzepts notwenig ist: Einerseits im Hinblick auf die Anwendungsvoraussetzungen und andererseits im Hinblick auf die Instrumente des Revenue Managements, damit verdeutlicht dargestellt wird, in welchen Branchen das RM Potentiale liefert. In den nachfolgenden zwei Kapiteln wird dieser Empfehlung gefolgt.

In der Literatur wird der Begriff des RM unterschiedlich definiert. \cite{friege1996yield} bezeichnet das RM als \textit{Preis-Mengen-Steuerung}, \cite{daudel1992yield} als \textit{Preis-Kapazitäts-Steuerung} und \cite{talluri2004theory} verstehen es als das gesamte \textit{Management der Nachfrage}. Die beiden ersteren Definitionen können als Synonym für eines der Instrumente des RM stehen und daher finden diese für das gesamte Konzept keine weitere Verwendung.\footnote{Vgl. \cite{Petrick:2009aa}, S. 9-13.} Nachfolgend wird die Definition von \citeauthor{klein2001revenue} (2001, S. 248) aufgegriffen:

\begin{quote}
\glqq Revenue Management umfasst eine Reihe von quantitativen Methoden zur Entscheidung über Annahme oder Ablehnung unsicherer, zeitlich verteilt eintreffender Nachfrage unterschiedlicher Wertigkeit. Dabei wird das Ziel verfolgt, die in einem begrenzten Zeitraum verfügbare, unflexibel Kapazität möglichst effizient zu nutzen.\grqq
\end{quote}

\cite{Petrick:2009aa} definiert das RM als Ziel eines Unternehmens die Gesamterlöse zu maximieren, die sich aufgrund der speziellen Anwendungsgebiete ergeben. Weiter beschreibt \cite{Petrick:2009aa} das RM als Zusammenfassung aller Interaktionen eines Unternehmens, die mit dem Markt, also der Absatz- oder Nachfrageseite, zusammenhängen. Im Kern lassen sich drei wichtige Perspektiven für eine Definition des RM nach \cite{Petrick:2009aa}, \cite{stuhlmann2000kapazitatsgestaltung},  \cite{corsten1999yield} übernehmen:
\begin{enumerate}
	\item Ziel ist es die Gesamterlöse unter möglichst optimaler Auslastung der vorhandenen Kapazitäten zu maximieren.
	\item Durch eine aktive Preispolitik wird das reine Kapazitäts- oder Auslastungsmanagement unterstützt.
	\item Für die erfolgreiche Implementierung des Revenue Managements ist eine umfangreiche Informationsbasis notwendig. Es muss u. a. eine möglichst gute Prognose über die zukünftige Nachfrage und Preisbereitschaft der Kunden vorhanden sein.
\end{enumerate}

\cite{kimms2005revenue} weisen darauf hin, dass eine differenzierte Betrachtung des Konzepts notwenig ist: Einerseits im Hinblick auf die \textbf{Anwendungsvoraussetzungen} und andererseits im Hinblick auf die \textbf{Instrumente}, damit verdeutlicht dargestellt wird, in welchen Branchen das RM Potentiale liefert. Dabei sollten branchenspezifische Besonderheiten, neben den zahlreichen Ähnlichkeiten Berücksichtigung finden, sowie das begrenzte Kapazitätskontingent, damit die Potentiale des RM zur Maximierung der Gesamterlöse in den Dienstleistungsbranchen erfolgen kann.\footnote{Vgl. z. B. \cite{Martens:2009aa}, S. 11-24.} In dem nachfolgenden Kapitel werden die Anwendungsvoraussetzungen sowie Instrumente des RM vorgestellt.

\section{Anwendungsvoraussetzungen und Instrumente}

\cite{Petrick:2009aa} weist da\-rauf hin, dass anhand von speziellen Anwendungsvoraussetzungen geprüft wird, ob das RM für die jeweilige Situation des Unternehmens (oder die gesamte Branche) zur Maximierung des Gesamterlöses beiträgt. \cite{kimes1989yield} definiert die in der Literatur häufigsten Anwendungsvoraussetzungen:\footnote{Vgl. u. a. \cite{friege1996yield}, S. 616-622, und \cite{weatherford1992taxonomy}, 831-832.}
\begin{itemize}
	\item \glqq weitgehend fixe\grqq\;Kapazitäten,
	\item \glqq Verderblichkeit\grqq\;bzw. \glqq Nichtlagerfähigkeit\grqq\;der Kapazitäten und der Leistung,
	\item Möglichkeit zur Vorausbuchung von Leistungen,
	\item stochastisch, schwankende Nachfrage,
	\item hohe Fixkosten für die Bereitstellung der gesamten Kapazitäten bei vergleichsweise geringen variablen Kosten für Produktion einer Leistungseinheit und
	\item Möglichkeit zur Marktsegmentierung und im Ergebnis dessen zur segmentorientierten Preisdifferenzierung.
\end{itemize}
\vspace{0.2cm}

\cite{Klein:2008aa} setzen sich mit den Anwendungsvoraussetzungen von mehreren Autoren auseinander. Sie konnten Gemeinsamkeiten innerhalb der Definitionen der Autoren finden, aber zeigten auch die Unterschiede und ihre Kritiken auf. In ihrer Arbeit übernehmen sie die Anwendungsvoraussetzung von \cite{corsten1998yield}: \glqq Marktseitige Anpassungserfordernis steht unternehmensseitig unzureichendes Flexibilitätspotential hinsichtlich der Kapazität -- bezogen auf Mittel- oder Zeitaufwand -- gegenüber\grqq. Zugleich weisen sie jedoch darauf hin, dass zum Verständnis eines komplexen und interdisziplinären Ansatzes auch die Definitionen anderer Autoren im Hinblick auf das Verständnis der Anwendungsvoraussetzungen beitragen.

Auf Grundlage der von \cite{friege1996yield} beschriebenen Anwendungsvoraussetzungen hat \cite{Petrick:2009aa} drei Instrumente des RM bestimmt. Die Instrumente benötigen als Grundlage \textit{Daten der Prognose}, damit sie zur Anwendung kommen.\footnote{Die Prognose zählt laut \cite{Petrick:2009aa} nicht als eigenständiges Instrument des RM.} Zu den Instrumenten zählen die \textbf{segmentorientierte Preisdifferenzierung}, die \textbf{Kapazitäten\-steuerung} und die \textbf{Über\-buchungssteuerung}. Die Instrumente weisen dabei unterschiedliche Ab\-hängigkeit\-en untereinander auf, da z. B. die Kapazitä\-tensteu\-erung auf den Ergebnissen der Preisdifferenzierung aufbaut und die Überbuchungssteuerung selten ohne Kapazitätensteuerung gelöst werden kann. Eine genauere Betrachtung der Instrumente liefern \cite{talluri2004theory} oder \cite{Petrick:2009aa} in ihren jeweiligen Arbeiten.

Aufgrund der Anwendungsvoraussetzungen des RM kann das Konzept auch auf die Auftragsfertigung bzw. MTO-Prozesse übertragen werden.\footnote{Vgl. \cite{hintsches2010revenue}, S.176-178; \cite{kimes1989yield}, S. 349-351.} Bei der Auftragsfertigung wird ein bestimmter Buchungszeitraum betrachtet, in dem weitestgehend von fixen Kapazitäten der Ressourcen ausgegangen werden kann. Durch Ablauf des Buchungshorizonts verfällt die Kapazität, da es sich hauptsächlich um erneuerbare Kapazitäten handelt (Arbeitskraft, Maschinenkapazität, usw.). Anders formuliert, die Ressourcen werden zur nachfolgenden Leistungserstellung erneuert und sofern sie in dem Betrachtungszeitraum keine Verwendung fanden, verfallen die Kapazitäten der Ressourcen zum nächsten Betrachtungszeitraum. Ein Betrachtungszeitraum versteht sich im Konzept des RM einen sogenannten Buchungshorizont $T$, indem die Anfragen bis zum Zeitpunkt der Erstellung der Leistung eintreffen können. Dabei beschreibt $t$ einen Zeitpunkt des Buchungshorizonts. Die Gesamtlänge des Buchungshorizonts entspricht dem Parameter $T$ und verläuft rückwärts in der Zählweise bis zur Leistungserstellung. Der Buchungshorizont im klassischen RM ist mit genau einer Leistungserstellung $i$ gekoppelt. Mit der Leistungserstellung $i$ werden die durch die Auftragsannahme reservierten Kapazitäten des Buchungshorizonts durch die Produktion beansprucht. Mit dem Start der Leistungserstellung erfolgt ebenfalls die Vorausbuchungszeit $T$ für die nachfolgende Leistungserstellung $i+1$. Mit der Leistungserstellung $i+1$ sind die Kapazitäten der Ressourcen regeneriert. Abbildung \ref{B0} zeigt den Zusammenhang von Buchungshorizont, Leistungserstellung und abnehmender Ressourcenkapazität. %Nachfolgend wird das mathematische Grundmodell für diese klassische Auffassung des RM beschrieben.

\begin{figure}[h!]
  \begin{center}
    \includegraphics[width=140mm]{Bilder/Kapaverbrauch.pdf}
    \caption{Grafische Darstellung des Buchungshorizonts und der jeweiligen Zeitpunkte der Leistungserstellung in der Auftragsfertigung}  \label{B0}
    {\footnotesize \textbf{In Anlehnung an:} \cite{lars}.} 
  \end{center}
\end{figure}


\section{Das dynamisch, stochastische Grundmodell}\label{Grundmodell}
Im Folgenden wird das dynamisch, stochastische Grundmodell des RM nach \cite{talluri2004revenue} beschrieben. Ein Dienstleistungsnetzwerk eines Anbieters benötigt jeweils zur Erstellung der Dienstleistungen ein bestimmtes Kontingent an Ressourcen aus einer Menge an Ressourcen $\mathcal{H} = \{1,...,l \}$. Der Index $h$ beschreibt dabei eine jeweilige Ressource und der Index $l$ die gesamte Anzahl an möglichen Ressourcen. Die jeweilige Kapazität einer Ressource $h \in \mathcal{H}$ ist durch den Parameter $c_{h}$ beschrieben und die gesamten Kapazitäten der Ressourcen ist als Vektor $\textbf{c}=(c_{1},...,c_{h},...,c_{l})$ formuliert. Eine Anfrage nach einem Produkt (Dienstleistung) im Netzwerk ist durch den Parameter $j$ aus der Menge an möglichen Produktanfragen $\mathcal{J} = \{1,...,n \}$ %für die Menge der Ressourcen $\mathcal{H}$
beschrieben. Die gesamte Anzahl an Produktanfragen ist durch den Parameter $n$ definiert. Sobald eine Produktanfrage $j\in \mathcal{J}$ akzeptiert und somit abgesetzt ist, fällt für den Absatz der Ertrag $r_{j}$ an. Der jeweilige Verbrauch einer Ressource $h$ durch Annahme einer Anfrage nach einem Produkt $j$ ist anhand des Parameters $a_{hj}$ beschrieben. Durch Vektorschreibweise kann der Ressourcenverbrauch für eine Produktanfrage $j$ als Vektor $\textbf{a}_{j}=(a_{1j},...,a_{hj},...,a_{lj})$ formuliert werden. Der Buchungshorizont entspricht $T$ Perioden und die Aufteilung kann jeweils in einzelne Perioden $t=1,...,T$ erfolgen. Dabei muss Beachtung finden, dass der Buchungshorizont $T$ gegenläufig verläuft. Die Wahrscheinlichkeit der Nachfrage eines Produkts $j$ in der Periode $t$ entspricht $p_{j}(t)$ und die Wahrscheinlichkeit, dass keine Nachfrage in der Periode $t$ eintrifft, entspricht $p_{0}(t)$. Es gilt $\sum_{j\in \mathcal{J}}p_{j}(t)+p_{0}(t)=1$ und somit kann $p_{0}(t)$ durch den Term $p_{0}(t)=1-\sum_{j\in \mathcal{J}}p_{j}(t)$ für die Periode $t$ ermittelt werden.\footnote{Vgl. \citeauthor{talluri2004revenue}, S. 18.} %Die noch erwartete Nachfrage $D_{jt}$ für ein bestimmtes Produkt $j$ für eine beliebige Periode $t$ lässt sich durch $\sum_{\tau=1}^{t}p_{j}(\tau)$ aggregieren. 


Mit den vorangegangenen Parametern kann der maximal erwartete Ertragswert $V(\textbf{c},t)$ für eine Periode $t$ bei einer noch vorhandenen Ressourcenkapazität $\textbf{c}$ als Bellman-Gleichung formuliert werden:\footnote{Vgl. \cite{Petrick:2009aa}, S. 185-187; \cite{doi:10.1287/trsc.37.3.257.16047}, S. 261.}
\begin{equation}\label{DP}
V(\textbf{c},t)=\sum_{j\in\mathcal{J}}p_{j}(t)\max[ \underbrace{V(\textbf{c},t-1)}_{\text{Ablehnung}},\; \underbrace{r_{j}+V(\textbf{c}-\textbf{a}_{j},t-1)}_{\text{Auftragsannahme}}]+p_{0}(t)\underbrace{V(\textbf{c},t-1)}_{\text{Kein Auftrag}}
\end{equation}


Es handelt sich hier um das stochastisch, dynamische Grundmodell im Netzwerk RM. Das Konzept dieser Modellformulierung ist von \citeauthor{bellman1954theory} entwickelt und dient der Ermittlung der optimalen Politik in Bezug auf den aktuellen Zustand eines Systems.\footnote{Vgl. \cite{bellman1954theory}, S. 4-5.} Dabei bildet jeder Erwartungswert $V(\textbf{c},t)$ mit der Ressourcenkapazität $\textbf{c}$ zum Zeitpunkt $t$ einen Systemzustand des Netzwerks ab. Eine derartige Formulierung eines Optimierungsproblems wird als sogenannte \textit{Bellman'sche Funktionsgleichung} oder \textit{Bellman-Gleichung} bezeichnet.

Der Erwartungswert $V(\textbf{c},t)$ mit einer Kapazität $\textbf{c}$ zur Buchungsperiode $t$ ist die Summe aller Entscheidung über die Produktanfragen $\mathcal{J}$, ob eine Produktanfrage $j$ abgelehnt ($V(\textbf{c},t-1)$) oder akzeptiert wird ($r_{j}+V(\textbf{c}-\textbf{a}_{j},t-1)$). Sofern keine Akzeptanz der Produktanfrage $j$ erfolgt, verstreicht eine Buchungsperiode ohne Inanspruchnahme der Kapazität. Wird jedoch die Produktanfrage $j$ akzeptiert, ist die Kapazität $\textbf{c}$ durch den Ressourcenverbrauch $\textbf{a}_j$ zur nächsten Buchungsperiode $t$ reduziert und der jeweilige Ertrag $r_j$ aus der Produktanfrage $j$ erzielt. Die Entscheidung über Ablehnung oder Annahme einer Produktanfrage $j$ erfolgt anhand des Erwartungswerts für nachfolgende Perioden $t\in T$. Siehe in Gleichung \eqref{DP} die Maximierungsfunktion. Dabei ist die Entscheidung mit der jeweiligen Eintrittswahrscheinlichkeit $p_j(t)$ der Produktanfrage $j$ zur Periode $t$ gewichtet. Zusätzlich ist der Erwartungswert $V(\textbf{c},t)$ um die Wahrscheinlichkeit des Nichteintreffens einer Produktanfrage $j$ erhöht ($p_{0}(t)V(\textbf{c},t-1)$). Der Erwartungswert $V(\textbf{c},t)$ weist dabei die Grenzbedingungen
\begin{equation}\label{GB1}
V(\textbf{c},0)=0, \text{ wenn } \textbf{c}\ge0, \text{ sowie }
\end{equation}
\begin{equation}\label{GB2}
V(\textbf{c},t)=-\infty, \text{ wenn } c_{h}<0 \;\forall h\in\mathcal{H},
\end{equation}
auf, da eine jeweilig verbleibende Kapazität nach Bereitstellung des Produkts wertlos und eine negative Ressourcenkapazität nicht möglich ist. 

Die Gleichung \eqref{DP} lässt sich umformen, indem die Entscheidungen über die Annahme und Ablehnung einer Produktanfrage separiert wird. Da $\sum_{j\in \mathcal{J}}p_{j}(t)+p_{0}(t)=1$ gilt, kann die Gleichung vereinfacht werden zur nachfolgenden Gleichung:\footnote{Vgl. \cite{Spengler:2007aa}, S. 161.}
\begin{equation}\label{DP2}
\begin{alignat*}{2}
V(\textbf{c},t)=\;& \sum_{j\in\mathcal{J}}p_{j}(t) V(\textbf{c},t-1)+ \sum_{j\in\mathcal{J}}p_{j}(t) \max[0,r_{j}-V(\textbf{c},t-1)\\
&+V(\textbf{c}-\textbf{a}_{j},t-1)]+p_{0}(t)V(\textbf{c},t-1)\\
=\;& V(\textbf{c},t-1) + \sum_{j\in\mathcal{J}}p_{j}(t) \max[0,\\
&  r_{j}-V(\textbf{c},t-1)+V(\textbf{c}-\textbf{a}_{j},t-1)]
\end{alignat*}
\end{equation}

Eine eintreffende Produktanfrage $j$ ist demnach dann akzeptiert, wenn der Ertrag $r_{j}$ größer gleich der Differenz des Erwartungswertes unter der Prämisse der Ablehnung und dem Erwartungswert unter der Prämisse der Annahme einer Produktanfrage $j$ ist:
\begin{equation}\label{r}
r_{j} \ge V(\textbf{c},t-1)-V(\textbf{c}-\textbf{a}_{j},t-1)
\end{equation}

Sofern der Ertrag $r_j$ diese Bedingung erfüllt, handelt es sich bei der Annahme der Produktanfrage $j$ um die optimale Politik. D. h. sofern diese Anfrage eintrifft, sollte diese unter Berücksichtigung der aktuellen Kapazität $\textbf{c}$ zur Periode $t$ akzeptiert werden. Dabei kann der rechte Term der Bedingung \eqref{r} als Opportunitätskosten (OK) der Auftragsannahme angesehen werden:
\begin{equation}\label{OC}
OC_{j} = V(\textbf{c},t-1)-V(\textbf{c}-\textbf{a}_{j},t-1)
\end{equation}

Somit erfolgt die Akzeptanz einer Produktanfrage $j\in\mathcal{J}$ ausschließlich nur dann, sofern die OK des Ressourcenverbrauchs niedriger als der Ertrag $r_j$ ist. Der maximal mögliche Erwartungswert unter Beachtung der Kapazität $\textbf{c}$ zum Zeitpunkt $t$ ist damit die Addition des Erwartungswerts unter der Prämisse der Ablehnung der Anfragen und die Summe der um die jeweiligen OK reduzierten Erträge $r_j$ der akzeptierten Produktanfragen $j\in\mathcal{J}$. Mathematisch lässt sich dies wie folgt definieren:
\begin{equation}\label{DPoc}
V(\textbf{c},t)=V(\textbf{c},t-1) + \sum_{j\in\mathcal{J}}p_{j}(t) \max[0,r_{j}-OC_{j}]
\end{equation}

%Die optimale Politik des Netzwerks zum Zeitpunkt $t$ ist damit die Annahme der Anfrage nach Produkt $j$ mit dem höchsten Ertrag $r_{j}$ abzüglich der anfallenden $OC_{j}$:

%begin{equation}\label{OP}
%OP_{\textbf{c}, t}:= \{ \; j\; | \max_{j\in\mathcal{J}} [r_{j}-OC_{j}] \} 
%\end{equation}

Zur Veranschaulichung des Netzwerk RM wird ein Netzwerk mit zwei möglichen Produktanfragen $j\in\mathcal{J}$ und zwei Ressourcen $h\in\mathcal{H}$ betrachtet. Die Ressource $h=1$ hat eine Kapazität von $c_{1}=2$ und die Ressource $h=2$ hat eine Kapazität von $c_{2}=1$. Zur Ausführung der Produktanfrage $j=1$ wird die Ressource $h=1$ mit einer Einheit und zur Ausführung der Produktanfrage $j=2$ wird wiederum eine Einheit der Ressource $h=2$ benötigt. Damit gilt $a_{11}=1$ und $a_{22}=1$. Durch Annahme einer Anfrage $j=1$ wird der Ertrag $r_{1}=100$ und durch Annahme von Anfrage $j=2$ der Ertrag $r_{2}=200$ generiert. Der Buchungshorizont entspricht $T=4$. Die Wahrscheinlichkeiten des Eintreffens einer Produktanfrage $j=1$ über die Buchungsperioden $t\in T$ lässt sich als Vektor $p_{1}(t)=(0.5, 0.5, 0.5, 0.5)$ beschreiben. Analog lassen sich die Wahrscheinlichkeiten für das Eintreffen der Produktanfragen $j=2$ als Vektor $p_{2}(t)=(0.1, 0.1, 0.1, 0.1)$ definieren. Die Gegenwahrscheinlichkeiten, dass keine Anfragen eintreffen, lässt sich mit $p_{0}(t)=1-\sum_{j\in \mathcal{J}}p_{j}(t)$ berechnen und bilden den Vektor $p_{0}(t)=(0.4, 0.4, 0.4, 0.4)$.

Die Parameter lassen sich damit abschließend wie folgt definieren:
\begin{center}
$j = \{1, 2\}, \; h = \{1, 2\}, \; r_{1} = 100, \; r_{2} = 200, \; \text{Startperiode } t=4$,
\end{center}
\[
    \textbf{c}=\begin{pmatrix} 2 \\ 1 \end{pmatrix}, \;
    \textbf{a}_1=\begin{pmatrix} 1 \\ 0 \end{pmatrix}, \;
     \textbf{a}_2=\begin{pmatrix} 0 \\ 1 \end{pmatrix}, \;
     p_{1}(t)=\begin{pmatrix} 0.5\\ 0.5\\ 0.5\\ 0.5  \end{pmatrix}, \;
     p_{2}(t)=\begin{pmatrix} 0.1\\ 0.1\\ 0.1\\ 0.1  \end{pmatrix}
  \]

Das stochastisch, dynamische Optimierungsproblem aus Gleichung \eqref{DP2} lässt sich als Graph darstellen, indem die einzelnen Erwartungswerte als Systemzustände des Netzwerks aufgeführt sind.\footnote{Vgl. \cite{demiguel2006multistage}, S. 8-13.} Der Erwartungswert $V(\textbf{c},t)$ ist abhängig vom Erwartungswert $V(\textbf{c},t-1)$ und vom Erwartungswert $V(\textbf{c}-\textbf{a}_{j},t-1)$. Der Erwartungswert $V(\textbf{c},t-1)$ ist wiederum abhängig vom Erwartungswert $V(\textbf{c},t-2)$ und vom Erwartungswert $V(\textbf{c}-\textbf{a}_{j},t-2)$, usw. Abbildung \ref{B0} zeigt die erste rekursive Folge für die Gleichung \eqref{DP2} als gerichteten Graphen und im nachfolgenden wird auf die Notation eingegangen.
\begin{figure}[h!]
  \begin{center}
    \includegraphics[width=80mm]{Bilder/Beispiel0.pdf}
    \caption{Beispielhafte Darstellung der ersten rekursiven Folge einer Problemstellung im Netzwerk RM}  \label{B0}
  \end{center}
\end{figure}

Ein Knoten repräsentiert einen Systemzustand des Netzwerks mit den vorhandenen Kapazitäten $\textbf{c}$ zum Zeitpunkt $t$. Bei dem Systemzustand handelt es sich um ein Teilproblem der \textit{Bellman'schen Funktionsgleichung}. Bei der Benennung eines solchen Knotens wird eine Zahlenfolge verwendet, bei dem die ersten Einträge die Ressourcenkapazität $\textbf{c}$ in Länge der Ressourcen $h$ entsprechen und der letzte Eintrag den Zeitpunkt $t$ aufzeigt. Bspw. zeigt der Startknoten des Beispiels die Zahlenfolge $[2\;1\;4]$, da das Netzwerk noch die volle Ressourcenkapazität $\textbf{c}=(2,1)$ aufweist und sich im Zeitpunkt $t=4$ befindet. Von diesem Systemzustand können jetzt nachfolgende Systemzustände abhängig der Produktanfragen erreicht werden. In diesem Netzwerk gibt es zwei Anfragen $j$ und durch Betrachtung der Gleichung \eqref{DP2} wird klar, dass drei Optionen zum Erreichen des nachfolgenden Systemzustands zum Zeitpunkt $t-1=3$ möglich sind. Diese Optionen bilden die Kanten des Graphen. Sofern keine Produktanfrage $j$ eintrifft, wird der nachfolgende Systemzustand $[2\;1\;3]$ erreicht. Alternativ können Produktanfragen $j=1$ oder $j=2$ eintreffen und dementsprechend werden die vorhandenen Kapazitäten $\textbf{c}$ um den Ressourcenverbrauch $a_{11}=1$ bzw. $a_{22}=1$ reduziert. Daraus folgt, dass der Systemzustand $[1\;1\;3]$ bzw. $[2\;0\;3]$ im Netzwerk erreichbar ist. Die optimale Politik eines solchen Graphen wird abgebildet durch die Kanten. Die optimale Politik einer Produktanfrage $j$ wird im Graphen als durchgezogener Pfeil dargestellt, sofern die Entscheidung der Annahme der Produktanfrage den Erwartungswert maximiert. Erfolgt keine Maximierung des Erwartungswerts durch die Auftragsannahme der Produktanfrage $j$, wird der Übergang in Form der Kante als gestrichelter Pfeil dargestellt. Ein Übergang über die Kante zum nächsten Systemzustand wäre theoretisch möglich, aber entspricht nicht der optimalen Politik. Sofern eine solche unrentable Produktanfrage $j$ eintrifft, sollte unter der Zielsetzung der Maximierung des Gesamtertrags die Ablehnung der Anfrage erfolgen.

Aufbauend auf dieser rekursiven Logik wird ein gerichteter und gewichteter Multigraph aufgebaut. Er zeigt alle möglichen Systemzustände des Netzwerks. Die Rekursion wird abgebrochen, sofern ein Systemzustand aufgrund der Grenzbedingungen aus den Gleichungen \eqref{GB1} oder \eqref{GB2} nicht möglich ist. Abbildung \ref{B1} zeigt die möglichen Systemzustände für das eingeführte Beispiel.\\[.5cm]
\begin{figure}[h!]
  \begin{center}
    \includegraphics[width=150mm]{Bilder/Beispiel1.pdf}
    \caption{Darstellung der Systemzustände der beispielhaften Problemstellung im Netzwerk RM}  \label{B1}
  \end{center}
\end{figure}

Anhand dieses Graphen mit den möglichen Systemzuständen aus der Abbildung \ref{B1} wird ersichtlich, welche Erwartungswerte (Teilprobleme) für das Entscheidungsproblem der Auftragsannahme durch Rückwärtsinduktion gelöst werden müssen.\footnote{Vgl. \cite{puterman2009markov}, S. 92-93.} Dafür wird im ersten Schritt die Grenzbedingung aus Gleichung \eqref{GB1} betrachtet. Alle Systemzustände zum Zeitpunkt $t=0$ nehmen den Erwartungswert $V(\textbf{c}, t=0)=0$ an. Mit dieser Bedingung lassen sich die zeitlich zuvorkommenden Systemzustände mit $t=1$ berechnen. Es wird die Gleichung \eqref{DP2} angewendet, wobei beachtet werden muss, dass nicht für alle Systemzustände mit $t=1$ alle Produktanfragen möglich sind. Dies folgt aus der Grenzbedingung aus Gleichung \eqref{GB2}. Zur Veranschaulichung wird der Erwartungswert des Systemzustands $[1\;0\;1]$ mittels der Gleichung \eqref{DP2} berechnet.
\begin{alignat*}{2}
V(1,0,1)=\;&V(1,0,0)+p_{1}(1)\max[r_{1}-V(1,0,0)+V(0,0,0),0]\\
&+p_{2}(1)\max[r_{2}-V(1,0,0)+V(0,-1,0),0]\\
=\;&0+0,5\cdot\max[100-0+0,0]+0,1\cdot\max[200-0+(-\infty),0]\\
=\;&0+0,5\cdot 100+0,1\cdot0\\
=\;&50\\
\end{alignat*}
Nach dieser Vorgehensweise der Rückwärtsinduktion erfolgt die Ermittlung alle Erwartungswerte der möglichen Systemzustände. Tabelle \ref{Tab1} zeigt für alle möglichen Systemzustände $[c_1\;c_2\;t]$ den Erwartungswert (ExpValue) und die optimale Politik. Die optimale Politik ist in den letzten zwei Spalten der Tabelle \ref{Tab1} jeweils für die zwei möglichen Anfragen $j\in\mathcal{J}$ aufgeführt. Sofern eine Anfrage nach dem Produkt $j$ eintrifft, zeigt die Tabelle mit dem Wert $\glqq 1\grqq$, dass die Anfrage unter der Berücksichtigung der vorhandenen Kapazitäten $\textbf{c}$ und den verbleibenden Perioden $t\in T$ angenommen werden soll. Sofern die Anfrage $j$ nicht der optimalen Politik entspricht oder in diesem Systemzustand nicht möglich ist, dann ist in den Spalten eine $\glqq 0\grqq$ vermerkt.
\begin{table}
\begin{footnotesize}
    \caption{Ergebnistabelle für die beispielhafte Problemstellung im Netzwerk RM} \label{Tab1}
    \vspace*{3mm}
    \begin{center}
\csvautotabular{data/beispiel1.csv}
      %{\footnotesize \textbf{In Anlehnung an:} \cite{gonsch2013using}, S. 113.} 
            \end{center}
\end{footnotesize}
\end{table}

Die optimale Politik lässt sich anhand der Gleichung \eqref{DPoc} für jeden Systemzustand ermitteln. Für jede Kante des Graphens in Abbildung \ref{B1} kann damit der Ertrag abzgl. der OK hinterlegt werden ($r_{j}-OC_{j}$). Die optimale Politik für eine Produktanfrage $j$ im betrachteten Graphen der Abbildung \ref{B1} ist demnach die Kante bei dem der Ertrag abzgl. der OK den Erwartungswert maximiert. Abbildung \ref{B1a} zeigt alle Systemzustände und Übergänge als Graphen mit den jeweiligen Erwartungswerten als Knoten und den Wert $r_{j}-OC_{j}$ als Kante.
\begin{figure}[h!]
  \begin{center}
    \includegraphics[width=150mm]{Bilder/Beispiel1a.pdf}
    \caption{Darstellung der Systemzustände der beispielhaften Problemstellung im Netzwerk RM (Erwartungswerte und optimale Politik)}  \label{B1a}
  \end{center}
\end{figure}

Bei einem Fallbeispiel mit solchen Parameterwerten ist damit nur eine Auswertung des Erwartungswertes über den gesamten Verlauf des Buchungshorizonts $T$ möglich. Dies folgt aus der Tatsache, dass keine Anfragen zu einem beliebigen Systemzustand abgelehnt werden. Bei einem derartigen Ergebnis können jedoch strategische Handlungsempfehlungen abgeleitet werden. Z. B. kann die Anpassung der Instrumente der Marktbearbeitung vom betrachteten Unternehmen erfolgen, damit der beste Pfad der möglichen Auftragseingänge abgearbeitet wird. Der beste Pfad für das Eintreffen der Produktanfragen lautet $[2\;1\;4] \rightarrow_{j={2}}[2\;0\;3] \rightarrow_{j={1}}[1\;0\;2]\rightarrow_{j={1}}[0\;0\;1]\rightarrow_{KA}[0\;0\;0]$. Mit geschickter Ausgestaltung der Instrumente der Marktbearbeitung wird die prognostizierte Verteilung der Auftragseingänge zum Irrtum und der maximal mögliche Gesamtertrag ist generiert.

Anhand des Beispiels wird klar, dass die Prognose über die Wahrscheinlichkeiten des Eintreffens einer Anfrage nach den Produkten $j\in\mathcal{J}$ die Erwartungswerte stark beeinflussen, sowie der potentielle Ertrag $r_{j}$ und die $OC_{j}$. Dies kann durch das Abwandeln der Parameter gezeigt werden:
\begin{center}
$j = \{1, 2\}, \; h = \{1\}, \; r_{1} = 100, \; r_{2} = 200, \; T=4$
\end{center}
\[
    c_{1}= 2, \;
    a_{11}=1, \;
     a_{12}=1, \;
     p_{1}(t)=\begin{pmatrix} 0.2\\ 0.2\\ 0.2\\ 0.2  \end{pmatrix}, \;
     p_{2}(t)=\begin{pmatrix} 0.8\\ 0.8\\ 0.8\\ 0.8  \end{pmatrix}
  \]
  
Bei dieser beispielhaften Ausgestaltung der Parameter konkurrieren die Typen der Produktanfragen $j$ mit der einzigen verfügbaren Ressource $h$. Die nachfolgende Abbildung \ref{B2} zeigt den zugehörigen Graphen mit den Systemzuständen, den möglichen Entscheidungen sowie der grafischen Darstellung der optimalen Politik und die Tabelle \ref{Tab2} zeigt die berechneten Erwartungswerte sowie die optimale Politik für jeden Systemzustand $[c_1\;t]$.
\begin{figure}[h!]
  \begin{center}
    \includegraphics[width=120mm]{Bilder/Beispiel2.pdf}
    \caption{Darstellung der Systemzustände der Problemstellung mit konkurrierenden Anfragen im Netzwerk RM}  \label{B2}
  \end{center}
\end{figure}

\begin{table}
\begin{footnotesize}
    \caption{Ergebnistabelle für die Problemstellung mit konkurrierenden Anfragen im Netzwerk RM} \label{Tab2}
          \begin{center}
    \vspace*{3mm}
\csvautotabular{data/beispiel2.csv}
      %{\footnotesize \textbf{In Anlehnung an:} \cite{gonsch2013using}, S. 113.} 
            \end{center}
\end{footnotesize}
\end{table}

Mit dem Beispiel wird klar, dass aufgrund des höheren Ertrags $r_j$ und der höheren Wahrscheinlichkeit $p_j(t)$ über den gesamten Verlauf des Buchungshorizonts $T$ die Annahme des Produktauftrags $j=2$ im Systemzustand $[3\;4]$ die optimale Politik ist. Sofern eine Produktanfrage $j=1$ zu dieser Periode $t$ eintrifft, sollte die Ablehnung erfolgen. Dies resultiert aus der Tatsache, dass die $OC_j$ den Ertrag $r_j$ des Auftrags übersteigen. In den $OC_j$ des Auftrags $j=2$ sind die nachfolgenden Erwartungswerte gebündelt. Aufgrund der Parameterausprägung des Netzwerks (Wahrscheinlichkeitsverteilung $p_j(t)$ und der potentielle Ertrag $r_j$ für die Anfragen $j\in\mathcal{J}$) ist die potentielle Annahme einer Anfrage $j=2$ möglich und für den Gesamterlös des Netzwerks ertragreicher. Jedoch ist in der Abbildung \ref{B2} ebenfalls zu erkennen, dass sofern am Anfang des Buchungshorizonts keine Anfrage eintrifft, die optimale Politik sich verändert. Die $OC_j$ einer Produktanfrage $j=1$ nehmen im Zeitverlauf ab und die Entscheidung über die Annahme einer Anfrage $j=1$ wird der optimalen Politik ergänzt. Für eine strategische Entscheidung ist der beste Pfad des Netzwerks $[2\;4] \rightarrow_{j=2} [1\;3] \rightarrow_{j=2} [0\;2] \rightarrow_{KA} [0\;1]\rightarrow_{LA} [0\;0]$. Ein Unternehmen muss damit das Ziel verfolgen, dass die Verteilung der Auftragseingänge sich zu Gunsten der Produktanfrage $j=2$ bewahrheiten. Damit wäre ein Gesamterlös in Höhe von $300$ GE generiert. 










