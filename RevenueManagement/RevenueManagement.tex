\chapter{Das Konzept des Revenue Managements bei der Annahme von Aufträgen}
\markboth{2 Revenue Management in der Auftragsannahme}{}
\setcounter{footnote}{4}  %um durchgehende Fußnotennummerierung zu haben, hier die Anzahl der bisherigen Fußnoten eintragen


\section{Herkunft des Revenue Managements}
Der Begriff \textit{Revenue Management} wird im deutschsprachigen Raum meist mit \textit{Ertrags}\-\textit{management} oder \textit{Erlösmanagement} übersetzt.\footnote{Vgl. z. B. \cite{zehle1991yield}, S. 486} Yield Management wird als Synonym benutzt.\footnote{Vgl. z. B. \cite{kolisch2006revenue}, S. 319} Dabei greift der Begriff zu kurz, da  Yield im Luftverkehr den Erlös je Passagier und geflogener Meile bezeichnet.\footnote{Vgl. z. B. \cite{weatherford1998tutorial}, S. 69} Daher hat sich der Term \textit{Revenue Management} gegenüber Yield Management durchgesetzt.\footnote{Vgl. \cite{Klein:2008aa}, S. 6} Erste Ansätze des RM wurden in der Praxis entwickelt. Durch die Deregulierung des amerikanischen Luftverkehrsmarktes im Jahr 1978 mussten die traditionellen Fluggesellschaften ihre Wettbewerbsfähigkeit gegenüber Billiganbietern erhöhen und entwickelten das frühe RM.\footnote{Vgl. \cite{Petrick:2009aa}, S. 1-3} In der Literatur ist der Begriff des RM unterschiedlich definiert. \cite{friege1996yield} bezeichnet das RM als \textit{Preis-Mengen-Steuerung}, \cite{daudel1992yield} als \textit{Preis-Kapazitäts-Steuerung} und \cite{talluri2004theory} verstehen es als das gesamtes \textit{Management der Nachfrage}. \citeauthor{klein2001revenue} (2001, S. 248) definiert RM als:

\begin{quote}
\glqq Revenue Management umfasst eine Reihe von quantitativen Methoden zur Entscheidung über Annahme oder Ablehnung unsicherer, zeitlich verteilt eintreffender Nachfrage unterschiedlicher Wertigkeit. Dabei wird das Ziel verfolgt, die in einem begrenzten Zeitraum verfügbare, unflexibel Kapazität möglichst effizient zu nutzen.\grqq
\end{quote}

\cite{Petrick:2009aa} definiert das RM als Ziel einer Unternehmung die Gesamterlöse zu maximieren, die sich aufgrund der speziellen Anwendungsgebiete ergeben. Damit definiert \cite{Petrick:2009aa} das RM als Zusammenfassung aller Interaktionen eines Unternehmens, die mit dem Markt, also der Absatz- oder Nachfrageseite, zusammenhängen. \cite{kimms2005revenue} weisen darauf hin, dass eine differenzierte Betrachtung des Konzepts notwenig ist: Einerseits im Hinblick auf die \textbf{Anwendungsvoraussetzungen} und andererseits im Hinblick auf die \textbf{Instrumente des Revenue Managements}, damit verdeutlicht dargestellt ist, in welchen Branchen das RM Potentiale liefert. Dabei sollten branchenspezifische Besonderheiten, neben den zahlreichen Ähnlichkeiten Berücksichtigung finden, sowie das begrenzte Kapazitätenkontingent, damit die Potentiale des RM zur Maximierung der Gesamterlöse in den Dienstleistungsbranchen erfolgen kann.\footnote{Vgl. z. B. \cite{Martens:2009aa}, S. 11-24}\\

\section{Anwendungsvoraussetzungen und Instrumente des Revenue Managements}

Es wurden typische Anwendungsgebiete für das RM aufgezeigt. (???) Jedoch bereitet die Definition weitere Schwierigkeiten. \cite{kimms2005revenue} versuchen durch eine umfangreiche Diskussion einige Erklärungsansätze aufzuzeigen. Zum einen hat das RM vor allem aus dem älteren, englischsprachigen Bereich einen engen Bezug zu konkreten Anwendungsgebieten. Weiter versuchen viele Autoren das komplexe Konzept des Revenue Managements in einer kurzen Erklärung zu verdeutlichen. Dieses läuft letztlich darauf hinaus, dass diese Autoren einige situative Merkmale und Instrumente des Managements vermischen, gleichzeitig aber versuchen, die Zielsetzung festzulegen und das Anwendungsgebiet auf bestimmte Branchen zu beschränken. 

  Die beiden ersteren Definitionen können als Synonym für eines der Instrumente des RM stehen und daher finden diese für das gesamte Konzept keine weitere Verwendung.\footnote{Vgl. z. B. \cite{Petrick:2009aa}}\\

Im Kern lassen sich drei wichtige Perspektiven für eine Definition des Revenue Managements nach \cite{Petrick:2009aa}, \cite{stuhlmann2000kapazitatsgestaltung},  \cite{corsten1999yield} übernehmen:
\begin{itemize}
	\item Ziel ist es die Gesamterlöse unter möglichst optimaler Auslastung der vorhandenen Kapazitäten zu maximieren.
	\item Durch eine aktive Preispolitik wird das reine Kapazitäts- oder Auslastungsmanagement unterstützt.
	\item Für die erfolgreiche Implementierung des Revenue Managements ist eine umfangreiche Informationsbasis notwendig. Es muss u. a. eine möglichst gute Prognose über die zukünftige Nachfrage und Preisbereitschaft der Kunden vorhanden sein.
\end{itemize}
\vspace{0.2cm}
Wie im vorherigen Abschnitt beschrieben, müssen bestimmte Voraussetzungen bestehen, damit die Instrumente des RM zur Anwendung kommen können.

\cite{Petrick:2009aa} weist da\-rauf hin, dass anhand von Anwendungsvoraussetzungen geprüft wird, ob das RM für die jeweilige Situation des Unternehmens (oder die gesamte Branche) zur Maximierung des Gesamterlöses beiträgt. \cite{kimes1989yield} definiert die in der Literatur häufigsten Anwendungsvoraussetzungen:\footnote{Vgl. u. a. \cite{friege1996yield}, S. 616-622, und \cite{weatherford1992taxonomy}, 831-832}
\begin{itemize*}
	\item \glqq weitgehend fixe\grqq\;Kapazitäten
	\item \glqq Verderblichkeit\grqq\;bzw. \glqq Nichtlagerfähigkeit\grqq\;der Kapazitäten und der Leistung
	\item Möglichkeit zur Vorausbuchung von Leistungen
	\item stochastische, schwankende Nachfrage
	\item hohe Fixkosten für die Bereitstellung der gesamten Kapazitäten bei vergleichsweise geringen variablen Kosten für Produktion einer Leistungseinheit
	\item Möglichkeit zur Marktsegmentierung und im Ergebnis dessen zur segmentorientierten Preisdifferenzierung
\end{itemize*}
\vspace{0.2cm}
\cite{Klein:2008aa} setzen sich mit den Anwendungsvoraussetzungen von mehreren Autoren auseinander. Sie konnten Gemeinsamkeiten innerhalb der Definitionen der Autoren finden, aber zeigten auch die Unterschiede und die Kritiken auf. In ihrer Arbeit übernehmen sie die Anwendungsvoraussetzung von \cite{corsten1998yield}: "Marktseitige Anpassungserfordernis steht unternehmesseitigig unzureichendes Flexibilitätspotential hinsichtlich der Kapazität - bezogen auf Mittel- oder Zeitaufwand gegenüber". Zugleich weisen sie jedoch darauf hin, dass zum Verständnis eines komplexen und interdisziplinären Ansatzes auch die Definitionen anderer Autoren im Hinblick auf das Verständnis der Anwendungsvoraussetzungen beitragen.\\

Auf Grundlage der von \cite{friege1996yield} beschriebenen Anwendungsvoraussetzungen hat \cite{Petrick:2009aa} drei Instrumente des RM bestimmt. Die Instrumente benötigen als Grundlage \textit{Daten der Prognose}, damit sie zur Anwendung kommen.\footnote{Die Prognose zählt laut \cite{Petrick:2009aa} nicht als eigenständiges Instrument des RM.} Zu den Instrumenten zählen die \textbf{segmentorientierte Preisdifferenzierung}, die \textbf{Kapazitäten\-steuerung} und die \textbf{Über\-buchungssteuerung}. Es lassen sich unterschiedliche Ab\-hängigkeit\-en der Instrumente untereinander ermitteln.\footnote{Als Beispiel baut die Kapazitätensteu\-erung auf den Ergebnissen der Preisdifferenzierung auf und die Überbuchungssteuerung kann selten ohne Kapazitätensteuerung gelöst werden.}

\section{Mathematische Modellformulierung des Revenue Managements beim Entscheidungsproblem der Auftragsannahme von Instandhaltungsprozessen}
Im Folgenden wird das dynamisch, stochastische Grundmodell des RM nach \citeauthor{talluri2004revenue} (2004, S. 18-19) beschrieben. Ein Dienstleistungsnetzwerk eines Anbieters benötigt jeweils zur Erstellung einer Dienstleistung eine bestimmte Kombination an Ressourcen aus der Menge der Ressourcen $\mathcal{H} = \{1,...,l \}$. Der Index $h$ beschreibt dabei eine jeweilige Ressource und der Index $l$ die gesamte Anzahl an möglichen Ressourcen. Die jeweilig verbleibende Kapazität einer Ressource $h \in \mathcal{H}$ ist durch den Parameter $c_{h}$ beschrieben und die gesamten Kapazitäten der Ressourcen ist als Vektor $\textbf{c}=(c_{1},...,c_{h},...,c_{l})$ formuliert. Ein Produkt in dem Netzwerk ist durch den Parameter $j$ aus der Menge an Produkten $\mathcal{J} = \{1,...,n \}$ %für die Menge der Ressourcen $\mathcal{H}$
beschrieben. Die gesamte Anzahl an Produkten ist durch den Parameter $n$ definiert. Sobald ein Produkt $j\in \mathcal{J}$ abgesetzt ist, fällt für den Verkauf der Ertrag $r_{j}$ an. Der Buchungshorizont entspricht $T$ Perioden und kann jeweils in einzelne Perioden $t=1,...,T$ aufgeteilt werden. Dabei muss Beachtung finden, dass der Buchungshorizont $T$ gegenläufig verläuft. Die Wahrscheinlichkeit der Nachfrage eines Produkts $j$ in der Periode $t$ entspricht $p_{j}(t)$ und die Wahrscheinlichkeit, dass keine Nachfrage in der Periode $t$ eintrifft, entspricht $p_{0}(t)$. Es gilt $\sum_{j\in \mathcal{J}}p_{j}(t)+p_{0}(t)=1$ und somit kann $p_{0}(t)$ durch den Term $p_{0}(t)=1-\sum_{j\in \mathcal{J}}p_{j}(t)$ für die Periode $t$ ermittelt werden.\footnote{Vgl. \citeauthor{talluri2004revenue}, S. 18} Die noch erwartete Nachfrage $D_{jt}$ für ein bestimmtes Produkt $j$ für eine beliebige Periode $t$ lässt sich durch $\sum_{\tau=1}^{t}p_{j}(\tau)$ aggregieren.\\

Die bisherige Notation ist analog der Formulierung des Grundmodells nach \citeauthor{talluri2004revenue} (2004, S. 18-19). Nachfolgend wird die Modellerweiterung nach \cite{gonsch2013using} beschrieben. Sofern ein Anbieter opake Produkte in sein Produktportfolio integriert, muss die Menge $\mathcal{M}_{j}\subseteq\mathcal{I}$ eingeführt werden. Mit dieser Menge ist die Erfassung des differenzierten Ressoucenverbrauchs der spezifischen und opaken Produkte $j\in\mathcal{J}$ möglich. Sie ist eine Teilmenge der Indexmenge $\mathcal{I}\subseteq\mathbb{N}^{+}$, die alle produktspezifischen Kombinationen für die Menge der Ressourcen $\mathcal{H}$ beschreibt. Die Menge $\mathcal{I}$ beschreibt alle möglichen Kombinationen von Produkten und Ressourcen. Für das weitere Vorgehen genügt das Betrachten der jeweiligen möglichen Ausführungsmodi $\mathcal{M}_{j}$ eines Produkts $j\in\mathcal{J}$. Eine einzelne produktspezifische Kombination der verfügbaren Ressourcen ist durch den Parameter $m\in\mathcal{M}_{j}$ beschrieben. Der jeweilige Verbrauch einer Ressource $h$ im Ausführungsmodus $m$ durch Annahme einer Anfrage nach einem Produkt $j$ ist anhand des Parameters $a_{hm}$ beschrieben. Durch Vektorschreibweise kann der Ressourcenverbrauch einer produktspezifischen Kombination als $\textbf{a}_{m}=(a_{1m},...,a_{hm},...,a_{lm})$ formuliert werden. Die Tabelle \ref{Tabelle0} verdeutlicht den Zusammenhang der Produkte $j\in\mathcal{J}$ und der Ressourcen $h\in\mathcal{H}$ mit dem dazugehörigen produktspezifischen Ausführungsmodus $m\in\mathcal{M}_{j}$ in einer Matrix. \\

\begin{table}[h!]
 \renewcommand{\arraystretch}{1.5}
  \setlength{\tabcolsep}{4mm}
  \begin{center}
    \caption{Beispiel einer Produkt-Ressourcen-Matrix}  \label{Tabelle0}
    \vspace*{3mm}
    \begin{tabular}{c|c|c|c|c|c|c|c}   %hier die Spaltenausrichtung, -breite, -begrenzung und -anzahl eintragen
     \multirow{3}{*}{Produkt $j$} & \multicolumn{5}{c|}{Ressurcenverbrauch $a_{hm}$} & \multirow{3}{*}{Ausführungsmodus $m$} & \multirow{3}{*}{Erlös $r_{j}$}\\
           & \multicolumn{5}{c|}{für jeweilige Ressource $h$} &  & \\ 
           & $1$ & $2$ & $3$ & $4$ &$5$ &  \\ \hline
         
         \multirow{4}{*}{$1$}  & $a_{11}$ & -- & $a_{31}$ & $a_{41}$ & -- & 1 &  \multirow{4}{*}{$100$}  \\ %\cline{2-7}
                & \cellcolor[gray]{0.9}--& \cellcolor[gray]{0.9}$a_{22}$ & \cellcolor[gray]{0.9}$a_{32}$ & \cellcolor[gray]{0.9}$a_{42}$ & \cellcolor[gray]{0.9}--& \cellcolor[gray]{0.9}2 & \\ %\cline{2-7}
                & $a_{13}$& --& --& $a_{43}$ & --& 3 &   \\ %\cline{2-7}
          &\cellcolor[gray]{0.9}--& \cellcolor[gray]{0.9}$a_{24}$& \cellcolor[gray]{0.9}-- & $\cellcolor[gray]{0.9}a_{44}$ & \cellcolor[gray]{0.9}--& \cellcolor[gray]{0.9}4 &   \\ \hline
                              2  & $a_{15}$ &-- & -- &--& $a_{55}$ & 5 &  150 \\ \hline
    \end{tabular} \\[3mm]
  \end{center}
\end{table}

Das Beispiel in Tabelle \ref{Tabelle0} zeigt einen Reiseveranstalter mit fünf Ressourcen. Bei den ersten zwei Ressourcen handelt es sich um einen 1.-Klasse-Flug ($h=1$) und um einen 2.-Klasse-Flug ($h=2$). Bei der Ressource $3$ handelt es sich jeweils um eine mögliche Überführungsfahrt ($h=3$) zu einem Hotel am Strand ($h=4$). Bei der letzten Ressource handelt es sich um ein Business-Hotel direkt am Flughafen ($h=5$). Die verschiedenen Ressourcen sind in dem Beispiel unterschiedlich zu Produkten $j\in\mathcal{J}$ kombiniert. Dies könnte einerseits daran liegen, dass einige Kombinationen für einen Anbieter nicht rentabel sind oder andererseits keine Nachfrage erhalten. Damit zeigt die Produkt-Ressourcen-Matrix nicht alle möglichen Kombinationsmöglichkeiten $i\in\mathcal{I}$ für die Menge an Ressourcen $\mathcal{H}$. Die Matrix zeigt keinen Wert für den Ressourcenverbrauch an, sofern $a_{hm}=0$ entspricht. Damit wird zur Erstellung des Produkts $j$ in dem Ausführungsmodus $m$ die jeweilige Ressource $h$ nicht benötigt. In dem Beispiel ist das Produkt $j=1$ ein opakes Produkt mit den Ausführungsmodi $m\in\mathcal{M}_{1}=\{1,2,3,4\}$. Dies liegt an der frei gewählten Angebotsstruktur. Der Anbieter könnte jeden Ausführungsmodus $m\in\mathcal{M}_{j}$ für ein eigenständiges Produkt nutzen. Bspw. handelt es sich bei dem Produkt $j=2$ um ein spezifisches Produkt, das der Anbieter nur in einem Ausführungsmodus $m\in\mathcal{M}_{2}=\{5\}$ anbietet. Somit sind spezifische Produkte eines Netzwerks als Sonderfall von opaken Produkten anzusehen, die nur einen Ausführungsmodus aufweisen ($|\mathcal{M}_{j'}|=1$).\footnote{Vgl. \cite{gonsch2013using}, S. 96}\\

Mit den vorangegangenen Parametern kann der maximal erwartete Ertragswert $V(\textbf{c},t)$ für eine Periode $t$ bei einer noch vorhandenen Ressourcenkapazität $\textbf{c}$ als Bellman-Gleichung formuliert werden (\textbf{DP-op}):\footnote{Vgl. \cite{gonsch2013using}, S. 97}

\begin{equation}\label{DPop}
V(\textbf{c},t)=\sum_{j\in\mathcal{J}}p_{j}(t)\max\left( V(\textbf{c},t-1),\; r_{j}+\max_{m\in\mathcal{M}_{j}}V(\textbf{c}-\textbf{a}_{m},t-1)\right)+p_{0}(t)V(\textbf{c},t-1)
\end{equation}

Es handelt sich hier um die Modellformulierung der dynamischen Programmierung im RM opaker Produkte. Die Gleichung weist die Grenzbedingungen $V(\textbf{c},0)=0$ für $\textbf{c}\ge0$ sowie sonst $V(\textbf{c},0)=-\infty$ auf, da eine jeweilig verbleibende Kapazität nach Bereitstellung des Produkts wertlos und eine negative Ressourcenkapazität nicht möglich ist. Die Standardformulierung der dynamischen Programmierung wird mit dem Term $\max_{m\in\mathcal{M}_{j}}V(\textbf{c}-\textbf{a}_{m},t-1)$ erweitert. Damit ist sichergestellt, dass eine Anfrage nach einem opaken Produkt $j$ nur im Ausführungsmodus $m$ mit dem höchsten Ertragswert gewählt wird. Der Gesamtertrag des Anbieters ist maßgeblich durch die Entscheidung der gewählten Ausführungs\-modi $m\in\mathcal{M}_{j}$ abhängig, da das Modell durch eine jede Entscheidung bzgl. der weiteren möglichen opaken Produkte neu gelöst werden muss. Eine eintreffende Anfrage nach einem opaken Produkt $j$ ist demnach dann akzeptiert, wenn gilt:
\begin{equation}\label{r}
r_{j}\ge\min_{m\in\mathcal{M}_{j}}\bigl\{V(\textbf{c},t-1)-V(\textbf{c}-\textbf{a}_{m},t-1)\bigr\}
\end{equation}


Somit erfolgt die Akzeptanz einer Anfrage nach einem opaken Produkt $j\in\mathcal{J}$ ausschließlich nur dann, sofern die OK des Ressourcenverbrauchs niedriger als der Ertrag ist. Zusätzlich wird der Ausführungsmodus $m\in\mathcal{M}_{j}$ mit den niedrigsten OK gewählt, wodurch das Maximum des gesamten Ertragswerts gewährleistet bleibt. Ein potentieller Ausführungs\-modus $m^{*}$ mit minimalen OK ($V(\textbf{c},t-1)-V(\textbf{c}-\textbf{a}_{m^{*}},t-1)$) ist gewählt und die Kapazität werden dementsprechend reduziert. Bei spezifischen Produkten existiert nur ein Ausführungsmodus $|\mathcal{M}_{j'}|=1$ und daher ist die Maximalfunktion in der Gleichung \eqref{DPop} und die Minimalfunktion in der Gleichung \eqref{r} nicht notwendig.