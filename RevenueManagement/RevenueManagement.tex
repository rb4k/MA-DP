% !TEX encoding = UTF-8 Unicode
\chapter{Das Konzept des Revenue Managements bei der Annahme von Aufträgen}
\markboth{2 Revenue Management in der Auftragsannahme}{}
\setcounter{footnote}{4}  %um durchgehende Fußnotennummerierung zu haben, hier die Anzahl der bisherigen Fußnoten eintragen


\section{Das Revenue Managements bei der Auftragsannahme}
Zur Entscheidungsunterstützung bei der Annahme von Kundenaufträgen wird in der aktuellen Forschung vermehrt auf das Konzept des Revenue Managements zurückgegriffen.\footnote{Vgl. \cite{klein2001revenue}, S. 246.} Da eine kurzfristige Anpassung der mittelfristig bereitgestellten Kapazitäten einer Dienstleistungsproduktion an eine unsichere und schwankende Nachfrage nicht möglich ist, wird mit dem Konzept eine effizientere Auslastung der bestehenden Kapazitäten ermöglicht.\footnote{Vgl. \cite{ing2005revenue}, S. 124.}

Der Begriff \textit{Revenue Management} wird im deutschsprachigen Raum meist mit \textit{Ertrags}\-\textit{management} oder \textit{Erlösmanagement} übersetzt.\footnote{Vgl. z. B. \cite{zehle1991yield}, S. 486} Yield Management wird als Synonym benutzt.\footnote{Vgl. z. B. \cite{kolisch2006revenue}, S. 319} Dabei greift der Begriff \textit{Yield} zu kurz, da damit in der Luftverkehrsbranche der Erlös je Passagier und geflogener Meile bezeichnet wird.\footnote{Vgl. z. B. \cite{weatherford1998tutorial}, S. 69} Der Term \textit{Revenue Management} hat sich jedoch gegenüber Yield Management durchgesetzt, da der Yield (Durchschnittsertrag) sich theoretisch durch nur einen Passagier maximieren lässt und somit die Maximierung als Zielsetzung nicht sinnvoll ist.\footnote{Vgl. \cite{Klein:2008aa}, S. 6; \cite{ing2005revenue}, S. 124-125.} Erste Ansätze des RM sind in der Praxis entwickelt. Durch die Deregulierung des amerikanischen Luftverkehrsmarktes im Jahr 1978 mussten die traditionellen Fluggesellschaften ihre Wettbewerbsfähigkeit gegenüber Billiganbietern erhöhen und entwickelten das frühe RM.\footnote{Vgl. \cite{Petrick:2009aa}, S. 1-3}

\cite{kimms2005revenue} versuchen durch eine umfangreiche Diskussion einige Erklärungsansätze aufzuzeigen (Warum? Wovon?). Zum einen hat das RM vor allem aus dem älteren, englischsprachigen Bereich einen engen Bezug zu konkreten Anwendungsgebieten. Sie zeigen auf, dass viele Autoren versuchen das komplexe Konzept des Revenue Managements in einer kurzen Erklärung zu überführen. Dieses läuft letztlich darauf hinaus, dass diese Autoren einige situative Merkmale und Instrumente des Managements vermischen, gleichzeitig aber versuchen, die Zielsetzung festzulegen und das Anwendungsgebiet auf bestimmte Branchen zu beschränken. %\cite{kimms2005revenue} weisen darauf hin, dass eine differenzierte Betrachtung des Konzepts notwenig ist: Einerseits im Hinblick auf die Anwendungsvoraussetzungen und andererseits im Hinblick auf die Instrumente des Revenue Managements, damit verdeutlicht dargestellt wird, in welchen Branchen das RM Potentiale liefert. In den nachfolgenden zwei Kapiteln wird dieser Empfehlung gefolgt.

Weiter wird in der Literatur der Begriff des RM unterschiedlich definiert. \cite{friege1996yield} bezeichnet das RM als \textit{Preis-Mengen-Steuerung}, \cite{daudel1992yield} als \textit{Preis-Kapazitäts-Steuerung} und \cite{talluri2004theory} verstehen es als das gesamtes \textit{Management der Nachfrage}. Die beiden ersteren Definitionen können als Synonym für eines der Instrumente des RM stehen und daher finden diese für das gesamte Konzept keine weitere Verwendung.\footnote{Vgl. z. B. \cite{Petrick:2009aa}} Nachfolgend wird die Definition von \citeauthor{klein2001revenue} (2001, S. 248) aufgegriffen:

\begin{quote}
\glqq Revenue Management umfasst eine Reihe von quantitativen Methoden zur Entscheidung über Annahme oder Ablehnung unsicherer, zeitlich verteilt eintreffender Nachfrage unterschiedlicher Wertigkeit. Dabei wird das Ziel verfolgt, die in einem begrenzten Zeitraum verfügbare, unflexibel Kapazität möglichst effizient zu nutzen.\grqq
\end{quote}

\cite{Petrick:2009aa} definiert das RM als Ziel einer Unternehmung die Gesamterlöse zu maximieren, die sich aufgrund der speziellen Anwendungsgebiete ergeben. Damit definiert \cite{Petrick:2009aa} das RM als Zusammenfassung aller Interaktionen eines Unternehmens, die mit dem Markt, also der Absatz- oder Nachfrageseite, zusammenhängen. Im Kern lassen sich drei wichtige Perspektiven für eine Definition des Revenue Managements nach \cite{Petrick:2009aa}, \cite{stuhlmann2000kapazitatsgestaltung},  \cite{corsten1999yield} übernehmen:
\begin{enumerate}
	\item Ziel ist es die Gesamterlöse unter möglichst optimaler Auslastung der vorhandenen Kapazitäten zu maximieren.
	\item Durch eine aktive Preispolitik wird das reine Kapazitäts- oder Auslastungsmanagement unterstützt.
	\item Für die erfolgreiche Implementierung des Revenue Managements ist eine umfangreiche Informationsbasis notwendig. Es muss u. a. eine möglichst gute Prognose über die zukünftige Nachfrage und Preisbereitschaft der Kunden vorhanden sein.
\end{enumerate}

\cite{kimms2005revenue} weisen darauf hin, dass eine differenzierte Betrachtung des Konzepts notwenig ist: Einerseits im Hinblick auf die \textbf{Anwendungsvoraussetzungen} und andererseits im Hinblick auf die \textbf{Instrumente des Revenue Managements}, damit verdeutlicht dargestellt ist, in welchen Branchen das RM Potentiale liefert. Dabei sollten branchenspezifische Besonderheiten, neben den zahlreichen Ähnlichkeiten Berücksichtigung finden, sowie das begrenzte Kapazitätenkontingent, damit die Potentiale des RM zur Maximierung der Gesamterlöse in den Dienstleistungsbranchen erfolgen kann.\footnote{Vgl. z. B. \cite{Martens:2009aa}, S. 11-24} In dem nachfolgenden Kapitel wird dieser Empfehlung gefolgt und die Anwendungsvoraussetzungen sowie Instrumente des RM vorgestellt.

\section{Anwendungsvoraussetzungen und Instrumente des Revenue Managements}

\cite{Petrick:2009aa} weist da\-rauf hin, dass anhand von speziellen Anwendungsvoraussetzungen geprüft wird, ob das RM für die jeweilige Situation des Unternehmens (oder die gesamte Branche) zur Maximierung des Gesamterlöses beiträgt. \cite{kimes1989yield} definiert die in der Literatur häufigsten Anwendungsvoraussetzungen:\footnote{Vgl. u. a. \cite{friege1996yield}, S. 616-622, und \cite{weatherford1992taxonomy}, 831-832}
\begin{itemize}
	\item \glqq weitgehend fixe\grqq\;Kapazitäten
	\item \glqq Verderblichkeit\grqq\;bzw. \glqq Nichtlagerfähigkeit\grqq\;der Kapazitäten und der Leistung
	\item Möglichkeit zur Vorausbuchung von Leistungen
	\item stochastische, schwankende Nachfrage
	\item hohe Fixkosten für die Bereitstellung der gesamten Kapazitäten bei vergleichsweise geringen variablen Kosten für Produktion einer Leistungseinheit
	\item Möglichkeit zur Marktsegmentierung und im Ergebnis dessen zur segmentorientierten Preisdifferenzierung
\end{itemize}
\vspace{0.2cm}
\cite{Klein:2008aa} setzen sich mit den Anwendungsvoraussetzungen von mehreren Autoren auseinander. Sie konnten Gemeinsamkeiten innerhalb der Definitionen der Autoren finden, aber zeigten auch die Unterschiede und die Kritiken auf. In ihrer Arbeit übernehmen sie die Anwendungsvoraussetzung von \cite{corsten1998yield}: "Marktseitige Anpassungserfordernis steht unternehmesseitigig unzureichendes Flexibilitätspotential hinsichtlich der Kapazität -- bezogen auf Mittel- oder Zeitaufwand -- gegenüber". Zugleich weisen sie jedoch darauf hin, dass zum Verständnis eines komplexen und interdisziplinären Ansatzes auch die Definitionen anderer Autoren im Hinblick auf das Verständnis der Anwendungsvoraussetzungen beitragen.

Auf Grundlage der von \cite{friege1996yield} beschriebenen Anwendungsvoraussetzungen hat \cite{Petrick:2009aa} drei Instrumente des RM bestimmt. Die Instrumente benötigen als Grundlage \textit{Daten der Prognose}, damit sie zur Anwendung kommen.\footnote{Die Prognose zählt laut \cite{Petrick:2009aa} nicht als eigenständiges Instrument des RM.} Zu den Instrumenten zählen die \textbf{segmentorientierte Preisdifferenzierung}, die \textbf{Kapazitäten\-steuerung} und die \textbf{Über\-buchungssteuerung}. Es lassen sich unterschiedliche Ab\-hängigkeit\-en der Instrumente untereinander ermitteln.\footnote{Als Beispiel baut die Kapazitätensteu\-erung auf den Ergebnissen der Preisdifferenzierung auf und die Überbuchungssteuerung kann selten ohne Kapazitätensteuerung gelöst werden.}

Erklärung segmentorientierte Preisdifferenzierung, Kapazitätssteuerung, Überbuchungssteuerung

Übergang zu Instandhaltung

\section{Mathematische Modellformulierung des Revenue Managements}
Im Folgenden wird das dynamisch, stochastische Grundmodell des RM nach \citeauthor{talluri2004revenue} (2004, S. 18-19) beschrieben. Ein Dienstleistungsnetzwerk eines Anbieters benötigt jeweils zur Erstellung der Dienstleistungen ein bestimmtes Kontingent an Ressourcen aus einer Menge an Ressourcen $\mathcal{H} = \{1,...,l \}$. Der Index $h$ beschreibt dabei eine jeweilige Ressource und der Index $l$ die gesamte Anzahl an möglichen Ressourcen. Die jeweilige Kapazität einer Ressource $h \in \mathcal{H}$ ist durch den Parameter $c_{h}$ beschrieben und die gesamten Kapazitäten der Ressourcen ist als Vektor $\textbf{c}=(c_{1},...,c_{h},...,c_{l})$ formuliert. Eine Anfrage nach einem Produkt (Dienstleistung) in dem Netzwerk ist durch den Parameter $j$ aus der Menge an möglichen Produktanfragen $\mathcal{J} = \{1,...,n \}$ %für die Menge der Ressourcen $\mathcal{H}$
beschrieben. Die gesamte Anzahl an Produktanfragen ist durch den Parameter $n$ definiert. Sobald eine Produktanfrage $j\in \mathcal{J}$ akzeptiert und somit abgesetzt ist, fällt für den Absatz der Ertrag $r_{j}$ an. Der jeweilige Verbrauch einer Ressource $h$ durch Annahme einer Anfrage nach einem Produkt $j$ ist anhand des Parameters $a_{hj}$ beschrieben. Durch Vektorschreibweise kann der Ressourcenverbrauch für eine Anfrage nach einem Produkt $j$ als Vektor $\textbf{a}_{j}=(a_{1j},...,a_{hj},...,a_{lj})$ formuliert werden. Der Buchungshorizont entspricht $T$ Perioden und kann jeweils in einzelne Perioden $t=1,...,T$ aufgeteilt werden. Dabei muss Beachtung finden, dass der Buchungshorizont $T$ gegenläufig verläuft. Die Wahrscheinlichkeit der Nachfrage eines Produkts $j$ in der Periode $t$ entspricht $p_{j}(t)$ und die Wahrscheinlichkeit, dass keine Nachfrage in der Periode $t$ eintrifft, entspricht $p_{0}(t)$. Es gilt $\sum_{j\in \mathcal{J}}p_{j}(t)+p_{0}(t)=1$ und somit kann $p_{0}(t)$ durch den Term $p_{0}(t)=1-\sum_{j\in \mathcal{J}}p_{j}(t)$ für die Periode $t$ ermittelt werden.\footnote{Vgl. \citeauthor{talluri2004revenue}, S. 18} Die noch erwartete Nachfrage $D_{jt}$ für ein bestimmtes Produkt $j$ für eine beliebige Periode $t$ lässt sich durch $\sum_{\tau=1}^{t}p_{j}(\tau)$ aggregieren. 


Mit den vorangegangenen Parametern kann der maximal erwartete Ertragswert $V(\textbf{c},t)$ für eine Periode $t$ bei einer noch vorhandenen Ressourcenkapazität $\textbf{c}$ als Bellman-Gleichung formuliert werden (\textbf{DP-op}):\footnote{???}
\begin{equation}\label{DP}
V(\textbf{c},t)=\sum_{j\in\mathcal{J}}p_{j}(t)\max[ V(\textbf{c},t-1),\; r_{j}+V(\textbf{c}-\textbf{a}_{j},t-1)]+p_{0}(t)V(\textbf{c},t-1)
\end{equation}


Es handelt sich hier um die Modellformulierung der dynamischen Programmierung im Netzwerk Revenue Management. 

Was ist mit Grundlagen DP?

Die Gleichung weist die Grenzbedingungen
\begin{equation}\label{GB1}
V(\textbf{c},0)=0 \text{ wenn } \textbf{c}\ge0 \text{ sowie }
\end{equation}
\begin{equation}\label{GB2}
V(\textbf{c},t)=-\infty \text{ wenn } c_{j}<0 \;\forall j\in\mathcal{J}
\end{equation}
auf, da eine jeweilig verbleibende Kapazität nach Bereitstellung des Produkts wertlos und eine negative Ressourcenkapazität nicht möglich ist. 

Die Gleichung \eqref{DP} lässt sich umformen, indem die Entscheidungen über die Annahme und Ablehnung einer Produktanfrage separiert wird. Da $\sum_{j\in \mathcal{J}}p_{j}(t)+p_{0}(t)=1$ gilt, kann die Gleichung weiter vereinfacht werden:\footnote{Vgl. \cite{Spengler:2007aa}, S. 161.}
\begin{equation*}
V(\textbf{c},t)=\sum_{j\in\mathcal{J}}p_{j}(t) V(\textbf{c},t-1) + \sum_{j\in\mathcal{J}}p_{j}(t) \max[r_{j}-V(\textbf{c},t-1)+V(\textbf{c}-\textbf{a}_{j},t-1,0)]+p_{0}(t)V(\textbf{c},t-1)
\end{equation*}
\begin{equation}\label{DP2}
=V(\textbf{c},t-1) + \sum_{j\in\mathcal{J}}p_{j}(t) \max[r_{j}-V(\textbf{c},t-1)+V(\textbf{c}-\textbf{a}_{j},t-1),0]
\end{equation}

Eine eintreffende Anfrage nach einem Produkt $j$ ist demnach dann akzeptiert, wenn der Ertrag $r_{j}$ größer gleich der Differenz des Erwartungswertes des Ertrags unter der Prämisse der Annahme der Produktanfrage und des Erwartungswertes des Ertrag unter der Prämisse der Ablehnung der Produktanfrage ist:
\begin{equation}\label{r}
r_{j} \ge V(\textbf{c},t-1)-V(\textbf{c}-\textbf{a}_{j},t-1)
\end{equation}

Dabei kann der rechte Term \eqref{r} als Opportunitätskosten (OK) der Auftragsannahme angesehen werden:
\begin{equation}\label{OC}
OC_{j} = V(\textbf{c},t-1)-V(\textbf{c}-\textbf{a}_{j},t-1)
\end{equation}

Somit erfolgt die Akzeptanz einer Anfrage nach einem Produkt $j\in\mathcal{J}$ ausschließlich nur dann, sofern die OK des Ressourcenverbrauchs niedriger als der Ertrag ist. Der maximal mögliche Erwartungswert unter Beachtung des Kapazität $\textbf{c}$ zum Zeitpunkt $t$ ist damit der Erwartungswert unter der Prämisse der Ablehnung der Anfrage zum nächsten Zeitpunkt $t-1$ inkl. der Summe der Erträge abzgl. der OK durch Annahme der möglichen Anfragen über aller Produkte $j\in\mathcal{J}$ und lässt sich mathematisch wie folgt definieren:
\begin{equation}\label{DPoc}
V(\textbf{c},t)=V(\textbf{c},t-1) + \sum_{j\in\mathcal{J}}p_{j}(t) \max[r_{j}-OC_{j},0]
\end{equation}

Die optimale Politik des Netzwerks zum Zeitpunkt $t$ ist damit die Annahme der Anfrage nach Produkt $j$ mit dem höchsten Ertrag $r_{j}$ bzgl. der anfallenden $OC_{j}$:

\begin{equation}\label{OP}
OP_{\textbf{c}, t}:= \{ \; j\; | \max_{j\in\mathcal{J}} [r_{j}-OC_{j}] \} 
\end{equation}

Zur Veranschaulichung des Netzwerk RM wird ein Netzwerk mit zwei Produkten $j\in\mathcal{J}$ und zwei Ressourcen $h\in\mathcal{H}$ betrachtet. Die Ressource $h=1$ hat eine Kapazität von $c_{1}=2$ und die Ressource $h=2$ hat eine Kapazität von $c_{2}=1$. Zur Ausführung des Produkt $j=1$ wird die Ressource $h=1$ mit einer Einheit benötigt und zur Ausführung des Produkt $j=2$ wird wiederum eine Einheit der Ressource $h=2$ gebraucht. Damit gilt $a_{11}=1$ und $a_{22}=1$. Durch Annahme einer Anfrage nach Produkt $j=1$ wird der Ertrag $r_{1}=100$ und durch Annahme von Produkt $j=2$ eine Ertrag von Ertrag $r_{2}=200$ generiert. Der Buchungshorizont entspricht $T=4$. Die Wahrscheinlichkeiten des Eintreffens eine Anfrage nach Produkt $j=1$ über die Buchungsperioden $t\in T$ lässt sich als Vektor $p_{1}(t)=(0.5, 0.5, 0.5, 0.5)$ beschreiben. Analog lassen sich die Wahrscheinlichkeiten für das Eintreffen der Produktanfragen $j=2$ als Vektor $p_{2}(t)=(0.1, 0.1, 0.1, 0.1)$ definieren. Die Gegenwahrscheinlichkeiten, dass keine Anfragen eintreffen, lassen sich mit $p_{0}(t)=1-\sum_{j\in \mathcal{J}}p_{j}(t)$ berechnen und bilden den Vektor $p_{0}(t)=(0.4, 0.4, 0.4, 0.4)$. Durch Vereinfachung der Gleichung \eqref{DP} zur Gleichung \eqref{DP2} werden die Gegenwahrscheinlichkeiten $p_{0}(t)$ nicht mehr benötigt und im weiteren Verlauf der Arbeit nicht mehr berücksichtigt.

Die Parameter lassen sich damit abschließend wie folgt definieren:
\begin{center}
$j = \{1, 2\}, \; h = \{1, 2\}, \; r_{1} = 100, \; r_{2} = 200, \; T=4$
\end{center}
\[
    \textbf{c}=\begin{pmatrix} 2 \\ 1 \end{pmatrix}, \;
    \textbf{a}_1=\begin{pmatrix} 1 \\ 0 \end{pmatrix}, \;
     \textbf{a}_2=\begin{pmatrix} 0 \\ 1 \end{pmatrix}, \;
     p_{1}(t)=\begin{pmatrix} 0.5\\ 0.5\\ 0.5\\ 0.5  \end{pmatrix}, \;
     p_{2}(t)=\begin{pmatrix} 0.1\\ 0.1\\ 0.1\\ 0.1  \end{pmatrix}
  \]

Die mathematische Modellformulierung des stochastisch, dynamisches Programms aus Gleichung \eqref{DP2} lässt sich als Graph darstellen, da es sich um eine rekursive Form handelt. Der Erwartungswert $V(\textbf{c},t)$ ist abhängig vom Erwartungswert $V(\textbf{c},t-1)$ und vom Erwartungswert $V(\textbf{c}-\textbf{a}_{j},t-1)$. Der Erwartungswert $V(\textbf{c},t-1)$ ist wiederum abhängig vom Erwartungswert $V(\textbf{c},t-2)$ und vom Erwartungswert $V(\textbf{c}-\textbf{a}_{j},t-2)$, usw. Abbildung \ref{B0} zeigt die erste rekursive Folge für die Gleichung \eqref{DP2} als Graphen auf und im nachfolgenden wird auf die Notation eingegangen.
\begin{figure}[h!]
  \begin{center}
    \includegraphics[width=80mm]{Bilder/Beispiel0.pdf}
    \caption{Beispielhafte Darstellung einer rekursive Folge eines Netzwerk RM}  \label{B0}
  \end{center}
\end{figure}
Ein Knoten repräsentiert einen Systemzustand des Netzwerks mit den vorhandenen Kapazitäten $\textbf{c}$ zum Zeitpunkt $t$. Dabei wird als Benennung für den Knoten eine Zahlenfolge verwendet, bei dem die ersten Einträge die Ressourcenkapazität $\textbf{c}$ in Länge der Ressourcen $h$ entsprechen und der letzte Eintrag den Zeitpunkt $t$ aufzeigt. Bspw. zeigt der Startknoten die Zahlenfolge $[2\;1\;4]$, da das Netzwerk noch die volle Ressourcenkapazität $\textbf{c}=(2,1)$ aufweist und sich im Zeitpunkt $t=4$ befindet. Von diesem Systemzustand können jetzt nachfolgende Systemzustände abhängig der Produktanfragen erreicht werden. In diesem Netzwerk gibt es zwei Produkte $j$ und durch betrachten der Gleichung \eqref{DP2} wird klar, dass drei Optionen zum Erreichen des nachfolgenden Systemzustands zum Zeitpunkt $t-1=3$ möglich sind. Diese Optionen bilden die Kanten des Graphen. Es kann keine Anfrage nach einem Produkt $j$ eintreffend, dann wird der nachfolgende Systemzustand $[2\;1\;3]$ erreicht. Alternativ können Anfragen nach Produkt $j=1$ oder $j=2$ eintreffen und dementsprechend müssen die vorhanden Kapazitäten $\textbf{c}$ um den Ressourcenverbrauch $a_{11}=1$ bzw. $a_{22}=1$ reduziert werden. Daraus folgt, dass der Systemzustand $[1\;1\;3]$ bzw. $[2\;0\;3]$ im Netzwerk erreicht werden kann.

Aufbauen auf dieser rekursiven Logik wird ein kompletter Entscheidungsbaum (Gerichteter und gewichteter Multigraph) aufgebaut. Er zeigt alle möglichen Systemzustände des Netzwerks. Die Rekursion wird abgebrochen, sofern ein Systemzustand aufgrund der Grenzbedingungen aus den Gleichungen \eqref{GB1} oder \eqref{GB1} nicht möglich ist. Abbildung \ref{B1} zeigt den Entscheidungsbaum für das eingeführte Beispiel.
\begin{figure}[h!]
  \begin{center}
    \includegraphics[width=150mm]{Bilder/Beispiel1.pdf}
    \caption{Beispielhafte Darstellung eines Entscheidungsbaums eines Netzwerk RM}  \label{B1}
  \end{center}
\end{figure}

Betrachten wird den Entscheidungsbaum mit den möglichen Systemzuständen aus der Abbildung \ref{B1} weiter, dann kann das Entscheidungsproblem der Auftragsannahme durch Rückwärtsinduktion gelöst werden.\footnote{Vgl. ???, S. ???.} Dafür wird im ersten Schritt die Grenzbedingung aus Gleichung \eqref{GB1} betrachtet. Alle Systemzustände zum Zeitpunkt $t=0$ nehmen den Erwartungswert $V(\textbf{c}, t=0)=0$ an. Mit dieser Bedingung lassen sich die zeitlich zuvorkommenden Systemzustände mit $t=1$ berechnen. Es wird die Gleichung \eqref{DP2} angewendet, wobei beachtet werden muss, dass nicht für alle Systemzustände mit $t=1$ alle Produktanfragen möglich sind. Dies folgt aus der Grenzbedingung aus Gleichung \eqref{GB2}. Zur Veranschaulichung wird der Erwartungswert des Systemzustands $[1\;0\;1]$ mittels der Gleichung \eqref{DP2} berechnet.
\begin{alignat*}{2}
V(1,0,1)=\;&V(1,0,0)+p_{1}(1)\max[r_{1}-V(1,0,0)+V(0,0,0),0]\\
&+p_{2}(1)\max[r_{2}-V(1,0,0)+V(0,-1,0),0]\\
=\;&0+0,5\cdot\max[100-0+0,0]+0,1\cdot\max[200-0+(-\infty),0]\\
=\;&0+0,5\cdot 100+0,1\cdot0\\
=\;&50\\
\end{alignat*}

Nach dieser Vorgehensweise der Rückwärtsinduktion erfolgt die Ermittlung alle Erwartungswerte der möglichen Systemzustände des Entscheidungsbaum. Tabelle \ref{Tab1} zeigt für den Entscheidungsbaum alle möglichen Systemzustände (Cap1, Cap2, Time) mit den Erwartungswerten (ExpValue) und den möglichen Nachfolgern (Successor).
\begin{table}
\begin{footnotesize}
    \caption{Ergebnistabelle für das beispielhafte Netzwerk RM} \label{Tab1}
    \vspace*{3mm}
\csvautotabular{data/beispiel1.csv}
      %{\footnotesize \textbf{In Anlehnung an:} \cite{gonsch2013using}, S. 113.} 
\end{footnotesize}
\end{table}

Zusätzlich zeigt die Tabelle die optimale Politik für jeden Systemzustand (Best Order, Best Successor, r[j]-OC[j]). Die optimale Politik $OP_{\textbf{c}, t}$ lässt sich anhand der Gleichung \eqref{OP} für jeden Systemzustand ermitteln. Für jede Kante im Entscheidungsbaum der Abbildung \ref{B1} kann damit der Ertrag abzgl. der OK hinterlegt werden ($r_{j}-OC_{j}$). Die optimale Politik $OP_{\textbf{c}, t}$ im betrachteten Systemzustand ist demnach die Kante bei der höchste Ertrag erzielt wird. Abbildung ??? zeigt den Entscheidungsbaum mit den jeweiligen Erwartungswerten und der optimalen Politiken für jeden Systemzustand.
\begin{figure}[h!]
  \begin{center}
    \includegraphics[width=150mm]{Bilder/Beispiel1a.pdf}
    \caption{Beispielhafte Darstellung eines Entscheidungsbaums eines Netzwerk RM (optimale Politiken)}  \label{B1a}
  \end{center}
\end{figure}

Mit diesem Ergebnis kann eine strategische Politik abgeleitet werden. Diese strategische Politik kann dem Unternehmen helfen seine Instrumente der Marktbearbeitung zu optimieren. In diesem kleinen Betrachtungsfeld $T=4$ sollte das Unternehmen zum Zeitpunkt $t=4$ versuchen seine Instrumente in der Form auszurichten, dass eine Anfrage nach Produkt $j=2$ eintrifft. Nachfolgend sollte das Unternehmen zum Zeitpunkt $t=3$ und $t=2$ den Absatz von Produkt $j=1$ vorantreiben. Damit wäre der optimale Pfad des Entscheidungsbaums gefunden ($[2\;1\;4] \rightarrow_{j=2} [2\;0\;3] \rightarrow_{j=1} [1\;0\;2] \rightarrow_{j=1} [0\;0\;1]\rightarrow_{j=0} [0\;0\;0]$). Mit dem optimalen Pfad ist ein Ertrag in Höhe von 400 GE generiert. Bei Abweichung des Pfads, z. B. durch nicht eintreffen der Anfrage $j=2$ zum Zeitpunkt $t=4$, gibt das Modell einen anderen Pfad mithilfe der optimalen Politiken an. Bspw. trifft zum Zeitpunkt $t=4$ nur die Anfrage nach Produkt $j=1$ ein. Dann muss das Unternehmen seine Instrumente an den neuen Pfad anpassen ($[2\;1\;4] \rightarrow_{j=1} [1\;1\;3] \rightarrow_{j=2} [0\;1\;2] \rightarrow_{j=1} [0\;0\;1]\rightarrow_{j=0} [0\;0\;0]$). Damit ist trotz der Abweichung ein Gesamtertrag von 400 GE möglich.

Anhand des Beispiels wird klar, dass die Prognose über die Wahrscheinlichkeiten des Eintreffens einer Anfrage nach den Produkten $j\in\mathcal{J}$ eine enorme Relevanz aufweist. Sofern die Wahrscheinlichkeiten verändert werden, dann ergeben sich andere optimale Politiken. Nachfolgend wird das Beispiel mit veränderten Parametern betrachtet:
\begin{center}
$j = \{1, 2\}, \; h = \{1, 2\}, \; r_{1} = 100, \; r_{2} = 200, \; T=4$
\end{center}
\[
    \textbf{c}=\begin{pmatrix} 2 \\ 1 \end{pmatrix}, \;
    \textbf{a}_1=\begin{pmatrix} 1 \\ 0 \end{pmatrix}, \;
     \textbf{a}_2=\begin{pmatrix} 0 \\ 1 \end{pmatrix}, \;
     p_{1}(t)=\begin{pmatrix} 0.9\\ 0.9\\ 0.9\\ 0.9  \end{pmatrix}, \;
     p_{2}(t)=\begin{pmatrix} 0.01\\ 0.01\\ 0.01\\ 0.01  \end{pmatrix}
  \]

