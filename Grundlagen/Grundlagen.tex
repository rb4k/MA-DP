\chapter{Grundlagen zu auftragsbezogenen Instandhaltungsprozessen}
\markboth{2 Grundlegende Begriffe}{}
\setcounter{footnote}{4}  %um durchgehende Fußnotennummerierung zu haben, hier die Anzahl der bisherigen Fußnoten eintragen

\section{Einordnung in die Produktionswirtschaft}

Bei Instandhaltungsprozessen (engl. maintenance-repair-and-overhaul (MRO)) handelt es sich um einen nachgelagerten Prozess im gesamten Wertschöpfungsprozess der Produktionswirtschaft\footnote{Vgl. \cite{???}, S. ???}. Bei der Produktionswirtschaft handelt es sich um alle ökonomischen Aktivitäten eines Unternehmens, die auf die Produktion ausgerichtet sind.\footnote{Vgl. \cite{???}, S. ???} Dazu zählt die Planung und Steuerung des Produktionsprogramms und der Produktionsprozesse....

Das Ziel einer jeden Unternehmung sollte es sein Wertschöpfung zu betreiben\footnote{Vgl. \cite{???}, S. ??}. Der konzeptionelle Rahmen dieser Zielsetzung bildet die Wertschöpfungslehre (engl. suppy chain management (SCM))\footnote{Vgl. \cite{???}, S. ??}. Bei diesem Konzept erfolgt die Transformation von materiellen und nichtmateriellen Inputgütern (Produktionsfaktoren) durch bestimmte unternehmerische Verfahrensweisen hin zu Outputgütern\footnote{Vgl. \cite{tempelmeier1994produktion}, S. 6}.

\section{Charakteristika}

\section{Relevanz für betriebliche Entscheidungen}
