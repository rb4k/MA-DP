% !TEX encoding = UTF-8 Unicode
\chapter{Bestehende Ansätze zur Annahme von Aufträgen in der Auftragsfertigung und bei Instandhaltungsprozessen}\label{Review}
\markboth{4 Bestehende Ansätze}{}
\setcounter{footnote}{54}  %um durchgehende Fußnotennummerierung zu haben, hier die Anzahl der bisherigen Fußnoten eintragen

Das Konzept des RM zur Annahme von Aufträgen bei Dienstleistungen findet bereits über mehrere Dekanen in der wissenschaftlichen Literatur Anwendung.\footnote{Vgl. Klein (2001), S. 246.} Neue Veröffentlichungen versuchen das Konzept auf die Problemstellung der Annahme von Aufträgen der Kundeneinzelfertigung zu übertragen. Wie in Kapitel \ref{Instandhaltung} dargelegt, kann der Instandhaltungsprozess eines Dienstleistungsunternehmens einer Auftragsfertigung gleichgesetzt werden. \cite{kimms2005revenue} geben einen Überblick über das traditionelle Konzept des RM über verschiedene Branchen. Hierzu schreiben die Autoren, dass das Konzept des RM vermehrt Anwendung findet, damit Unternehmen eine Unterstützung in der Entscheidungsfindung erhalten, welche Aufträge zur Auftragsfertigung akzeptiert werden sollen.\footnote{Vgl. \cite{kimms2005revenue}, S. 1.} \cite{quante2009management} gibt einen Überblick über relevante Literatur des traditionellen RM. Tabelle \ref{Überblick} zeigt eine Anlehnung der Übersicht von \cite{quante2009management} mit den Publikationen zum traditionellen RM in der Fertigungsindustrie, erweitert um neue Veröffentlichungen in dem Forschungsbereich. Die Tabelle zeigt jeweils zur Publikation den betrachteten KEP, die Anzahl der berücksichtigen Konsumerklassen des Modells und die Methodik der Arbeit. Die Konsumerklassen resultieren aus der für das Konzept des RM notwendigen Marktsegmentierung von Kunden bzw. Auftragstypen. Einige Konzepte und Modelle berücksichtigen daher explizit die Anzahl solcher Klassen.

\renewcommand{\arraystretch}{1.2}
\begin{table}[h!]
  \begin{center}
    \caption{Überblick über Publikationen des traditionellen Konzepts des Revenue Managements in der Fertigungsindustrie}  \label{Überblick}
    \vspace*{3mm}
    \begin{tabular}{llll}   %hier die Spaltenausrichtung, -breite, -begrenzung und -anzahl eintragen
     Autoren & KEP  & \#Klassen & Methode  \\ \hline
     \cite{deBHarris1995299} &      ATO          &  2  &  K, M \\
      \cite{Kalyan:2002aa}      &      MTO/ATO/MTS          &  --  &  K \\
                \cite{rehkopf:2005aa}   &      MTO          &  mehrere &  K, M \\
                      \cite{rehkopf2007revenue}    &      MTO          &  --  &  K, M, L \\
                               \cite{Spengler:2007aa}   &    MTO            & mehrere & M  \\
                               \cite{Volling20121021} & MTO & -- & K, M \\
          \cite{kimms2005branchenverg} & MTO & -- & K \\
          \cite{guhlich2015revenue} & ATO & -- & M, L \\
          \cite{kolisch2006revenue} & MTO & -- & K \\
              \cite{DECI:DECI074}  &      MTO          &  mehrere  &  M \\
    % Kumar and Frederick            &      MTO/MTS          &  3  &  F \\
          \cite{kuhn2004revenue} & MTO & 2 & F \\
        \cite{Specht:2008aa} &        ATO    &  --  & K, F  \\
        \cite{quante2009revenue} & MTO & -- & L \\ 
        \cite{cheraghi2010revenue} & MTO/MTS & -- & L \\
        \cite{sucky2009revenue} & MTO & -- & M, F \\ \hline
    \end{tabular} \\[3mm]
    {\footnotesize \textbf{In Anlehnung an:} \cite{quante2009management}, S. 44.}\\
        {\footnotesize \textbf{Legende:} KEP: Kundenentkopplungspunkt, F: Fallstudie, K: Konzeption, L: Literaturüberblick, M: Simulations-/Analysemodell. }   %footnotesize liefert Schrift in Größe 10pt
  \end{center}
\end{table}

\cite{deBHarris1995299} beziehen die RM-Komponenten der differenzierten Preispolitik und eine multiklassen Kapazitätsallokation in ihr Modell mit ein. Sie zeigen, dass es zu Lagerfehlbeständen bei Niedrig-Preis-Segmenten führt, sofern stochastische Nachfrage auf fixe und kurzfristige Kapazität trifft. Jedoch rechtfertigen diese Lagerfehlbestände eine Premiumpreisstrategie bei den Kundengruppen mit höherer Preisbereitschaft, was letztendlich zu Umsatzsteigerungen führt.\footnote{Vgl. \cite{deBHarris1995299}, S. 307-308.} \cite{Kalyan:2002aa} beschäftigt sich mit der Bestätigung der Anwendungsvoraussetzungen des RM für die Auftragsfertigung. Das vom Autor beschriebene Konzept sieht die Einführung eines minimalen Akzeptanzwert vor. Sofern dieser Wert bekannt ist, kann ein Unternehmen bei jedem Auftragseingang die Entscheidung treffen, welche Anfrage angenommen oder abgelehnt werden soll.

Der vom Autor \cite{Kalyan:2002aa} eingeführte minimale Akzeptanzwert wird in der wissenschaftlichen Literatur im Kontext des Konzepts des RM als sogenannter Bid-Preis bezeichnet. Bei dem Bid-Preis handelt es sich um einen variierenden Parameter in Abhängigkeit der Zeit (bzw. der Periode) und der verfügbaren Ressourcenkapazität des betrachteten Netzwerks. Er kann als statische Information der optimalen Lösung für die eindimensionalen Probleme angesehen werden. Die Ermittlung des Bid-Preises erfolgt bei den traditionelle Modellformulierungen des RM anhand eines \textit{deterministischen linearen Programms} unter deterministisch eintreffender Nachfrage.\footnote{Vgl. \cite{talluri2004revenue}, S. 107-108.} %Der Bid-Preis korrespondiert zur Entscheidungsvariable $x_{jm}$ des DLP, da er verbunden ist mit der Kapazität einer jeden Ressource $h$.\footnote{Vgl. \cite{gonsch2013using}, S. 98}
Er fungiert als Schwellenpreis für eine jede Ressource im Netzwerk und ist normalerweise beschrieben als geschätzte marginale Kosten aufgrund des nächsten sukzessiven Verbrauchs einer Einheit der Ressourcenkapazität.\footnote{Vgl. \cite{talluri2004theory}, S. 89.} Laut \cite{gonsch2013using} erfolgt der Ansatz erstmalig von \cite{talluri2001airline} in Verbindung der Optimierung von Passagierrouten. Das Verfahren des Bid-Preises ist ein einfacher Weg die Kapazitäten der Ressourcen in einem Netzwerk eines Anbieters zu kontrollieren.\footnote{Vgl. \cite{talluri2004theory}, S. 86-87\label{RMH}.} Viele Konzepte der neueren Veröffentlichungen im Bereich des Netzwerk RM sehen die Verwendung des Bid-Preises vor.\footnote{Vgl. \cite{petrick2010dynamic}, S. 2028; \cite{gonsch2013using}, S. 98-100.}

%Der Bid-Preis stellt das dynamische Minimum dar, zu dem eine Unter- nehmung bereit ist eine Leistung zu erstellen. Vor allem für Probleme vernetzter Leis- tungserstellung ist die Methode des Bid-Preises besonders geeignet. Tscheulin/Lindenmeier (2003), S. 639.

\cite{rehkopf:2005aa} zeigen durch das Lösen eines linearen Modells die Kapazitätsallokation für die Problemformulierung des Netzwerk RM im Fall von MTO-Prozessen. Die Autoren fokussieren sich dabei auf die Branche der Eisen- sowie Stahlindustrie und veröffentlichen zwei Publikationen mit ihren Forschungsergebnissen. Dabei findet der im vorherigen Absatz definierte Bid-Preis Anwendung. 

\cite{rehkopf2007revenue} verfasste nach Veröffentlichung der ersten Publikation eine Schrift zu einem umfangreichen RM-Konzept zur Auftragsannahme bei kundenindividueller Produktion. Auch hier wird als Beispiel ein Anwendungsfall aus der Branche der Eisen und Stahl erzeugenden Industrie untersucht. Es werden zwei Fallstudien betrachtet. Zum einen eine taktisch-operative Allokation der Kapazität und zum anderen eine operative Annahmeentscheidung. Bei der Entwicklung einer geeigneten RM-Methodik in der Stahl erzeugenden Industrie kommt zum Tragen, dass in dieser Branche eine strikt divergente Produktionsstruktur vorliegt und dadurch die Auftragsfertigung in einstufige Leistungserstellung zerlegt werden kann.\footnote{Vgl. \cite{rehkopf2007revenue}, S. 113.} Dabei konnte die Fallstudie zeigen, dass bei der taktisch-operativen Allokation der Kapazität das vorgestellte Verfahren des RM mit einem FCFS-Verfahren eine ausgesprochene Dominanz gegenüber einem deterministischen Ansatz mit FCFS-Verfahren aufweist.\footnote{Vgl. \cite{rehkopf2007revenue}, S. 127-134.} Für die operative Entscheidungsunterstützung formuliert der Autor ein Modell des Netzwerk RM, welches durch ein spezielles Verfahren die Restkapazität bestimmt. Das Verfahren berechnet die Restkapazität nach einem real-time Ansatz für den zusicherbaren Bestand\footnote{Vgl. den Ansatz \glqq Available-to-promise (ATP)\grqq.} aus der kapazitierten Hauptproduktionsprogrammplanung und den bereits eingeplanten Aufträgen des Netzwerks.\footnote{Vgl. \cite{rehkopf2007revenue}, S. 140; \cite{Spengler:2007aa}, S. 160.} Ebenso wird hier der Ansatz des Bid-Preises eingesetzt, welcher durch ein duales lineares Optimierungsmodell ermittelt wird.\footnote{Vgl. \cite{rehkopf2007revenue}, S. 144.} Dabei konnte er in der Fallstudie das Potential der Anwendung eines Bid-Preises und des angewandten Ansatzes zur Ermittlung der Restkapazität feststellen.\footnote{Vgl. \cite{rehkopf2007revenue}, S. 177.} Die Ergebnisse aus dieser Monografie fließen in die zweite Veröffentlichung von \cite{Spengler:2007aa} ein. In dieser Arbeit wird in einer Fallstudie ebenfalls gezeigt, dass durch eine Bid-Preis-Strategie der Gesamtdeckungsbeitrag verbessert wird.\footnote{Vgl. \cite{Spengler:2007aa}, S. 157–171.} Dabei werden in der Publikation die Bid-Preise jeweils mit einer linearen Modellformulierung sowie einer multi-dimensionalen Knapsack-Modell\-for\-mu\-lier\-ung berechnet und verglichen. Dabei wird deutlich, dass beide Verfahren den Deckungsbeitrag ähnlich verbessern, wobei sich das Lösen anhand der Knapsack-Modell\-for\-mu\-lier\-ung als robuster darstellt.\footnote{Vgl. \cite{Spengler:2007aa}, S. 168-169.} In der Arbeit zeigten die Autoren, dass durch Anwenden der Heuristik sich der Gesamterlös im Vergleich zu einer einfachen Reihenfolgeannahme (FCFS) um 5,3\% erhöhen lässt.\footnote{Vgl. \cite{Spengler:2007aa}, S. 170.}

\cite{Volling20121021} beschäftigen sich ebenfalls mit dem Auftragsannahmeproblem von MTO-Prozessen. Dabei legt die Arbeit u. a. den Fokus auf die Ermittlung und den Zeitpunkt der Neuberechnung des Bid-Preises für das Netzwerk RM. Es werden mögliche Szenarien gezeigt, zu welchem Zeitpunkt im Buchungsverlauf der Bid-Preis neuberechnet werden kann. Dabei erfolgt insbesondere die Untersuchung der Anwendbarkeit bei MTO-Prozessen. Die Notwendigkeit der Neuberechnung resultiert aus der Vereinfachung der Problemstellung des Auftragsannahmeproblems im Netzwerk RM. Im ersten Schritt des in der Arbeit präsentierten Verfahrens erfolgt die Berechnung der anfänglichen Bid-Preise. Im zweiten Schritt werden szenarioabhängige Bid-Preise berechnet, die für den weiteren Verlauf des Buchungshorizonts gelten. Die Autoren nutzen zur Neuberechnung dieser szenarioabhängigen Bid-Preise ein Verfahren der \glqq Neuronalen Netze{\grqq}. Dabei wird im ersten Schritt anhand der beobachteten und gebuchten Nachfrage die weitere erwartete Nachfrage berechnet. Anschließend wird diese Kennzahl mit der übrigen Ressourcenkapazität ins Verhältnis gesetzt. Sofern die erwartete Restnachfrage größer als die übrige Kapazität des Netzwerks ist, erfolgt die Neuberechung der Bid-Preise. Dabei wird der Bid-Preis und die erwartete Nachfrage in eine nicht lineare Beziehung gesetzt, bis beide Betrachtungsobjekte eine gleiche Nachfrageverteilung aufweisen.\footnote{Vgl. \cite{Volling20121021}, S. 1026.} Laut der in der Arbeit durchgeführten numerischen Untersuchung wird der Deckungsbeitrag durch Anwenden des Verfahrens um 7,8-20,1\% gegenüber eines FCFS-Verfahrens verbessert.

Einen umfassenden Branchenüberblick über Anwendungsmöglichkeiten des RM liefern \cite{kimms2005branchenverg} in ihrer Veröffentlichung. Dabei wird exemplarisch das RM-Instrument der Kapazitätssteuerung untersucht, indem jeweils ein Entscheidungsmodell formuliert wird. In dem Branchenüberblick werden Konzepte zur Modellformulierung für das Luftverkehrswesen (Passagier und Fracht), der Touristikbranche (Hotellerie, Gastronomie und Autovermietung) und eben der Fertigungsindustrie vorgestellt. Das Modell für die Auftragsfertigung sieht explizit keine Lagerhaltung von Kapazitäten vor, wobei die eintreffenden Kundenaufträge über unterschiedliche Ausführungsmodi realisiert werden. Mit den unterschiedlichen Modi wird in dem Modell z. B. eine unterschiedliche Betriebsintensität zur Ausführung einer angenommenen Anfrage verstanden. Mit der Arbeit legten die Autoren dar, welche branchenübergreifenden Voraussetzungen zur Anwendung von RM erforderlich sind. %U. a. die Einschränkung operativer Flexibilität und die Heterogenität des Nachfrageverhaltens.
Dabei definieren die Autoren das RM im weitesten Sinne auch als mögliches Instrument der strategisch-taktische Entscheidungsunterstützung.\footnote{Vgl. \cite{kimms2005branchenverg}, S. 24.} 

\cite{kolisch2006revenue} diskutieren in ihrer Arbeit die Voraussetzungen für die Anwendung von RM in der Vermarktung von Sachleistungen für Geschäftskunden. Zusätzlich werden die Komponenten eines RM-Sysmtems dargestellt und Ergebnisse einer empirischen Studie zum Einsatz von RM in der Prozessindustrie präsentiert. Zu den Komponenten eines RM-Systems zählen lt. den Autoren die Datenanalyse, die Nachfrageprognose, die Optimierung und die Steuerung. Letzteren zwei beziehen sich auf das Lösen des mathematischen Optimierungsmodells zur Ermittlung des Bid-Preises. Durch diese Systemkomponenten kann eine verbesserte Kapazitäts- und Preispolitik erfolgen. Die präsentierte empirischen Studie belegt, dass Unternehmen den Einsatz von RM-Instrumenten positiv gegenüber stehen und einen zunehmenden Einsatz bestätigen. Dabei geben ca. 80\% der befragten Unternehmen an, dass der Einsatz über Systemlösungen erfolgt und dass das Preismanagement gegenüber des reinen Kapazitätsmanagements in den letzten Jahren an Bedeutung gewonnen hat.\footnote{Vgl. \cite{kolisch2006revenue}, S. 40-41.}

Anders als die vorher aufgeführten Autoren befassen sich \cite{DECI:DECI074} nicht nur mit dem Auftragsannahmeproblem und der Kapazitätsallokation, welches sich aus dem Konzept des RM ergibt, sondern auch um die Planung und der Bestimmung des genauen Zeitpunkts der Fertigung. Die Autoren stellen eine Heuristik vor, die Aufträge in verschiedene Lose sortiert. Dabei werden mehrere Konsumerklassen beachtet. Die Basisidee des Verfahrens ist die Beachtung der relativen Gewinnspannen der Aufträge, damit der Gesamtdeckungsbeitrag erhöht wird.\footnote{Vgl. \cite{DECI:DECI074}, S. 291.} Unter Einsatz der Heuristik zeigen die Autoren, dass ein höherer Gewinn aufgrund einer effizienten Nutzung der verfügbaren Kapazität erzielt wird. Dies kommt zustande, da eine Unterscheidung der Aufträge in Bezug auf verschiedene Produktklassen mit unterschiedlichen Deckungsbeiträge erfolgt.\footnote{Vgl. \cite{DECI:DECI074}, S. 310.}

Auch die Autoren \cite{guhlich2015revenue} beschäftigen sich zusätzlich zur optimalen Kapazitätsallokation mit der Ermittlung der möglichen Fertigstellungszeit eines Auftrags im Kontext des RM-Konzepts. Das von den Autoren vorgestellte Modell berücksichtig bei jeder Anfrage einen Zeitpunkt der angebotenen Fertigstellung. Dabei betrachtet das Modell Buchungsperioden innerhalb von Planungszeiträumen.\footnote{Vgl. \cite{guhlich2015revenue}, S. 10.} Dadurch ist es dem Modell gestattet, zum Anfragezeitpunkt den angebotenen Fertigstellungszeitpunkt ebenfalls für die Auftragsproblematik zu berücksichtigen. Dies erfolgt in der Form, dass sich zur Herstellung des gewünschten Produkts ein Zeitraum der Produktion ergibt. Damit kann eine Entscheidung getroffen werden, in Abhängigkeit der Kapazitäten und der Fertigstellungszeit, zu welchem Zeitpunkt die Herstellung starten soll. Dabei ist mit der Annahme der Anfrage der angebotene Fertigstellungszeitpunkt für den Kunden bestätigt. Das Modell berücksichtigt auch Lager- und Fertigstellungsrückstandskosten in dem möglichen Produktionszeitraum, wobei im Vorfeld die Ermittlung der möglichen Durchführbarkeit erfolgt. Für die Durchführungsprüfung wird ein lineares Modell gelöst. Dies ist notwendig, damit der Zeitpunkt der angebotenen Fertigstellungszeitpunkt und Planungsgrundlage geprüft wird.\footnote{Vgl. \cite{guhlich2015revenue}, S. 12-14.} Auch hier werden die Opportunitätskosten durch Ermittlung eines Bid-Preises approximiert.\footnote{Vgl. \cite{guhlich2015revenue}, S. 18-19.} In einer numerischen Untersuchung ist als Feststellung dargelegt, dass das Modell der Autoren im Vergleich zu anderen Algorithmen eine niedrigere Abweichung zur optimalen Lösung aufweist.\footnote{Vgl. \cite{guhlich2015revenue}, S. 22-24.}\\[.5cm]

\cite{kuhn2004revenue} beschreiben die Anwendung des RM anhand eines Papierherstellers, der eine Auftragsfertigung mit zwei unterschiedlichen Klassen anbietet. Eine Klasse ist dabei eine mit höheren und die andere eine mit niedrigen Erlösen. Dabei nehmen die Autoren an, dass durch Annahme der Aufträge die Maschinen eine Belegungszeit zur Fertigung des geforderten Produkts haben. Außerdem unterscheiden sie die Aufträge durch ihre individuelle Lieferzeit. Durch diese Einschränkungen kommt es zu eine differenzierten Möglichkeit der Auftragsannahme, da untersucht werden muss, ob Aufträge aufgrund blockierter Maschinen innerhalb der geforderten Lieferzeit möglich sind und ob genügend Kapazität verfügbar ist. In der Fallstudie wird das Konzept mit der linearen Programmierung der optimalen Auftragsannahmepolitik verglichen. Auch in dieser Fallstudie konnten die Autoren die Anwendung des RM für MTO-Prozesse durch eine Gesamterlösverbesserung bestätigen.

\cite{Specht:2008aa} schreibt in ihrem Artikel über die Anwendung des RM im Bereich der Automobilindustrie. Dabei wird als Betrachtungselement das Unternehmen \glqq Ford Motor {Company\grqq} herangezogen, wobei die Autoren zum Zeitpunkt der Erstellung der Publikation keinen Zugriff auf interne Dokumente des Unternehmens hatten. Die Autoren beschreiben die Anwendung von mehreren Teilsystemen, die Kundenwünsche systematisch abfragen. Dem Unternehmen ist es daraufhin möglich, sehr nah an den Bedürfnissen des Marktes zu produzieren.\footnote{Vgl. \cite{Specht:2008aa}, S. 66.} Es konnten aber keine Anhaltspunkte über eine mögliche Verbesserung der Kapazitätsallokation der Werke ermittelt werden. \cite{Specht:2008aa} kommen daher zum Ergebnis, dass es sich eher um preisbasierte und nicht um kapazitätsbasierte RM-Systeme handeln muss. Sie konnten aber grundsätzlich die Anwendbarkeit des RM in der Automobilindustrie bestätigen.

Der Problematik der wachsenden Anzahl an möglichen Anwendungsfeldern, der unterschiedlichen Modelle und der verfügbaren Software in Bezug zum RM nehmen sich die Autoren \cite{quante2009revenue} in einem zusammenführenden Überblick an. Zur den Anwendungsfelder zählen aus der Sicht der Autoren die anwendungsorientierte Lieferkettenmerkmale und die Anwendungstypen. Bei den Lieferkettenmerkmalen werden die einzelnen Systeme der Lieferkette in Bezug auf die Merkmale des RM untersucht. Dabei wird eine beispielhafte Übersicht aufgezeigt, welche Merkmale des RM sich in Abhängigkeit des Anwendungsfeld ergeben. Danach lässt sich für den Anwendungsfall eine Charakterisierung der relevanten Merkmale für das RM ermitteln. Die in dieser Arbeit betrachtete Auftragsfertigung bzw. Instandhaltung kann die in der Veröffentlichung von \cite{quante2009revenue} beschriebenen Fertigung von Maschinen gleichgesetzt werden. Bei der Maschinenfertigung ist der KEP vor der Produktion, daher handelt es sich um einen klassischen MTO-Prozess. Die Flexibilität der Kapazität ist hoch. Diese Merkmalsausprägung resultiert aufgrund der Betrachtung der Verwendung der Kapazität innerhalb des SCM-Systems der Produktion. Bei den Maschinen handelt es sich um langlebige Produkte mit einem langen Lebenszyklus. Bei dem Anwendungsfall der Auftragsfertigung ist eine Preisflexibilität ab dem Zeitpunkt des Auftragseingangs möglich. Die Autoren bestimmen des Weiteren die Profitheterogenität. Bei einem MTO-Prozess der Auftragsfertigung ist der Profit lt. den Autoren abhängig vom Zeitpunkt des Eingangs und der Spezifikationen des Kundenauftrags. Zusammengesetzt wird er durch den erzielten Ertrag, der ebenfalls vom Zeitpunkt und Auftrag abhängig ist. Die Ertrag wird um die Kosten reduziert und um einen Wert der strategischen Bedeutung erhöht. Diese beiden Werte sind wiederum nur abhängig vom Kundenauftrag.\footnote{Vgl. \cite{quante2009revenue}, S. 37.}

Ebenfalls wird in der Arbeit eine mögliche Klassifizierung der Modelle des RM anhand der Lieferkette getätigt. Dabei werden zu den einzelnen Systemen der Lieferkette mögliche Ausprägungen der Parameter der Modelle angegeben. Es werden verschiedene Modelle in ihren Ausprägungen der eingesetzten Parametern verglichen, wie z. B. bei den Modellen der stochastischen Lagerbestandskontrolle, des \glqq vielversprechensten Auftrags{\grqq} und des traditionellen RM.\footnote{Vgl. \cite{quante2009revenue}, S. 43-44.} Als letzten Teil der Übersicht werden verschiedene softwarebasierte Lösungen für das RM aufgeführt. Anschließend erfolgt jeweils für die Branchen der Serviceindustrie, des Handels und der Fertigung ein zusammenfassendes Ergebnis der Anwendungsfelder, der Modelle und der Softwarelösungen. Für MTO-Prozesse kommen die Autoren zu dem Schluss, dass die verfügbaren Modelle und Softwarelösungen nicht das Problem darstellen, sondern die Lösungsqualität und Zuverlässigkeit sowie die kurze Reaktionszeiten der Auftragsannahme.\footnote{Vgl. \cite{quante2009revenue}, S. 56-57.}

Aufgrund erhöhter Unternehmensforschung im Bereich des RM verfassen die Autoren \cite{cheraghi2010revenue} ein Literaturüberblick über RM-Systeme in der Fertigungsindustrie anhand der Dimensionen des Kapazitäts- und Preismanagements sowie der Marktsegmentierung. Damit bilden die Autoren neben der Kapazitätssteuerung auch Verfahren der Segmentierung und der Bepreisung ab, da aus ihrer Sicht die richtige Ansprache der unterschiedlichen Kundengruppen eines Fertigungsunternehmens hohe Relevanz aufweist.\footnote{Vgl. \cite{cheraghi2010revenue}, S. 64-65. \label{testenvorher}} Durch die Marktsegmentierung ist es einem Unternehmen möglich, die Produkte unter unterschiedlichen Rahmenbedingungen anzubieten. Infolgedessen kann ein verbessertes Preismanagement erfolgen. Dabei stellen die Autoren die Besonderheit des Preismanagements in der Fertigung heraus. Anders als bei Fluggesellschaften, verfällt die Kapazität nicht aufgrund eines ablaufenden Planungshorizonts. Die verbleibenden Kapazitäten könnten weiterhin abgesetzt werden. Eine Verderblichkeit der Kapazität erfolgt jedoch in der Hinsicht, da ein möglicher Abschluss eines Verkaufsgeschäfts mit einem potentiellen Kunden nicht eingetroffen ist.\footref{testenvorher} In Bezug des Literaturüberblick zum Kapazitätsmanagement fassen die Autoren ihre Ergebnisse in den Abschnitten \glqq Auftragswahl{\grqq} und \glqq Kapazitätsallokation{\grqq} zusammen. Bei dem Überblick zu den Modellen kann insbesondere die Arbeit von \cite{Defregger:2007aa} genannt werden, da sich die Arbeit mit der Annahme von Aufträgen eines Unternehmens mit endlichem Lagerbestand beschäftigt. In dieser Arbeit wird ein Modell für MTO-Prozesse modelliert, bei dem das Auftragsannahmeproblem mittels zeitdiskreter Markov-Entscheidungsprozess und einem heuristischen Verfahren untersucht wird. Das in der Arbeit vorgestellte Modell basiert dabei auf Sequenzen von Entscheidungen, in welchen Fällen eine Anfrage akzeptiert und zur Reduktion des Lagerbestands führt. Dabei werden die Auftragsarten in Klassen sortiert und mittels einer Heuristik untersucht. Sofern eine Anfrage zu der Klasse mit niedrigen Erträgen gehört, wird diese abgelehnt. Dabei konnte sich die Heuristik in einer numerischen Untersuchung gegenüber eines FCFS-Verfahrens behaupten.\footnote{Vgl.  \cite{Defregger:2007aa}, S. 150-154.}

\cite{sucky2009revenue} zeigt das Potential vom RM in der Auftragsfertigung anhand eines einfachen Beispiels eines Modells zur Kapazitätssteuerung. Dabei erfolgt die Kapazitätssteuerung auf einem Konzept der Beachtung eines Buchungslimits. Damit versucht der Autor konkrete Anwendungsfälle des RM in der Auftragsfertigung zu ermitteln. Dabei führt der Autor auch die möglichen Hemmnisse des RM in der Auftragsfertigung auf. Zum Beispiel wird der Fall aufgeführt, dass Anfragen von wichtigen Geschäftskunden aufgrund niedrigerer Stückdeckungsbeiträge abgelehnt werden, da diese tendenziell weniger für die gleiche Leistung zahlen.\footnote{Vgl. \cite{sucky2009revenue}, S. 89.}
