\documentclass[a4paper,12pt,normalheadings,footexclude,headinclude,liststotoc,nochapterprefix,onecolumn,oneside,parskip,pointlessnumbers]{scrreprt}

\usepackage[utf8]{inputenc}
\usepackage[ngerman]{babel}
\usepackage[T1]{fontenc}
\usepackage{graphicx}
\usepackage{setspace,natbib}
\usepackage{natbib}
\usepackage{abstract,amsmath,amssymb}
\usepackage[notindex,nottoc]{tocbibind}  %Inhaltsverzeichnisse erstellen
\usepackage[labelsep=space,justification=centering]{caption}
\usepackage{remreset}
\usepackage[lined,commentsnumbered]{algorithm2e}

\usepackage{amsthm}
\newtheorem{mydef}{Definition}

\renewcommand{\familydefault}{\sfdefault}

%Helvetica
%\renewcommand{\familydefault}{\sfdefault}
%\usepackage{helvet}

\usepackage[table]{xcolor}
\usepackage{colortbl}
\usepackage{multirow}

\usepackage{longtable} 
\usepackage{array} 
\usepackage{ragged2e} 
\usepackage{lscape} 
\usepackage{multirow} 
\usepackage{booktabs} 

\usepackage{csvsimple}

%\usepackage{datatool}
%\usepackage[showframe,paperwidth=32cm]{geometry}

\SetAlgorithmName{Algorithmus}{Algorithmenverzeichnis}

% Fussnoten
\usepackage{chngcntr}
\counterwithout{footnote}{chapter}

%\usepackage[Sonny]{fncychap.sty}
\setlength{\parindent}{0pt}
\newcommand{\url}{\;}

\topmargin -0.9cm
\textheight 24cm
\textwidth 14cm
\oddsidemargin 1.5cm
\footskip 1cm

\onehalfspacing        % anderthalbzeilig

\renewcommand{\sectionmark}[1]{\markright{\footnotesize \sffamily\slshape \thesection\ #1}}

%Kopfzeilendefinition
\usepackage[fit]{truncate}
\usepackage{fancyhdr}
\pagestyle{fancy}
\cfoot{}
\lhead{\slshape\leftmark}
\rhead{\thepage}       % Eintrag rechts oben in Kopfzeile

%Kopfzeilendefinition bei neuem Kapitel
\makeatletter
\renewcommand\ps@plain{
 \renewcommand\@oddfoot{}  %leere Fusszeile
  \let\@evenfoot\@oddfoot
  \renewcommand\@oddhead{\hfil \normalfont \textrm{\thepage}}
  \let\@evenhead\@oddhead}
\makeatother

%Spaltendefinition f\"{u}r Tabellen
\usepackage{array}

%Formelnummerierung
\makeatletter
\@removefromreset{equation}{chapter}
\makeatother
\renewcommand*{\theequation}{\arabic{equation}}

%Tabellennummerierung
\makeatletter
\@removefromreset{table}{chapter}
\makeatother
\renewcommand*{\thetable}{\arabic{table}}

%Abbildungsnummerierung
\makeatletter
\@removefromreset{figure}{chapter}
\makeatother
\renewcommand*{\thefigure}{\arabic{figure}}


\begin{document}
\newpage\thispagestyle{empty}\null\newpage
\pagenumbering{roman}  %r\"{o}mische Seitennummerierung

%Titelseite
\begin{titlepage}
\begin{sffamily}
\begin{center}
  {\Huge{\bf Auftragsannahme- und}}\\[0.5cm]
    {\Huge{\bf Lagerhaltungsentscheidungen}}\\[0.5cm]
  {\Huge{\bf bei auftragsbezogenen}} \\[0.5cm]
    {\Huge{\bf Instandhaltungsprozessen}} \\[3cm]
  {\huge Masterarbeit} \\[1cm]
  {\Large zur Erlangung des akademischen Grades \\
  \glqq Master of Science (M.Sc.)\grqq\;im Studiengang Wirtschaftswissenschaft} \\[8mm]
  {\Large der Wirtschaftswissenschaftlichen Fakult\"{a}t \\
  der Leibniz Universit\"{a}t Hannover} \\[2cm]
  {\Large vorgelegt von} \\[8mm]
  {\Large\bf Robert Matern} \\[5mm]
  {\Large geb. am 7. März 1987 in Tschimkent } \\[2mm]
  {\Large (Matrikel-Nr. 2798160)} \\[2cm]
  {\Large\bf Erstpr\"{u}fer: Prof. Dr. Stefan Helber} \\[6mm]
  {\Large Hannover, den 11. September 2015}
  \end{center}
  \end{sffamily}
\end{titlepage}
\newpage


% Inhaltsverzeichnis
\tableofcontents

%Abkürzungsverzeichnis
\chapter*{Abkürzungsverzeichnis}
\addcontentsline{toc}{chapter}{Abkürzungsverzeichnis}
\begin{table}[h!]
    \vspace*{-3mm}
    \hspace*{2mm}
  \renewcommand{\arraystretch}{1,5}
  \begin{flushleft}
    \begin{tabular}{lp{11.5cm}}  %hier die Spaltenausrichtung und Anzahl eintragen
        APS & Advanced Planning and Scheduling Systeme \\
        ATO & Assemble-to-Order (engl. Begriff für kundenindividuelle Fertigung mit standardisierten Komponenten)\\
        DLP			& Deterministisch-lineares Programm\\
        DP  			 & Dynamisches Programm zur Annahme von Aufträgen\\
        DP-storage		    & Dynamisches Programm zur Annahme von Aufträgen unter Beachtung von Lagerbeständen\\
        %GAMS 		& General Algebraic Modeling System\\
        GE & Geldeinheiten \\
        MRO & Maintenance-Repair-and-Overhaul (engl. Begriff für Instandhaltungsprozess)\\
     MTO & Make-to-Order (engl. Begriff für Auftragsfertigung) \\
     MTS & Make-to-Stock (engl. Begriff für Lagerfertigung)\\
     OK		    & Opportunitätskosten\\
     RM		    & Revenue Management\\
     SCM & Suppy Chain Management (engl. Begriff für Wertschöpfungslehre) \\
    u. B. d. R. & unter Berücksichtigung der Randbedingungen\\
	\end{tabular}
	\end{flushleft}
\end{table}

%Symbolverzeichnis
\chapter*{Symbolverzeichnis}
\addcontentsline{toc}{chapter}{Symbolverzeichnis}
\begin{table}[h!]
    \vspace*{-3mm}
    \hspace*{2mm}
  \renewcommand{\arraystretch}{1,5}
  \begin{flushleft}
    \begin{tabular}{lp{11.5cm}}  %hier die Spaltenausrichtung und Anzahl eintragen
    $\textbf{a}_{j}$ & Vektor des Ressourcenverbrauchs für Produktanfrage $j$\\
    $a_{hj}$ &	Verbrauch der Ressource $h$ für Produktanfrage $j$\\
   $\textbf{c}$ & Vektor der Ressourcenkapazität\\
   $c_{h}$ & Kapazität der Ressource $h$\\
    $D_{jt}$ & Aggregierte erwartete Nachfrag nach Produkt $j$ zur Periode $t$\\
    $h$	& Ressource aus der Menge $\mathcal{H}$\\
    $\mathcal{H}$ & Menge an Ressourcen\\
    $i$ & ... $\mathcal{I}$\\
    $\mathcal{I}$ & Gesamtmenge an möglichen Kombinationen ...\\
    $j$ & Produktanfrage aus der Menge $\mathcal{J}$\\
$\mathcal{J}$ & Menge an möglichen Produktanfragen \\   
$m$ & ... Ausführungsmodus aus der Menge $\mathcal{M}_{j}$\\
    $\mathcal{M}_{j}$ & Menge an produktspezifisch-möglichen Ausführungsmodi für ein Produkt $j$\\
 $\pi_{h}$ & Bid-Preis der Ressource $h$\\
   $p_{j}(t)$ & Wahrscheinlichkeit der Nachfrage nach Produkt $j$ in Periode $t$\\
    $r_{j}$ & Erlös des Produkts $j$\\
    $t$ & Periode des Buchungshorizonts $T$\\
        $\tau$ & Endperiode $t=1$ im Buchungshorizont $T$\\ 
        $T$ & Buchungshorizont\\
   $V(\textbf{c},t)$ & Ertragsfunktion in Abhängigkeit der Kapazitäten $\textbf{c}$ in der Periode $t$\\
    $x_{jm}$ & Anzahl der akzeptierten Anfragen nach Produkt $j$ im Ausführungsmodus $m$\\
	\end{tabular}
	\end{flushleft}
\end{table}






%Tabellenverzeichnis
\begingroup
\renewcommand*{\addvspace}[1]{}
\listoftables
\endgroup


%Abbildungsverzeichnis
\begingroup
\renewcommand*{\addvspace}[1]{}
\listoffigures
\endgroup


%\listofalgorithms
\newpage
\pagenumbering{arabic}   %ab hier arabische Seitenzahlen beginnend mit 1

\addcontentsline{toc}{chapter}{Vorwort}
\chapter*{Vorwort}
\markboth{Vorwort}{}


\chapter{Einleitung}
\markboth{1 Einleitung}{}

\section{Problemstellung}
%Unternehmen müssen aufgrund der heutigen dynamischen Veränderungen der Marktgegebenheiten die schnelle Entwicklung ihrer Produkte vorantreiben, sowie deren Platzierung und Bepreisung optimieren.\footnote{Vgl. \cite{gonsch2013using}, S. 94-95\label{dingens1}}
%Das Revenue Management (RM) ist daher für viele Dienstleistungsbereiche unabdinglich geworden. Durch dieses Managementkonzept und den daraus resultierenden Planungsinstrumenten können bei speziellen Anwendungsgebieten die Kapazitätsauslastungen optimiert werden. Damit lassen sich für Unternehmen mit Kapazitätsbe\-schränkungen die Absatzmärkte erfolgreich abschöpfen und die Gesamterlöse maximieren. In dieser Arbeit wird der Ansatz des Verkaufs von opaken Produkten im RM zur besseren Kapazitätsauslastung thematisiert. %\footnote{Klassische Ansätze des RM beschäftigen sich mit der Optimierung der Kapazitätsauslastung bei der Erstellung von Produkten oder Dienstleistungen.}
%Bei opaken Produkten handelt es sich um Produkte, bei denen der Anbieter einige Eigenschaften des Produkts bis zum Abschluss der Transaktion verbirgt.\footref{dingens1} %Anders als bei spezifischen Produkten, bei dem sämtliche Bestandteile der Produkteigenschaften vor dem Kauf bekannt sind.
%Die vorgestellten Produktarten werden i. d. R. in der Dienstleistungsindustrie von multiplen Anbietern angeboten.\footnote{Vgl. \cite{Klein:2008aa}, S. 4} Vereinfacht gesagt bedeutet dies, dass diese Anbieter ihre Dienstleistungen über eine Mehrmarkenstrategie distribuieren und ggf. ihr eigenes Produktportfolio mit anderen Dienstleistungen von Partnerfirmen erweitern. Sie bündeln aus dieser Menge an verfügbaren unterschiedlichen Dienstleistungen ein neues zusammenhängendes Produkt und vertreiben diese Art von Produkten über eine einzelne Marke an potentielle Konsumenten. Aus einem solchen Netzwerk an verfügbaren Produkten kann der Anbieter opake Produkte formen.\footnote{Vgl. \cite{anderson2009effectiveness}, S. 307-309}\textsuperscript{,}\footnote{Siehe Kapitel \ref{DP}}\\

Die Entscheidung über die Annahme von Kundenaufträgen zur Instandsetzung von Gütern ist von zentrale Bedeutung für Reparaturdienstleister. Abhängig des Kundenauftrags, indem der Zustand des Gutes beschrieben ist sowie die für die Reparatur notwendigen Prozessschritte beschrieben sind, generiert der Dienstleister unterschiedliche Erträge. Die jeweiligen Prozessschritte zur Instandsetzung des Gutes geben zusätzlich den notwendigen Ressourcenbedarf für die auszuführende Dienstleistung an, die notwendig ist um das Gut in seinen ursprünglichen bzw. geforderten Zustand zurückzuversetzen. Ressourcen zur Instandsetzung von Gütern können z. B. Material oder Personalstunden sein. Abhängig des möglichen Ertrags und des für den Auftrag notwendigen Ressourcenbedarf muss der Reparaturdienstleister die Entscheidung über Annahme oder Ablehnung des Kundenauftrags treffen. Sofern nur der einfache Fall betrachtet wird, bei dem nur der einzelne Kundenauftrag zur Auswahl steht, ist die Entscheidung für den Dienstleister einfach getroffen. Der Kundenauftrag wird angenommen, sofern der Aufwand des Ressourceneinsatzes niedriger als der erziele Umsatz ist (sofern von einer Vollkostenrechnung ausgegangen wird). Sofern der Reparaturdienstleister eine begrenzte Ressourcenkapazität zur Instandsetzung der Güter besitzt, muss zusätzlich der absolute Ressourcenverbrauch des Auftrags für Annahmeentscheidung geprüft werden. Mit Annahme des Auftrags ist ein individueller Ertrag erzielt und ein auftragsbezogener Ressourcenverbrauch eingetreten. Nachdem diese Entscheidung getroffen ist, wird der zeitlich darauffolgenden Kundenauftrag betrachtet.

Für die Entscheidung über die Annahme oder Ablehnung eines Kundenauftrags (KA) zur Instandhaltung von Gütern bedarf es einer umfassenderen Betrachtung als nur die kurzsichtige Entscheidung über einen einzelnen Auftrag. Angenommen ein Reparaturdienstleister besitzt ein bestimmtes Kontingent an unterschiedlichen Ressourcen über einen bestimmten Zeitraum zur Erfüllung seiner angebotenen Dienstleistung. In diesem betrachteten Zeitraum treffen jetzt unterschiedliche Kundenaufträge mit unterschiedlicher Wertigkeit ein. Zur Maximierung seiner Erträge über diesen Zeitraum unter Beachtung der vorhanden Ressourcenkapazität kann es sinnvoll sein, Anfragen mit niedrigem Ertrag abzulehnen, sofern im weiteren Verlauf des betrachteten Zeitraums Aufträge mit höherem Ertrag eintreffen.

Bei der Problemformulierung der Auftragsannahmeentscheidung bei auftragsbezogenen Instandhaltungsprozessen handelt es sich um ein stochastisch-dynamisches Optimierungsmodell. Eine mögliche grafische Darstellungsform dieser Problemstellung erfolgt als sogenannter Entscheidungsbaum. Bei dem Entscheidungsbaum handelt es sich um ein gerichteten azyklischen Graphen. Abhängig der eintreffenden Anfragen, der vorhandenen Kapazitäten, der möglichen Entscheidungen und des betrachteten Zeithorizont hat der Entscheidungsbaum eine unterschiedliche Anzahl an Kanten und Knoten. Ein Knoten bildet jeweils einen neuen Zustand des Systems ab und eine Kante die mögliche Entscheidung in einen neuen Systemzustand zu gelangen.

In dieser Arbeit wird aber zusätzlich zur Annahmeentscheidung eines Auftrags zur Instandhaltung eines Gutes auch die Entscheidung über eine mögliche Lagerhaltung von Gütern getroffen. Durch Annahme der Entscheidung zur Lagerhaltung des defekten Gutes wird dieses in die Lagerhaltung des Reparaturdienstleisters übernommen und durch ein bereits repariertes Gut ausgetauscht. Das reparierte Gut entspricht den vom jeweiligen Auftrag geforderten Instandhaltungszustand. Anders formuliert bedeutet dies, dass ein Reparaturdienstleister nicht nur die Entscheidungsmöglichkeit über die Instandsetzung des Gutes hat, sonder auch die Möglichkeit hat in Abhängigkeit des verfügbaren Lagerbestandes bereits reparierte Güter zur Befriedigung der Kundenaufträge zu verwenden.

Das stochastisch-dynamische Optimierungsmodell muss demnach die Entscheidung treffen, ob die Auftragsannahme zur Instandsetzung des Gutes, die Lagerhaltung des defekten Gutes sowie die Herausgabe eines bereits reparierten Gutes oder die Ablehnung des Kundenauftrags erfolgen soll. Diese Entscheidung erfolgt in Abhängigkeit der verfügbaren restlichen Ressourcenkapazität, die zur Instandsetzung der Güter notwendig ist, des aktuell-vorhandenen Lagerbestandes der bereits reparierten Güter und der noch potentiell eintreffenden Anfragen zur Instandhaltung von Gütern.

\section{Zielsetzung}

Die mathematische Darstellung des Optimierungsmodells der Auftragsannahme- und Lagerhaltungsentscheidungen bei auftragsbezogenen Instandhaltungsprozessen erfolgt als \textit{dynamische Programmierung (DP}).
Die Aufgabe des DP ist laut \cite{talluri2004theory} die Entscheidungsfindung zu unterstützen, damit der Gesamtertrag des Dienstleisters maximiert wird. Der Gesamtertrag wird bewertet in Geldeinheiten (GE). 
Das Grundmodell zur Annahme von Kundenaufträgen bzw. -anfragen kommt aus der wissenschaftlichen Betrachtung des Revenue Management von Dienstleistungsunternehmen mit beschränkten Ressourcenkapazitäten.
%Bei den betrachteten Ressourcen handelt es sich um zeitgebundene Ressourcen mit einer jeweiligen vordefinierten Ressourcenkapazität über den gesamten Zeithorizont. 

Die Zielsetzung der Arbeit ist demnach das Grundmodell des Revenue Managements zur Annahme von Kundenaufträgen mit der Möglichkeit der Entscheidung der Lagerhaltung der Güter zu erweitern.


\section{Aufbau der Arbeit}

Der Aufbau der Arbeit ist demnach,

die Grundlagen über auftragsbezogenen Instandhaltungsprozessen zu definieren,

das Konzept des Revenue Management bei der Annahme von Kundenaufträgen vorzustellen,

bestehende Ansätze zum Lösen des stochastisch-dynamischen Optimierungsmodells bei der Annahme von Kundenaufträgen in der Auftragsproduktion und bei Instandhaltungsprozessen aufzuführen

sowie eine dynamische Programmierung für die Auftragsannahme- und Lagerhaltungsentscheidungen bei auftragsbezogenen Instandhaltungsprozessen zu formulieren und exakt zu lösen.
\chapter{Grundlagen zu auftragsbezogenen Instandhaltungsprozessen}
\markboth{2 Grundlegende Begriffe}{}
\setcounter{footnote}{4}  %um durchgehende Fußnotennummerierung zu haben, hier die Anzahl der bisherigen Fußnoten eintragen

\section{Einordnung in die Produktionswirtschaft}

\section{Charakteristika}

\section{Relevanz für betriebliche Entscheidungen}

\chapter{Das Konzept des Revenue Managements bei der Annahme von Aufträgen}
\markboth{2 Revenue Management in der Auftragsannahme}{}
\setcounter{footnote}{4}  %um durchgehende Fußnotennummerierung zu haben, hier die Anzahl der bisherigen Fußnoten eintragen


\section{Herkunft des Revenue Managements}
Der Begriff \textit{Revenue Management} wird im deutschsprachigen Raum meist mit \textit{Ertrags}\-\textit{management} oder \textit{Erlösmanagement} übersetzt.\footnote{Vgl. z. B. \cite{zehle1991yield}, S. 486} Yield Management wird als Synonym benutzt.\footnote{Vgl. z. B. \cite{kolisch2006revenue}, S. 319} Dabei greift der Begriff zu kurz, da  Yield im Luftverkehr den Erlös je Passagier und geflogener Meile bezeichnet.\footnote{Vgl. z. B. \cite{weatherford1998tutorial}, S. 69} Daher hat sich der Term \textit{Revenue Management} gegenüber Yield Management durchgesetzt.\footnote{Vgl. \cite{Klein:2008aa}, S. 6} Erste Ansätze des RM wurden in der Praxis entwickelt. Durch die Deregulierung des amerikanischen Luftverkehrsmarktes im Jahr 1978 mussten die traditionellen Fluggesellschaften ihre Wettbewerbsfähigkeit gegenüber Billiganbietern erhöhen und entwickelten das frühe RM.\footnote{Vgl. \cite{Petrick:2009aa}, S. 1-3} In der Literatur ist der Begriff des RM unterschiedlich definiert. \cite{friege1996yield} bezeichnet das RM als \textit{Preis-Mengen-Steuerung}, \cite{daudel1992yield} als \textit{Preis-Kapazitäts-Steuerung} und \cite{talluri2004theory} verstehen es als das gesamtes \textit{Management der Nachfrage}. \citeauthor{klein2001revenue} (2001, S. 248) definiert RM als:

\begin{quote}
\glqq Revenue Management umfasst eine Reihe von quantitativen Methoden zur Entscheidung über Annahme oder Ablehnung unsicherer, zeitlich verteilt eintreffender Nachfrage unterschiedlicher Wertigkeit. Dabei wird das Ziel verfolgt, die in einem begrenzten Zeitraum verfügbare, unflexibel Kapazität möglichst effizient zu nutzen.\grqq
\end{quote}

\cite{Petrick:2009aa} definiert das RM als Ziel einer Unternehmung die Gesamterlöse zu maximieren, die sich aufgrund der speziellen Anwendungsgebiete ergeben. Damit definiert \cite{Petrick:2009aa} das RM als Zusammenfassung aller Interaktionen eines Unternehmens, die mit dem Markt, also der Absatz- oder Nachfrageseite, zusammenhängen. \cite{kimms2005revenue} weisen darauf hin, dass eine differenzierte Betrachtung des Konzepts notwenig ist: Einerseits im Hinblick auf die \textbf{Anwendungsvoraussetzungen} und andererseits im Hinblick auf die \textbf{Instrumente des Revenue Managements}, damit verdeutlicht dargestellt ist, in welchen Branchen das RM Potentiale liefert. Dabei sollten branchenspezifische Besonderheiten, neben den zahlreichen Ähnlichkeiten Berücksichtigung finden, sowie das begrenzte Kapazitätenkontingent, damit die Potentiale des RM zur Maximierung der Gesamterlöse in den Dienstleistungsbranchen erfolgen kann.\footnote{Vgl. z. B. \cite{Martens:2009aa}, S. 11-24}\\

\section{Anwendungsvoraussetzungen und Instrumente des Revenue Managements}

Es wurden typische Anwendungsgebiete für das RM aufgezeigt. (???) Jedoch bereitet die Definition weitere Schwierigkeiten. \cite{kimms2005revenue} versuchen durch eine umfangreiche Diskussion einige Erklärungsansätze aufzuzeigen. Zum einen hat das RM vor allem aus dem älteren, englischsprachigen Bereich einen engen Bezug zu konkreten Anwendungsgebieten. Weiter versuchen viele Autoren das komplexe Konzept des Revenue Managements in einer kurzen Erklärung zu verdeutlichen. Dieses läuft letztlich darauf hinaus, dass diese Autoren einige situative Merkmale und Instrumente des Managements vermischen, gleichzeitig aber versuchen, die Zielsetzung festzulegen und das Anwendungsgebiet auf bestimmte Branchen zu beschränken. 

  Die beiden ersteren Definitionen können als Synonym für eines der Instrumente des RM stehen und daher finden diese für das gesamte Konzept keine weitere Verwendung.\footnote{Vgl. z. B. \cite{Petrick:2009aa}}\\

Im Kern lassen sich drei wichtige Perspektiven für eine Definition des Revenue Managements nach \cite{Petrick:2009aa}, \cite{stuhlmann2000kapazitatsgestaltung},  \cite{corsten1999yield} übernehmen:
\begin{itemize}
	\item Ziel ist es die Gesamterlöse unter möglichst optimaler Auslastung der vorhandenen Kapazitäten zu maximieren.
	\item Durch eine aktive Preispolitik wird das reine Kapazitäts- oder Auslastungsmanagement unterstützt.
	\item Für die erfolgreiche Implementierung des Revenue Managements ist eine umfangreiche Informationsbasis notwendig. Es muss u. a. eine möglichst gute Prognose über die zukünftige Nachfrage und Preisbereitschaft der Kunden vorhanden sein.
\end{itemize}
\vspace{0.2cm}
Wie im vorherigen Abschnitt beschrieben, müssen bestimmte Voraussetzungen bestehen, damit die Instrumente des RM zur Anwendung kommen können.

\cite{Petrick:2009aa} weist da\-rauf hin, dass anhand von Anwendungsvoraussetzungen geprüft wird, ob das RM für die jeweilige Situation des Unternehmens (oder die gesamte Branche) zur Maximierung des Gesamterlöses beiträgt. \cite{kimes1989yield} definiert die in der Literatur häufigsten Anwendungsvoraussetzungen:\footnote{Vgl. u. a. \cite{friege1996yield}, S. 616-622, und \cite{weatherford1992taxonomy}, 831-832}
\begin{itemize*}
	\item \glqq weitgehend fixe\grqq\;Kapazitäten
	\item \glqq Verderblichkeit\grqq\;bzw. \glqq Nichtlagerfähigkeit\grqq\;der Kapazitäten und der Leistung
	\item Möglichkeit zur Vorausbuchung von Leistungen
	\item stochastische, schwankende Nachfrage
	\item hohe Fixkosten für die Bereitstellung der gesamten Kapazitäten bei vergleichsweise geringen variablen Kosten für Produktion einer Leistungseinheit
	\item Möglichkeit zur Marktsegmentierung und im Ergebnis dessen zur segmentorientierten Preisdifferenzierung
\end{itemize*}
\vspace{0.2cm}
\cite{Klein:2008aa} setzen sich mit den Anwendungsvoraussetzungen von mehreren Autoren auseinander. Sie konnten Gemeinsamkeiten innerhalb der Definitionen der Autoren finden, aber zeigten auch die Unterschiede und die Kritiken auf. In ihrer Arbeit übernehmen sie die Anwendungsvoraussetzung von \cite{corsten1998yield}: "Marktseitige Anpassungserfordernis steht unternehmesseitigig unzureichendes Flexibilitätspotential hinsichtlich der Kapazität - bezogen auf Mittel- oder Zeitaufwand gegenüber". Zugleich weisen sie jedoch darauf hin, dass zum Verständnis eines komplexen und interdisziplinären Ansatzes auch die Definitionen anderer Autoren im Hinblick auf das Verständnis der Anwendungsvoraussetzungen beitragen.\\

Auf Grundlage der von \cite{friege1996yield} beschriebenen Anwendungsvoraussetzungen hat \cite{Petrick:2009aa} drei Instrumente des RM bestimmt. Die Instrumente benötigen als Grundlage \textit{Daten der Prognose}, damit sie zur Anwendung kommen.\footnote{Die Prognose zählt laut \cite{Petrick:2009aa} nicht als eigenständiges Instrument des RM.} Zu den Instrumenten zählen die \textbf{segmentorientierte Preisdifferenzierung}, die \textbf{Kapazitäten\-steuerung} und die \textbf{Über\-buchungssteuerung}. Es lassen sich unterschiedliche Ab\-hängigkeit\-en der Instrumente untereinander ermitteln.\footnote{Als Beispiel baut die Kapazitätensteu\-erung auf den Ergebnissen der Preisdifferenzierung auf und die Überbuchungssteuerung kann selten ohne Kapazitätensteuerung gelöst werden.}

\section{Mathematische Modellformulierung des Revenue Managements beim Entscheidungsproblem der Auftragsannahme von Instandhaltungsprozessen}
Im Folgenden wird das dynamisch, stochastische Grundmodell des RM nach \citeauthor{talluri2004revenue} (2004, S. 18-19) beschrieben. Ein Dienstleistungsnetzwerk eines Anbieters benötigt jeweils zur Erstellung einer Dienstleistung eine bestimmte Kombination an Ressourcen aus der Menge der Ressourcen $\mathcal{H} = \{1,...,l \}$. Der Index $h$ beschreibt dabei eine jeweilige Ressource und der Index $l$ die gesamte Anzahl an möglichen Ressourcen. Die jeweilig verbleibende Kapazität einer Ressource $h \in \mathcal{H}$ ist durch den Parameter $c_{h}$ beschrieben und die gesamten Kapazitäten der Ressourcen ist als Vektor $\textbf{c}=(c_{1},...,c_{h},...,c_{l})$ formuliert. Ein Produkt in dem Netzwerk ist durch den Parameter $j$ aus der Menge an Produkten $\mathcal{J} = \{1,...,n \}$ %für die Menge der Ressourcen $\mathcal{H}$
beschrieben. Die gesamte Anzahl an Produkten ist durch den Parameter $n$ definiert. Sobald ein Produkt $j\in \mathcal{J}$ abgesetzt ist, fällt für den Verkauf der Ertrag $r_{j}$ an. Der Buchungshorizont entspricht $T$ Perioden und kann jeweils in einzelne Perioden $t=1,...,T$ aufgeteilt werden. Dabei muss Beachtung finden, dass der Buchungshorizont $T$ gegenläufig verläuft. Die Wahrscheinlichkeit der Nachfrage eines Produkts $j$ in der Periode $t$ entspricht $p_{j}(t)$ und die Wahrscheinlichkeit, dass keine Nachfrage in der Periode $t$ eintrifft, entspricht $p_{0}(t)$. Es gilt $\sum_{j\in \mathcal{J}}p_{j}(t)+p_{0}(t)=1$ und somit kann $p_{0}(t)$ durch den Term $p_{0}(t)=1-\sum_{j\in \mathcal{J}}p_{j}(t)$ für die Periode $t$ ermittelt werden.\footnote{Vgl. \citeauthor{talluri2004revenue}, S. 18} Die noch erwartete Nachfrage $D_{jt}$ für ein bestimmtes Produkt $j$ für eine beliebige Periode $t$ lässt sich durch $\sum_{\tau=1}^{t}p_{j}(\tau)$ aggregieren.\\

Die bisherige Notation ist analog der Formulierung des Grundmodells nach \citeauthor{talluri2004revenue} (2004, S. 18-19). Nachfolgend wird die Modellerweiterung nach \cite{gonsch2013using} beschrieben. Sofern ein Anbieter opake Produkte in sein Produktportfolio integriert, muss die Menge $\mathcal{M}_{j}\subseteq\mathcal{I}$ eingeführt werden. Mit dieser Menge ist die Erfassung des differenzierten Ressoucenverbrauchs der spezifischen und opaken Produkte $j\in\mathcal{J}$ möglich. Sie ist eine Teilmenge der Indexmenge $\mathcal{I}\subseteq\mathbb{N}^{+}$, die alle produktspezifischen Kombinationen für die Menge der Ressourcen $\mathcal{H}$ beschreibt. Die Menge $\mathcal{I}$ beschreibt alle möglichen Kombinationen von Produkten und Ressourcen. Für das weitere Vorgehen genügt das Betrachten der jeweiligen möglichen Ausführungsmodi $\mathcal{M}_{j}$ eines Produkts $j\in\mathcal{J}$. Eine einzelne produktspezifische Kombination der verfügbaren Ressourcen ist durch den Parameter $m\in\mathcal{M}_{j}$ beschrieben. Der jeweilige Verbrauch einer Ressource $h$ im Ausführungsmodus $m$ durch Annahme einer Anfrage nach einem Produkt $j$ ist anhand des Parameters $a_{hm}$ beschrieben. Durch Vektorschreibweise kann der Ressourcenverbrauch einer produktspezifischen Kombination als $\textbf{a}_{m}=(a_{1m},...,a_{hm},...,a_{lm})$ formuliert werden. Die Tabelle \ref{Tabelle0} verdeutlicht den Zusammenhang der Produkte $j\in\mathcal{J}$ und der Ressourcen $h\in\mathcal{H}$ mit dem dazugehörigen produktspezifischen Ausführungsmodus $m\in\mathcal{M}_{j}$ in einer Matrix. \\

\begin{table}[h!]
 \renewcommand{\arraystretch}{1.5}
  \setlength{\tabcolsep}{4mm}
  \begin{center}
    \caption{Beispiel einer Produkt-Ressourcen-Matrix}  \label{Tabelle0}
    \vspace*{3mm}
    \begin{tabular}{c|c|c|c|c|c|c|c}   %hier die Spaltenausrichtung, -breite, -begrenzung und -anzahl eintragen
     \multirow{3}{*}{Produkt $j$} & \multicolumn{5}{c|}{Ressurcenverbrauch $a_{hm}$} & \multirow{3}{*}{Ausführungsmodus $m$} & \multirow{3}{*}{Erlös $r_{j}$}\\
           & \multicolumn{5}{c|}{für jeweilige Ressource $h$} &  & \\ 
           & $1$ & $2$ & $3$ & $4$ &$5$ &  \\ \hline
         
         \multirow{4}{*}{$1$}  & $a_{11}$ & -- & $a_{31}$ & $a_{41}$ & -- & 1 &  \multirow{4}{*}{$100$}  \\ %\cline{2-7}
                & \cellcolor[gray]{0.9}--& \cellcolor[gray]{0.9}$a_{22}$ & \cellcolor[gray]{0.9}$a_{32}$ & \cellcolor[gray]{0.9}$a_{42}$ & \cellcolor[gray]{0.9}--& \cellcolor[gray]{0.9}2 & \\ %\cline{2-7}
                & $a_{13}$& --& --& $a_{43}$ & --& 3 &   \\ %\cline{2-7}
          &\cellcolor[gray]{0.9}--& \cellcolor[gray]{0.9}$a_{24}$& \cellcolor[gray]{0.9}-- & $\cellcolor[gray]{0.9}a_{44}$ & \cellcolor[gray]{0.9}--& \cellcolor[gray]{0.9}4 &   \\ \hline
                              2  & $a_{15}$ &-- & -- &--& $a_{55}$ & 5 &  150 \\ \hline
    \end{tabular} \\[3mm]
  \end{center}
\end{table}

Das Beispiel in Tabelle \ref{Tabelle0} zeigt einen Reiseveranstalter mit fünf Ressourcen. Bei den ersten zwei Ressourcen handelt es sich um einen 1.-Klasse-Flug ($h=1$) und um einen 2.-Klasse-Flug ($h=2$). Bei der Ressource $3$ handelt es sich jeweils um eine mögliche Überführungsfahrt ($h=3$) zu einem Hotel am Strand ($h=4$). Bei der letzten Ressource handelt es sich um ein Business-Hotel direkt am Flughafen ($h=5$). Die verschiedenen Ressourcen sind in dem Beispiel unterschiedlich zu Produkten $j\in\mathcal{J}$ kombiniert. Dies könnte einerseits daran liegen, dass einige Kombinationen für einen Anbieter nicht rentabel sind oder andererseits keine Nachfrage erhalten. Damit zeigt die Produkt-Ressourcen-Matrix nicht alle möglichen Kombinationsmöglichkeiten $i\in\mathcal{I}$ für die Menge an Ressourcen $\mathcal{H}$. Die Matrix zeigt keinen Wert für den Ressourcenverbrauch an, sofern $a_{hm}=0$ entspricht. Damit wird zur Erstellung des Produkts $j$ in dem Ausführungsmodus $m$ die jeweilige Ressource $h$ nicht benötigt. In dem Beispiel ist das Produkt $j=1$ ein opakes Produkt mit den Ausführungsmodi $m\in\mathcal{M}_{1}=\{1,2,3,4\}$. Dies liegt an der frei gewählten Angebotsstruktur. Der Anbieter könnte jeden Ausführungsmodus $m\in\mathcal{M}_{j}$ für ein eigenständiges Produkt nutzen. Bspw. handelt es sich bei dem Produkt $j=2$ um ein spezifisches Produkt, das der Anbieter nur in einem Ausführungsmodus $m\in\mathcal{M}_{2}=\{5\}$ anbietet. Somit sind spezifische Produkte eines Netzwerks als Sonderfall von opaken Produkten anzusehen, die nur einen Ausführungsmodus aufweisen ($|\mathcal{M}_{j'}|=1$).\footnote{Vgl. \cite{gonsch2013using}, S. 96}\\

Mit den vorangegangenen Parametern kann der maximal erwartete Ertragswert $V(\textbf{c},t)$ für eine Periode $t$ bei einer noch vorhandenen Ressourcenkapazität $\textbf{c}$ als Bellman-Gleichung formuliert werden (\textbf{DP-op}):\footnote{Vgl. \cite{gonsch2013using}, S. 97}

\begin{equation}\label{DPop}
V(\textbf{c},t)=\sum_{j\in\mathcal{J}}p_{j}(t)\max\left( V(\textbf{c},t-1),\; r_{j}+\max_{m\in\mathcal{M}_{j}}V(\textbf{c}-\textbf{a}_{m},t-1)\right)+p_{0}(t)V(\textbf{c},t-1)
\end{equation}

Es handelt sich hier um die Modellformulierung der dynamischen Programmierung im RM opaker Produkte. Die Gleichung weist die Grenzbedingungen $V(\textbf{c},0)=0$ für $\textbf{c}\ge0$ sowie sonst $V(\textbf{c},0)=-\infty$ auf, da eine jeweilig verbleibende Kapazität nach Bereitstellung des Produkts wertlos und eine negative Ressourcenkapazität nicht möglich ist. Die Standardformulierung der dynamischen Programmierung wird mit dem Term $\max_{m\in\mathcal{M}_{j}}V(\textbf{c}-\textbf{a}_{m},t-1)$ erweitert. Damit ist sichergestellt, dass eine Anfrage nach einem opaken Produkt $j$ nur im Ausführungsmodus $m$ mit dem höchsten Ertragswert gewählt wird. Der Gesamtertrag des Anbieters ist maßgeblich durch die Entscheidung der gewählten Ausführungs\-modi $m\in\mathcal{M}_{j}$ abhängig, da das Modell durch eine jede Entscheidung bzgl. der weiteren möglichen opaken Produkte neu gelöst werden muss. Eine eintreffende Anfrage nach einem opaken Produkt $j$ ist demnach dann akzeptiert, wenn gilt:
\begin{equation}\label{r}
r_{j}\ge\min_{m\in\mathcal{M}_{j}}\bigl\{V(\textbf{c},t-1)-V(\textbf{c}-\textbf{a}_{m},t-1)\bigr\}
\end{equation}


Somit erfolgt die Akzeptanz einer Anfrage nach einem opaken Produkt $j\in\mathcal{J}$ ausschließlich nur dann, sofern die OK des Ressourcenverbrauchs niedriger als der Ertrag ist. Zusätzlich wird der Ausführungsmodus $m\in\mathcal{M}_{j}$ mit den niedrigsten OK gewählt, wodurch das Maximum des gesamten Ertragswerts gewährleistet bleibt. Ein potentieller Ausführungs\-modus $m^{*}$ mit minimalen OK ($V(\textbf{c},t-1)-V(\textbf{c}-\textbf{a}_{m^{*}},t-1)$) ist gewählt und die Kapazität werden dementsprechend reduziert. Bei spezifischen Produkten existiert nur ein Ausführungsmodus $|\mathcal{M}_{j'}|=1$ und daher ist die Maximalfunktion in der Gleichung \eqref{DPop} und die Minimalfunktion in der Gleichung \eqref{r} nicht notwendig.
\chapter{Bestehende Ansätze zur Annahme von Aufträgen in der Auftragsfertigung und bei Instandhaltungsprozessen}
\markboth{4 Bestehende Ansätze}{}
\setcounter{footnote}{4}  %um durchgehende Fußnotennummerierung zu haben, hier die Anzahl der bisherigen Fußnoten eintragen

Das Konzept des Revenue Managements (RM) zur Annahme von Aufträgen im Dienstleistungsgewerbe findet bereits über mehrere Dekanen in der wissenschaftlichen Literatur Anwendung.\footnote{Vgl. Klein (2001), S. 246 NACHLESEN SONST FALSCH!!!!!} Neue Veröffentlichungen versuchen das Konzept auf die Problemstellung der Annahme von Anfragen der Auftragsfertigung bzw. Kundeneinzelfertigung zu übertragen.\footnote{Vgl. ???} Wie in Kapitel \ref{Instandhaltung} dargelegt, kann der Instandhaltungsprozess eines Dienstleistungsunternehmens einer Auftragsfertigung gleichgesetzt werden. \cite{kimms2005revenue} geben einen Überblick über das traditionelle Konzept des RM über verschiedene Branchen. Dabei schreiben die Autoren, dass das Konzept des RM vermehrt Anwendung findet, damit Unternehmen eine Unterstützung in der Entscheidungsfindung erhalten, welche Aufträge zur Auftragsfertigung akzeptiert werden sollen.\footnote{Vgl. \cite{kimms2005revenue}, S. 1} \cite{quante2009management} gibt einen Überblick über relevante Literatur des traditionellen RM. Tabelle \ref{Überblick} zeigt eine Anlehnung der Übersicht von \cite{quante2009management} mit den Publikationen zum traditionellen RM in der Fertigungsindustrie. Die Tabelle zeigt jeweils zur Publikation den Kundenauftragskoppelpunkt, die Anzahl der berücksichtigen Konsumerklassen und die Methode aufgeführt ist. Die Konsumerklassen resultieren aus der für das Konzepts des Revenue Management notwendigen Marktsegmentierung von Kunden bzw. Auftragstypen. Einige Konzepte und Modelle berücksichtigen daher explizit die Anzahl dieser Klassen.


\begin{table}[h!]
  \begin{center}
    \caption{Überblick über Publikationen des traditionellen Konzepts des Revenue Managements in der Fertigungsindustrie}  \label{Überblick}
    \vspace*{3mm}
    \begin{tabular}{llll}   %hier die Spaltenausrichtung, -breite, -begrenzung und -anzahl eintragen
     Autoren & KAKP  & \#Klassen & Methode  \\ \hline
     \cite{deBHarris1995299} &      ATO          &  2  &  K, M \\
      \cite{Kalyan:2002aa}      &      MTO/ATO/MTS          &  --  &  K \\
                \cite{rehkopf:2005aa};   &      MTO          &  mehrere &  K, M, F \\
                               \cite{Spengler:2007aa}   &                &  &   \\
            \cite{petrick2012using}      &      MTO          &  ?  &  M \\
              \cite{DECI:DECI074}  &      MTO          &  mehrere  &  M \\
               &      MTO          &  --  &  K \\
                   &      MTO          &  mehrere  &  M \\
               &      MTO          &  --  &  K \\
               &      MTO          &  --  &  K, M, L \\
                &      MTO/MTS          &  3  &  F \\
                   &      MTO          &  mehrere  &  M \\
                  &      --          &  --  &  L \\
                &        ATO    &  --  & K, F  \\    \hline
    \end{tabular} \\[3mm]
    {\footnotesize \textbf{In Anlehnung an:} \cite{quante2009management}, S. 44.}\\
        {\footnotesize \textbf{Legende:} KAKP: Kundenauftragskoppelpunkt, F: Fallstudie, K: Konzeption, L: Literaturüberblick, M: Simulations-/Analysemodell. }   %footnotesize liefert Schrift in Größe 10pt
  \end{center}
\end{table}

\cite{deBHarris1995299} beziehen die RM-Komponenten der differenzierten Preispolitik und eine multiklassen Kapazitätsallokation in ihr Modell mit ein. Sie zeigen, dass sofern stochastische Nachfrage auf fixe und kurzfristige Kapazität trifft, dass es zu Lagerfehlbeständen bei Niedrig-Preis-Segmenten führt. Jedoch rechtfertigen diese Lagerfehlbestände eine Premiumpreisstrategie bei den Kundengruppen mit höherer Preisbereitschaft, was letztendlich zu Umsatzsteigerungen führt.\footnote{Vgl. \cite{deBHarris1995299}, S. 307-308.} \cite{Kalyan:2002aa} beschäftigt sich mit der Bestätigung der Anwendungsvoraussetzungen des RM für die Auftragsfertigung. Das vom Autor beschriebene Konzept sieht die Einführung eines minimalen Akzeptanzwert vor. Sofern dieser Wert bekannt ist, kann ein Unternehmen bei jedem Auftragseingang die Entscheidung treffen, welche Anfrage angenommen oder abgelehnt werden soll.

Der vom Autoren \cite{Kalyan:2002aa} eingeführte minimalen Akzeptanzwert wird in der wissenschaftlichen Literatur im Kontext des Konzepts des Revenue Managements als sogenannten Bid-Preis bezeichnet. Bei dem Bid-Preis handelt es sich um einen variierenden Parameter in Abhängigkeit der Zeit (bzw. der Periode) und der verfügbaren Ressourcenkapazität des betrachteten Netzwerks. Er kann als statische Information der optimalen Lösung für die eindimensionalen Probleme angesehen werden. Die Ermittlung des Bid-Preises erfolgt bei den traditionelle Modellformulierungen des RM anhand des \textit{deterministische lineare Programm (DLP)} unter deterministisch eintreffenden Nachfragen.\footnote{Vgl. \cite{talluri2004revenue}, S. 107-108} %Der Bid-Preis korrespondiert zur Entscheidungsvariable $x_{jm}$ des DLP, da er verbunden ist mit der Kapazität einer jeden Ressource $h$.\footnote{Vgl. \cite{gonsch2013using}, S. 98}
Er fungiert als Schwellenpreis für eine jede Ressource im Netzwerk und ist normalerweise beschrieben als geschätzte marginale Kosten aufgrund des nächsten sukzessiven Verbrauchs einer Einheit der Ressourcenkapazität.\footnote{Vgl. \cite{talluri2004theory}, S. 89} Laut \cite{gonsch2013using} erfolgt der Ansatz erstmalig von \cite{talluri2001airline} in Verbindung der Optimierung von Passagierrouten. Das Verfahren des Bid-Preises ist ein einfacher Weg, die Kapazitäten der Ressourcen in einem Netzwerk eines Anbieters zu kontrollieren.\footnote{Vgl. \cite{talluri2004theory}, S. 86-87\label{RMH}} Viele Konzepte der neueren Veröffentlichungen im Bereich des Netzwerk RM sehen die Verwendung des Bid-Preise vor.\footnote{Vgl. \cite{petrick2010dynamic}, S. 2028; \cite{gonsch2013using}, S. 98-100.}

\cite{rehkopf:2005aa} zeigen durch das Lösen eines linearen Modells die Kapazitätsallokation für die Problemformulierung des Netzwerk-RM im Fall von MTO-Prozessen. Die Autoren fokussieren sich dabei auf die Branche der Eisen- sowie Stahlindustrie und veröffentlichen zwei Publikationen mit ihren Forschungsergebnissen. Dabei findet der im vorherigen Absatz definierte Bid-Preis Anwendung. In der Fallstudie wird gezeigt, dass durch eine Bid-Preis-Strategie der Gesamtdeckungsbeitrag verbessert wird.\footnote{Vgl. \cite{Spengler:2007aa}, S. 157–171} Dabei werden die Bid-Preise jeweils mit einer linearen Modellformulierung sowie einer multi-dimensionalen Knaksack-Modell\-for\-mu\-lier\-ung berechnet und verglichen. Dabei wird deutlich, dass beide Verfahren den Deckungsbeitrag ähnlich verbessern, wobei sich das Lösen anhand der Knaksack-Modell\-for\-mu\-lier\-ung als robuster darstellt. 

Das Modell von \cite{petrick2012using} zeigt eine Anwendung des RM mit flexiblen Produkten, welche auf MTO/MRO-Prozesse überführt werden können.\footnote{Vgl. \cite{petrick2012using}, S. 218.} ....

Anders als die vorher aufgeführten Autoren befassen sich die Autoren \cite{DECI:DECI074} nicht nur mit dem Auftragsannahmeproblem, welches sich im Konzept des RM ergibt, sondern auch um die Planung und der Bestimmung des genauen Zeitpunkts der Fertigung. Die Autoren stellen eine Heuristik vor, die Aufträge in verschiedene Lose sortiert. Dabei werden mehrere Konsumerklassen beachtet. Die Basisidee des Verfahrens ist die Beachtung der relativen Gewinnspannen der Aufträge, damit der Gesamtdeckungsbeitrag sich erhöht.\footnote{Vgl. \cite{DECI:DECI074}, S. 291} Unter Einsatz der Heuristik zeigen die Autoren, dass ein höherer Gewinn aufgrund einer effizienten Nutzung der verfügbaren Kapazität erzielt wird. Dies kommt zustande, da eine Unterscheidung der Aufträge in Bezug von verschiedene Produktklassen mit unterschiedlichen Deckungsbeiträge erfolgt.





% !TEX encoding = UTF-8 Unicode
\chapter{Ein exaktes Lösungsverfahren zur Auftragsannahme- und Lagerhaltungsentscheidung bei auftragsbezogenen Instandhaltungsprozessen}
\markboth{3 Ein exaktes Lösungsverfahren zur Auftragsannahme- und Lagerhaltungsentscheidung}{}
\setcounter{footnote}{7}

\section{Mathematische Modellformulierung}

Aufbauend auf der Modelldarstellung der Auftragsannahme mittels Revenue Management erfolgt in diesem Kapitel die Erweiterung um die Lagerhaltungsentscheidung. Instandhalter haben die Möglichkeit Anfragen nach MRO-Prozessen entweder durch Kapazitätsinanspruchnahme oder durch Lagerentnahme eines neuwertigen Produkts zu akzeptieren. Abhängig der Entscheidung, in welchem dieser Modi die Anfrage befriedigt wird, erfolgt entweder eine Kapazitäts- oder Lagerreduktion.

Richtiger Text !!!

\subsection{Lagerentnahmeentscheidung}

Das Gleichung \eqref{DP} wird um den Parameter des Lagerbestands $y_{j}$ erweitert. Bei dieser Modellerweiterung fungiert der Parameter $y_{j}$ als Lagerbestand von Produkten $j\in\mathcal{J}$. D. h. es sind Bündel an bereits verbrauchten Ressourcen $h\in\mathcal{H}$ gemeint, die direkt den Nachfragern zur Verfügung gestellt werden. Der Parameter lässt sich als Vektor $\textbf{y}$ interpretieren, wobei die Länge des Vektors die Anzahl an Produkten $j\in\mathcal{J}$ und der sortierte Eintrag des Vektors dem Lagerbestands des Produkts $j$ entspricht. Sofern eine Anfrage über den Lagerbestand befriedigt wird, erfolgt eine Reduktion des Lagerbestand durch den Parameter $s_{jj}$. Da in dieser Modellannahme eine Anfrage nach einem Produkt $j$ auch nur mit dem Lagerbestand des Produkts $j$ angenommen werden kann, entspricht $s_{j=j}=1$ und alle anderen $s_{j\neq j}=0$. Damit lässt sich ein Vektor $\textbf{s}_j$ formen, der als Lagerentnahme für eine Anfrage nach dem Produkt $j$ dient. Weiter können die einzelnen Vektoren $s_{j}$ als Matrix $\textbf{S}$ aufgebaut werden. Die Matrix $\textbf{S}$ entspricht einer Einheitsmatrix $I_{n}\in\mathbb{R}^{n\times n}$ wobei $n=j$.

Die Gleichung \eqref{DP} lässt sich damit wie folgt formulieren:

\begin{alignat*}{2}
V(\textbf{c}, \textbf{y}, t) =\;& \sum_{j \in \mathcal{J}}p_{j}(t)\max[V(\textbf{c}, \textbf{y}, t-1), \\
& r_{j} + V(\textbf{c}-\textbf{a}_j, \textbf{y}, t-1),\\
& r_{j} + V(\textbf{c}, \textbf{y}-\textbf{s}_j, t-1)] \\
& + p_{0}(t)V(\textbf{c}, \textbf{y}, t-1)\\[10pt] 
%= \;& \sum_{j \in \mathcal{J}}p_{j}(t)V(\textbf{c}, \textbf{y}, t-1)\\
%& + \sum_{j \in \mathcal{J}}p_{j}(t)\max[r_{j} - V(\textbf{c}, \textbf{y}, t-1) + V(\textbf{c}-\textbf{a}_j, \textbf{y}, t-1),0]\\
%& +  \sum_{j \in \mathcal{J}}p_{j}(t)\max[r_{j} - V(\textbf{c}, \textbf{y}, t-1) + V(\textbf{c}, \textbf{y}-\textbf{s}_j, t-1),0]\\
%&+ p_{0}(t)V(\textbf{c}, \textbf{y}, t-1)\\[10pt] 
= \;& \sum_{j \in \mathcal{J}}p_{j}(t)V(\textbf{c}, \textbf{y}, t-1)\\
&+ \sum_{j \in \mathcal{J}}p_{j}(t)[\max[r_{j} - V(\textbf{c}, \textbf{y}, t-1)+ V(\textbf{c}-\textbf{a}_j, \textbf{y}, t-1),0]\\
&+ \max[r_{j} - V(\textbf{c}, \textbf{y}, t-1) + V(\textbf{c}, \textbf{y}-\textbf{s}_j, t-1),0]]\\
&+ p_{0}(t)V(\textbf{c}, \textbf{y}, t-1)\\
\end{alignat*}
\begin{equation}\label{stock}
\begin{alignat*}{2}
V(\textbf{c}, \textbf{y}, t) = \;& V(\textbf{c}, \textbf{y}, t-1)\\
&+ \sum_{j \in \mathcal{J}}p_{j}(t)[\max[r_{j} - V(\textbf{c}, \textbf{y}, t-1) + V(\textbf{c}-\textbf{a}_j, \textbf{y}, t-1),0]\\
&+ \max[r_{j} - V(\textbf{c}, \textbf{y}, t-1) + V(\textbf{c}, \textbf{y}-\textbf{s}_j, t-1),0]]\\
\end{alignat*}
\end{equation}




Bei der Modellformulierung $V(\textbf{c}, \textbf{y}, t)$ der \textit{Bellman'schen Funktionsgleichung} des RM mit Lagerentnahme ist der Term $ r_{j} + V(\textbf{c}, \textbf{y}-\textbf{s}_j, t-1)]$ integriert, der die Annahme mittels des Lagerbestands beschreibt. Damit ist es dem Unternehmen möglich entweder die Kapazität oder den Lagerbestand in Anspruch zu nehmen. Diese Optionen werden in dieser Arbeit durch einen weiteren Index $m$ bei dem Parameter für den Produktauftrag $j_{m}$ kenntlich gemacht. Sofern es sich um eine Auftragsannahme mittels Kapazitätsinanspruchnahme (AA) handelt, erfolgt die Annahme des Produktauftrags $j_{AA}$. Handelt es sich um eine Annahme des Auftrags mittels des Lagerentnahme (LE), dann wird der Parameter $j_{LE}$ aufgeführt.

Für die Modellerweiterung gelten die Grenzbedingungen \eqref{GB1} sowie \eqref{GB2} und es gilt zusätzlich
\begin{equation}\label{GB3}
     OP_{\textbf{c}, t}:=\left\{\begin{array}{ll} j_{AA}, & \text{für } r_{j_{AA}} - OC_{j_{AA}} \ge r_{j_{LE}} - OC_{j_{LE}}\\
         j_{LE}, & \text{für } r_{j_{AA}} - OC_{j_{AA}} < r_{j_{LE}} - OC_{j_{LE}}\end{array}\right. ,
\end{equation}
da das Unternehmen vorrangig versucht seine Kapazitäten auszulasten. Die Funktionsweise der Modellformulierung wird im nachfolgenden Beispiel verdeutlicht.
\begin{center}
$j = \{1, 2\}, \; h = \{1\}, \; r_{1} = 100, \; r_{2} = 200, \; \text{Startperiode } t=4$,
\end{center}
\[
    c_{1}=1, \;
    a_{11}=1, \;
     a_{12}=1, \;
     p_{1}(t)=\begin{pmatrix} 0.5\\ 0.5\\ 0.5\\ 0.5  \end{pmatrix}, \;
     p_{2}(t)=\begin{pmatrix} 0.1\\ 0.1\\ 0.1\\ 0.1  \end{pmatrix},
  \]
  \[
    \textbf{y}=\begin{pmatrix} 1 \\ 0 \end{pmatrix}, \;
    \textbf{s}_1=\begin{pmatrix} 1 \\ 0 \end{pmatrix}, \;
     \textbf{s}_2=\begin{pmatrix} 0 \\ 1 \end{pmatrix} \;
  \]

\begin{figure}[h!]
  \begin{center}
    \includegraphics[width=130mm]{Bilder/Beispiel3.pdf}
    \caption{Darstellung des Entscheidungsbaums des Netzwerk RM mit Möglichkeit der Lagerentnahme}  \label{B3}
    {\footnotesize \textbf{Legende:} Die Zahlen stehen für den Produktauftrag $j$, AA='Auftragsannahme', LE='Lagerentnahme', KA='Kein Auftrag'} 
  \end{center}
\end{figure}

Abbildung \ref{B3} zeigt den Entscheidungsbaum mit den möglichen Systemzuständen aufgrund der vorher beschrieben Parameter. Dabei beschreibt ein Knoten weiterhin den Systemzustand. Ein Systemzustand wird durch die Zahlenfolge definiert, wobei die ersten Einträge der Zahlenfolge die Ressoucenkapazität $\textbf{c}$ entspricht. Da in diesem Beispiel nur eine Ressource vorhanden ist, beschreibt der erste Eintrag der Zahlenfolge die Ressourcenkapazität $c_{h}=1$ von $h=1$. Das Beispiel verfügt über zwei unterschiedliche Produkte $j\in\mathcal{J}$, die von den Nachfragern angefragt werden können. Somit existieren zwei Parameter für den Lagerbestand $\textbf{y}=(y_{1},y_{2})$. In dem vereinfachten Beispiel weist jedoch nur das Produkt $j=1$ einen Lagerbestand in Höhe von $y_{1}=1$ auf. Vom Produkt $j=2$ gibt es in diesem Beispiel keinen Lagerbestand. Damit entsprechen die nachfolgenden Zahlen in der Zahlenfolge des Systemzustands den möglichen Lagerbeständen $\textbf{y}$. Der letzter Wert der Zahlenfolge ist weiterhin der Zeitpunkt bzw. die Periode $t$. Der Systemzustand lässt damit wie folgt definieren: $[c_{1}\; y_{1}\; y_{2}\;t]$. Dem Unternehmen ist es möglich, unter Beachtung der vorausgehenden Parameter, entweder eine Anfrage $j_{AA}=1$ oder $j_{AA}=2$ mittels der Kapazität von $c_{1}$ anzunehmen (Auftragsannahme) oder eine Anfrage $j_{LE}=1$ mittels des Lagerbestands $y_{1}$ zu erfüllen (Lagerentnahme). Des Weiteren ist das Eintreffen keiner Anfrage zum Zeitpunkt $t$ möglich (Kein Auftrag). Damit sind die in dem Graphen aus der Abbildung \ref{B3} möglichen Systemzustände (Knoten) und Übergänge (Kanten) möglich.

\begin{table}
\begin{footnotesize}
    \caption{Ergebnistabelle für das beispielhafte Netzwerk RM mit Möglichkeit der Lagerentnahme} \label{Tab3}
    \vspace*{3mm}
\csvautotabular{data/beispiel3.csv}
\begin{center}
      {\footnotesize \textbf{Legende:} AA='Auftragsannahme', LE='Lagerentnahme', KA='Kein Auftrag'} 
      \end{center}
\end{footnotesize}
\end{table}

Die Tabelle \ref{Tab3} zeigt für das Beispiel die berechneten Erwartungswerte des noch möglichen Ertrags für jeden Systemzustand. Ebenfalls ist die optimale Politik in Form des besten Auftrags $j^*$ mit zugehörigem Ausführungsmodus $m$ aufgeführt. Dabei beschreibt hier der Ausführungsmodus die Auftragsannahme (AA), Lagerentnahme (LE) oder ob ein Auftrag keine Annahme erhält (KA). Zusätzlich ist der Wert $r_{j}-OC_{j}$ für den besten Auftrag $j^*$ je Systemzustand angegeben. Mit diesen Werten lässt sich der optimale Pfad des Beispiels ermitteln: $[2\;1\;0\;4] \rightarrow_{j_{AA}=2} [1\;1\;0\;3] \rightarrow_{j_{AA}=2} [0\;1\;0\;2] \rightarrow_{j_{LE}=1} [0\;0\;0\;1]\rightarrow_{j=0} [0\;0\;0\;0]$. Auch hier gilt weiterhin, dass der optimale Pfad abhängig der tatsächlich eintreffenden Anfragen ist. Er kann nur aber bei strategischen Entscheidung herangezogen werden.

Durch die Modellerweiterung wird gezeigt, dass ein Lagerbestand an Produkten $j\in\mathcal{J}$ den Gesamtertrag des Unternehmens erhöhen kann. Dies erfolgt jedoch aufgrund des Mechanismus, dass ein Lagerbestand eine Kapazitätserhöhung für das Unternehmen entspricht. Somit handelt es hier um eine andere Darstellung der Modellformulierung des Netzwerk RM mit verschiedenen Ausführungsmodi für die Produktanfragen $j\in\mathcal{J}$. %Eine abschließende Beurteilung der Modellerweiterung erfolgt mit Abschluss des nachfolgenden Abschnitts.

\subsection{Lagerentnahme- und Lagerproduktionsentscheidung}

Im vorhergehenden Abschnitt ist die Gleichung \eqref{DP} um die Eigenschaft der Lagerentnahme erweitert. Damit sind die Entscheidungen über die Annahme eines Auftrags via Kapazitäts- oder Lagerparameter möglich. Die nachfolgende Modellerweiterung soll die Gleichung \eqref{stock} mit der Entscheidung über die gewollte Ablehnung einer Anfrage erweitern, damit die Kapazitäten für die Lagererhöhung Verwendung finden. D. h. die Kapazitäten $\textbf{c}$ werden um den Ressourcenverbrauch $\textbf{a}_{j}$ reduziert, damit der Lagerbestand $\textbf{y}$ um den Parameter für die Lagerveränderung $\textbf{s}_{j}$ für ein Produkt $j$ erhöht werden kann. Die Modellformulierung lautet wie folgt:
\begin{alignat*}{2}
V(\textbf{c}, \textbf{y}, t) =\;& \sum_{j \in \mathcal{J}}p_{j}(t)\max[V(\textbf{c}, \textbf{y}, t-1),\\
&r_{j} + V(\textbf{c}-\textbf{a}_j, \textbf{y}, t-1),\\
&r_{j} + V(\textbf{c}, \textbf{y}-\textbf{s}_j, t-1),\\
&V(\textbf{c}-\textbf{a}_j, \textbf{y}+\textbf{s}_j, t-1)]\\
&+ p_{0}(t)V(\textbf{c}, \textbf{y}, t-1) \\
\end{alignat*}
\begin{equation}\label{storage}
\begin{alignat*}{2}
= \;& V(\textbf{c}, \textbf{y}, t-1)\\
&+ \sum_{j \in \mathcal{J}}p_{j}(t)[\max[r_{j} - V(\textbf{c}, \textbf{y}, t-1) + V(\textbf{c}-\textbf{a}_j, \textbf{y}, t-1),0] \\
&+ \max[r_{j} - V(\textbf{c}, \textbf{y}, t-1) + V(\textbf{c}, \textbf{y}-\textbf{s}_j, t-1),0]\\
&+ \max[V(\textbf{c}-\textbf{a}_j, \textbf{y}+\textbf{s}_j, t-1) - V(\textbf{c}, \textbf{y}, t-1) ,0]]\\
\end{alignat*}
\end{equation}

Neben der Entscheidungen über die Auftragsannahme mittels Kapazitätsinanspruchnahme (AA) und der Lagerentnahme eines Produkts (LE) ist in dieser Modellerweiterung der Gleichung \eqref{storage} die Produktion eines Produkts $j$ auf Lager $y_{j}$ möglich. Dafür wird der Parameter $s_{j}$ als Lagerveränderung interpretiert. Zu beachten ist jedoch, dass bei der Entscheidung der Lagerproduktion eines Produkts $j_{LP}$ kein Ertrag $r_{j_{LE}}$ erzielt wird. Des Weiteren wird ein Parameter für einen maximalen Lagerbestand $y_{j}^{max}$ für jedes Produkt $j\in\mathcal{J}$ definiert. Eine derartige Modellformulierung bringt jedoch eine spezielle Funktionsweise mit, wie nachfolgendes Beispiel verdeutlichen soll:
\begin{center}
$j = \{1, 2\}, \; h = \{1\}, \; r_{1} = 100, \; r_{2} = 200, \; \text{Startperiode } t=3$,
\end{center}
\[
    c_{1}=2, \;
    a_{11}=1, \;
     a_{12}=2, \;
     p_{1}(t)=\begin{pmatrix} 0.5\\ 0.5\\ 0.5  \end{pmatrix}, \;
     p_{2}(t)=\begin{pmatrix} 0.1\\ 0.1\\ 0.1  \end{pmatrix},
  \]
  \[
    \textbf{y}=\begin{pmatrix} 0 \\ 0 \end{pmatrix}, \;
    \textbf{y}^{max}=\begin{pmatrix} 2 \\ 1 \end{pmatrix}, \;
    \textbf{s}_1=\begin{pmatrix} 1 \\ 0 \end{pmatrix}, \;
     \textbf{s}_2=\begin{pmatrix} 0 \\ 1 \end{pmatrix} \;
  \]
\begin{figure}[h!]
  \begin{center}
    \includegraphics[width=130mm]{Bilder/Beispiel4.pdf}
    \caption{Darstellung des Entscheidungsbaums des Netzwerk RM mit Möglichkeit der Lagerentnahme und Lagerproduktion}  \label{B4}
    {\footnotesize \textbf{Legende:} Die Zahlen stehen für den Produktauftrag $j$, AA='Auftragsannahme', LE='Lagerentnahme', LP='Lagerproduktion', KA='Kein Auftrag'} 
  \end{center}
\end{figure}

Die Abbildung \ref{B4} mit dem Entscheidungsbaum für das Beispiel zeigt alle möglichen Systemzustände und Optionen. Ein Systemzustand ist definiert als Zeichenfolge $[c_{1}\; y_{1}\; y_{2}\;t]$. Es gelten die Grenzbedingungen \eqref{GB1} sowie \eqref{GB2} und
\begin{equation}\label{GB4}
     OP_{\textbf{c}, t}:=\left\{\begin{array}{lll} j_{AA}, & \text{für } r_{j_{AA}} - OC_{j_{AA}} \ge r_{j_{LE}} - OC_{j_{LE}}\\
         j_{LE}, & \text{für } r_{j_{AA}} - OC_{j_{AA}} < r_{j_{LE}} - OC_{j_{LE}}\\
         j_{LP}, & \text{sonst}\end{array}\right. .
\end{equation}
Sofern die Modellerweiterung in dieser Form definiert ist, wird niemals $OP_{\textbf{c}, t}:=j_{LP}$ gelten. Dies resultiert aus der Tatsache, dass sofern genügend Kapazitäten $\textbf{c}$ zur Produktion eines Produkts $j$ vorhanden sind, die Kapazitäten für die direkte Annahme der Produktanfrage $j$ verwendet werden. Die Entscheidung über die Produktion eines Produkts $j_{LP}$ ist für das Unternehmen nur dann sinnvoll, wenn keine Anfragen zum Zeitpunkt $t$ eintreffen und die Kapazität im weiteren Verkauf verfallen würde. Die Tabelle \ref{Tab4} zeigt die berechneten Werte für das hier aufgeführte Beispiel.
\begin{table}
\begin{footnotesize}
    \caption{Ergebnistabelle für das beispielhafte Netzwerk RM mit Möglichkeit der Lagerentnahme und Lagerproduktion} \label{Tab4}
    \vspace*{3mm}
\csvautotabular{data/beispiel4.csv}
\begin{center}
      {\footnotesize \textbf{Legende:} AA='Auftragsannahme', LE='Lagerentnahme', LP='Lagerproduktion', KA='Kein Auftrag'} 
      \end{center}
\end{footnotesize}
\end{table}

\subsection{Aufarbeitung von Ressourcen}

Bei der nachfolgenden Modellerweiterung wird nicht mehr von einem Produktlager ausgegangen, sondern von einem Ressourcenlager. Damit existiert für jede Ressource $h\in\mathcal{H}$ ein Lagerbestand $y_h^{res}$. Die Obergrenze des Lagerbestands wird durch den Parameter $y_{h}^{max,res}$ beschrieben. Die einzelnen Lagerbestände können als Vektor $\textbf{y}^{res}$ zusammengefasst werden. In der Modellerweiterung wird davon ausgegangen, dass Kapazitäten $c_{h}$ einer Ressource $h$ verwendet werden können, damit der Lagerbestand an Ressourcen $y_{h}$ erhöht wird. Es handelt sich damit um eine Art der Vorarbeit der Leistung. Die einzelnen Bestandteile des Produkts werden auf Lager gelegt, damit diese zu einem späteren Zeitpunkt Verwendung finden. In dieser Modellerweiterung wird davon ausgegangen, dass ein komplettes Bündel des Produkts $j$ auf Lager gelegt wird. Daher kann der Parameter $\textbf{a}_{j}$ als Ressourcenveränderung angesehen werden. Das Modell schichtet damit die Kapazität der Ressourcen $h\in\mathcal{H}$ vom Parameter $c_{h}$ zu dem Lagerbestand $y_{h}^{res}$ um. Das mathematische Modell lässt sich wie folgt beschreiben:

\begin{alignat*}{2}
V(\textbf{c}, \textbf{y}^{res}, t) =\;& \sum_{j \in \mathcal{J}}p_{j}(t)\max[V(\textbf{c}, \textbf{y}^{res}, t-1),\\
&r_{j} + V(\textbf{c}-\textbf{a}_j, \textbf{y}^{res}, t-1),\\
&r_{j} + V(\textbf{c}, \textbf{y}^{res}-\textbf{a}_j, t-1),\\
&V(\textbf{c}-\textbf{a}_j, \textbf{y}^{res}+\textbf{a}_j, t-1)]\\
&+ p_{0}(t)V(\textbf{c}, \textbf{y}^{res}, t-1) \\
\end{alignat*}
\begin{equation}\label{workup}
\begin{alignat*}{2}
= \;& V(\textbf{c}, \textbf{y}^{res}, t-1)\\
&+ \sum_{j \in \mathcal{J}}p_{j}(t)[\max[r_{j} - V(\textbf{c}, \textbf{y}^{res}, t-1) + V(\textbf{c}-\textbf{a}_j, \textbf{y}^{res}, t-1),0] \\
&+ \max[r_{j} - V(\textbf{c}, \textbf{y}^{res}, t-1) + V(\textbf{c}, \textbf{y}^{res}-\textbf{a}_j, t-1),0]\\
&+ \max[V(\textbf{c}-\textbf{a}_j, \textbf{y}^{res}+\textbf{a}_j, t-1) - V(\textbf{c}, \textbf{y}^{res}, t-1) ,0]]\\
\end{alignat*}
\end{equation}

Weiterhin gelten die Grenzbedingungen \eqref{GB1} sowie \eqref{GB2} und Gleichung \eqref{GB4}. Bei dieser Modellformulierung der Gleichung \eqref{workup} wird die Einschränkung aufgehoben, bei der die Anzahl an notwendigen Ressourcen die Entscheidung der Lagerproduktion dominiert. Dies wird durch das nachfolgende Beispiel verdeutlicht:
\begin{center}
$j = \{1, 2\}, \; h = \{1\}, \; r_{1} = 100, \; r_{2} = 5000, \; \text{Startperiode } t=3$,
\end{center}
\[
    c_{1}=2, \;
    a_{11}=1, \;
     a_{12}=7, \;
     p_{1}(t)=\begin{pmatrix} 0.5\\ 0.5\\ 0.5  \end{pmatrix}, \;
     p_{2}(t)=\begin{pmatrix} 0.1\\ 0.1\\ 0.1  \end{pmatrix},
  \]
  \[
    y_{1}^{res}= 5, \;
    y_{1}^{max,res}=7
      \]
\begin{figure}[h!]
  \begin{center}
    \includegraphics[width=130mm]{Bilder/Beispiel5.pdf}
    \caption{Darstellung des Entscheidungsbaums des Netzwerk RM mit Möglichkeit der Aufarbeitung als Lagerproduktion}  \label{B5}
    {\footnotesize \textbf{Legende:} Die Zahlen stehen für den Produktauftrag $j$, AA='Auftragsannahme', LE='Lagerentnahme', LP='Lagerproduktion', KA='Kein Auftrag'} 
  \end{center}
\end{figure}

Abbildung \ref{B5} zeigt den Entscheidungsbaum für das aufgeführte Beispiel. Die Systemzustände sind beschrieben als Zahlenfolge $[c_{1}\; y_{1}^{res}\;t]$, da bei diesem Beispiel nur eine Ressource $h$ existiert. Für die Ressource $h=1$ beträgt die Kapazität $c_{1}=2$ und der Lagerbestand $y_{1}^{res}=5$. Als Obergrenze für den Lagerbestand wird $y_{1}^{max,res}=7$ festgelegt. Aufgrund der Ressourcenveränderungsparameter $\textbf{a}_{j}$ für jedes Produkt $j\in\mathcal{J}$ gibt es die Entscheidungsmöglichkeit Auftragsannahme (AA), Lagerentnahme (LE) sowie Lagerproduktion aufgrund von Aufarbeitung (LP) für jedes Produkt $j\in\mathcal{J}$. Weiterhin ist die Option zu jedem Systemzustand möglich, dass keine Anfrage eintrifft und der Wechsel in die nächste Periode ohne Kapazitäts- oder Lagerreduktion erfolgt.

Aufgrund der der Restriktionen der Kapazität $c_{1}$ und des Lagerbestands $y_{1}^{res}$ sind nicht alle möglichen Entscheidungsalternativen zu jedem Systemzustand möglich. Beispielsweise ist im Systemzustand $[2\;5\;3]$ die AA, LE sowie LP des Produkts $j=2$ nicht möglich. Jedoch kann in diesem Systemzustand die Annahme einer Anfrage nach Produkt $j=1$ erfolgen, entweder durch Kapazitäts- oder Lagerreduktion. Um eine Anfrage nach Produkt $j=2$ akzeptieren zu können, werden aufgrund des Parameters $a_{12}$ genau 7 Einheiten von der Ressource $h=1$ benötigt. Entweder müssen genügen Kapazitäten vorhanden sein, was jedoch im Beispiel nicht gegeben ist, oder der Lagerbestand $y_{1}^{res}$ muss auf 7 Einheiten der Ressource $h=1$ erhöht werden. Sofern die Erträge $r_{1}=100$ und $r_{2}=5000$ für das Beispiel angenommen sind, erhalten wir die berechneten Werte der Tabelle \ref{Tab5}.
\begin{table}
\begin{footnotesize}
    \caption{Ergebnistabelle für das beispielhafte Netzwerk RM mit Möglichkeit der Aufarbeitung als Lagerproduktion} \label{Tab5}
    \vspace*{3mm}
\csvautotabular{data/beispiel5.csv}
    \begin{center}
      {\footnotesize \textbf{Legende:} AA='Auftragsannahme', LE='Lagerentnahme', LP='Lagerproduktion', KA='Kein Auftrag'} 
      \end{center}
\end{footnotesize}
\end{table}

Ist der Ertrag $r_{j}$ eines Produkts $j\in\mathcal{J}$ ausreichend groß bei gegebener Eintrittswahrscheinlichkeit $p_{j}(t)$ zum Zeitpunkt $t$, dann erfolgt in dem Modell der Gleichung \eqref{workup} eine Überführung der Ressourcenkapazität $c_{h}$ hin zu dem Lagerbestand $y_{h}^{res}$, sofern dies eine andere Ressourceninanspruchnahme $a_{j}$ ermöglicht. Das Beispiel zeigt eine solche optimale Politik.

Für den anfänglichen Systemzustand $[2\;5\;3]$ existieren die Systemübergänge $j_{LE}$, $j_{LP}$ und $j_{AA}$ für Anfragen nach Produkt $j=1$. Außerdem ist das Eintreffen keiner Anfrage möglich. Damit lässt sich in diesem Systemzustand ein Ertrag $r_{j}=100$ generieren. Mit der Entscheidung $j_{LE}=1$ gelangt das System zum Zustand $[2\;4\;2]$, da für eine Annahme einer Anfrage nach Produkt $j=1$ der Lagerbestand $y_{1}^{res}$ reduziert wird. Sofern die Entscheidung $j_{AA}=1$ eintrifft, gelangt das System durch Inanspruchnahme der Kapazität $c_{1}$ zum Zustand $[1\;5\;2]$. Außerdem ist der Systemwechsel zum Zustand $[1\;6\;2]$ durch $j_{LP}=1$ möglich. Aufgrund der berechneten OK ist unter Beachtung der Parameter ein Systemwechsel durch die Lagerproduktion optimal (Vgl. Tabelle \ref{Tab5}). Abbildung \ref{B5} zeigt diesen und alle anderen optimalen Politiken für jeden Systemzustand. Daraus lässt dich die beste Pfad ermitteln:  $[2\;5\;3] \rightarrow_{j_{LP}=1} [1\;6\;2] \rightarrow_{j_{LP}=1} [0\;7\;1] \rightarrow_{j_{LE}=2} [0\;0\;0]$.

Es ist damit gezeigt, dass die Kapazität $c_{1}$ der Ressource $h=1$ durch den Ausführungsparameter $a_{1}$ des Produkts $j$ verwendet wird, damit der Lagerbestand $y_{1}^{res}$ jener Ressource $h=1$ erhöht wird. Damit ist zu einem späteren Zeitpunkt $t$ die Annahme der Anfrage nach Produkt $j=2$ durch die Lagerentnahme (LE) des Lagerbestands $y_{1}^{res}$ der Ressource $h=1$ möglich. Damit ist ein Gesamtertrag von $5000$ GE generiert. 

\subsection{Erneuerung von Ressourcen innerhalb des Buchungshorizonts}

Wie die vergangenen Modellerweiterungen zeigen, erweitert der Lagerbestand $\textbf{y}$ bzw. $\textbf{y}^{res}$ die Kapazitäten des Netzwerks. Sofern es sich um ein Ressourcenlager $\textbf{y}^{res}$ handelt, werden die notwendigen Kapazitäten $c_{h}$ der Ressourcen $h\in\mathcal{H}$ für ein Produkt $j$ mit niedrigem Ertrag auf den Lagerbestandsparameter $\textbf{y}^{res}$ umverteilt, damit die Akzeptanz eine nachfolgende Anfrage nach einem anderweitigen Produkt $j\in\mathcal{J}$ mit höherem Ertrag $r_{j}$ über den Lagerparameter $\textbf{y}^{res}$ erfolgen kann. Damit ist gezeigt, dass mithilfe der Modellerweiterung der \textit{Bellman'schen Funktionsgleichung} für das Netzwerk RM die optimale Politik ist, Anfragen mit niedrigem Ertrag zugunsten zukünftiger Anfragen mit höherem Ertrag abzulehnen. Mit dieser Ablehnung geht die Kapazitätserhöhung des Lagerbestands $\textbf{y}^{res}$ mit den dafür notwendigen Ressourcen $h\in\mathcal{H}$ einher, die für die Produktanfrage $j$ mit hohen Ertrag $r_j$ notwendig ist.

Wie im vorherigen Abschnitt gezeigt, müssen jedoch gewisse Rahmenbedingungen für eine Ablehnung von Produktanfragen $j\in\mathcal{J}$ mit niedrigem Ertrag gegeben sein. Die Kapazitäten $\textbf{c}$ des Netzwerks müssen einerseits ausreichend groß sein, damit Anfragen nach dem Produkt $j$ mit niedrigem Ertrag $r_{j}$ abgelehnt und auf das Lager $\textbf{y}^{res}$ umverteilt werden können, aber anderseits niedrigen als der notwendige Bedarf $\textbf{a}_{j}$ für das Produkt $j$ mit höherem Ertrag. Zusätzlich muss die Differenz zwischen den vorhanden Kapazitäten $\textbf{c}$ und der notwendigen Kapazitäten $\textbf{a}_{j}$ für die Produktanfrage $j$ mit hohem Ertrag auf dem Lager $\textbf{y}^{res}$ vorhanden sein. Wären ausreichend Kapazitäten $\textbf{c}$ für beide Arten der Produktanfragen $j\in\mathcal{J}$ möglich, dann wäre eine Umverteilung auf das Lager aufgrund der Gleichung \eqref{GB4} unnötig. Die notwendigen Kapazitäten $c_{h}$ der Ressourcen $h\in\mathcal{H}$ würden direkt für die Anfrage nach dem Produkt $j$ mit hohem Ertrag Verwendung finden und eingeplant. Die optimale Politik wäre das Produkt $j$ mit hohem Ertrag $r_{j}$ frühzeitig einzuplanen, sofern Anfragen für das Produkt unter Beachtung der Wahrscheinlichkeiten $p_{j}(t)$ zu den Zeitpunkten $t\in T$ eintreffen. Anders formuliert, die Option der Annahme einer Anfrage nach dem Produkt $j$ mit hohem Ertrag wäre im Systemzustand mit ausreichendem Kapazitäten $c_{h}$ zum Zeitpunkt $t$ möglich und, sofern der Ertrag aufgrund der OK des Netzwerks nicht zu stark beeinflussen wird, optimal.

Die Aufhebung der vorhergehenden Restriktion erfolgt durch das Einführen von regenerativen Ressourcen $h\in\mathcal{H}$ innerhalb des Buchungshorizonts $T$. Erst durch die Modellerweiterung der regenerativen Ressourcen $h\in\mathcal{H}$ erhält die Modellformulierung unter Beachtung einer Lagerhaltung von Ressourcen $\textbf{y}^{res}$ innhalb des Buchungshorizonts $T$ eine Sinnhaftigkeit. Für die Modellerweiterung wird der Parameter $\tilde{t}\in\tilde{T}$ eingeführt, wobei $\tilde{T}\subset T$ gilt. Der Parameter $\tilde{t}$ zeigt den Zeitpunkt der Regeneration der Ressourcen $h\in\mathcal{H}$ an. Durch die Regeneration erhöht sich der Wert des Kapazitätsparameters auf $\textbf{c}^{max}$. Die \textit{Bellman'schen Funktionsgleichung} für ein Netzwerk RM mit regenerativen Ressourcen innerhalb des Buchungshorizonts wird wie folgt definiert:

\begin{equation}\label{reg}
     V^{reg}(\textbf{c}, \textbf{y}^{res}, t)=\left\{\begin{array}{ll} V(\textbf{c}, \textbf{y}^{res}, t), & \forall t\neq\tilde{t}\\
         V(\textbf{c}^{max}, \textbf{y}^{res}, t), &\forall t=\tilde{t}\end{array}\right. .
\end{equation}

Es gelten weiterhin Gleichungen \eqref{GB1}, \eqref{GB2} und \eqref{GB4}. Dabei erfolgt die Ermittlung der Erwartungswerte aus der Gleichung \eqref{reg} abhängig des betrachteten Zeitpunkts $t$. Die Berechnung des Erwartungswerts $V(\textbf{c}, \textbf{y}^{res}, t)$ $\forall t\neq\tilde{t}$ erfolgt wie in Gleichung \eqref{workup} und die Berechnung des Erwartungswerts $V(\textbf{c}^{max}, \textbf{y}^{res}, t)$ $\forall t=\tilde{t}$ wie nachfolgende Gleichung zeigt:
\begin{alignat*}{2}
 V(\textbf{c}^{max}, \textbf{y}^{res}, t) = \;& V(\textbf{c}^{max}, \textbf{y}^{res}, t-1)\\
&+ \sum_{j \in \mathcal{J}}p_{j}(t)[\max[r_{j} - V(\textbf{c}^{max}, \textbf{y}^{res}, t-1)\\
&+ V(\textbf{c}^{max}-\textbf{a}_j, \textbf{y}^{res}, t-1),0] \\
&+ \max[r_{j} - V(\textbf{c}^{max}, \textbf{y}^{res}, t-1) + V(\textbf{c}^{max}, \textbf{y}^{res}-\textbf{a}_j, t-1),0]\\
&+ \max[V(\textbf{c}^{max}-\textbf{a}_j, \textbf{y}^{res}+\textbf{a}_j, t-1) - V(\textbf{c}^{max}, \textbf{y}^{res}, t-1) ,0]]\\
\end{alignat*}

Die Funktionsweise des Modells soll an einem Beispiel verdeutlicht werden:
\begin{center}
$j = \{1, 2\}, \; h = \{1\}, \; r_{1} = 100, \; r_{2} = 5000, \; \text{Startperiode } t=4, \; \tilde{t}=\{2\} $,
\end{center}
\[
    c_{1}=1, \;
    a_{11}=1, \;
     a_{12}=2, \;
     p_{1}(t)=\begin{pmatrix} 0.5\\ 0.5\\ 0.5\\ 0.5  \end{pmatrix}, \;
     p_{2}(t)=\begin{pmatrix} 0.1\\ 0.1\\ 0.1\\ 0.1  \end{pmatrix},
  \]
  \[
    y_{1}^{res}= 0, \;
    y_{1}^{max,res}=2
      \]
      
Es existieren zwei Produkte $j\in\mathcal{J}$ und eine Ressource $h\in\mathcal{H}$. Sofern eine Anfrage nach dem Produkt $j=1$ akzeptiert wird, generiert sich ein Ertrag in Höhe von $r_1=100$ GE. Bei Akzeptanz einer Produktanfrage $j=2$ erzielt das Unternehmen einen Ertrag von $r_1=5000$ GE. Beide Produktanfragen benötigen die Ressource $h=1$, wobei die Kapazität der Ressource bei $c_1=1$ liegt. Ein Anfrage nach Produkt $j=1$ benötigt $a_{11}=1$ Kapazitäten und eine Anfrage nach Produkt $j=1$ benötigt $a_{12}=2$. Der Buchungshorizont entspricht $T=4$ Perioden und zum Zeitpunkt $\tilde{t}=2$ erfolgt eine Regeneration der Ressourcen. Die Wahrscheinlichkeit des Eintreffens einer Anfragen zum Zeitpunkt $t$ entspricht $ p_{1}(t)=(0.5, 0.5, 0.5, 0.5)$ bzw. $ p_{2}(t)=(0.1, 0.1, 0.1, 0.1)$. Die Lagerkapazität $y_1^{max,res}$ ist auf 2 Einheiten beschränkt und es existiert zur Starperiode $t=4$ kein Anfangsbestand ($y_1^{res}=0$). Abbildung \ref{B6} zeigt alle möglichen Systemzustände mit den einzelnen Übergängen für das Beispiel. 

\begin{figure}[h!]
  \begin{center}
    \includegraphics[width=140mm]{Bilder/Beispiel6.pdf}
    \caption{Darstellung des Entscheidungsbaums des Netzwerk RM mit regenerativen Ressourcen}  \label{B6}
    {\footnotesize \textbf{Legende:} Die Zahlen stehen für den Produktauftrag $j$, AA='Auftragsannahme', LE='Lagerentnahme', LP='Lagerproduktion', KA='Kein Auftrag', $\cdots$='Anfrage ablehnen'} 
  \end{center}
\end{figure}

Ein Systemzustand im vorausgehenden Beispiel ist definiert als Zahlenfolge $[c_1\;y^{res}_1$ $t]$. Zu beachten ist, dass in diesem Netzwerk zum Zeitpunkt $t=2$ die Ressourcen erneuert sind und dementsprechend für die betreffenden Systemzustände die Zahlenfolge $[c_1^{max}\;y^{res}_1\;2]$ gilt. Wird das Gesamtnetzwerks in Abbildung \ref{B6} betrachtet, dann ergibt sich der bester Pfad $[1\;0\;4] \rightarrow_{j_{LP}=1} [0\;1\;3] \rightarrow_{KA} [1\;1\;2] \rightarrow_{j_{LP}=1} [0\;2\;1] \rightarrow_{j_{LE}=2} [0\;0\;0]$ und ein Gesamtertrag in Höhe von $5000$ GE. Damit ist für das hier betrachtete Netzwerk die Lagerproduktion (LP) der Ressource $h=1$ optimal, damit eine Anfrage nach einem Produkt $j=2$ akzeptiert werden kann. Des Weiteren ist unter diesen Bedingungen im Systemzustand $[1\;0\;4]$ die Ablehnung der Anfrage nach einem Produkt $j=1$ die beste Politik, sofern nur diese Anfrage zum Zeitpunkt $t=4$ eintrifft. Durch diese Entscheidung gelangt das Netzwerk zum Systemzustand $[1\;0\;3]$, bei dem in weiteren Verlauf das Erreichen des Systemzustands $[1\;1\;2]$ möglich ist, bei dem die Kapazitäten regeneriert sind und der Lagerbestand $y_1^{res}$ eine Einheit der Ressource $h=1$ beinhaltet. Von diesem Systemzustand ist ein weiterer Verlauf möglich, bei dem die Anfrage nach einem Produkt $j=2$ möglich ist.

Wird im Gegensatz im Systemzustand $[1\;0\;4]$ die Anfrage nach dem Produkt $j=1$ angenommen, dann gelangt das Netzwerk in den Systemzustand $[0\;0\;3]$ und erzielt einen Ertrag in Höhe von 100 GE. Vom Systemzustand $[0\;0\;3]$ ist keine Akzeptanz mehr möglich, da keine Kapazität und kein Lagerbestand vorhanden sind. Das Netzwerk gelangt dann in den Systemzustand $[1\;0\;2]$ mit einem regenerierten Kapazitätsbestand $c_1=1$. Im weiteren Verlauf ist nur noch die Akzeptanz einer Anfrage nach Produkt $j=1$ möglich. Damit ist ein Gesamtertrag in Höhe von 200 GE generiert. Damit verstößt die Annahme der Anfrage $j=1$ im Systemzustand $[1\;0\;4]$ gegen die Bedingung \eqref{OC}.

Damit ist gezeigt, dass die Umverlagerung der Kapazitäten hin zu einem Lagerbestand das Ergebnis bzw. den Gesamtertrag verbessern kann. Die ermittelten Werte des Beispiels ist in der Tabelle \ref{Tab6} aufgeführt.

\begin{table}
\begin{footnotesize}
    \caption{Ergebnistabelle für das beispielhafte Netzwerk RM mit regenerativen Ressourcen} \label{Tab6}
    \vspace*{3mm}
\csvautotabular{data/beispiel6.csv}
    \begin{center}
      {\footnotesize \textbf{Legende:} AA='Auftragsannahme', LE='Lagerentnahme', LP='Lagerproduktion', KA='Kein Auftrag'} 
      \end{center}
\end{footnotesize}
\end{table}

Zu beachten ist jedoch, dass durch das Betrachtung der regenerierten Ressourcen als Gesamtkapazität des Netzwerks sich ein gleiches Ergebnis unter einer anderer Betrachtungsweise ergibt. Es gilt $\textbf{c}=(1+max(\tilde{T}))\cdot \textbf{c}^{max}$, dann ergeben sich die optimalen Politiken aus Tabelle ... . Da jedoch von Anfang an genügen Kapazitäten zur Annahme alle Produktanfragen $j\in\mathcal{J}$ möglich sind, ist die Lagerproduktion nicht optimale Politik sofern eine beliebige Anfrage eintrifft. Die Auftragsannahme (AA) dominiert die Lagerproduktion (LP). Die Lagerentnahme wiederum kann dann optimale Politik werden, wenn keine Anfragen eintreffen und eine Umlagerung der Kapazitäten hin zum Lager erfolgen muss. Die Betrachtung von regenerativen Ressourcen $h\in\mathcal{H}$ im Buchungshorizont $T$ hat nur zur Vereinfachung des Netzwerks einen Nutzen.



\subsection{Inanspruchnahme der Kapazitäten zur Aufstockung eines beliebigen Lagerbestands}

Das Modell des Netzwerk RM wird jetzt um eine weitere Eigenschaft erweitert. Sofern angenommen wird, dass es sich bei den Ressourcen um flexible und regenerative Ressourcen handelt, dann ist eine beliebige Aufstockung des Lagerbestands möglich. Anders formuliert, eine Ressource $h\in\mathcal{H}$ für eine Produktanfrage $j\in\mathcal{J}$ mit dem notwendigen Verbrauch bzw. Ressourceneinsatz $\textbf{a}_j$ wird verwendet um einen beliebigen Lagerbestand an Ressourcen $\textbf{y}^{res}$ zu erhöhen. Zur Verdeutlichung wird ein einfaches Fallbeispiel gebildet. Sei die Ressource $h=1$ eine Arbeitsstunde eines Werksarbeiter, der in dieser Stunde für die Fertigung eines bestimmten Produkts $j=1$ zuständig ist. Das Produkt $j=1$ kann z. B. ein Taschenrechner sein. Die Produktion eines Taschenrechners bedarf einer Stunde der Ressource $h=1$. Jetzt trifft im Zeitverlauf des betrachteten Buchungshorizonts auch eine Anfrage nach einem Produkt $j=2$ ein. Bei dem Produkt handelt es sich um einen hochwertigen Computer, der einen höheren Ertrag $r_{2}$ erzielt. Für das Beispiel wird angenommen, dass der Werksarbeiter die Fertigkeiten besitzt diesen Computer zu fertigen, obwohl dieses bei der Ressourcenplanung des Unternehmens nicht vorgesehen ist. Zur Annahme der Anfrage werden dafür zwei Komponenten der Ressource $h=2$ benötigt. Damit gilt ein Ressourcenverbauch von $a_12=2$ zur Annahme der Anfrage nach Produkt $j=2$. In diesem einfachen Beispiel gehen wird davon aus, dass die Arbeitsstunden der Ressource $h=1$ aufgewendet werden, damit die erforderlichen Komponenten hergestellt und auf Lager gelegt werden. Diese Transformation erfolgt im Ausmaß des Ressourcenverbrauchs $a_{j}$. D. h. es kann nur so viel auf das Lager transferiert werden, wie der Ressourcenverbrauch $a_{j}$ ermöglicht. Die unterschiedlichen Ressourcen sind damit substituierbar.

Sofern angenommen wird, dass der Buchungshorizont $T$ und die Kapazität $c_1$ ausreichend groß sind, sowie kein Lagerbestand $y^{res}_1$ für die Ressource $h=1$ vorhanden ist, dann ist die optimale Politik des Beispiels den Anfragen nach dem Taschenrechner nicht nachzugehen und die Anfrage nach dem Computer zu befriedigen. Dies erfolg durch aufwenden der Ressource $h=2$ zur Erhöhung des Lagerbestands $y^{res}_{2}$ der Ressource $h=2$.

Für die Formulierung dieses Modellerweiterung wird der Vektor $\textbf{m}$ eingeführt. Bei dem Vektor handelt es sich um die Lagerbestandserhöhung in Abhängigkeit des Ressourceneinsatzes $\textbf{a}_{j}$ der Produktanfrage $j\in\mathcal{J}$. Für jedes Produkt $j\in\mathcal{J}$ existiert eine Menge $\mathcal{M}_j$ an möglichen Modi $\textbf{m}$, die den Lagerbestand $y^{res}_h$ in Höhe des Ressourcenverbrauchs $a_{j}$ erhöhen können. Die Menge $\mathcal{M}_j$ beinhaltet damit alle möglichen Transformationen des Ressourceneinsatzes $a_{j}$ eines Produkts $j\in\mathcal{J}$. Sei $\hat{j}$ eine Produktanfrage mit dem Ressourcenverbrauch $a_{\hat{j}}=(2,0,0,0)$, dann existiert eine Menge $\mathcal{M}_{\hat{j}}$ mit allen möglichen Ausführungsmodi $\textbf{m}$ der Transformation des Ressourcenverbauchs. Für $a_{\hat{j}}=(2,0,0,0)$ beinhaltet die Menge $\mathcal{M}_{\hat{j}}$ die Ausführungsmodi $\textbf{m}$: $\{ (2, 0, 0, 0), (1, 1, 0, 0), (1, 0, 1, 0),$ $(1, 0, 0, 1), (0 ,2, 0, 0), (0, 1, 1, 0), (0, 1, 0, 1), (0, 0, 2, 0),$ $(0, 0, 1, 1), (0, 0, 0, 2)\}$.

Damit lässt sich die \textit{Bellman'schen Funktionsgleichung} für das Netwerk RM mit der Inanspruchnahme der Kapazitäten zur Aufstockung eines beliebigen Lagerbestands formulieren:
\begin{equation}\label{dif}
     V^{dif}(\textbf{c}, \textbf{y}^{res}, t)=\left\{\begin{array}{ll} V(\textbf{c}, \textbf{y}^{res}, t), & \forall t\neq\tilde{t}\\
         V(\textbf{c}^{max}, \textbf{y}^{res}, t), &\forall t=\tilde{t}\end{array}\right. .
\end{equation}
\begin{alignat*}{2}
 V(\textbf{c}, \textbf{y}^{res}, t) = \;& V(\textbf{c}, \textbf{y}^{res}, t-1)+ \sum_{j \in \mathcal{J}}p_{j}(t)[\max[r_{j} - V(\textbf{c}, \textbf{y}^{res}, t-1)\\
&+ V(\textbf{c}-\textbf{a}_j, \textbf{y}^{res}, t-1),0] \\
&+ \max[r_{j} - V(\textbf{c}, \textbf{y}^{res}, t-1) + V(\textbf{c}, \textbf{y}^{res}-\textbf{a}_j, t-1),0]\\
&+ \max_{\textbf{m}\in\mathcal{M}_{j}}[V(\textbf{c}-\textbf{a}_j, \textbf{y}^{res}+\textbf{m}, t-1) - V(\textbf{c}, \textbf{y}^{res}, t-1) ,0]]\\
\end{alignat*}
Für den Erwartungswert $V(\textbf{c}^{max}, \textbf{y}^{res}, t)$ gilt analog die vorhergehende Gleichung mit $\textbf{c}=\textbf{c}^{max}$. Die Modellformulierung des Auftragsannahmeproblems hat damit die Optionen der Ablehnung des Auftrags (KA), der Annahme des Auftrags, entweder durch Kapazitäts- oder Lagerreduktion (AA bzw. LE), sowie die Option der Transformation der Ressourcen bzw. der Produktion von Ressourcen durch Aufwendung der notwendigen Ressourcen einer Produktanfrage (LP). Zur Verdeutlichung des Modells wird ebenfalls ein Beispiel eingeführt und berechnet:

\begin{center}
$j = \{1, 2, 3\}, \; h = \{1,2,3\}, \; r_{1} = 100, \; r_{2} = 200, \; r_{3} = 5000,$ \\
$\text{Startperiode } t=3, \; \tilde{t}=\{\} $,
\end{center}
\[
    \textbf{c}=\begin{pmatrix} 1\\ 1\\ 0  \end{pmatrix}, \;
    \textbf{a}_{1}=\begin{pmatrix} 1\\ 0\\ 0  \end{pmatrix}, \;
     \textbf{a}_{2}=\begin{pmatrix} 0\\ 1\\ 0  \end{pmatrix}, \;
       \textbf{a}_{3}=\begin{pmatrix} 0\\ 0\\ 2  \end{pmatrix}, \;
            p_{j}(t)=
       \begin{pmatrix}
       0.9 & 0 & 0 \\
0 & 0.9 & 0 \\
0 & 0 & 0.9
\end{pmatrix}, 
  \]
  \[
    \textbf{y}^{res}= \begin{pmatrix} 0\\ 0\\ 0  \end{pmatrix}, \;
    \textbf{y}^{max,res}=\begin{pmatrix} 0\\ 0\\ 2  \end{pmatrix}
      \]

Bei dem vorausgehenden Beispiel werden drei Produkte $j\in\mathcal{J}$ mit drei unterschiedlichen Ressourcen $h\in\mathcal{H}$ betrachtet. Die Produkte generieren einen Ertrag in Höhe von $r_1=100, r_2=200 und r_3=5000$. Die Kapazitäten für die Ressourcen betragen $\textbf{c}=(1,1,0)$ und die Ressourcenverbräuche für die Produktanfragen betragen $\textbf{a}_1=(1,0,0),\; \textbf{a}_2=(0,1,0)$ sowie $\textbf{a}_3=(0,0,2)$. Der Buchungshorizont beträgt $T=3$ und die Ressourcen werden darin nicht regeneriert $(\tilde{t}=\{\})$. Es existiert kein Lagerbestand $\textbf{y}^{res}$ und es kann nur ein maximaler Lagerbestand $\textbf{y}^{max,res}=(0, 0, 2)$ aufgebaut werden. In diesem Beispiel sind nicht alle Produktanfragen $j\in\mathcal{J}$ zu jedem Zeitpunkt $t\in T$ möglich. Dies zeigt die Matrix $p_{j}(t)$ mit den Eintrittswahrscheinlichkeiten für jedes Produkt $j$ zum jeweiligen Zeitpunkt $t$. Für das Produkt $j=1$ treffen Anfragen zu den Zeitpunkten $t\in T$ mit den Wahrscheinlichkeiten $p_{1}(t)=(0.9, 0, 0)$ ein. Für die Produkte $j=2$ und $j=3$ betragen die Wahrscheinlichkeiten $p_{2}(t)=(0, 0.9, 0)$ bzw. $p_{3}(t)=(0, 0, 0.9)$. Bei den Wahrscheinlichkeiten $p_j(t)$ der Produktanfrage $j$ zum Zeitpunkt $t$ muss beachtet werden, dass der Buchungshorizont rückwärts verläuft. Damit gilt $p_{j}(t)=(p_{j}(3), p_{j}(2), p_{j}(1))$. Abbildung \ref{B7} zeigt für das Beispiel den Entscheidungsbaum mit allen Optionen. Dabei wird ein Systemzustand im Netzwerk als Zahlenfolge $[c_1,c_2,c_3,y_1,y_2,y_3,t]$ beschrieben.
 
\begin{figure}[h!]
  \begin{center}
    \includegraphics[width=200mm, angle=90]{Bilder/Beispiel7.pdf}
    \caption{Darstellung des Entscheidungsbaums des Netzwerk RM mit Inanspruchnahme der Kapazitäten zur Aufstockung eines beliebigen Lagerbestands}  \label{B7}
    {\footnotesize \textbf{Legende:} Die Zahlen stehen für den Produktauftrag $j$, AA='Auftragsannahme', LE='Lagerentnahme', LP='Lagerproduktion', KA='Kein Auftrag', $\cdots$='Anfrage ablehnen'} 
  \end{center}
\end{figure}

Wie in der Abbildung \ref{B7} zu erkennen ist, wäre die optimale Politik im Systemzustand $[1,1,0,0,0,0,3]$ die Lagerproduktion im Ausführungsmodus $\textbf{m}=(0,0,1)$ durchzuführen. Dadurch transformiert sich die Ressource $h=1$ durch $\textbf{a}_{1}$ hin zu einer Ressource $h=3$. Damit ist im nächsten Systemzustand ein Lagerbestand $\textbf{y}^{res}=(0,0,1)$ generiert. Eine andere Transformation ist aufgrund der Eintrittswahrscheinlichkeiten $p_{j}(t)$ zum Zeitpunkt $t=3$ nicht möglich. Bei dem nächsten Systemzustand handelt es sich um die Zahlenfolge $[0,1,0,0,0,1,2]$. In diesem Systemzustand ist die Transformation bzw. Lagerproduktion von $h=2$ mit dem Ressourcenverbauch $a_{2}$ hin zu der Ressource $h=3$ möglich. Hierfür wird der Ausführungsmodus $\textbf{m}=(0,0,1)\in\mathcal{M}_2$ verwendet. Damit wird der Systemzustand $[0,0,0,0,0,2,1]$ mit einem Lagerbestand $\textbf{y}^{res}=(0,0,2)$ erreicht. In diesem Systemzustand ist aufgrund der Eintrittswahrscheinlichkeiten $p_2(t)$ und des vorhanden Lagerbestand $\textbf{y}^{res}=(0,0,2)$ eine Lagerentnahme der Ressource $h=3$ möglich. Dies erfolgt aufgrund der möglichen Annahme einer Anfrage nach Produkt $j=3$ mit einem Ressourcenverzehr von $a_{3}=(0,0,2)$. Unter Einhaltung dieser optimalen Politik wird in dem Netzwerk ein Ertrag in Höhe von $5000$ GE erzielt. Tabelle \ref{Tab7} zeigt die berechneten Erwartungswerte für alle Systemzustände des Beispiels. 

\begin{table}
\begin{footnotesize}
    \caption{Ergebnistabelle für das beispielhafte Netzwerk RM mit regenerativen Ressourcen} \label{Tab7}
    \vspace*{3mm}
\csvautotabular{data/beispiel7.csv}
    \begin{center}
      {\footnotesize \textbf{Legende:} AA='Auftragsannahme', LE='Lagerentnahme', LP='Lagerproduktion', KA='Kein Auftrag'} 
      \end{center}
\end{footnotesize}
\end{table}

Diese Modellformulierung zeigt, dass eine Transformation der Ressourcen sinnvoll ist, damit Produktanfragen mit höherem Ertrag akzeptiert werden. Abschließend muss bei der Modellformulierung kritisiert werden, dass es sich um eine andere Darstellung des Netzwerk RM mit opaken Produkten handelt. Eine Transformation von Ressourcen kann auch durch unterschiedliche Ausführungsmodi der Produktanfragen formuliert werden und eine Akzeptanz der Produktanfragen kann folglich direkt über die Kapazität geschehen. Das hier vorgestellte Modell ermöglich jedoch eine flexible Bestimmung aller möglichen Ausführungsmodi, da nicht nur vordefinierte Ausführungsmodi untersucht werden.

\subsection{Inanspruchnahme der Kapazitäten zur Aufstockung eines Lagerbestands für nachfolgende Produktanfragen}

Bei der Modellerweiterung der Inanspruchnahme der Kapazitäten zur Aufstockung eines Lagerbestands für nachfolgende Produktanfragen wird die \textit{Bellman'schen Funktionsgleichung} \ref{dif} mit der Eigenschaft erweitert, dass die Ressourcen $h\in\mathcal{H}$ und die zugehörigen Kapazitäten $c_{h}$ für bestimmte Leistungsperioden $\hat{t}\in \hat{T}$ vorgesehen sind. D. h. dem Parametern $h$ kann der Index ??? der zugehörigen Leistungsperiode $\hat{t}$ ergänzt werden ($\exists{!\hat{t}}\in\hat{T}\; \forall\; h^{\hat{t}}\in\mathcal{H}$). Mit dieser Eigenschaft können Anfragen betrachtet werden, die speziell für bestimmte zusammenhängende Buchungsperioden eintreffen. Damit sind bestimmte Produktanfragen nicht über den gesamten Buchungshorizont verfügbar. Mit Veränderung der Möglichkeiten des Auftragseingangs von Produktanfragen wird die Leistungsperiode $\hat{t}$ bestimmt. Die Leistungsperioden $\hat{t}\in \hat{T}$ lassen sich somit als zusammenhängenden Buchungsperioden $t\in T$ definieren, wie Abbildung \ref{LP} anhand eines Beispiels aufzeigt. In dem Beispiel wird ein Netzwerk mit drei Ressourcen $h^{\hat{t}}\in\mathcal{H}$ betrachtet, die somit den drei Leistungsperioden $\hat{t}\in\hat{T}$ des Beispiels entsprechen. Der Buchungshorizont $T$ entspricht 15 Perioden, die sich auf die Leistungsperioden gleichmäßig verteilen. Diese Verteilung resultiert aus der Möglichkeit des Auftragseingangs der drei Produktanfragen $j\in\mathcal{J}$ (Wahrscheinlichkeitsverteilung $p_j(t)\; \forall\; j\in\mathcal{J}$).

\begin{figure}[h!]
  \begin{center}
    \includegraphics[width=130mm]{Bilder/Leistungsperioden.pdf}
    \caption{Zusammenhand von Buchungs- und Leistungsperiode bei der Inanspruchnahme von Kapazität zur Aufstockung eines Lagerbestands für nachfolgende Produktanfragen}  \label{LP}
    {\footnotesize \textbf{In Anlehnung an:} ????} 
  \end{center}
\end{figure}

Aufgrund dieser Eigenschaft erfolgt die Modifizierung der Menge an möglichen Ausführungsmodi $\textbf{m}\in\mathcal{M}_{j}$. Die Ressourcen $h^{\hat{t}}\in\mathcal{H}$ sind speziell für die jeweilige Leistungsperiode $\hat{t}\in \hat{T}$ vorgesehen und daher kann eine Transformation der Ressource $h^{\hat{t}}$ aus der aktuellen Leistungsperiode $\hat{t}$ nur zu den nächstfolgenden Leistungsperioden $\hat{t}\in \hat{T}$ erfolgen. Sei der Parameter $\tilde{j}$ alle möglichen Nachfolger der Produktanfrage $j$, dann ist die Menge aller möglichen Ausführungsmodi $\mathcal{M}_{\tilde{j}>j}\subseteq\mathcal{M}_j$. Damit existieren zum Beispiel für die Anfrage nach $\hat{j}$ aufgrund des Ressourcenverbrauchs $a_{\hat{j}}=(2,0,0,0)$ die Ausführungsmodi $\mathcal{M}_{\tilde{j}>\hat{j}}$: $\{ (0 ,2, 0, 0), (0, 1, 1, 0), (0, 1, 0, 1),$ $(0, 0, 2, 0),$ $(0, 0, 1, 1), (0, 0, 0, 2)\}$. Zur Veranschaulichung der Modellerweiterung wird ein Beispiel eingeführt:

\begin{center}
$j = \{1, 2, 3\}, \; h = \{1,2,3\}, \; r_{1} = 100, \; r_{2} = 200, \; r_{3} = 5000,$ \\
$\text{Startperiode } t=3, \; \tilde{t}=\{\} $,
\end{center}
\[
    \textbf{c}=\begin{pmatrix} 1\\ 1\\ 1  \end{pmatrix}, \;
    \textbf{a}_{1}=\begin{pmatrix} 1\\ 0\\ 0  \end{pmatrix}, \;
     \textbf{a}_{2}=\begin{pmatrix} 0\\ 1\\ 0  \end{pmatrix}, \;
       \textbf{a}_{3}=\begin{pmatrix} 0\\ 0\\ 2  \end{pmatrix}, \;
            p_{j}(t)=
       \begin{pmatrix}
       0.3 & 0.3 & 0.3 \\
0 & 0.5 & 0.5 \\
0 & 0 & 0.2
\end{pmatrix}, 
  \]
  \[
    \textbf{y}^{res}= \begin{pmatrix} 0\\ 0\\ 0  \end{pmatrix}, \;
    \textbf{y}^{max,res}=\begin{pmatrix} 1\\ 1\\ 2  \end{pmatrix}
      \]


\begin{figure}[h!]
  \begin{center}
    \includegraphics[width=200mm, angle=90]{Bilder/Beispiel8.pdf}
    \caption{Darstellung des Entscheidungsbaums des Netzwerk RM mit Inanspruchnahme von Kapazität zur Aufstockung eines Lagerbestands für nachfolgende Produktanfragen}  \label{B8}
    {\footnotesize \textbf{Legende:} Die Zahlen stehen für den Produktauftrag $j$, AA='Auftragsannahme', LE='Lagerentnahme', LP='Lagerproduktion', KA='Kein Auftrag', $\cdots$='Anfrage ablehnen'} 
  \end{center}
\end{figure}


\section{Implementierung mittels IPython Notebook}

\section{Numerische Untersuchung}
% !TEX encoding = UTF-8 Unicode
\chapter{Schlussbemerkung}
\markboth{5 Schlussbemerkung}{}
\setcounter{footnote}{9}

\section*{Zusammenfassung}

Die Arbeit zeigt, dass eine mögliche Lagerhaltung die Inanspruchnahme der Kapazität von endlichen Ressourcen verbessert und den erwarteten Ertrag erhöht. Die verbesserte Kapazitätsallokation kommt zustande, da die Ressourcenkapazitäten zur Erhöhung eines Lagerbestands Verwendung finden. Dies erfolgt indem Anfragen zur Instandsetzung von defekten Produkten mit niedrigem Ertrag abgelehnt und durch die Entscheidungen einer Lagerproduktion zur Erhöhung eines Lagerbestands von bereits reparierten Produkten ersetzt werden. Bei dem Lagerbestand handelt es sich um Kapazitäten, die mit hoher Wahrscheinlichkeit für spätere Anfragen mit höherem Ertragswert Beanspruchung finden müssen, damit eine Maximierung des erwarteten Ertragswerts möglich ist. Durch die Verwendung der Kapazität erfolgt kein Verlust der endlichen Ressource, sondern die Sicherung dieser Kapazität für nachfolgende Produktanfragen. Zusätzlich ermöglich das Modell aus dieser Arbeit auch das Betrachten einer möglichen Lagerproduktion von Produkten sofern keine Anfragen eintreffen. Damit erhält ein Unternehmen für die Planung der Kapazitätsverwendung mehr Flexibilität und erhält dementsprechend eine umfangreichere Politik der möglichen Auftragsannahmen für den Betrachtungszeitraum.

Die Planung der optimalen Politik ist dabei erheblich abhängig von der Wahrscheinlichkeitsverteilung des Eintreffens und der möglichen Erträge der Produktanfragen. Dies resultiert aus der Gleichung \eqref{GB4}. Für jede mögliche Entscheidungsalternative der Modellgleichung gibt es bestimmte OK die durch diese Parameter beeinflusst werden. Im Verlauf des Buchungshorizonts eines beispielhaften Netzwerks verändern sich die Wahrscheinlichkeiten des möglichen Eintreffens der Produktanfragen und damit die OK. Je wahrscheinlicher der Verfall einer Ressource wird, desto höher ist die Wahrscheinlichkeit das eine Lagerproduktion der Ressource den noch möglichen Ertrag maximiert. Der mögliche Ertrag durch Annahme einer Anfrage inkl. der weitere Erwartungswert der noch möglichen Erträge ist nicht mehr ausreichend rentabel und das Modell empfiehlt für die Verwendung der Kapazität die Lagerproduktion. Damit ist im weiteren Verlauf ein höherer Ertrag für das Netzwerk möglich. Anders formuliert bedeutet dies, dass ab einem solchen Punkt bzw. ab einem solchen Systemzustand der Übergang in einen anderen Systemzustand mit einem vorhandenen Lagerbestand nicht gegen die Grenzbedingungnen verstößt und im weiteren Verlauf einen höheren Ertragswert verspricht. Es existiert  damit ein besserer Pfad im Netzwerk-Graphen, der potentiell mehr Ertrag verspricht.

Die numerische Untersuchung zeigt, dass die Anzahl der Ausprägungen der optimalen Politik mit der Wahrscheinlichkeitsverteilung der Produktanfragen variiert. Es gab in der Untersuchung kein Szenario, dass eine höhere Anzahl der optimalen Politik der Lagerproduktion gegenüber der Auftragsannahme aufweist. Auch der Fall, dass die optimale Politik der Lagerproduktion die Auftragsannahme zumindest für eine bestimmte Art von Produktanfragen dominiert konnte nicht gezeigt werden. Dieser Sachverhalt wird dadurch erklärt, dass für die optimale Politik der Auftragsannahme (und auch für die Lagerentnahme) die jeweiligen Erträge der Produktanfragen bei dem Entscheidungsmodell Berücksichtigung finden. Sofern eine bestimmte Reihenfolge der Auftragseingänge unterstellt wird, kann die Untersuchung des Verlaufs der optimalen Politik erfolgen. Die Szenarien zeigen unterschiedliche Verläufe für die optimale Politik, sofern keine Anfragen eintreffen und sofern eine jeweilige Produktanfrage eintrifft. Abhängig der Szenarien ist dabe dargelegt, dass der Zeitpunkt des Verderbens einer Ressourcenkapazität erheblichen Einfluss auf die optimale Politik hat. Abhängig der Eintrittswahrscheinlichkeit und der möglichen Erträge erfolgt die Verwendung der Kapazität zur Lagerproduktion eines Produkts in unterschiedlicher Weise in Bezug auf Inanspruchnahme dieser Option und dem Zeitpunkt des Auftretens dieser Alternative.

\section*{Limitation}

In dieser Arbeit wird die Lagerproduktion eines Produkts betrachtet, bei der die Instandsetzung von defekten Produkten aufgrund der Inanspruchnahme der Kapazität einer einzelnen Ressource erfolgt. Das Modell ermöglicht die Berücksichtigung von unterschiedlichen Ressourcen, wie z. B. Maschinen- und Personaleinsatzstunden. Für die numerische Untersuchung ist die Vereinfachung auf nur eine Ressource jedoch zielführend. Damit war zum einen die schnelle Berechnung der Szenarien möglich und zum anderen ist dadurch die Funktionsweise des Modells im Grundsatz explorierbar. Jedoch sollte für spätere Analysen auch die Berechnung von umfangreicheren Szenarien mit unterschiedlichen Ressourcen zur Erstellung differenzierter Produkte in Betracht kommen.

In der Modellerweiterung \eqref{time} ist die Entscheidung der Lagerproduktion bei eintreffenden Anfragen nur für den jeweiligen Ausführungsmodi des Produktanfrage möglich. Damit steht dem Netzwerk jeweils nur die Produktionskapazität einer abgelehnten Anfrage zur Verfügung. D. h. ein Unternehmen ist immer gezwungen die Ressourcenkapazität für die Lagererhöhung zu verwendet, die aktuell als Auftrag vorliegt. Damit geht jedoch viel Flexibilität verloren. Daher sollte bei der Modellformulierung \eqref{time} die Möglichkeit bestehen, dass sofern eine Anfrage abgelehnt wird, ein beliebiger Ausführungsmodi aller möglichen Produktanfragen für die Lagerproduktion genutzt werden können. Damit wäre ebenfalls bei dieser Entscheidung über die Auftragsannahme einer eintreffenden Produktanfrage das Maximum über alle Produktanfragen zu bilden. Bei der in dieser Arbeit umgesetzten Modellerweiterung ist dieses nicht berücksichtigt.

Die Implementierung in das Computersystem bietet weiteres Optimierungspotential. Der Algorithmus ist zwar mit einigen Funktionen des Softwarepakets \texttt{Numpy} entwickelt, aber eine komplette Implementierung in die Programmiersprache \texttt{C++} oder \texttt{Java} könnte die Leistung weiter verbessern bzw. die Rechenzeit reduzieren. Sofern weiter an dieser Implementierung festgehalten wird, empfiehlt es sich die Memofunktion zu überarbeiten. In der aktuellen Fassung der Implementierung basiert die Memofunktion auf der Programmiersprache \texttt{Python}. Auch hier sollte ein Versuch getätigt werden, eine Memofunktion mit dem Softwarepaket \texttt{Numpy} zu entwickeln. Ggf. kann dadurch die vorherige Berechnung aller möglichen Systemzustände des Netzwerks entfallen. Diese Berechnung aller möglichen Systemzustände ist zwar ein sehr effizienter Algorithmus, aber die Funktion zur Berechnung hat einen hohen Hauptspeicherbedarf. Sofern die Berechnung von umfangreichere Szenarien erfolgen soll, muss diese Funktion parallelisiert werden. Dadurch erfolg die Berechnung der Funktion auf mehreren Knoten (Computern) des Clustersystems unter Nutzung des jeweiligen Speichers. Die Umsetzung dieser Leistungsverbesserungen des Algorithmus konnte in dieser Arbeit nicht mehr erfolgen.

Weiter ist die verwendete Implementierung des DP in rekursiver Form programmiert. Ein iterativer Lösungweg könnte die Leistung des Algorithmus verbessern, aber auch eine mögliche Parallelisierung der Teilprobleme des Auftragsannahmeproblems sollte weiter untersucht werden. In dem begrenzten Zeitraum der Erstellung dieser Arbeit erfolgte keine abschließende Umsetzung einer möglichen Parallelisierung der Problemstellung. Die Implementierung nutzt zwar viele aktuelle Funktionen in Bezug des wissenschaftlichen Rechnens, aber es ist der erste Versuch einer Implementierung des Auftragsannahmeproblems des Netzwerk RM unter Berücksichtigung einer Lagerhaltung. Es besteht daher Potential zur Verbesserung der Implementierung.

\section*{Ausblick}

Der Schwerpunkt weiter Forschung in Bezug des Auftragsannahmeproblems im Netzwerk RM mit Lagerhaltungsentscheidung sollte die Konkretisierung der Problemstellung sein. Bei der hier gezeigten Modellerweiterung handelt es sich um eine einfache Umformung des Grundmodells des Netzwerk RM (siehe Gleichung \eqref{DP} im Vergleich zu \eqref{time}). Es berücksichtig z. B. keine Lagerkosten oder eine zeitgebundene Erhöhung von Ressourcen. Daher sollte nachfolgend ein Modell entwickelt werden, welches aus einem tatsächlichen Anwendungsfall resultiert und die Problemstellung umfangreicher abbildet.

Jedoch zeigt die aktuelle Entwicklung im Forschungsbereich des Netzwerk RM in der Auftragsfertigung, dass ein exaktes Lösungsverfahren keine reellen Probleme in einer angemessenen Berechnungszeit löst. Die Integration eines Verfahrens der Approximation der Problemstellung sollte daher ebenfalls vorangetrieben werden. Durch Integration von Verfahren, wie z. B. dem Verfahren des Bid-Preises, reduziert die Berechnungszeit und ermöglicht die Übertragung in moderne Anwendungssystemen für das RM. Ebenfalls können die Ansätze aus dem Literaturüberblick aus Kapitel \ref{Review} auf mögliche Umsetzung untersucht werden. Vielversprechend erscheint die Ermittlung des Bid-Preises anhand eines Knappsack-Verfahrens.

Diese Arbeit zeigt eine Modellformulierung für den Anwendungsfalls des Auftragsannahmeproblems im Netzwerk RM mit Lagerhaltungsentscheidungen. 



%Sofern eine Lagerhaltung berücksichtigt wird, verstößt es gegen die Anwendungsvoraussetzung der \glqq Nichtlagerfähigkeit{\grqq} Diese 

%Anhang
\begin{appendix}
\chapter*{Anhang}
\markboth{Anhang}{}
\addcontentsline{toc}{chapter}{Anhang} %Anhang im Inhaltsverzeichnis

\section{DP-Implementierung des Auftragsannahmeproblems im Netzwerk Revenue Management}\label{CodeA}

\lstinputlisting[language=Python, caption=Parameter Definition für die Dynamische Programmierung im Netzwerk Revenue Management, style=Listing, label=Parameter]{/Users/Superuser/DP-RM-with-storage/cluster/Muster/Parameter.py}
%firstline=102, lastline=114, 

\lstinputlisting[language=Python, caption=Python-Algorithmus für die Dynamische Programmierung im Netzwerk Revenue Management, style=Listing, label=DyProgramm]{/Users/Superuser/DP-RM-with-storage/cluster/Muster/DynamicProgramm_Stock.py}

\section{Szenarien}

\subsection{Rechnung}

%\applyCSVfile{/Users/Superuser/DP-RM-with-storage/cluster/DP_N_G_V/Table_Optimal2015-08-15.csv}
\end{appendix}

%Literaturverzeichnis
\bibliographystyle{Prod_Seminar}
\bibliography{Literatur}




%Selbstst\"{a}ndigkeitserkl\"{a}rung

\newpage
\thispagestyle{empty}
\begin{center}
  \vspace*{\stretch{1.5}}
  {\Large\bf Erkl\"{a}rung} \\ [2.5cm]
\end{center}
\begin{flushleft}
  Hiermit versichere ich, dass ich die vorliegende Arbeit selbstst\"{a}ndig verfasst und keine anderen als die angegebenen Quellen und Hilfsmittel
  benutzt habe, dass alle Stellen der Arbeit, die w\"{o}rtlich oder sinngem\"{a}{\ss} aus anderen Quellen \"{u}bernommen wurden, als solche kenntlich gemacht
  sind und dass die Arbeit in gleicher oder \"{a}hnlicher Form noch keiner Pr\"{u}fungsbeh\"{o}rde vorgelegt wurde.\\[3cm]
  Hannover, 11. September 2015
  \vspace*{\stretch{1.5}}
\end{flushleft}
\newpage\thispagestyle{empty}\null\newpage

\end{document}
