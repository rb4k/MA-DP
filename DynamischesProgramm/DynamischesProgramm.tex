% !TEX encoding = UTF-8 Unicode
\chapter{Ein exaktes Lösungsverfahren zur Auftragsannahme- und Lagerhaltungsentscheidung bei auftragsbezogenen Instandhaltungsprozessen}
\markboth{3 Ein exaktes Lösungsverfahren zur Auftragsannahme- und Lagerhaltungsentscheidung}{}
\setcounter{footnote}{7}

\section{Mathematische Modellformulierung}

Aufbauend auf der Modelldarstellung der Auftragsannahme mittels Revenue Management erfolgt in diesem Kapitel die Erweiterung um die Lagerhaltungsentscheidung. Instandhalter haben die Möglichkeit Anfragen nach MRO-Prozessen entweder durch Kapazitätsinanspruchnahme oder durch Lagerentnahme eines neuwertigen Produkts zu akzeptieren. Abhängig der Entscheidung, in welchem dieser Modi die Anfrage befriedigt wird, erfolgt entweder eine Kapazitäts- oder Lagerreduktion.

\subsection{Lagerentnahmeentscheidung}

Das Gleichung \eqref{DP} wird um den Parameter des Lagerbestands $y_{j}$ erweitert. Bei dieser Modellerweiterung fungiert der Parameter $y_{j}$ als Lagerbestand von Produkten $j\in\mathcal{J}$. D. h. es sind Bündel an Ressourcen $h\in\mathcal{H}$ gemeint, die direkt den Nachfragern zur Verfügung gestellt werden. Der Parameter lässt sich als Vektor $\textbf{y}$ interpretieren, wobei die Länge des Vektors die Anzahl an Produkten $j\in\mathcal{J}$ und der sortierte Eintrag des Vektors dem Lagerbestands des Produkts $j$ entspricht. Sofern eine Anfrage über den Lagerbestand befriedigt wird, erfolgt eine Reduktion des Lagerbestand durch den Parameter $s_{jj}$. Da in dieser Modellannahme eine Anfrage nach einem Produkt $j$ auch nur mit dem Lagerbestand des Produkts $j$ angenommen werden kann, entspricht $s_{j=j}=1$ und alle anderen $s_{j\neq j}=0$. Damit lässt sich ein Vektor $\textbf{s}_j$ formen, der als Lagerentnahme für eine Anfrage nach dem Produkt $j$ dient. Weiter können die einzelnen Vektoren $s_{j}$ als Matrix $\textbf{S}$ aufgebaut werden. Die Matrix $\textbf{S}$ entspricht einer Einheitsmatrix $I_{n}\in\mathbb{R}^{n\times n}$ wobei $n=j$.

Die Gleichung \eqref{DP} lässt sich damit wie folgt formulieren:

\begin{alignat*}{2}
V(\textbf{c}, \textbf{y}, t) =\;& \sum_{j \in \mathcal{J}}p_{j}(t)\max[V(\textbf{c}, \textbf{y}, t-1), \\
& r_{j} + V(\textbf{c}-\textbf{a}_j, \textbf{y}, t-1),\\
& r_{j} + V(\textbf{c}, \textbf{y}-\textbf{s}_j, t-1)] \\
& + p_{0}(t)V(\textbf{c}, \textbf{y}, t-1)\\[10pt] 
= \;& \sum_{j \in \mathcal{J}}p_{j}(t)V(\textbf{c}, \textbf{y}, t-1)\\
& + \sum_{j \in \mathcal{J}}p_{j}(t)\max[r_{j} - V(\textbf{c}, \textbf{y}, t-1) + V(\textbf{c}-\textbf{a}_j, \textbf{y}, t-1),0]\\
& +  \sum_{j \in \mathcal{J}}p_{j}(t)\max[r_{j} - V(\textbf{c}, \textbf{y}, t-1) + V(\textbf{c}, \textbf{y}-\textbf{s}_j, t-1),0]\\
&+ p_{0}(t)V(\textbf{c}, \textbf{y}, t-1)\\[10pt] 
= \;& \sum_{j \in \mathcal{J}}p_{j}(t)V(\textbf{c}, \textbf{y}, t-1)\\
&+ \sum_{j \in \mathcal{J}}p_{j}(t)[\max[r_{j} - V(\textbf{c}, \textbf{y}, t-1)+ V(\textbf{c}-\textbf{a}_j, \textbf{y}, t-1),0]\\
&+ \max[r_{j} - V(\textbf{c}, \textbf{y}, t-1) + V(\textbf{c}, \textbf{y}-\textbf{s}_j, t-1),0]]\\
&+ p_{0}(t)V(\textbf{c}, \textbf{y}, t-1)\\
\end{alignat*}
\begin{equation}\label{stock}
\begin{alignat*}{2}
V(\textbf{c}, \textbf{y}, t) = \;& V(\textbf{c}, \textbf{y}, t-1)\\
&+ \sum_{j \in \mathcal{J}}p_{j}(t)[\max[r_{j} - V(\textbf{c}, \textbf{y}, t-1) + V(\textbf{c}-\textbf{a}_j, \textbf{y}, t-1),0]\\
&+ \max[r_{j} - V(\textbf{c}, \textbf{y}, t-1) + V(\textbf{c}, \textbf{y}-\textbf{s}_j, t-1),0]]\\
\end{alignat*}
\end{equation}




Bei der Modellformulierung $V(\textbf{c}, \textbf{y}, t)$ der \textit{Bellman'schen Funktionsgleichung} des RM mit Lagerentnahme ist der Term $ r_{j} + V(\textbf{c}, \textbf{y}-\textbf{s}_j, t-1)]$ integriert, der die Annahme mittels des Lagerbestands beschreibt. Damit ist es dem Unternehmen möglich entweder die Kapazität oder den Lagerbestand in Anspruch nehmen. Diese Optionen werden in dieser Arbeit durch einen weiteren Indiz $m$ bei dem Parameter für den Produktauftrag $j_{m}$ kenntlich gemacht. Sofern es sich um eine Auftragsannahme mittels Kapazitätsinanspruchnahme (AA) handelt, erfolgt die Annahme des Produktauftrags $j_{AA}$. Handelt es sich um eine Annahme des Auftrags mittels des Lagerentnahme (LE), dann wird der Parameter $j_{LE}$ aufgeführt.

Für die Modellerweiterung gelten die Grenzbedingungen \eqref{GB1} sowie \eqref{GB2} und es gilt zusätzlich
\begin{equation}\label{GB3}
     OP_{\textbf{c}, t}:=\left\{\begin{array}{ll} j_{AA}, & \text{für } r_{j_{AA}} - OC_{j_{AA}} \ge r_{j_{LE}} - OC_{j_{LE}}\\
         j_{LE}, & \text{für } r_{j_{AA}} - OC_{j_{AA}} < r_{j_{LE}} - OC_{j_{LE}}\end{array}\right. ,
\end{equation}
da das Unternehmen vorrangig versucht seine Kapazitäten auszulasten. Die Funktionsweise der Modellformulierung wird im nachfolgenden Beispiel verdeutlicht.
\begin{center}
$j = \{1, 2\}, \; h = \{1\}, \; r_{1} = 100, \; r_{2} = 200, \; T=4$,
\end{center}
\[
    c_{1}=1, \;
    a_{11}=1, \;
     a_{12}=1, \;
     p_{1}(t)=\begin{pmatrix} 0.5\\ 0.5\\ 0.5\\ 0.5  \end{pmatrix}, \;
     p_{2}(t)=\begin{pmatrix} 0.1\\ 0.1\\ 0.1\\ 0.1  \end{pmatrix},
  \]
  \[
    \textbf{y}=\begin{pmatrix} 1 \\ 0 \end{pmatrix}, \;
    \textbf{s}_1=\begin{pmatrix} 1 \\ 0 \end{pmatrix}, \;
     \textbf{s}_2=\begin{pmatrix} 0 \\ 1 \end{pmatrix} \;
  \]

\begin{figure}[h!]
  \begin{center}
    \includegraphics[width=130mm]{Bilder/Beispiel3.pdf}
    \caption{Darstellung des Entscheidungsbaums des Netzwerk RM mit Möglichkeit der Lagerentnahme}  \label{B3}
    {\footnotesize \textbf{Legende:} Die Zahlen stehen für den Produktauftrag $j$, AA='Auftragsannahme', LE='Lagerentnahme', KA='Kein Auftrag'} 
  \end{center}
\end{figure}

Abbildung \ref{B3} zeigt den Entscheidungsbaum mit den möglichen Systemzuständen aufgrund der vorher beschrieben Parameter. Dabei beschreibt ein Knoten weiterhin der Systemzustand. Ein Systemzustand wird durch die Zahlenfolge definiert, wobei die ersten Einträge der Zahlenfolge die Ressoucenkapazität $\textbf{c}$ entspricht. Da in diesem Beispiel nur eine Ressource vorhanden ist, beschreibt der erste Eintrag der Zahlenfolge die Ressourcenkapazität $c_{h}$=1 von $h=1$. Das Beispiel verfügt über zwei unterschiedliche Auftragsarten von Produkten $j\in\mathcal{J}$. Somit existieren zwei Parameter für den Lagerbestand $y_{j}$. In dem vereinfachten Beispiel weist jedoch nur das Produkt $j=1$ einen Lagerbestand in Höhe von $y_{1}=1$ auf. Vom Produkt $j=2$ gibt es in diesem Beispiel keinen Lagerbestand. Damit entsprechen die nachfolgenden Zahlen in der Zahlenfolge der Systemzustände den möglichen Lagerbestand $\textbf{y}$. Der letzter Wert der Zahlenfolge ist weiterhin der Zeitpunkt bzw. die Periode $t$. Dem Unternehmen ist es damit möglich, unter Beachtung der vorausgehenden Parameter, entweder eine Anfrage von $j=1$ oder $j=2$ mittels der Kapazität von $c_{1}$ anzunehmen (Auftragsannahme) oder eine Anfrage nach Produkt $j=1$ mittels des Lagerbestands $y_{1}$ zu erfüllen (Lagerentnahme). Des Weiteren ist das Eintreffen keiner Anfrage zum Zeitpunkt $t$ möglich (Kein Auftrag). Damit sind die in dem Graphen aus der Abbildung \ref{B3} möglichen Systemzustände (Knoten) und Übergänge (Kanten) möglich.

\begin{table}
\begin{footnotesize}
    \caption{Ergebnistabelle für das beispielhafte Netzwerk RM mit Möglichkeit der Lagerentnahme} \label{Tab3}
    \vspace*{3mm}
\csvautotabular{data/beispiel3.csv}
\begin{center}
      {\footnotesize \textbf{Legende:} AA='Auftragsannahme', LE='Lagerentnahme', KA='Kein Auftrag'} 
      \end{center}
\end{footnotesize}
\end{table}

Die Tabelle \ref{Tab3} zeigt für das Beispiel die berechneten Erwartungswerte des noch möglichen Ertrags für jeden Systemzustand. Ebenfalls ist die optimale Politik in Form des besten Auftrags mit zugehörigem Ausführungsmodus aufgeführt. Dabei beschreibt hier der Ausführungsmodus die Auftragsannahme (AA), Lagerentnahme (LE) oder, ob ein Auftrag keine Annahme erhält (KA). Zusätzlich ist der Wert $r_{j}-OC_{j}$ für den besten Auftrag je Systemzustand angegeben. Mit diesen Werten lässt sich der optimale Pfad des Beispiels ermitteln: $[2\;1\;0\;4] \rightarrow_{j_{AA}=2} [1\;1\;0\;3] \rightarrow_{j_{AA}=2} [0\;1\;0\;2] \rightarrow_{j_{LE}=1} [0\;0\;0\;1]\rightarrow_{j=0} [0\;0\;0\;0]$. Auch hier gilt weiterhin, dass der optimale Pfad abhängig ist der tatsächlich eintreffenden Anfragen. Er kann nur aber bei strategischen Entscheidung herangezogen werden.

Durch die Modellerweiterung wird gezeigt, dass ein Lagerbestand an Produkten $j\in\mathcal{J}$ den Gesamtertrag des Unternehmens erhöhen kann. Dies erfolgt jedoch aufgrund des Mechanismus, dass ein Lagerbestand eine Kapazitätserhöhung für das Unternehmen bedeutet. Somit handelt es hier um eine andere Darstellung der Modellformulierung des Netzwerk RM mit verschiedenen Ausführungsmodi für die Produktanfragen $j\in\mathcal{J}$. %Eine abschließende Beurteilung der Modellerweiterung erfolgt mit Abschluss des nachfolgenden Abschnitts.

\subsection{Lagerentnahme- und Lagerproduktionsentscheidung}

Im vorhergehenden Abschnitt ist die Gleichung \eqref{DP} um die Eigenschaft der Lagerentnahme erweitert. Damit sind die Entscheidungen über die Annahme eines Auftrags via Kapazitäts- oder Lagerparameter möglich. Die nachfolgende Modellerweiterung soll die Gleichung \eqref{stock} mit der Entscheidung über die gewollte Ablehnung einer Anfrage erweitern, damit die Kapazitäten für die Lagererhöhung Verwendung finden. D. h. die Kapazitäten $\textbf{c}$ werden um den Ressourcenverbrauch $\textbf{a}_{j}$ reduziert, damit der Lagerbestand $\textbf{y}$ um den Parameter für die Lagerveränderung $\textbf{s}_{j}$ für ein Produkt $j$ erhöht werden kann. Die Modellformulierung lautet wie folgt:
\begin{alignat*}{2}
V(\textbf{c}, \textbf{y}, t) =\;& \sum_{j \in \mathcal{J}}p_{j}(t)\max[V(\textbf{c}, \textbf{y}, t-1),\\
&r_{j} + V(\textbf{c}-\textbf{a}_j, \textbf{y}, t-1),\\
&r_{j} + V(\textbf{c}, \textbf{y}-\textbf{s}_j, t-1),\\
&V(\textbf{c}-\textbf{a}_j, \textbf{y}+\textbf{s}_j, t-1)]\\
&+ p_{0}(t)V(\textbf{c}, \textbf{y}, t-1) \\
\end{alignat*}
\begin{equation}\label{storage}
\begin{alignat*}{2}
= \;& V(\textbf{c}, \textbf{y}, t-1)\\
&+ \sum_{j \in \mathcal{J}}p_{j}(t)[\max[r_{j} - V(\textbf{c}, \textbf{y}, t-1) + V(\textbf{c}-\textbf{a}_j, \textbf{y}, t-1),0] \\
&+ \max[r_{j} - V(\textbf{c}, \textbf{y}, t-1) + V(\textbf{c}, \textbf{y}-\textbf{s}_j, t-1),0]\\
&+ \max[V(\textbf{c}-\textbf{a}_j, \textbf{y}+\textbf{s}_j, t-1) - V(\textbf{c}, \textbf{y}, t-1) ,0]]\\
\end{alignat*}
\end{equation}

Neben der Entscheidungen über die Auftragsannahme mittels Kapazitätsinanspruchnahme (AA) und der Lagerentnahme eines Produkts (LE) ist in dieser Modellerweiterung der Gleichung \eqref{storage} die Produktion eines Produkts $j$ auf Lager möglich. Dafür wird der Parameter $s_{j}$ als Lagerveränderung interpretiert. Zu beachten ist jedoch, dass bei der Entscheidung der Lagerproduktion eines Produkts $j_{LP}$ kein Ertrag $r_{j_{LE}}$ erzielt wird. Des Weiteren wird ein Parameter für einen maximalen Lagerbestand $y_{j}^{max}$ für jede möglichen Anfragentyp der Produkte $j\in\mathcal{J}$ definiert. Eine derartige Modellformulierung birgt jedoch eine spezielle Funktionsweise mit, wie nachfolgendes Beispiel verdeutlichen soll:
\begin{center}
$j = \{1, 2\}, \; h = \{1\}, \; r_{1} = 100, \; r_{2} = 200, \; T=3$,
\end{center}
\[
    c_{1}=2, \;
    a_{11}=1, \;
     a_{12}=2, \;
     p_{1}(t)=\begin{pmatrix} 0.5\\ 0.5\\ 0.5  \end{pmatrix}, \;
     p_{2}(t)=\begin{pmatrix} 0.1\\ 0.1\\ 0.1  \end{pmatrix},
  \]
  \[
    \textbf{y}=\begin{pmatrix} 0 \\ 0 \end{pmatrix}, \;
    \textbf{y}^{max}=\begin{pmatrix} 2 \\ 1 \end{pmatrix}, \;
    \textbf{s}_1=\begin{pmatrix} 1 \\ 0 \end{pmatrix}, \;
     \textbf{s}_2=\begin{pmatrix} 0 \\ 1 \end{pmatrix} \;
  \]
\begin{figure}[h!]
  \begin{center}
    \includegraphics[width=130mm]{Bilder/Beispiel4.pdf}
    \caption{Darstellung des Entscheidungsbaums des Netzwerk RM mit Möglichkeit der Lagerentnahme und Lagerproduktion}  \label{B4}
    {\footnotesize \textbf{Legende:} Die Zahlen stehen für den Produktauftrag $j$, AA='Auftragsannahme', LE='Lagerentnahme', LP='Lagerproduktion', KA='Kein Auftrag'} 
  \end{center}
\end{figure}
Die Abbildung \ref{B4} mit dem Entscheidungsbaum für das Beispiel zeigt alle möglichen Systemzustände und Optionen. Es gelten die Grenzbedingungen \eqref{GB1} sowie \eqref{GB2} und
\begin{equation}\label{GB3}
     OP_{\textbf{c}, t}:=\left\{\begin{array}{lll} j_{AA}, & \text{für } r_{j_{AA}} - OC_{j_{AA}} \ge r_{j_{LE}} - OC_{j_{LE}}\\
         j_{LE}, & \text{für } r_{j_{AA}} - OC_{j_{AA}} < r_{j_{LE}} - OC_{j_{LE}}\\
         j_{LP}, & \text{sonst}\end{array}\right. .
\end{equation}
Sofern die Modellerweiterung in dieser Form definiert ist, wird niemals $OP_{\textbf{c}, t}:=j_{LP}$ gelten. Dies resultiert aus der Tatsache, dass sofern genügend Kapazitäten $\textbf{c}$ zur Produktion eines Produkts $j$ vorhanden sind, die Kapazitäten für die direkte Annahme der Produktanfrage $j$ verwendet werden. Die Entscheidung über die Produktion eines Produkts $j_{LP}$ ist für das Unternehmen nur dann sinnvoll, wenn keine Anfragen zum Zeitpunkt $t$ eintreffen. Die Tabelle \ref{Tab4} zeigt die berechneten Werte für das hier aufgeführte Beispiel.
\begin{table}
\begin{footnotesize}
    \caption{Ergebnistabelle für das beispielhafte Netzwerk RM mit Möglichkeit der Lagerentnahme} \label{Tab4}
    \vspace*{3mm}
\csvautotabular{data/beispiel4.csv}
\begin{center}
      {\footnotesize \textbf{Legende:} AA='Auftragsannahme', LE='Lagerentnahme', LP='Lagerproduktion', KA='Kein Auftrag'} 
      \end{center}
\end{footnotesize}
\end{table}


\section{Implementierung mittels IPython Notebook}

\section{Numerische Untersuchung}