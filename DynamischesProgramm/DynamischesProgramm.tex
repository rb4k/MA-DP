% !TEX encoding = UTF-8 Unicode
\chapter{Ein exaktes Lösungsverfahren zur Auftragsannahme- und Lagerhaltungsentscheidung bei auftragsbezogenen Instandhaltungsprozessen}
\markboth{3 Ein exaktes Lösungsverfahren zur Auftragsannahme- und Lagerhaltungsentscheidung}{}
\setcounter{footnote}{7}

\section{Mathematische Modellformulierung}

Aufbauend auf der Modelldarstellung der Auftragsannahme mittels Revenue Management erfolgt in diesem Kapitel die Erweiterung um die Lagerhaltungsentscheidung. Instandhalter haben die Möglichkeit Anfragen nach MRO-Prozessen entweder durch Kapazitätsinanspruchnahme oder durch Lagerentnahme eines neuwertigen Produkts zu akzeptieren. Abhängig der Entscheidung, in welchem dieser Modi die Anfrage befriedigt wird, erfolgt entweder eine Kapazitäts- oder Lagerreduktion.

\subsection{Lagerentnahmeentscheidung}

Das Gleichung \eqref{DP} wird um den Parameter des Lagerbestands $y_{j}$ erweitert. Bei dieser Modellerweiterung fungiert der Parameter $y_{j}$ als Lagerbestand von Produkten $j\in\mathcal{J}$. D. h. es sind Bündel an bereits verbrauchten Ressourcen $h\in\mathcal{H}$ gemeint, die direkt den Nachfragern zur Verfügung gestellt werden. Der Parameter lässt sich als Vektor $\textbf{y}$ interpretieren, wobei die Länge des Vektors die Anzahl an Produkten $j\in\mathcal{J}$ und der sortierte Eintrag des Vektors dem Lagerbestands des Produkts $j$ entspricht. Sofern eine Anfrage über den Lagerbestand befriedigt wird, erfolgt eine Reduktion des Lagerbestand durch den Parameter $s_{jj}$. Da in dieser Modellannahme eine Anfrage nach einem Produkt $j$ auch nur mit dem Lagerbestand des Produkts $j$ angenommen werden kann, entspricht $s_{j=j}=1$ und alle anderen $s_{j\neq j}=0$. Damit lässt sich ein Vektor $\textbf{s}_j$ formen, der als Lagerentnahme für eine Anfrage nach dem Produkt $j$ dient. Weiter können die einzelnen Vektoren $s_{j}$ als Matrix $\textbf{S}$ aufgebaut werden. Die Matrix $\textbf{S}$ entspricht einer Einheitsmatrix $I_{n}\in\mathbb{R}^{n\times n}$ wobei $n=j$.

Die Gleichung \eqref{DP} lässt sich damit wie folgt formulieren:

\begin{alignat*}{2}
V(\textbf{c}, \textbf{y}, t) =\;& \sum_{j \in \mathcal{J}}p_{j}(t)\max[V(\textbf{c}, \textbf{y}, t-1), \\
& r_{j} + V(\textbf{c}-\textbf{a}_j, \textbf{y}, t-1),\\
& r_{j} + V(\textbf{c}, \textbf{y}-\textbf{s}_j, t-1)] \\
& + p_{0}(t)V(\textbf{c}, \textbf{y}, t-1)\\[10pt] 
%= \;& \sum_{j \in \mathcal{J}}p_{j}(t)V(\textbf{c}, \textbf{y}, t-1)\\
%& + \sum_{j \in \mathcal{J}}p_{j}(t)\max[r_{j} - V(\textbf{c}, \textbf{y}, t-1) + V(\textbf{c}-\textbf{a}_j, \textbf{y}, t-1),0]\\
%& +  \sum_{j \in \mathcal{J}}p_{j}(t)\max[r_{j} - V(\textbf{c}, \textbf{y}, t-1) + V(\textbf{c}, \textbf{y}-\textbf{s}_j, t-1),0]\\
%&+ p_{0}(t)V(\textbf{c}, \textbf{y}, t-1)\\[10pt] 
= \;& \sum_{j \in \mathcal{J}}p_{j}(t)V(\textbf{c}, \textbf{y}, t-1)\\
&+ \sum_{j \in \mathcal{J}}p_{j}(t)[\max[r_{j} - V(\textbf{c}, \textbf{y}, t-1)+ V(\textbf{c}-\textbf{a}_j, \textbf{y}, t-1),0]\\
&+ \max[r_{j} - V(\textbf{c}, \textbf{y}, t-1) + V(\textbf{c}, \textbf{y}-\textbf{s}_j, t-1),0]]\\
&+ p_{0}(t)V(\textbf{c}, \textbf{y}, t-1)\\
\end{alignat*}
\begin{equation}\label{stock}
\begin{alignat*}{2}
V(\textbf{c}, \textbf{y}, t) = \;& V(\textbf{c}, \textbf{y}, t-1)\\
&+ \sum_{j \in \mathcal{J}}p_{j}(t)[\max[r_{j} - V(\textbf{c}, \textbf{y}, t-1) + V(\textbf{c}-\textbf{a}_j, \textbf{y}, t-1),0]\\
&+ \max[r_{j} - V(\textbf{c}, \textbf{y}, t-1) + V(\textbf{c}, \textbf{y}-\textbf{s}_j, t-1),0]]\\
\end{alignat*}
\end{equation}




Bei der Modellformulierung $V(\textbf{c}, \textbf{y}, t)$ der \textit{Bellman'schen Funktionsgleichung} des RM mit Lagerentnahme ist der Term $ r_{j} + V(\textbf{c}, \textbf{y}-\textbf{s}_j, t-1)]$ integriert, der die Annahme mittels des Lagerbestands beschreibt. Damit ist es dem Unternehmen möglich entweder die Kapazität oder den Lagerbestand in Anspruch zu nehmen. Diese Optionen werden in dieser Arbeit durch einen weiteren Index $m$ bei dem Parameter für den Produktauftrag $j_{m}$ kenntlich gemacht. Sofern es sich um eine Auftragsannahme mittels Kapazitätsinanspruchnahme (AA) handelt, erfolgt die Annahme des Produktauftrags $j_{AA}$. Handelt es sich um eine Annahme des Auftrags mittels des Lagerentnahme (LE), dann wird der Parameter $j_{LE}$ aufgeführt.

Für die Modellerweiterung gelten die Grenzbedingungen \eqref{GB1} sowie \eqref{GB2} und es gilt zusätzlich
\begin{equation}\label{GB3}
     OP_{\textbf{c}, t}:=\left\{\begin{array}{ll} j_{AA}, & \text{für } r_{j_{AA}} - OC_{j_{AA}} \ge r_{j_{LE}} - OC_{j_{LE}}\\
         j_{LE}, & \text{für } r_{j_{AA}} - OC_{j_{AA}} < r_{j_{LE}} - OC_{j_{LE}}\end{array}\right. ,
\end{equation}
da das Unternehmen vorrangig versucht seine Kapazitäten auszulasten. Die Funktionsweise der Modellformulierung wird im nachfolgenden Beispiel verdeutlicht.
\begin{center}
$j = \{1, 2\}, \; h = \{1\}, \; r_{1} = 100, \; r_{2} = 200, \; \text{Startperiode } t=4$,
\end{center}
\[
    c_{1}=1, \;
    a_{11}=1, \;
     a_{12}=1, \;
     p_{1}(t)=\begin{pmatrix} 0.5\\ 0.5\\ 0.5\\ 0.5  \end{pmatrix}, \;
     p_{2}(t)=\begin{pmatrix} 0.1\\ 0.1\\ 0.1\\ 0.1  \end{pmatrix},
  \]
  \[
    \textbf{y}=\begin{pmatrix} 1 \\ 0 \end{pmatrix}, \;
    \textbf{s}_1=\begin{pmatrix} 1 \\ 0 \end{pmatrix}, \;
     \textbf{s}_2=\begin{pmatrix} 0 \\ 1 \end{pmatrix} \;
  \]

\begin{figure}[h!]
  \begin{center}
    \includegraphics[width=130mm]{Bilder/Beispiel3.pdf}
    \caption{Darstellung des Entscheidungsbaums des Netzwerk RM mit Möglichkeit der Lagerentnahme}  \label{B3}
    {\footnotesize \textbf{Legende:} Die Zahlen stehen für den Produktauftrag $j$, AA='Auftragsannahme', LE='Lagerentnahme', KA='Kein Auftrag'} 
  \end{center}
\end{figure}

Abbildung \ref{B3} zeigt den Entscheidungsbaum mit den möglichen Systemzuständen aufgrund der vorher beschrieben Parameter. Dabei beschreibt ein Knoten weiterhin den Systemzustand. Ein Systemzustand wird durch die Zahlenfolge definiert, wobei die ersten Einträge der Zahlenfolge die Ressoucenkapazität $\textbf{c}$ entspricht. Da in diesem Beispiel nur eine Ressource vorhanden ist, beschreibt der erste Eintrag der Zahlenfolge die Ressourcenkapazität $c_{h}=1$ von $h=1$. Das Beispiel verfügt über zwei unterschiedliche Produkte $j\in\mathcal{J}$, die von den Nachfragern angefragt werden können. Somit existieren zwei Parameter für den Lagerbestand $\textbf{y}=(y_{1},y_{2})$. In dem vereinfachten Beispiel weist jedoch nur das Produkt $j=1$ einen Lagerbestand in Höhe von $y_{1}=1$ auf. Vom Produkt $j=2$ gibt es in diesem Beispiel keinen Lagerbestand. Damit entsprechen die nachfolgenden Zahlen in der Zahlenfolge des Systemzustands den möglichen Lagerbeständen $\textbf{y}$. Der letzter Wert der Zahlenfolge ist weiterhin der Zeitpunkt bzw. die Periode $t$. Der Systemzustand lässt damit wie folgt definieren: $[c_{1}\; y_{1}\; y_{2}\;t]$. Dem Unternehmen ist es möglich, unter Beachtung der vorausgehenden Parameter, entweder eine Anfrage $j_{AA}=1$ oder $j_{AA}=2$ mittels der Kapazität von $c_{1}$ anzunehmen (Auftragsannahme) oder eine Anfrage $j_{LE}=1$ mittels des Lagerbestands $y_{1}$ zu erfüllen (Lagerentnahme). Des Weiteren ist das Eintreffen keiner Anfrage zum Zeitpunkt $t$ möglich (Kein Auftrag). Damit sind die in dem Graphen aus der Abbildung \ref{B3} möglichen Systemzustände (Knoten) und Übergänge (Kanten) möglich.

\begin{table}
\begin{footnotesize}
    \caption{Ergebnistabelle für das beispielhafte Netzwerk RM mit Möglichkeit der Lagerentnahme} \label{Tab3}
    \vspace*{3mm}
\csvautotabular{data/beispiel3.csv}
\begin{center}
      {\footnotesize \textbf{Legende:} AA='Auftragsannahme', LE='Lagerentnahme', KA='Kein Auftrag'} 
      \end{center}
\end{footnotesize}
\end{table}

Die Tabelle \ref{Tab3} zeigt für das Beispiel die berechneten Erwartungswerte des noch möglichen Ertrags für jeden Systemzustand. Ebenfalls ist die optimale Politik in Form des besten Auftrags $j^*$ mit zugehörigem Ausführungsmodus $m$ aufgeführt. Dabei beschreibt hier der Ausführungsmodus die Auftragsannahme (AA), Lagerentnahme (LE) oder ob ein Auftrag keine Annahme erhält (KA). Zusätzlich ist der Wert $r_{j}-OC_{j}$ für den besten Auftrag $j^*$ je Systemzustand angegeben. Mit diesen Werten lässt sich der optimale Pfad des Beispiels ermitteln: $[2\;1\;0\;4] \rightarrow_{j_{AA}=2} [1\;1\;0\;3] \rightarrow_{j_{AA}=2} [0\;1\;0\;2] \rightarrow_{j_{LE}=1} [0\;0\;0\;1]\rightarrow_{j=0} [0\;0\;0\;0]$. Auch hier gilt weiterhin, dass der optimale Pfad abhängig der tatsächlich eintreffenden Anfragen ist. Er kann nur aber bei strategischen Entscheidung herangezogen werden.

Durch die Modellerweiterung wird gezeigt, dass ein Lagerbestand an Produkten $j\in\mathcal{J}$ den Gesamtertrag des Unternehmens erhöhen kann. Dies erfolgt jedoch aufgrund des Mechanismus, dass ein Lagerbestand eine Kapazitätserhöhung für das Unternehmen bedeutet. Somit handelt es hier um eine andere Darstellung der Modellformulierung des Netzwerk RM mit verschiedenen Ausführungsmodi für die Produktanfragen $j\in\mathcal{J}$. %Eine abschließende Beurteilung der Modellerweiterung erfolgt mit Abschluss des nachfolgenden Abschnitts.

\subsection{Lagerentnahme- und Lagerproduktionsentscheidung}

Im vorhergehenden Abschnitt ist die Gleichung \eqref{DP} um die Eigenschaft der Lagerentnahme erweitert. Damit sind die Entscheidungen über die Annahme eines Auftrags via Kapazitäts- oder Lagerparameter möglich. Die nachfolgende Modellerweiterung soll die Gleichung \eqref{stock} mit der Entscheidung über die gewollte Ablehnung einer Anfrage erweitern, damit die Kapazitäten für die Lagererhöhung Verwendung finden. D. h. die Kapazitäten $\textbf{c}$ werden um den Ressourcenverbrauch $\textbf{a}_{j}$ reduziert, damit der Lagerbestand $\textbf{y}$ um den Parameter für die Lagerveränderung $\textbf{s}_{j}$ für ein Produkt $j$ erhöht werden kann. Die Modellformulierung lautet wie folgt:
\begin{alignat*}{2}
V(\textbf{c}, \textbf{y}, t) =\;& \sum_{j \in \mathcal{J}}p_{j}(t)\max[V(\textbf{c}, \textbf{y}, t-1),\\
&r_{j} + V(\textbf{c}-\textbf{a}_j, \textbf{y}, t-1),\\
&r_{j} + V(\textbf{c}, \textbf{y}-\textbf{s}_j, t-1),\\
&V(\textbf{c}-\textbf{a}_j, \textbf{y}+\textbf{s}_j, t-1)]\\
&+ p_{0}(t)V(\textbf{c}, \textbf{y}, t-1) \\
\end{alignat*}
\begin{equation}\label{storage}
\begin{alignat*}{2}
= \;& V(\textbf{c}, \textbf{y}, t-1)\\
&+ \sum_{j \in \mathcal{J}}p_{j}(t)[\max[r_{j} - V(\textbf{c}, \textbf{y}, t-1) + V(\textbf{c}-\textbf{a}_j, \textbf{y}, t-1),0] \\
&+ \max[r_{j} - V(\textbf{c}, \textbf{y}, t-1) + V(\textbf{c}, \textbf{y}-\textbf{s}_j, t-1),0]\\
&+ \max[V(\textbf{c}-\textbf{a}_j, \textbf{y}+\textbf{s}_j, t-1) - V(\textbf{c}, \textbf{y}, t-1) ,0]]\\
\end{alignat*}
\end{equation}

Neben der Entscheidungen über die Auftragsannahme mittels Kapazitätsinanspruchnahme (AA) und der Lagerentnahme eines Produkts (LE) ist in dieser Modellerweiterung der Gleichung \eqref{storage} die Produktion eines Produkts $j$ auf Lager $y_{j}$ möglich. Dafür wird der Parameter $s_{j}$ als Lagerveränderung interpretiert. Zu beachten ist jedoch, dass bei der Entscheidung der Lagerproduktion eines Produkts $j_{LP}$ kein Ertrag $r_{j_{LE}}$ erzielt wird. Des Weiteren wird ein Parameter für einen maximalen Lagerbestand $y_{j}^{max}$ für jedes Produkt $j\in\mathcal{J}$ definiert. Eine derartige Modellformulierung bringt jedoch eine spezielle Funktionsweise mit, wie nachfolgendes Beispiel verdeutlichen soll:
\begin{center}
$j = \{1, 2\}, \; h = \{1\}, \; r_{1} = 100, \; r_{2} = 200, \; \text{Startperiode } t=3$,
\end{center}
\[
    c_{1}=2, \;
    a_{11}=1, \;
     a_{12}=2, \;
     p_{1}(t)=\begin{pmatrix} 0.5\\ 0.5\\ 0.5  \end{pmatrix}, \;
     p_{2}(t)=\begin{pmatrix} 0.1\\ 0.1\\ 0.1  \end{pmatrix},
  \]
  \[
    \textbf{y}=\begin{pmatrix} 0 \\ 0 \end{pmatrix}, \;
    \textbf{y}^{max}=\begin{pmatrix} 2 \\ 1 \end{pmatrix}, \;
    \textbf{s}_1=\begin{pmatrix} 1 \\ 0 \end{pmatrix}, \;
     \textbf{s}_2=\begin{pmatrix} 0 \\ 1 \end{pmatrix} \;
  \]
\begin{figure}[h!]
  \begin{center}
    \includegraphics[width=130mm]{Bilder/Beispiel4.pdf}
    \caption{Darstellung des Entscheidungsbaums des Netzwerk RM mit Möglichkeit der Lagerentnahme und Lagerproduktion}  \label{B4}
    {\footnotesize \textbf{Legende:} Die Zahlen stehen für den Produktauftrag $j$, AA='Auftragsannahme', LE='Lagerentnahme', LP='Lagerproduktion', KA='Kein Auftrag'} 
  \end{center}
\end{figure}

Die Abbildung \ref{B4} mit dem Entscheidungsbaum für das Beispiel zeigt alle möglichen Systemzustände und Optionen. Ein Systemzustand ist definiert als Zeichenfolge $[c_{1}\; y_{1}\; y_{2}\;t]$. Es gelten die Grenzbedingungen \eqref{GB1} sowie \eqref{GB2} und
\begin{equation}\label{GB4}
     OP_{\textbf{c}, t}:=\left\{\begin{array}{lll} j_{AA}, & \text{für } r_{j_{AA}} - OC_{j_{AA}} \ge r_{j_{LE}} - OC_{j_{LE}}\\
         j_{LE}, & \text{für } r_{j_{AA}} - OC_{j_{AA}} < r_{j_{LE}} - OC_{j_{LE}}\\
         j_{LP}, & \text{sonst}\end{array}\right. .
\end{equation}
Sofern die Modellerweiterung in dieser Form definiert ist, wird niemals $OP_{\textbf{c}, t}:=j_{LP}$ gelten. Dies resultiert aus der Tatsache, dass sofern genügend Kapazitäten $\textbf{c}$ zur Produktion eines Produkts $j$ vorhanden sind, die Kapazitäten für die direkte Annahme der Produktanfrage $j$ verwendet werden. Die Entscheidung über die Produktion eines Produkts $j_{LP}$ ist für das Unternehmen nur dann sinnvoll, wenn keine Anfragen zum Zeitpunkt $t$ eintreffen. Die Tabelle \ref{Tab4} zeigt die berechneten Werte für das hier aufgeführte Beispiel.
\begin{table}
\begin{footnotesize}
    \caption{Ergebnistabelle für das beispielhafte Netzwerk RM mit Möglichkeit der Lagerentnahme und Lagerproduktion} \label{Tab4}
    \vspace*{3mm}
\csvautotabular{data/beispiel4.csv}
\begin{center}
      {\footnotesize \textbf{Legende:} AA='Auftragsannahme', LE='Lagerentnahme', LP='Lagerproduktion', KA='Kein Auftrag'} 
      \end{center}
\end{footnotesize}
\end{table}

\subsection{Aufarbeitung von Ressourcen}

Bei der nachfolgenden Modellerweiterung wird nicht mehr von einem Produktlager ausgegangen, sondern von einem Ressourcenlager. Damit existiert für jede Ressource $h\in\mathcal{H}$ ein Lagerbestand $y_h^{res}$. Die Obergrenze des Lagerbestands wird durch den Parameter $y_{h}^{max,rex}$ beschrieben. Die einzelnen Lagerbestände können als Vektor $\textbf{y}^{res}$ zusammengefasst werden. In der Modellerweiterung wird davon ausgegangen, dass Kapazitäten $c_{h}$ einer Ressource $h$ verwendet werden können, damit der Lagerbestand an Ressourcen $y_{h}$ erhöht wird. Es handelt sich damit um eine Art der Vorarbeit der Leistung. Die einzelnen Bestandteile des Produkts werden auf Lager gelegt, damit diese zu einem späteren Zeitpunkt Verwendung finden. In dieser Modellerweiterung wird davon ausgegangen, dass ein komplettes Bündel des Produkts $j$ auf Lager gelegt wird. Daher kann der Parameter $\textbf{a}_{j}$ als Ressourcenveränderung angesehen werden. Das Modell schichtet damit die Kapazität der Ressourcen $h\in\mathcal{H}$ vom Parameter $c_{h}$ zu dem Lagerbestand $y_{h}$ um. Das mathematische Modell lässt sich wie folgt beschreiben:

\begin{alignat*}{2}
V(\textbf{c}, \textbf{y}^{res}, t) =\;& \sum_{j \in \mathcal{J}}p_{j}(t)\max[V(\textbf{c}, \textbf{y}^{res}, t-1),\\
&r_{j} + V(\textbf{c}-\textbf{a}_j, \textbf{y}^{res}, t-1),\\
&r_{j} + V(\textbf{c}, \textbf{y}^{res}-\textbf{a}_j, t-1),\\
&V(\textbf{c}-\textbf{a}_j, \textbf{y}^{res}+\textbf{a}_j, t-1)]\\
&+ p_{0}(t)V(\textbf{c}, \textbf{y}^{res}, t-1) \\
\end{alignat*}
\begin{equation}\label{workup}
\begin{alignat*}{2}
= \;& V(\textbf{c}, \textbf{y}^{res}, t-1)\\
&+ \sum_{j \in \mathcal{J}}p_{j}(t)[\max[r_{j} - V(\textbf{c}, \textbf{y}^{res}, t-1) + V(\textbf{c}-\textbf{a}_j, \textbf{y}^{res}, t-1),0] \\
&+ \max[r_{j} - V(\textbf{c}, \textbf{y}^{res}, t-1) + V(\textbf{c}, \textbf{y}^{res}-\textbf{a}_j, t-1),0]\\
&+ \max[V(\textbf{c}-\textbf{a}_j, \textbf{y}^{res}+\textbf{a}_j, t-1) - V(\textbf{c}, \textbf{y}^{res}, t-1) ,0]]\\
\end{alignat*}
\end{equation}

Weiterhin gelten die Grenzbedingungen \eqref{GB1} sowie \eqref{GB2} und Gleichung \eqref{GB4}. Bei dieser Modellformulierung der Gleichung \eqref{workup} wird die Einschränkung aufgehoben, bei der die Anzahl an notwendigen Ressourcen die Entscheidung der Lagerproduktion dominiert. Dies wird durch das nachfolgende Beispiel verdeutlicht:
\begin{center}
$j = \{1, 2\}, \; h = \{1\}, \; r_{1} = 100, \; r_{2} = 5000, \; \text{Startperiode } t=3$,
\end{center}
\[
    c_{1}=2, \;
    a_{11}=1, \;
     a_{12}=7, \;
     p_{1}(t)=\begin{pmatrix} 0.5\\ 0.5\\ 0.5  \end{pmatrix}, \;
     p_{2}(t)=\begin{pmatrix} 0.1\\ 0.1\\ 0.1  \end{pmatrix},
  \]
  \[
    y_{1}^{res}= 5, \;
    y^{max,res}=7
      \]
\begin{figure}[h!]
  \begin{center}
    \includegraphics[width=130mm]{Bilder/Beispiel5.pdf}
    \caption{Darstellung des Entscheidungsbaums des Netzwerk RM mit Möglichkeit der Aufarbeitung als Lagerproduktion}  \label{B5}
    {\footnotesize \textbf{Legende:} Die Zahlen stehen für den Produktauftrag $j$, AA='Auftragsannahme', LE='Lagerentnahme', LP='Lagerproduktion', KA='Kein Auftrag'} 
  \end{center}
\end{figure}

Abbildung \ref{B5} zeigt den Entscheidungsbaum für das aufgeführte Beispiel. Die Systemzustände sind beschrieben als Zahlenfolge $[c_{1}\; y_{1}\;t]$, da bei diesem Beispiel nur eine Ressource $h$ existiert. Für die Ressource $h=1$ beträgt die Kapazität $c_{1}=2$ und der Lagerbestand $y_{1}^{res}=5$. Als Obergrenze für den Lagerbestand wird $y_{1}^{max,res}$ festgelegt. Aufgrund der Ressourcenveränderungsparameter $\textbf{a}_{j}$ für jedes Produkt $j\in\mathcal{J}$ gibt es die Entscheidungsmöglichkeit Auftragsannahme (AA), Lagerentnahme (LE) sowie Lagerproduktion aufgrund von Aufarbeitung (LP) für jedes Produkt $j\in\mathcal{J}$. Weiterhin ist die Option zu jedem Systemzustand möglich, dass keine Anfrage eintrifft und der Wechsel in die nächste Periode ohne Kapazitäts- oder Lagerreduktion erfolgt.

Aufgrund der der Restriktionen der Kapazität $c_{1}$ und des Lagerbestands $y_{1}^{res}$ sind nicht alle möglichen Entscheidungsalternativen zu jedem Systemzustand möglich. Beispielsweise ist im Systemzustand $[2\;5\;3]$ die AA, LE sowie LP zur Annahme einer Anfrage nach Produkt $j=2$ nicht möglich. Jedoch kann in diesem Systemzustand die Annahme einer Anfrage nach Produkt $j=1$ erfolgen, entweder durch Kapazitäts- oder Lagerreduktion. Um eine Anfrage nach Produkt $j=2$ akzeptieren zu können, werden aufgrund des Parameters $a_{12}$ 7 Einheiten von der Ressource $h=1$ benötigt. Entweder müssen genügen Kapazitäten vorhanden sein, was jedoch im Beispiel nicht gegeben ist, oder der Lagerbestand $y_{1}^{res}$ muss auf 7 Einheiten der Ressource $h=1$ erhöht werden. Sofern die Erträge $r_{1}=100$ und $r_{2}=5000$ für das Beispiel angenommen sind, erhalten wir die berechneten Werte der Tabelle \ref{Tab5}.
\begin{table}
\begin{footnotesize}
    \caption{Ergebnistabelle für das beispielhafte Netzwerk RM mit Möglichkeit der Aufarbeitung als Lagerproduktion} \label{Tab5}
    \vspace*{3mm}
\csvautotabular{data/beispiel5.csv}
    \begin{center}
      {\footnotesize \textbf{Legende:} AA='Auftragsannahme', LE='Lagerentnahme', LP='Lagerproduktion', KA='Kein Auftrag'} 
      \end{center}
\end{footnotesize}
\end{table}

Ist der Ertrag $r_{j}$ eines Produkts $j\in\mathcal{J}$ ausreichend groß bei gegebener Eintrittswahrscheinlichkeit $p_{j}(t)$ zum Zeitpunkt $t$, dann erfolgt in dem Modell der Gleichung \eqref{workup} eine Überführung der Ressourcenkapazität $c_{h}$ hin zu dem Lagerbestand $y_{h}^{res}$, sofern dies eine andere Ressourceninanspruchnahme $a_{j}$ ermöglicht. Das Beispiel zeigt eine solche optimale Politik.

Für den anfänglichen Systemzustand $[2\;5\;3]$ existieren die Systemübergänge $j_{LE}$, $j_{LP}$ und $j_{AA}$ für Anfragen nach Produkt $j=1$. Außerdem ist das Eintreffen keiner Anfrage möglich. Damit lässt sich in diesem Systemzustand ein Ertrag $r_{j}=100$ generieren. Mit der Entscheidung $j_{LE}=1$ gelangt das System zum Zustand $[2\;4\;2]$, da für eine Annahme einer Anfrage nach Produkt $j=1$ der Lagerbestand $y_{1}^{res}$ reduziert wird. Sofern die Entscheidung $j_{AA}=1$ eintrifft, gelangt das System durch Inanspruchnahme der Kapazität $c_{1}$ zum Zustand $[1\;5\;2]$. Außerdem ist der Systemwechsel zum Zustand $[1\;6\;2]$ durch $j_{LP}=1$ möglich. Aufgrund der berechneten OK ist unter Beachtung der Parameter ein Systemwechsel durch die Lagerproduktion optimal (Vgl. Tabelle \ref{Tab5}). Abbildung \ref{B5} zeigt diesen und alle anderen optimalen Politiken für jeden Systemzustand. Daraus lässt dich die beste Pfad ermitteln:  $[2\;5\;3] \rightarrow_{j_{LP}=1} [1\;6\;2] \rightarrow_{j_{LP}=1} [0\;7\;1] \rightarrow_{j_{LE}=2} [0\;0\;0]$.

Es ist damit gezeigt, dass die Kapazität $c_{1}$ der Ressource $h=1$ durch den Ausführungsparameter $a_{1}$ des Produkts $j$ verwendet wird, damit der Lagerbestand $y_{1}^{res}$ jener Ressource $h=1$ erhöht wird. Damit ist zu einem späteren Zeitpunkt $t$ die Annahme der Anfrage nach Produkt $j=2$ durch die Lagerentnahme (LE) des Lagerbestands $y_{1}^{res}$ der Ressource $h=1$ möglich. Damit ist ein Gesamtertrag von $5000$ GE generiert. 

\subsection{Erneuerung von Ressourcen innerhalb des Buchungshorizonts}

Wie die vergangenen Modellerweiterungen zeigen, erweitert der Lagerbestand $\textbf{y}$ bzw. $\textbf{y}^{res}$ die Kapazitäten des Netzwerks. Sofern es sich um ein Ressourcenlager $\textbf{y}^{res}$ handelt, werden die notwendigen Kapazitäten $c_{h}$ der Ressourcen $h\in\mathcal{H}$ für ein Produkt $j$ mit niedrigem Ertrag auf den Lagerbestandsparameter $\textbf{y}^{res}$ umverteilt, damit die Akzeptanz eine nachfolgende Anfrage nach einem anderweitigen Produkt $j\in\mathcal{J}$ mit höherem Ertrag $r_{j}$ über den Lagerparameter $\textbf{y}^{res}$ erfolgen kann. Damit ist gezeigt, dass mithilfe der Modellerweiterung der \textit{Bellman'schen Funktionsgleichung} für das Netzwerk RM die optimale Politik ist, Anfragen mit niedrigem Ertrag zugunsten zukünftiger Anfragen mit höherem Ertrag abzulehnen. Mit dieser Ablehnung geht die Kapazitätserhöhung des Lagerbestands $\textbf{y}^{res}$ mit den dafür notwendigen Ressourcen $h\in\mathcal{H}$ einher, die für die Produktanfrage $j$ mit hohen Ertrag $r_j$ notwendig ist.

Wie im vorherigen Abschnitt gezeigt, müssen jedoch gewisse Rahmenbedingungen für eine Ablehnung von Produktanfragen $j\in\mathcal{J}$ mit niedrigem Ertrag gegeben sein. Die Kapazitäten $\textbf{c}$ des Netzwerks müssen einerseits ausreichend groß sein, damit Anfragen nach dem Produkt $j$ mit niedrigem Ertrag $r_{j}$ abgelehnt und auf das Lager $\textbf{y}^{res}$ umverteilt werden können, aber anderseits niedrigen als der notwendige Bedarf $\textbf{a}_{j}$ für das Produkt $j$ mit höherem Ertrag. Zusätzlich muss die Differenz zwischen den vorhanden Kapazitäten $\textbf{c}$ und der notwendigen Kapazitäten $\textbf{a}_{j}$ für die Produktanfrage $j$ mit hohem Ertrag auf dem Lager $\textbf{y}^{res}$ vorhanden sein. Wären ausreichend Kapazitäten $\textbf{c}$ für beide Arten der Produktanfragen $j\in\mathcal{J}$ möglich, dann wäre eine Umverteilung auf das Lager aufgrund der Gleichung \eqref{GB4} unnötig. Die notwendigen Kapazitäten $c_{h}$ der Ressourcen $h\in\mathcal{H}$ würden direkt für die Anfrage nach dem Produkt $j$ mit hohem Ertrag Verwendung finden und eingeplant. Die optimale Politik wäre das Produkt $j$ mit hohem Ertrag $r_{j}$ frühzeitig einzuplanen, sofern Anfragen für das Produkt unter Beachtung der Wahrscheinlichkeiten $p_{j}(t)$ zu den Zeitpunkten $t\in T$ eintreffen. Anders formuliert, die Option der Annahme einer Anfrage nach dem Produkt $j$ mit hohem Ertrag wäre im Systemzustand mit ausreichendem Kapazitäten $c_{h}$ zum Zeitpunkt $t$ möglich und, sofern der Ertrag aufgrund der OK des Netzwerks nicht zu stark beeinflussen wird, optimal.

Die Aufhebung der vorhergehenden Restriktion erfolgt durch das Einführen von regenerativen Ressourcen $h\in\mathcal{H}$ innerhalb des Buchungshorizonts $T$. Erst durch die Modellerweiterung der regenerativen Ressourcen $h\in\mathcal{H}$ erhält die Modellformulierung unter Beachtung einer Lagerhaltung von Ressourcen $\textbf{y}^{res}$ innhalb des Buchungshorizonts $T$ eine Sinnhaftigkeit. Für die Modellerweiterung wird der Parameter $\tilde{t}\in\tilde{T}$ eingeführt, wobei $\tilde{T}\subset T$ gilt. Der Parameter $\tilde{t}$ zeigt den Zeitpunkt der Regeneration der Ressourcen $h\in\mathcal{H}$ an. Durch die Regeneration erhöht sich der Wert des Kapazitätsparameters auf $\textbf{c}^{max}$. Die \textit{Bellman'schen Funktionsgleichung} für ein Netzwerk RM mit regenerativen Ressourcen innerhalb des Buchungshorizonts wird wie folgt definiert:

\begin{equation}\label{reg}
     V^{reg}(\textbf{c}, \textbf{y}^{res}, t)=\left\{\begin{array}{ll} V(\textbf{c}, \textbf{y}^{res}, t), & \forall t\neq\tilde{t}\\
         V(\textbf{c}^{max}, \textbf{y}^{res}, t), &\forall t=\tilde{t}\end{array}\right. .
\end{equation}

Es gelten weiterhin Gleichungen \eqref{GB1}, \eqref{GB2} und \eqref{GB4}. Dabei erfolgt die Ermittlung der Erwartungswerte aus der Gleichung \eqref{reg} abhängig des betrachteten Zeitpunkts $t$. Die Berechnung des Erwartungswerts $V(\textbf{c}, \textbf{y}^{res}, t)$ $\forall t\neq\tilde{t}$ erfolgt wie in Gleichung \eqref{workup} und die Berechnung des Erwartungswerts $V(\textbf{c}^{max}, \textbf{y}^{res}, t)$ $\forall t=\tilde{t}$ wie nachfolgende Gleichung zeigt:
\begin{alignat*}{2}
 V(\textbf{c}^{max}, \textbf{y}^{res}, t) = \;& V(\textbf{c}^{max}, \textbf{y}^{res}, t-1)\\
&+ \sum_{j \in \mathcal{J}}p_{j}(t)[\max[r_{j} - V(\textbf{c}^{max}, \textbf{y}^{res}, t-1)\\
&+ V(\textbf{c}^{max}-\textbf{a}_j, \textbf{y}^{res}, t-1),0] \\
&+ \max[r_{j} - V(\textbf{c}^{max}, \textbf{y}^{res}, t-1) + V(\textbf{c}^{max}, \textbf{y}^{res}-\textbf{a}_j, t-1),0]\\
&+ \max[V(\textbf{c}^{max}-\textbf{a}_j, \textbf{y}^{res}+\textbf{a}_j, t-1) - V(\textbf{c}^{max}, \textbf{y}^{res}, t-1) ,0]]\\
\end{alignat*}

Die Funktionsweise des Modells soll an einem Beispiel verdeutlicht werden:
\begin{center}
$j = \{1, 2\}, \; h = \{1\}, \; r_{1} = 100, \; r_{2} = 5000, \; \text{Startperiode } t=4, \; \tilde{t}=\{2\} $,
\end{center}
\[
    c_{1}=1, \;
    a_{11}=1, \;
     a_{12}=2, \;
     p_{1}(t)=\begin{pmatrix} 0.5\\ 0.5\\ 0.5\\ 0.5  \end{pmatrix}, \;
     p_{2}(t)=\begin{pmatrix} 0.1\\ 0.1\\ 0.1\\ 0.1  \end{pmatrix},
  \]
  \[
    y_{1}^{res}= 0, \;
    y^{max,res}=2
      \]
      
Es existieren zwei Produkte $j\in\mathcal{J}$ und eine Ressource $h\in\mathcal{H}$. Sofern eine Anfrage nach dem Produkt $j=1$ akzeptiert wird, generiert sich ein Ertrag in Höhe von $r_1=100$ GE. Bei Akzeptanz einer Produktanfrage $j=2$ erzielt das Unternehmen einen Ertrag von $r_1=5000$ GE. Beide Produktanfragen benötigen die Ressource $h=1$, wobei die Kapazität der Ressource bei $c_1=1$ liegt. Ein Anfrage nach Produkt $j=1$ benötigt $a_11=1$ Kapazitäten und eine Anfrage nach Produkt $j=1$ benötigt $a_12=2$. Der Buchungshorizont entspricht $T=4$ Perioden und zum Zeitpunkt $\tilde{t}=2$ erfolgt eine Regeneration der Ressourcen. Die Wahrscheinlichkeit des Eintreffens einer Anfragen zum Zeitpunkt $t$ entspricht $ p_{1}(t)=(0.5, 0.5, 0.5, 0.5)$ bzw. $ p_{2}(t)=(0.1, 0.1, 0.1, 0.1)$. Die Lagerkapazität $y^{max,res}$ ist auf 2 Einheiten beschränkt und es existiert zur Starperiode $t=4$ kein Anfangsbestand ($y^{res}=0$). Abbildung \ref{B6} zeigt alle möglichen Systemzustände mit den einzelnen Übergängen für das Beispiel. 

\begin{figure}[h!]
  \begin{center}
    \includegraphics[width=140mm]{Bilder/Beispiel6.pdf}
    \caption{Darstellung des Entscheidungsbaums des Netzwerk RM mit regenerativen Ressourcen}  \label{B6}
    {\footnotesize \textbf{Legende:} Die Zahlen stehen für den Produktauftrag $j$, AA='Auftragsannahme', LE='Lagerentnahme', LP='Lagerproduktion', KA='Kein Auftrag', $\cdots$='Anfrage ablehnen'} 
  \end{center}
\end{figure}

Ein Systemzustand im vorausgehenden Beispiel ist definiert als Zahlenfolge $[c_1\;y^{res}_1\;t]$. Zu beachten ist, dass in diesem Netzwerk zum Zeitpunkt $t=2$ die Ressourcen erneuert sind und dementsprechend für die Systemzustände die Zahlenfolge $[c_1^{max}\;y^{res}_1\;2]$ gilt. Wird das Gesamtnetzwerks in Abbildung \ref{B6} betrachtet, dann ergibt sich der bester Pfad $[1\;0\;4] \rightarrow_{j_{LP}=1} [0\;1\;3] \rightarrow_{KA} [1\;1\;2] \rightarrow_{j_{LP}=1} [0\;2\;1] \rightarrow_{j_{LE}=2} [0\;0\;0]$ und ein Gesamtertrag in Höhe von $5000$ GE. Damit ist für das hier betrachtete Netzwerk die Lagerproduktion (LP) der Ressource $h=1$ optimal, damit eine Anfrage nach einem Produkt $j=2$ akzeptiert werden kann. Des Weiteren ist unter diesen Bedingungen im Systemzustand $[1\;0\;4]$ die Ablehnung der Anfrage nach einem Produkt $j=1$ die beste Politik, sofern nur diese Anfrage zum Zeitpunkt $t=4$ eintrifft. Durch diese Entscheidung gelangt das Netzwerk zum Systemzustand $[1\;0\;3]$, bei dem in weiteren Verlauf das Erreichen des Systemzustands $[1\;1\;2]$ möglich ist, bei dem die Kapazitäten regeneriert sind und der Lagerbestand $y_1^{res}$ eine Einheit der Ressource $h=1$ beinhaltet. Von diesem Systemzustand ist ein weiterer Verlauf möglich, bei dem die Anfrage nach einem Produkt $j=2$ möglich ist.

Wird im Gegensatz im Systemzustand $[1\;0\;4]$ die Anfrage nach dem Produkt $j=1$ angenommen, dann gelangt das Netzwerk in den Systemzustand $[0\;0\;3]$ und erzielt einen Ertrag in Höhe von 100 GE. Vom Systemzustand $[0\;0\;3]$ ist keine Akzeptanz mehr möglich, da keine Kapazität und kein Lagerbestand vorhanden sind. Das Netzwerk gelangt dann in den Systemzustand $[1\;0\;2]$ mit einem regenerierten Kapazitätsbestand $c_1=1$. Im weiteren Verlauf ist nur noch die Akzeptanz einer Anfrage nach Produkt $j=1$ möglich. Damit ist ein Gesamtertrag in Höhe von 200 GE generiert. Damit verstößt die Annahme der Anfrage $j=1$ im Systemzustand $[1\;0\;4]$ gegen die Bedingung \eqref{OC}.

Damit ist gezeigt, dass die Umverlagerung der Kapazitäten hin zu einem Lagerbestand das Ergebnis bzw. den Gesamtertrag verbessern kann. Die ermittelten Werte des Beispiels ist in der Tabelle \ref{Tab6} aufgeführt.

\begin{table}
\begin{footnotesize}
    \caption{Ergebnistabelle für das beispielhafte Netzwerk RM mit regenerativen Ressourcen} \label{Tab6}
    \vspace*{3mm}
\csvautotabular{data/beispiel6.csv}
    \begin{center}
      {\footnotesize \textbf{Legende:} AA='Auftragsannahme', LE='Lagerentnahme', LP='Lagerproduktion', KA='Kein Auftrag'} 
      \end{center}
\end{footnotesize}
\end{table}

Zu beachten ist jedoch, dass durch das Betrachtung der regenerierten Ressourcen als Gesamtkapazität des Netzwerks ein gleiches Ergebnis unter einer anderer Betrachtungsweise ergibt. Es gilt $\textbf{c}=(1+|\tilde{t}|)\cdot c^{max}$, dann ergeben sich die optimalen Politiken aus Tabelle ... . Da jedoch von Anfang an genügen Kapazitäten zur Annahme alle Produktanfragen $j\in\mathcal{J}$ möglich sind, ist die Lagerproduktion nicht optimale Politik sofern eine beliebige Anfrage eintrifft. Die Auftragsannahme (AA) dominiert die Lagerproduktion (LP). Die Lagerentnahme wiederum kann dann optimale Politik werden, wenn keine Anfragen eintreffen und eine Umlagerung der Kapazitäten hin zum Lager erfolgen muss. Die Betrachtung von regenerativen Ressourcen $h\in\mathcal{H}$ im Buchungshorizont $T$ hat nur zur Vereinfachung des Netzwerks einen Nutzen.

\section{Implementierung mittels IPython Notebook}

\section{Numerische Untersuchung}