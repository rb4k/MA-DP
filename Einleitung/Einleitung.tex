\chapter{Einleitung}
\markboth{1 Einleitung}{}

\section{Problemstellung}
%Unternehmen müssen aufgrund der heutigen dynamischen Veränderungen der Marktgegebenheiten die schnelle Entwicklung ihrer Produkte vorantreiben, sowie deren Platzierung und Bepreisung optimieren.\footnote{Vgl. \cite{gonsch2013using}, S. 94-95\label{dingens1}}
%Das Revenue Management (RM) ist daher für viele Dienstleistungsbereiche unabdinglich geworden. Durch dieses Managementkonzept und den daraus resultierenden Planungsinstrumenten können bei speziellen Anwendungsgebieten die Kapazitätsauslastungen optimiert werden. Damit lassen sich für Unternehmen mit Kapazitätsbe\-schränkungen die Absatzmärkte erfolgreich abschöpfen und die Gesamterlöse maximieren. In dieser Arbeit wird der Ansatz des Verkaufs von opaken Produkten im RM zur besseren Kapazitätsauslastung thematisiert. %\footnote{Klassische Ansätze des RM beschäftigen sich mit der Optimierung der Kapazitätsauslastung bei der Erstellung von Produkten oder Dienstleistungen.}
%Bei opaken Produkten handelt es sich um Produkte, bei denen der Anbieter einige Eigenschaften des Produkts bis zum Abschluss der Transaktion verbirgt.\footref{dingens1} %Anders als bei spezifischen Produkten, bei dem sämtliche Bestandteile der Produkteigenschaften vor dem Kauf bekannt sind.
%Die vorgestellten Produktarten werden i. d. R. in der Dienstleistungsindustrie von multiplen Anbietern angeboten.\footnote{Vgl. \cite{Klein:2008aa}, S. 4} Vereinfacht gesagt bedeutet dies, dass diese Anbieter ihre Dienstleistungen über eine Mehrmarkenstrategie distribuieren und ggf. ihr eigenes Produktportfolio mit anderen Dienstleistungen von Partnerfirmen erweitern. Sie bündeln aus dieser Menge an verfügbaren unterschiedlichen Dienstleistungen ein neues zusammenhängendes Produkt und vertreiben diese Art von Produkten über eine einzelne Marke an potentielle Konsumenten. Aus einem solchen Netzwerk an verfügbaren Produkten kann der Anbieter opake Produkte formen.\footnote{Vgl. \cite{anderson2009effectiveness}, S. 307-309}\textsuperscript{,}\footnote{Siehe Kapitel \ref{DP}}\\

Die Entscheidung über die Annahme von Kundenaufträgen zur Instandsetzung von Gütern ist von zentrale Bedeutung für Reparaturdienstleister. Abhängig des Kundenauftrags, indem der Zustand des Gutes beschrieben ist sowie die für die Reparatur notwendigen Prozessschritte beschrieben sind, generiert der Dienstleister unterschiedliche Erträge. Die jeweiligen Prozessschritte zur Instandsetzung des Gutes geben zusätzlich den notwendigen Ressourcenbedarf für die auszuführende Dienstleistung an, die notwendig ist um das Gut in seinen ursprünglichen bzw. geforderten Zustand zurückzuversetzen. Ressourcen zur Instandsetzung von Gütern können z. B. Material oder Personalstunden sein. Abhängig des möglichen Ertrags und des für den Auftrag notwendigen Ressourcenbedarf muss der Reparaturdienstleister die Entscheidung über Annahme oder Ablehnung des Kundenauftrags treffen. Sofern nur der einfache Fall betrachtet wird, bei dem nur der einzelne Kundenauftrag zur Auswahl steht, ist die Entscheidung für den Dienstleister einfach getroffen. Der Kundenauftrag wird angenommen, sofern der Aufwand des Ressourceneinsatzes niedriger als der erziele Umsatz ist (sofern von einer Vollkostenrechnung ausgegangen wird). Sofern der Reparaturdienstleister eine begrenzte Ressourcenkapazität zur Instandsetzung der Güter besitzt, muss zusätzlich der absolute Ressourcenverbrauch des Auftrags für Annahmeentscheidung geprüft werden. Mit Annahme des Auftrags ist ein individueller Ertrag erzielt und ein auftragsbezogener Ressourcenverbrauch eingetreten. Nachdem diese Entscheidung getroffen ist, wird der zeitlich darauffolgenden Kundenauftrag betrachtet.

Für die Entscheidung über die Annahme oder Ablehnung eines Kundenauftrags (KA) zur Instandhaltung von Gütern bedarf es einer umfassenderen Betrachtung als nur die kurzsichtige Entscheidung über einen einzelnen Auftrag. Angenommen ein Reparaturdienstleister besitzt ein bestimmtes Kontingent an unterschiedlichen Ressourcen über einen bestimmten Zeitraum zur Erfüllung seiner angebotenen Dienstleistung. In diesem betrachteten Zeitraum treffen jetzt unterschiedliche Kundenaufträge mit unterschiedlicher Wertigkeit ein. Zur Maximierung seiner Erträge über diesen Zeitraum unter Beachtung der vorhanden Ressourcenkapazität kann es sinnvoll sein, Anfragen mit niedrigem Ertrag abzulehnen, sofern im weiteren Verlauf des betrachteten Zeitraums Aufträge mit höherem Ertrag eintreffen.

Bei der Problemformulierung der Auftragsannahmeentscheidung bei auftragsbezogenen Instandhaltungsprozessen handelt es sich um ein stochastisch-dynamisches Optimierungsmodell. Eine mögliche grafische Darstellungsform dieser Problemstellung erfolgt als sogenannter Entscheidungsbaum. Bei dem Entscheidungsbaum handelt es sich um ein gerichteten azyklischen Graphen. Abhängig der eintreffenden Anfragen, der vorhandenen Kapazitäten, der möglichen Entscheidungen und des betrachteten Zeithorizont hat der Entscheidungsbaum eine unterschiedliche Anzahl an Kanten und Knoten. Ein Knoten bildet jeweils einen neuen Zustand des Systems ab und eine Kante die mögliche Entscheidung in einen neuen Systemzustand zu gelangen.

In dieser Arbeit wird aber zusätzlich zur Annahmeentscheidung eines Auftrags zur Instandhaltung eines Gutes auch die Entscheidung über eine mögliche Lagerhaltung von Gütern getroffen. Durch Annahme der Entscheidung zur Lagerhaltung des defekten Gutes wird dieses in die Lagerhaltung des Reparaturdienstleisters übernommen und durch ein bereits repariertes Gut ausgetauscht. Das reparierte Gut entspricht den vom jeweiligen Auftrag geforderten Instandhaltungszustand. Anders formuliert bedeutet dies, dass ein Reparaturdienstleister nicht nur die Entscheidungsmöglichkeit über die Instandsetzung des Gutes hat, sonder auch die Möglichkeit hat in Abhängigkeit des verfügbaren Lagerbestandes bereits reparierte Güter zur Befriedigung der Kundenaufträge zu verwenden.

Das stochastisch-dynamische Optimierungsmodell muss demnach die Entscheidung treffen, ob die Auftragsannahme zur Instandsetzung des Gutes, die Lagerhaltung des defekten Gutes sowie die Herausgabe eines bereits reparierten Gutes oder die Ablehnung des Kundenauftrags erfolgen soll. Diese Entscheidung erfolgt in Abhängigkeit der verfügbaren restlichen Ressourcenkapazität, die zur Instandsetzung der Güter notwendig ist, des aktuell-vorhandenen Lagerbestandes der bereits reparierten Güter und der noch potentiell eintreffenden Anfragen zur Instandhaltung von Gütern.

\section{Zielsetzung}

Die mathematische Darstellung des Optimierungsmodells der Auftragsannahme- und Lagerhaltungsentscheidungen bei auftragsbezogenen Instandhaltungsprozessen erfolgt als \textit{dynamische Programmierung (DP}).
Die Aufgabe des DP ist laut \cite{talluri2004theory} die Entscheidungsfindung zu unterstützen, damit der Gesamtertrag des Dienstleisters maximiert wird. Der Gesamtertrag wird bewertet in Geldeinheiten (GE). 
Das Grundmodell zur Annahme von Kundenaufträgen bzw. -anfragen kommt aus der wissenschaftlichen Betrachtung des Revenue Management von Dienstleistungsunternehmen mit beschränkten Ressourcenkapazitäten.
%Bei den betrachteten Ressourcen handelt es sich um zeitgebundene Ressourcen mit einer jeweiligen vordefinierten Ressourcenkapazität über den gesamten Zeithorizont. 

Die Zielsetzung der Arbeit ist demnach das Grundmodell des Revenue Managements zur Annahme von Kundenaufträgen mit der Möglichkeit der Entscheidung der Lagerhaltung der Güter zu erweitern.


\section{Aufbau der Arbeit}

Der Aufbau der Arbeit ist demnach,

die Grundlagen über auftragsbezogenen Instandhaltungsprozessen zu definieren,

das Konzept des Revenue Management bei der Annahme von Kundenaufträgen vorzustellen,

bestehende Ansätze zum Lösen des stochastisch-dynamischen Optimierungsmodells bei der Annahme von Kundenaufträgen in der Auftragsproduktion und bei Instandhaltungsprozessen aufzuführen

sowie eine dynamische Programmierung für die Auftragsannahme- und Lagerhaltungsentscheidungen bei auftragsbezogenen Instandhaltungsprozessen zu formulieren und exakt zu lösen.