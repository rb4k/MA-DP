\chapter{Einleitung}
\markboth{1 Einleitung}{}

\section{Problemstellung}
Unternehmen müssen aufgrund der heutigen dynamischen Veränderungen der Marktgegebenheiten die schnelle Entwicklung ihrer Produkte vorantreiben, sowie deren Platzierung und Bepreisung optimieren.\footnote{Vgl. \cite{gonsch2013using}, S. 94-95\label{dingens1}} Das Revenue Management (RM) ist daher für viele Dienstleistungsbereiche unabdinglich geworden. Durch dieses Managementkonzept und den daraus resultierenden Planungsinstrumenten können bei speziellen Anwendungsgebieten die Kapazitätsauslastungen optimiert werden. Damit lassen sich für Unternehmen mit Kapazitätsbe\-schränkungen die Absatzmärkte erfolgreich abschöpfen und die Gesamterlöse maximieren. In dieser Arbeit wird der Ansatz des Verkaufs von opaken Produkten im RM zur besseren Kapazitätsauslastung thematisiert. %\footnote{Klassische Ansätze des RM beschäftigen sich mit der Optimierung der Kapazitätsauslastung bei der Erstellung von Produkten oder Dienstleistungen.}
Bei opaken Produkten handelt es sich um Produkte, bei denen der Anbieter einige Eigenschaften des Produkts bis zum Abschluss der Transaktion verbirgt.\footref{dingens1} %Anders als bei spezifischen Produkten, bei dem sämtliche Bestandteile der Produkteigenschaften vor dem Kauf bekannt sind.
Die vorgestellten Produktarten werden i. d. R. in der Dienstleistungsindustrie von multiplen Anbietern angeboten.\footnote{Vgl. \cite{Klein:2008aa}, S. 4} Vereinfacht gesagt bedeutet dies, dass diese Anbieter ihre Dienstleistungen über eine Mehrmarkenstrategie distribuieren und ggf. ihr eigenes Produktportfolio mit anderen Dienstleistungen von Partnerfirmen erweitern. Sie bündeln aus dieser Menge an verfügbaren unterschiedlichen Dienstleistungen ein neues zusammenhängendes Produkt und vertreiben diese Art von Produkten über eine einzelne Marke an potentielle Konsumenten. Aus einem solchen Netzwerk an verfügbaren Produkten kann der Anbieter opake Produkte formen.\footnote{Vgl. \cite{anderson2009effectiveness}, S. 307-309}\textsuperscript{,}\footnote{Siehe Kapitel \ref{DP}}\\


Aufgrund der dynamischen Entscheidungsfindung, welche Ressourcen in Abhängigkeit der verfügbaren Kapazitäten für das opake Produkt %zur Maximierung des Ertrags
Verwendung finden, erfolgt die Darstellung des Problems als \textit{dynamische Programmierung (DP}). %Das traditionelle Optimierungsmodell im RM muss daher erweitert werden, da dieses zur automatischen Kapazitätskontrolle unter diesen Bedingungen keine Anwendung mehr findet.\footnote{?????}
Die Aufgabe des DP ist laut \cite{talluri2004theory} die Entscheidungsfindung zu unterstützen, damit der Gesamtertrag des Anbieters maximiert wird. Der Gesamtertrag wird bewertet in Geldeinheiten (GE). Bei den betrachteten Ressourcen handelt es sich um zeitgebundene Ressourcen mit einer jeweiligen vordefinierten Ressourcenkapazität über den gesamten Zeithorizont. Die Zielsetzung der Arbeit ist, Grundlagen über Begriffe des RM mit opaken Produkten zu definieren, das Grundmodell der dynamischen Programmierung im RM unter Beachtung von opaken Produkten vorzustellen sowie anhand eines Beispiels den Ansatz zur Zerlegung des dynamischen Programms vorzustellen. Durch diesen Zerlegungsansatz wird die Problemstellung in Teilprobleme vereinfacht, die einen geringeren Rechenaufwand benötigen und dadurch leicht in praktische Problemstellungen überführt werden können.\footnote{Vgl. \cite{gonsch2013using}, S. 98}

\section{Zielsetzung}

\section{Aufbau der Arbeit}