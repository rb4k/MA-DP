% !TEX encoding = UTF-8 Unicode
\chapter{Numerische Untersuchung}
\markboth{5 Numerische Untersuchung}{}
\setcounter{footnote}{97}  %um durchgehende Fußnotennummerierung zu haben, hier die Anzahl der bisherigen Fußnoten eintragen

Für die numerische Untersuchung wurden mehrere Berechnungen der optimalen Politik für bestimmte Szenerien mit unterschiedlichen Eigenschaften und verschiedener Eintrittswahrscheinlichkeiten der Produktanfragen durchgeführt. Ziel ist es zu untersuchen, ob durch die Möglichkeit der Lagerhaltung eine Veränderung in der optimalen Politik resultiert und sofern es eine Veränderung gibt, in welcher Form diese eintritt. Die Ergebnisse der Berechnungen, die in dieser Arbeit vorgestellt sind, wurden mithilfe des Clustersystems an der Leibniz Universität Hannover berechnet. Tabelle \ref{Hardware} zeigt die verwendete Rechnersysteme, wobei eine Berechnung jeweils auf einem Knoten der verfügbaren Rechnersysteme durchgeführt wurde.

\begin{table}[h!]
\renewcommand{\arraystretch}{1.5}
  \begin{center}
  \begin{small}
    \caption{Überblick der verwendeten Hardware des Clustersystems an der Leibniz Universität Hannover}  \label{Hardware}
    \vspace*{3mm}
    \begin{tabular}{llp{6cm}p{1.5cm}p{1.5cm}}   %hier die Spaltenausrichtung, -breite, -begrenzung und -anzahl eintragen
     Cluster & Knoten  & Prozessoren & Kerne/ Knoten  & Speicher/ Knoten (GB) \\  \hline
  Tane   & 96 & 2x Intel Westmere-EP Xeon X5670 (6-cores, 2.93GHz, 12MB Cache, 95W)  & 12 & 48 \\
   Taurus  & 54 & 2x Intel Westmere-EP Xeon X5650 (6-cores, 2.67GHz, 12MB Cache, 95W)  & 12 &  48 \\
   SMP  & 9 &4x Intel Westmere-EX Xeon E7-4830 (8-cores, 2.13GHz, 24MB Cache, 105W)   & 32 & 256  \\
    & 9 & 4x Intel Backton Xeon E7540 (6-cores, 2.00GHz, 18MB Cache, 105W)   & 24 & 256 \\
      & 3 & 4x Intel Westmere-EX Xeon E7-4830 (8-cores, 2.13GHz, 24MB Cache, 105W)   & 32 & 1024  \\ \hline
    \end{tabular} \\[3mm]
    \end{small}
  %  {\footnotesize \textbf{In Anlehnung an:} \cite{quante2009management}, S. 44.}\\
        % {\footnotesize \textbf{Quelle:} \url{http://www.rrzn.uni-hannover.de/scientific_computing_doku.html} }   %footnotesize liefert Schrift in Größe 10pt
  \end{center}
\end{table}

Die in der Arbeit verwendete Implementierung des Auftragsannahmeproblems des Netzwerk RM mit Lagerhaltungsentscheidung sieht die Berechnung der möglichen Systemzustände und des maximal möglichen Erwartungswerts vor. Die Memofunktion wird gespeist durch die vorher berechneten Systemzustände und ist wiederum bei der Berechnung der Erwartungswerte für die tatsächlich benötigten Systemzustände notwendig (vgl. Kapitel \ref{Implementierung}). Bei der Berechnung der Erwartungswerte wird ein angepasstes DP des Netzwerk RM verwendet. Im Anhang ist der verwendete Quellcode hinterlegt. Zusätzlich erfolgt mithilfe der Implementierung, nachdem die Berechnung der Erwartungswerte vollzogen ist, die Übertragung und Speicherung der optimalen Politik für jeden Systemzustand in eine Datenbank. Aufgrund der zwei Berechnungsschritte werden für beide Berechnungen jeweils die ermittelten Daten bzw. Berechnungskennzahlen angegeben. Auf dem vorgestellten Rechnersystem wurden folgende Versionen der verwendeten Software-Pakete genutzt:

\colorbox{hellgrau}{\parbox{14cm}{\texttt{Python Version 2.7.9 (default, Jul  2 2015, 11:24:04) [GCC 4.9.2]\\
Numpy Version 1.9.2\\
Matplotlib Version 1.4.3\\
Pandas Version 0.16.2
}}}

Nachfolgend sind alle berechneten Szenarien dargestellt. Je Szenario gibt es eine Anzahl an Buchungsperioden $T$ und eine Anzahl an Leistungserstellungszeitpunkten $\hat T$. Die Leistungserstellungszeitpunkte $\hat t \in \hat T$ teilen sich dabei gleichmäßig über den Buchungshorizont $T$ auf. In dieser Arbeit wird nur eine Ressource $h\in\mathcal{H}$ betrachtet, was sich ebenfalls in den Szenarien widerspiegelt. Für einen jeden Leistungserstellungszeitpunkt $\hat t \in \hat T$ gibt es eine Anzahl an Kapazitäten $c_h^{\hat t}$, einen Lagerbestand $y_h^{\hat t}$ und einen maximal möglichen Lagerbestand $y_h^{{\hat t},max}$ der Ressource $h=1$. Die unterschiedlichen Produktanfragen $j\in\mathcal{J}$ haben dabei einen individuellen Ertrag $r_j$, einen Parameter für die Inanspruchnahme bzw. Bestandsveränderung der Kapazitäten $a_{hj}$ der Ressource $h=1$ und einen Zeitpunkt der Leistungserstellung $\hat t$. Die Szenarien besitzen für die Buchungsperioden $t\in T$ eine bestimmte Wahrscheinlichkeitsverteilung $p_j(t)$ über den Auftragseingang einer jeden Produktanfrage $j\in\mathcal{J}$. Die Verteilungen ist jeweils beim Auftrag $j$ angegeben und über alle Produktaufträge $j\in\mathcal{J}$ des jeweiligen Szenarios als grafischer Kurvenverlauf abgebildet.

%%%%%%%%%%%%%%%%%% DP_L_AB
\begin{table}[h!]
\renewcommand{\arraystretch}{1.5}
  \begin{center}
    \caption{Szenario 1}  \label{S1}
    \vspace*{3mm}
    \begin{tabular}{l l l l l l}   %hier die Spaltenausrichtung, -breite, -begrenzung und -anzahl eintragen
    $T$ & $\hat T$  & $h$ & $c_h^{\hat t}\forall \hat{t}\in{\hat T}$ & $y_h^{\hat t}\forall \hat{t}\in{\hat T}$  & $y_h^{{\hat t},max}\forall \hat{t}\in{\hat T}$  \\  \hline
100 & 5 & 1 & 1 & 0 & 2  \\ \hline
    \end{tabular} \\[3mm]
        \begin{tabular}{p{1cm} p{1cm} p{1cm}  p{1cm} p{6cm}}   %hier die Spaltenausrichtung, -breite, -begrenzung und -anzahl eintragen
    $j$ & $r_j$  & $a_{1j}$ & $\hat t$ & Verteilung $p_j(t)$ \\  \hline
1 & 100 & 1 & 1 & Linear verlaufend (0.09)   \\
2 & 1000 & 1 & 1 & Linear verlaufend (0.09)  \\
3 & 100 & 1 & 2 & Linear verlaufend (0.09)  \\
4 & 1000 & 1 & 2 & Linear verlaufend (0.09)  \\
5 & 100 & 1 & 3 & Linear verlaufend (0.09)  \\
6 & 1000 & 1 & 3 & Linear verlaufend (0.09)  \\
7 & 100 & 1 & 4 & Linear verlaufend (0.09)  \\
8 & 1000 & 1 & 4 & Linear verlaufend (0.09)  \\
9 & 100 & 1 & 5 & Linear verlaufend (0.09)  \\
10 & 1000 & 1 & 5 & Linear verlaufend (0.09)  \\ \hline
    \end{tabular} \\[3mm]
     \begin{tabular}{p{7cm}p{5cm}} \hline
     Rechenzeit Systemzustände (h): & \texttt{0:00:00.098259} \\
     Anzahl möglicher Systemzustände: & \texttt{785376} \\
     Anzahl benötigter Systemzustände: & \texttt{35324} \\ 
     Rechenzeit DP (h): & \texttt{6:39:09.675572} \\ 
          Max. Erwartungswert (GE): & \texttt{4998.52} \\ \hline
         \end{tabular} \\[3mm]
  \end{center}
\end{table}

\textbf{Szenario 1}

Im ersten Szenario werden über einen Buchungszeitraum $T=100$ insgesamt zehn Produktanfragen $j\in\mathcal{J}$ betrachtet. Jeweils zwei Produktanfragen $j$ sind für eine der fünf möglichen Leistungserstellungen $\hat t\in\hat T$ vorgesehen und die Anfragen wechseln sich dabei jeweils zwischen dem möglichen Erträgen $r_j=100$ und $r_j=1000$ ab. Die Wahrscheinlichkeitsverteilung einer jeden Produktanfrage verläuft dabei linear über den Buchungshorizont $T$ mit dem konstanten Wert $p_j(t)=0,09$, wobei mit Überschreitung des Leistungserstellungszeitpunkts $\hat t$ die Eintrittswahrscheinlichkeit der zugehörigen Produktanfrage $j$ für alle nachfolgenden Perioden $t$ auf $p_j(t)=0$ sinkt (vgl. Abbildung \ref{SB1}). Tabelle \ref{S1} zeigt die verwendeten Parameter des Szenarios und das Ergebnis der Berechnung.

\begin{figure}[h!]
  \begin{center}
    \includegraphics[width=80mm, trim=300pt 180pt 300pt 0pt]{/Users/Superuser/DP-RM-with-storage/cluster/DP_L_Ab/Wverteilung.png}
    \caption{Wahrscheinlichkeitsverteilung Szenario 1}  \label{SB1}
  \end{center}
\end{figure}

\begin{table}[h!]
\renewcommand{\arraystretch}{1.5}
  \begin{center}
    \caption{Auswertung des Szenarios 1}  \label{AS1}
    \vspace*{3mm}
    %%%%%%%%%%%%%%% d[0]
    \begin{tabular}{l l l l l l l l l l l l }  \hline 
         $j$ & 0 & 1  & 2 & 3 & 4  & 5 & 6 & 7 & 8 & 9 & 10  \\ \hline
$d_{0}$ &  24151 &  1168 &  0 &  2787 &  0 &  3226 &  0 &   3628 &  0 &  0 &  0 \\
$d_{0}$ (\%) &  69.08 &  3.34 &  0 &  7.97 &  0 &  9.23 &  0 &  10.38 &  0 &  0 &  0 \\
\hline
    \end{tabular} \\[3mm]
        \begin{tabular}{ l l l l l l l l l}   \hline    %%%%%%%%%%%%%%% d[0]
    $d_j$ & \multicolumn{2}{c}{Ablehnung (0)} & \multicolumn{2}{c}{Annahme (1)}  & \multicolumn{2}{c}{Lagerentnahme (2)} & \multicolumn{2}{c}{Lagerproduktion (3)}\\
    & Anz. & \% & Anz. & \% & Anz. & \% & Anz. & \% \\ \hline 
$d_{1}$  &  33752 &  96.54 &     40 &   0.11 &   0 &    0 &  1168 &   3.33 \\
$d_{2}$  &  33464 &  95.72 &   1175 &   3.35 &   0 &    0 &   321 &   0.92 \\
$d_{3}$  &  32019 &  91.59 &     44 &   0.13 &   0 &    0 &  2897 &   8.26 \\
$d_{4}$  &  29915 &  85.57 &   3066 &   8.75 &  1092 &   3.12 &   887 &   2.53 \\
$d_{5}$  &  31401 &  89.82 &     44 &   0.13 &   0 &    0 &  3515 &  10.03 \\
$d_{6}$  &  24944 &  71.35 &   6015 &  17.17 &  3272 &   9.34 &   729 &   2.08 \\
$d_{7}$  &  30819 &  88.16 &     30 &   0.09 &   0 &    0 &  4111 &  11.73 \\
$d_{8}$  &  17557 &  50.22 &  10109 &  28.87 &  6966 &  19.88 &   328 &   0.94 \\
$d_{9}$  &  33780 &  96.62 &    880 &   2.51 &   300 &   0.86 &   0 &    0 \\
$d_{10}$ &   8720 &  24.94 &  16871 &  48.22 &  9369 &  26.74 &   0 &    0 \\
          \hline
   \end{tabular} \\[3mm]
     \end{center}
\end{table}

Tabelle \ref{AS1} zeigt die Auswertung der optimalen Politik die sich aufgrund der Parameter und der Wahrscheinlichkeitsverteilung des Szenarios 1 ergeben. Es handelt sich um eine Auswertung der absoluten Anzahl und der Relation der gewählten optimalen Politik des Szenarios. Jeweils sofern keine Anfrage ($d_0$) und sofern eine Anfrage eintrifft ($d_j\forall j \in\mathcal{J}$). Wie in der Tabelle \ref{AS1} zu erkennen ist, wird für die Produktanfragen $j=1$ eher die optimale Politik der Lagerproduktion gewählt, anstelle der Auftragsannahme. Für Produktanfragen $j=2$ gilt dies nicht, da der Ertrag dieser Anfragen höher ist und damit empfiehlt das Modell eher die Auftragsannahme. Der weitere Verlauf der Anzahl der gewählten optimalen Politik für ertragsarme Anfragen $j$ steigt über den Buchungshorizont dabei nicht merklich an. Sofern keine Anfragen eintreffen, ist die vom Modell gewählte optimale Politik $d_{0}$ die Verwendung von Anfragen mit niedrigem Ertrag für eine mögliche Lagerhaltung.

\begin{figure}[h!]     
\begin{center}
\includegraphics[width=35mm, trim=0pt 20pt 20pt 0pt, clip]{/Users/Superuser/DP-RM-with-storage/cluster/DP_L_Ab/OP-0.png}
\includegraphics[width=140mm, trim=125pt 0pt 110pt 10pt, clip]{/Users/Superuser/DP-RM-with-storage/cluster/DP_L_Ab/OP-J.png}
    \caption{Veränderung der optimale Politik für das Szenario 1 sofern keine Anfragen eintreffen ($d_0=0$) und je Produktanfrage bei konstanter Ressourcenkapazität ($d_j\text{ entspricht }1\text{ bis }10$)}  \label{SV1}
  \end{center}
\end{figure}

Damit die Veränderung der optimalen Politik $d_0$ und $d_j \forall j \in \mathcal{J}$ im Zeitverlauf untersucht werden kann, wird eine Auswertung für einen bestimmten Verlauf im System der jeweiligen Szenarien betrachtet. D. h. es werden bestimmte Auftragseingänge angenommen und es wird die optimale Politik die aufgrund dieses Verlaufs resultieren erfasst. Für dieses und die nachfolgenden Szenarien wird angenommen, dass über den gesamten Buchungshorizont $T$ keine Anfrage angenommen wird. Damit bleiben die Ressourcenkapazitäten $\textbf{c}^{\hat t}$ sowie der Lagerbestände $\textbf{y}^{\hat t}$ der Ressource $h=1$ für die Leistungserstellungszeitpunkte $\hat t \in \hat T$ über alle Perioden $t\in T$ konstant. Da es sich hier um ein dynamisches Optimierungsmodell handelt, ist eine solche Auswertung innerhalb des Netzwerks nur ein möglicher Verlauf der optimalen Politik. Sofern zu einer belieben Periode $t$ eine andere Entscheidung getroffen bzw. eine andere Kante des Graphen gewählt wird, könnte sich die optimale Politik des Netzwerks für alle Perioden $t$ ab dem Zeitpunkt der Entscheidung ändern. Aufgrund des vordefinierten Verlaufs ist auch für alle $t \in T$ die optimale Politik der Lagerentnahme nicht möglich ($d_j=2$). Dies resultiert aus der Tatsache, dass für eine solche optimale Politik ein Lagerbestand $\textbf{y}^{\hat t}$ notwendig ist.

Abbildung \ref{SV1} zeigt den Verlauf der Veränderung der optimalen Politik für das Szenario 1 sofern keine Anfragen eintreffen und je Produktanfrage bei konstanter Ressourcenkapazität. Der erste Verlauf der Abbildung zeigt die optimale Politik bzgl. der Entscheidung, ob eine Lagerproduktion durchgeführt werden soll wenn keine Anfrage eintrifft und zeigt damit die optimale Politik $d_0$. Damit lässt sich anhand dieses Kurvenverlaufs ablesen, durch welche Anfragen $j$ eine Lagerproduktion zur Ertragsmaximierung erfolgen soll. Wie zu erkenne ist, ist die optimale Politik $d_0$ für die Perioden $100\ge t \ge 81$ (gesamte Anzahl der Perioden $t$ bis zur Leistungserstellung $\hat t = 1$) anfangs den Modus der Produktanfrage $j=1$ für eine Lagerproduktion zu verwenden. D. h. der Lagerbestand wäre ab der Leistungsperiode $\hat t=2$ verfügbar. Sofern diese Anfrage $j=1$ nicht mehr möglich ist, wechselt das Modell für die Perioden $80\ge t \ge 61$ zur optimalen Politik $d_0=3$. Sofern auch diese Anfrage $j=3$ nicht mehr möglich ist, werden anfangs keine möglichen Produktanfragen $j\in\mathcal{J}$ für eine mögliche Lagerproduktion als optimale Politik ermittelt. Erst durch Annäherung an den Leistungserstellungszeitpunkt $\hat t = 3$ wird erneut die Lagerproduktion als optimale Politik vom Modell ermittelt. Es wird eine optimale Politik der Lagerproduktion anhand des Ausführungsmodus für die ertragsarme Produktanfrage $j=5$ verwendet. Zum Ausführungszeitpunkt $\hat t = 4$ ist ein ähnliches Bild im Bezug zur Produktanfrage $j=7$ zu erkennen, wobei der Zeitraum der optimale Politik zur Lagerproduktion erneut kürzer vor dem Leistungserstellungszeitpunkt ist.

Die weiteren Verläufe der Abbildung \ref{SV1} zeigen jeweils für die Produktanfrage $j\in\mathcal{J}$ die optimale Politik $d_j$. D. h. die möglichen Entscheidungen \glqq Auftragsablehnung ($d_j=0$){\grqq}, \glqq Auftragsannahme via Kapazität ($d_j=1$){\grqq}, \glqq Auftragsannahme via Lagerentnahme ($d_j=2$){\grqq} und \glqq Auftragsablehnung und Lagerproduktion ($d_j=3$){\grqq} für den vordefinierten Verlauf des Netzwerks über alle Perioden $t\in T$. Für Produktanfrage $j=1$ zeigt die optimale Politik für den produktspezifisch möglichen Zeitraum ausschließlich $d_1=3\forall t\in T$. Die Produktanfrage $j=2$ werden ausschließlich angenommen ($d_2=1\forall t \in T$). Bei Produktanfragen $=3$ erfolgt nur die optimale Politik der Lagerproduktion, aber nicht über die gesamten möglichen Perioden $t\in T$. Bei Produktanfragen $j=4$ wechselt sich die optimale Politik  der Auftragsannahme und der Lagerproduktion ab. Die Lagerproduktion ist ausschließlich optimale Politik für Anfragen $j=5$. Diese optimale Politik stellt sich jedoch kurz vor dem Verderben der Kapazität ein. Produktanfragen $j=6$ werden ausschließlich angenommen, aber nicht über den gesamt möglichen Zeitraum $t\in T$. Es gibt Intervalle an denen die Anfragen abgelehnt werden. Anfragen $j=7$ werden kurz vor dem Leistungserstellungszeitpunkt $\hat t=4$ auf Lager produziert. Die optimale Politik für Produktanfragen $j=8$ lautet $d_8=1$ über alle möglichen Perioden $t\in T$, wobei eine Unterbrechung in dem zum Leistungserstellungszeitpunkt $\hat t=1$ zugehörigen Buchungszeitraum $100\ge t \ge 81$ stattfindet. Für Produktanfragen $j=9$ kommt nur die Lagerentnahme kurz vor Verfall der Kapazität in Betracht. Die Auftragsannahme von Produktanfragen $j=10$ ist über den gesamten Buchungshorizont $T$ die optimale Politik.

\textbf{Szenario 2}

Das zweite Szenario zeichnet sich durch linear ansteigende Eintrittswahrscheinlichkeiten $p_j(t)$ der Produktanfragen $j\in\mathcal{J}$ aus. Dabei steigt die Eintrittswahrscheinlichkeit für eine jeweilige Anfrage $j$ bis zum Leistungserstellungszeitpunkt $\hat t$ auf $p_j(t)=0,3$ an, wobei die Steigung an jeweils unterschiedlichen Perioden startet. Die weiteren Eigenschaften der Produktanfragen basieren auf den des Szenarios 1. Tabelle \ref{S2} und Abbildung \ref{SB2} zeigen die Parameter für das Szenario 2.

%%%%%%%%% DP_A_Lang
\begin{table}[h!]
\renewcommand{\arraystretch}{1.5}
  \begin{center}
    \caption{Szenario 2}  \label{S2}
    \vspace*{3mm}
    \begin{tabular}{l l l l l l}   %hier die Spaltenausrichtung, -breite, -begrenzung und -anzahl eintragen
    $T$ & $\hat T$  & $h$ & $c_h^{\hat t}\forall \hat{t}\in{\hat T}$ & $y_h^{\hat t}\forall \hat{t}\in{\hat T}$  & $y_h^{{\hat t},max}\forall \hat{t}\in{\hat T}$  \\  \hline
100 & 5 & 1 & 1 & 0 & 2  \\ \hline
    \end{tabular} \\[3mm]
        \begin{tabular}{p{.5cm} p{.5cm} p{.5cm}  p{.5cm} p{9cm}}   %hier die Spaltenausrichtung, -breite, -begrenzung und -anzahl eintragen
    $j$ & $r_j$  & $a_{1j}$ & $\hat t$ & Verteilung $p_j(t)$ \\  \hline
1 & 100 & 1 & 1 & Linear ansteigend auf $p_j(t)=0,3$ von 100 bis 81   \\
2 & 1000 & 1 & 1 & Linear ansteigend auf $p_j(t)=0,3$ von 100 bis 81  \\
3 & 100 & 1 & 2 & Linear ansteigend auf $p_j(t)=0,3$ von 100 bis 61  \\
4 & 1000 & 1 & 2 & Linear ansteigend auf $p_j(t)=0,3$ von 100 bis 61  \\
5 & 100 & 1 & 3 & Linear ansteigend auf $p_j(t)=0,3$ von 60 bis 41 \\
6 & 1000 & 1 & 3 & Linear ansteigend auf $p_j(t)=0,3$ von 60 bis 41 \\
7 & 100 & 1 & 4 & Linear ansteigend auf $p_j(t)=0,3$ von 80 bis 21  \\
8 & 1000 & 1 & 4 & Linear ansteigend auf $p_j(t)=0,3$ von 80 bis 21  \\
9 & 100 & 1 & 5 & Linear ansteigend auf $p_j(t)=0,3$ von 40 bis 1  \\
10 & 1000 & 1 & 5 & Linear ansteigend auf $p_j(t)=0,3$ von 40 bis 1  \\ \hline
    \end{tabular} \\[3mm]
     \begin{tabular}{p{7cm}p{5cm}} \hline
     Rechenzeit Systemzustände (h): & \texttt{0:00:00.094716} \\
     Anzahl möglicher Systemzustände: & \texttt{785376} \\
     Anzahl benötigter Systemzustände: & \texttt{17667} \\ 
     Rechenzeit DP (h): & \texttt{1:36:44.968302} \\ 
          Max. Erwartungswert (GE): & \texttt{4998.84} \\ \hline
         \end{tabular} \\[3mm]
  %  {\footnotesize \textbf{In Anlehnung an:} \cite{quante2009management}, S. 44.}\\
        % {\footnotesize \textbf{Quelle:} \url{http://www.rrzn.uni-hannover.de/scientific_computing_doku.html} }   %footnotesize liefert Schrift in Größe 10pt
  \end{center}
\end{table}

\begin{figure}[h!]
  \begin{center}
    \includegraphics[width=80mm, trim=300pt 180pt 300pt 0pt]{/Users/Superuser/DP-RM-with-storage/cluster/DP_A_Lang/Wverteilung.png}
    \caption{Wahrscheinlichkeitsverteilung Szenario 2}  \label{SB2}
       % {\footnotesize \textbf{Quelle:} ????} 
    %{\footnotesize \textbf{Legende:} Annahme einer Produktauftrag entspricht '$j$', KA='Kein Auftrag'} 
  \end{center}
\end{figure}

\begin{table}[h!]
\renewcommand{\arraystretch}{1.5}
  \begin{center}
    \caption{Auswertung des Szenarios 2}  \label{AS2}
    \vspace*{3mm}
    %%%%%%%%%%%%%%% d[0]
    \begin{tabular}{l l l l l l l l l l l l }  \hline 
         $j$ & 0 & 1  & 2 & 3 & 4  & 5 & 6 & 7 & 8 & 9 & 10  \\  \hline
$d_{0}$ &  15297 &    31 &  0 &    97 &  0 &   644 &  0 &  1234 &  0 &  0 &  0 \\
$d_{0}$ (\%) &  88.41 &  0.18 &  0 &  0.56 &  0 &  3.72 &  0 &  7.13 &  0 &  0 &  0 \\
\hline
    \end{tabular} \\[3mm]
        \begin{tabular}{ l l l l l l l l l}   \hline    %%%%%%%%%%%%%%% d[0]
    $d_j$ & \multicolumn{2}{c}{Ablehnung (0)} & \multicolumn{2}{c}{Annahme (1)}  & \multicolumn{2}{c}{Lagerentnahme (2)} & \multicolumn{2}{c}{Lagerproduktion (3)}\\
    & Anz. & \% & Anz. & \% & Anz. & \% & Anz. & \% \\ \hline 
$d_{1}$  &  17272 &  99.82 &   0 &    0 &   0 &    0 &    31 &  0.18 \\
$d_{2}$  &  17255 &  99.72 &    48 &   0.28 &   0 &    0 &   0 &   0 \\
$d_{3}$  &  17204 &  99.43 &     1 &   0.01 &   0 &    0 &    98 &  0.56 \\
$d_{4}$  &  16931 &  97.85 &   232 &   1.33 &   140 &    0.8 &   0 &   0 \\
$d_{5}$  &  16652 &  96.24 &     7 &   0.04 &   0 &    0 &   644 &   3.7 \\
$d_{6}$  &  15909 &  91.94 &   944 &   5.43 &   450 &   2.59 &   0 &   0 \\
$d_{7}$  &  16057 &   92.8 &    12 &   0.07 &   0 &    0 &  1234 &  7.09 \\
$d_{8}$  &  11026 &  63.72 &  3664 &   21.1 &  2613 &  15.03 &   0 &   0 \\
$d_{9}$  &  16849 &  97.38 &   332 &   1.91 &   122 &    0.7 &   0 &   0 \\
$d_{10}$ &   6713 &   38.8 &  6934 &  39.98 &  3656 &  21.03 &   0 &   0 \\
          \hline
   \end{tabular} \\[3mm]
     \end{center}
\end{table}

Tabelle \ref{AS2} zeigt die optimale Politik für das Szenario 2. Aufgrund der Wahrscheinlichkeitsverteilung $p_j(t)$ und aufgrund des hohen Ertrags $r_j$ von Produktanfrage $j=8$ wird diese tendenziell gegenüber der anderen Produktanfragen $j<8$ öfters gewählt. In diesem Szenario steigt die Anzahl der Entscheidungen über eine Lagerproduktion mit ablaufenden Buchungshorizonts für Anfragen mit niedrigem Ertrag an. Die Anzahl an Systemzustände ist unter einer solchen Verteilung geringer, da nicht alle Anfragen zu jeder Periode verfügbar sind. Dies reduziert auch die Berechnungszeit des Algorithmus. Tendenziell erfolgt die Entscheidungen über eine Lagerproduktion nicht so häufig wie bei einer konstant linearen Verteilung. Jedoch wird in diesem Szenario keine Anfrage $j=1$ angenommen, sondern ausschließlich für die Lagerproduktion verwendet. Für die weiteren ertragsarmen Produktanfragen ist die Auftragsannahme eher vereinzelt die optimale Politik. Sofern keine Anfragen eintreffen, wird im Vergleich zu einem linearen Verlauf der Eintrittswahrscheinlichkeiten tendenziell weniger eine Lagerproduktion als optimale Politik gewählt. Nur mit ablaufenden Buchungshorizont wird die Politik der Lagerproduktion vom Modell häufiger als optimal bestimmt.

\begin{figure}[h!]     
\begin{center}
\includegraphics[width=35mm, trim=0pt 20pt 20pt 0pt, clip]{/Users/Superuser/DP-RM-with-storage/cluster/DP_A_Lang/OP-0.png}
\includegraphics[width=140mm, trim=125pt 0pt 110pt 10pt, clip]{/Users/Superuser/DP-RM-with-storage/cluster/DP_A_Lang/OP-J.png}
    \caption{Veränderung der optimale Politik für das Szenario 2 sofern keine Anfragen eintreffen ($d_0=0$) und je Produktanfrage bei konstanter Ressourcenkapazität ($d_j\text{ entspricht }1\text{ bis }10$)}  \label{SV2}
  \end{center}
\end{figure}

Die Auswertung des Verlaufs der optimalen Politik sofern keine Anfragen über den Buchungshorizont $T$ akzeptiert werden zeigt die Abbildung \ref{SV2} für das Szenario 2. Die optimale Politik $d_0$ ist jeweils die Lagerproduktion der ertragsarmen Produktanfragen des Szenarios. Wobei diese optimale Politik ungefähr im mittleren Bereich des zeitlich vorherigen Buchungsabschnitts vom produktspezifischen Leistungserstellungszeitpunkt $\hat t$ eintritt. In diesem Szenario ist die optimale Politik $d_j$ eindeutig für die ertragsarmen und ertragreichen Produktanfragen $j\in\mathcal{J}$. Produktanfragen mit hohem Ertrag $r_j$ werden angenommen und Produktanfragen mit wenig Ertrag $r_j$ werden auf Lager produziert. Jedoch zeigt sich kein stetiger Verlauf für die jeweilige Produktanfrage $j$. D. h. wie in Abbildung \ref{SV2} zu erkennen ist, gibt es des Öfteren Unterbrechungen bei den jeweiligen optimalen Politiken. Bspw. ist bei der Produktanfrage $j=3$ eine längere Unterbrechung zwischen den ersten beiden Buchungsabschnitten ermittelt. Bei der Produktanfrage $j=8$ ist wiederum eine kurze Unterbrechung vor der Leistungserstellung $\hat t = 3$ eingetreten. Die Unterbrechungen resultieren aufgrund der neu eintretenden Konkurrenzanfragen im Buchungshorizont.\\[.5cm]

\textbf{Szenario 3}

Bei dem Szenario 3 handelt es sich um normalverteilte Eintrittswahrscheinlichkeiten $p_j(t)$ für die Produktanfragen $j\in\mathcal{J}$. Dabei ist für jede Anfrage der Erwartungswert $\mu$ und die Standardabweichung $\sigma$ angegeben. Zu beachten ist, dass aufgrund des unter Umständen umfangreichen Zeitraums der Wahrscheinlichkeitsverteilung $p_j(t)$ die zugehörige Normalverteilung $\mathcal{N}(\mu,\sigma)$ einer Produktanfrage $j$ um einen Skala normiert ist. Andernfalls wären die Werte der Wahrscheinlichkeitsverteilung für solche Anfragen viel geringer als die für Anfragen mit einem kurzen Betrachtungszeitraum und einer niedrigen Standardabweichung $\sigma$. Dies spiegelt z. B. die Verteilung von Produkt 1 und Produkt 9 in Tabelle \ref{S3} wider. Weiter zeigt die Tabelle alle anderen Parameterausprägungen für das Szenario 3. Die weiteren Eigenschaften der Produktanfragen orientieren sich anhand der bereits beschriebenen Szenarien. Abbildung \ref{SB3} zeigt die Wahrscheinlichkeitsverteilung des Szenarios.

%%%%%%%%%%%%%%%%%%%%% DP_N_Lang
\begin{table}[h!]
\renewcommand{\arraystretch}{1.5}
  \begin{center}
    \caption{Szenario 3}  \label{S3}
    \vspace*{3mm}
    \begin{tabular}{l l l l l l}   %hier die Spaltenausrichtung, -breite, -begrenzung und -anzahl eintragen
    $T$ & $\hat T$  & $h$ & $c_h^{\hat t}\forall \hat{t}\in{\hat T}$ & $y_h^{\hat t}\forall \hat{t}\in{\hat T}$  & $y_h^{{\hat t},max}\forall \hat{t}\in{\hat T}$  \\  \hline
100 & 5 & 1 & 1 & 0 & 2  \\ \hline
    \end{tabular} \\[3mm]
        \begin{tabular}{p{1cm} p{1cm} p{1cm}  p{1cm} p{6cm}}   %hier die Spaltenausrichtung, -breite, -begrenzung und -anzahl eintragen
    $j$ & $r_j$  & $a_{1j}$ & $\hat t$ & Verteilung $p_j(t)$ \\  \hline
1 & 100 & 1 & 1 & Normalverteilt $\mathcal{N}(90, 2)$   \\
2 & 1000 & 1 & 1 & Normalverteilt $\mathcal{N}(90, 2)$  \\
3 & 100 & 1 & 2 & Normalverteilt $5\cdot\mathcal{N}(70, 10)$  \\
4 & 1000 & 1 & 2 & Normalverteilt $5\cdot\mathcal{N}(70, 10)$  \\
5 & 100 & 1 & 3 & Normalverteilt $5\cdot\mathcal{N}(50, 10)$ \\
6 & 1000 & 1 & 3 & Normalverteilt $5\cdot\mathcal{N}(50, 10)$ \\
7 & 100 & 1 & 4 & Normalverteilt $5\cdot\mathcal{N}(30, 10)$  \\
8 & 1000 & 1 & 4 & Normalverteilt $5\cdot\mathcal{N}(30, 10)$  \\
9 & 100 & 1 & 5 & Normalverteilt $15\cdot\mathcal{N}(10, 30)$  \\
10 & 1000 & 1 & 5 & Normalverteilt $15\cdot\mathcal{N}(10, 30)$  \\ \hline
    \end{tabular} \\[3mm]
     \begin{tabular}{p{7cm}p{5cm}} \hline
     Rechenzeit Systemzustände (h): & \texttt{0:00:00.052650} \\
     Anzahl möglicher Systemzustände: & \texttt{785376} \\
     Anzahl benötigter Systemzustände: & \texttt{35324} \\ 
     Rechenzeit DP (h): & \texttt{3:01:16.875038} \\ 
          Max. Erwartungswert (GE): & \texttt{4999.91} \\ \hline
         \end{tabular} \\[3mm]
  \end{center}
\end{table}

\begin{figure}[h!]
  \begin{center}
    \includegraphics[width=80mm, trim=300pt 180pt 300pt 0pt]{/Users/Superuser/DP-RM-with-storage/cluster/DP_N_Lang/Wverteilung.png}
    \caption{Wahrscheinlichkeitsverteilung Szenario 3}  \label{SB3}
  \end{center}
\end{figure}

\begin{table}[h!]
\renewcommand{\arraystretch}{1.5}
  \begin{center}
    \caption{Auswertung des Szenarios 3}  \label{AS3}
    \vspace*{3mm}
    %%%%%%%%%%%%%%% d[0]
    \begin{tabular}{l l l l l l l l l l l l }  \hline 
         $j$ & 0 & 1  & 2 & 3 & 4  & 5 & 6 & 7 & 8 & 9 & 10  \\  \hline
$d_{0}$ &  28843 &  561 &  0 &   931 &  0 &  1679 &  0 &  2946 &  0 &  0 &  0 \\
$d_{0}$ (\%)&   82.5 &  1.6 &  0 &  2.66 &  0 &   4.8 &  0 &  8.43 &  0 &  0 &  0 \\
\hline
    \end{tabular} \\[3mm]
        \begin{tabular}{ l l l l l l l l l}   \hline    %%%%%%%%%%%%%%% d[0]
    $d_j$ & \multicolumn{2}{c}{Ablehnung (0)} & \multicolumn{2}{c}{Annahme (1)}  & \multicolumn{2}{c}{Lagerentnahme (2)} & \multicolumn{2}{c}{Lagerproduktion (3)}\\
    & Anz. & \% & Anz. & \% & Anz. & \% & Anz. & \% \\ \hline 
$d_{1}$  &  34259 &  97.99 &    140 &    0.4 &   0 &    0 &   561 &   1.6 \\
$d_{2}$  &  33649 &  96.25 &   1198 &   3.42 &   0 &    0 &   113 &  0.32 \\
$d_{3}$  &  34006 &  97.27 &     22 &   0.06 &   0 &    0 &   932 &  2.66 \\
$d_{4}$  &  30576 &  87.46 &   2954 &   8.43 &  1251 &   3.57 &   179 &  0.51 \\
$d_{5}$  &  33148 &  94.82 &     22 &   0.06 &    60 &   0.17 &  1730 &  4.94 \\
$d_{6}$  &  25345 &   72.5 &   5909 &  16.87 &  3206 &   9.15 &   500 &  1.43 \\
$d_{7}$  &  31769 &  90.87 &     12 &   0.03 &    80 &   0.23 &  3099 &  8.84 \\
$d_{8}$  &  17642 &  50.46 &  10097 &  28.84 &  7024 &  20.05 &   197 &  0.56 \\
$d_{9}$  &  34316 &  98.16 &    454 &   1.29 &   190 &   0.54 &   0 &   0 \\
$d_{10}$ &   8677 &  24.82 &  16791 &     48 &  9492 &  27.09 &   0 &   0 \\
          \hline
   \end{tabular} \\[3mm]
     \end{center}
\end{table}

\begin{figure}[h!]     
\begin{center}
\includegraphics[width=35mm, trim=0pt 20pt 20pt 0pt, clip]{/Users/Superuser/DP-RM-with-storage/cluster/DP_N_Lang/OP-0.png}
\includegraphics[width=140mm, trim=125pt 0pt 110pt 10pt, clip]{/Users/Superuser/DP-RM-with-storage/cluster/DP_N_Lang/OP-J.png}
    \caption{Veränderung der optimale Politik für das Szenario 3 sofern keine Anfragen eintreffen ($d_0=0$) und je Produktanfrage bei konstanter Ressourcenkapazität ($d_j\text{ entspricht }1\text{ bis }10$)}  \label{SV3}
  \end{center}
\end{figure}

Sofern eine solche Wahrscheinlichkeitsverteilung $p_j(j)$ angenommen wird, verhält es sich die optimale Politik $d_j$ ähnlich der Verteilung mit einem linearen Verlauf der Eintrittswahrscheinlichkeiten (siehe Auswertung des Szenarios 1). Tabelle \ref{AS3} zeigt die Auswertung des Szenarios 3. Die optimale Politik bzgl. einer Lagerhaltungsentscheidung wird eher für Produktanfragen $j$ mit geringem Ertrag $r_j$ ermittelt. Auch bei möglichen Entscheidung der Lagerproduktion sofern keine Anfrage eintrifft bzw. für die optimalen Politik $d_0$ sind ähnliche Werte wie bei einem linearen Verlauf der Eintrittswahrscheinlichkeiten $p_j(t)$ festzustellen.

Abbildung \ref{SV3} zeigt den Verlauf der optimalen Politik für das Szenario 3. Auch hier gilt der vordefiniert Verlauf innerhalb des Netzwerks, dass keine Anfragen innerhalb des Buchungshorizonts $T$ angenommen werden. Die optimale Politik $d_0$ ist erneut die Verwendung der Ausführungsmodi der ertragsarmen Produktanfragen $j$ für die Lagerproduktion. Wobei unter Annahme dieser Wahrscheinlichkeitsverteilung $p_0(t)$ über alle Perioden $t\in T$ im Buchungsabschnitt $60\ge t \ge 41$ die optimale Politik $d_0=5$ einige Unterbrechungen aufweist. Die Produktanfrage $j=4$ und $j=6$ zeigen in diesem Szenario eine interessante optimale Politik $d_j$. Für Produktanfragen $j=4$ ist die optimale Politik $d_4$ zum Anfang des Buchungshorizonts $T$ die Auftragsannahme und im späteren Verlauf die Lagerproduktion. Die Entscheidung der Lagerproduktion folgt damit aufgrund des anstehenden Verderbens der Ressourcenkapazität und der Wahrscheinlichkeitsverteilung. Bei der optimalen Politik der Produktanfragen $j=6$ ist ein ähnliches Bild erkennbar. Im Intervall mit den größten Wahrscheinlichkeitswerten $p_6(t)$ ist die optimale Politik $d_6=3$. Sofern die Wahrscheinlichkeitswerte $p_6(t)$ jedoch eher gering sind, scheint die Auftragsannahme ($d_6=1$) oftmals die optimale Politik zu sein. 

%%%%%%%%%%%%%%%%%%%%% DP_C_Lang
\begin{table}[h!]
\renewcommand{\arraystretch}{1.5}
  \begin{center}
    \caption{Szenario 4}  \label{S4}
    \vspace*{3mm}
    \begin{tabular}{l l l l l l}   %hier die Spaltenausrichtung, -breite, -begrenzung und -anzahl eintragen
    $T$ & $\hat T$  & $h$ & $c_h^{\hat t}\forall \hat{t}\in{\hat T}$ & $y_h^{\hat t}\forall \hat{t}\in{\hat T}$  & $y_h^{{\hat t},max}\forall \hat{t}\in{\hat T}$  \\  \hline
100 & 5 & 1 & 1 & 0 & 2  \\ \hline
    \end{tabular} \\[3mm]
        \begin{tabular}{p{1cm} p{1cm} p{1cm}  p{1cm} p{6cm}}   %hier die Spaltenausrichtung, -breite, -begrenzung und -anzahl eintragen
    $j$ & $r_j$  & $a_{1j}$ & $\hat t$ & Verteilung $p_j(t)$ \\  \hline
1 & 100 & 1 & 1 & Cauchyverteilt $5\cdot\mathcal{C}(81, 5)$   \\
2 & 1000 & 1 & 1 & Cauchyverteilt $5\cdot\mathcal{C}(81, 5)$  \\
3 & 100 & 1 & 2 & Cauchyverteilt $5\cdot\mathcal{C}(61, 5)$  \\
4 & 1000 & 1 & 2 & Cauchyverteilt $5\cdot\mathcal{C}(61, 5)$  \\
5 & 100 & 1 & 3 & Cauchyverteilt $5\cdot\mathcal{C}(41, 5)$ \\
6 & 1000 & 1 & 3 & Cauchyverteilt $5\cdot\mathcal{C}(41, 5)$ \\
7 & 100 & 1 & 4 & Cauchyverteilt $5\cdot\mathcal{C}(21, 5)$  \\
8 & 1000 & 1 & 4 & Cauchyverteilt $5\cdot\mathcal{C}(21, 5)$  \\
9 & 100 & 1 & 5 & Cauchyverteilt $5\cdot\mathcal{C}(1, 5)$  \\
10 & 1000 & 1 & 5 & Cauchyverteilt $5\cdot\mathcal{C}(1, 5)$  \\ \hline
    \end{tabular} \\[3mm]
     \begin{tabular}{p{7cm}p{5cm}} \hline
     Rechenzeit Systemzustände (h): & \texttt{0:00:00.156592} \\
     Anzahl möglicher Systemzustände: & \texttt{785376} \\
     Anzahl benötigter Systemzustände: & \texttt{35324} \\ 
     Rechenzeit DP (h): & \texttt{6:04:54.450184} \\ 
          Max. Erwartungswert (GE): & \texttt{4918.91} \\ \hline
         \end{tabular} \\[3mm]
  %  {\footnotesize \textbf{In Anlehnung an:} \cite{quante2009management}, S. 44.}\\
        % {\footnotesize \textbf{Quelle:} \url{http://www.rrzn.uni-hannover.de/scientific_computing_doku.html} }   %footnotesize liefert Schrift in Größe 10pt
  \end{center}
\end{table}

\newpage

\textbf{Szenario 4}

Für das Szenario 4 wird eine Cauchyverteilung für die Eintrittswahrscheinlichkeiten $p_j(t)$ der Produktanfragen $j\in\mathcal{J}$ angenommen. Damit wird eine Verteilung modelliert, die für eine jede Produktanfrage $j$ den Höhepunkt der Wahrscheinlichkeit des Eintreffens zur letztmöglichen Buchungsperiode $t$ vorsieht, aber mit geringer Wahrscheinlichkeit über die alle Buchungsperioden $t\in T$ möglich ist. D. h. die Wahrscheinlichkeit steigt zum Leistungserstellungszeitpunkt $\hat t$ für eine jede Produktanfrage $j$ stark an. Abbildung \ref{SB4} zeigt die beschriebene Verteilung anhand eines Kurvendiagramms. Auch in diesem Szenario ist die Verteilung eines Produkts $j\in\mathcal{J}$ durch einen jeweiligen Skala verstärkt, damit die Eintrittswahrscheinlichkeiten $p_j(t)$ größere Werte aufweisen. Die weiteren Eigenschaften des Szenarios orientieren sich an den bisherigen Szenarien. Tabelle \ref{S4} zeigt die Parameter zusammenfassend.

\begin{figure}[h!]
  \begin{center}
    \includegraphics[width=80mm, trim=300pt 180pt 300pt 0pt]{/Users/Superuser/DP-RM-with-storage/cluster/DP_C_Lang/Wverteilung.png}
    \caption{Wahrscheinlichkeitsverteilung Szenario 4}  \label{SB4}
       % {\footnotesize \textbf{Quelle:} ????} 
    %{\footnotesize \textbf{Legende:} Annahme einer Produktauftrag entspricht '$j$', KA='Kein Auftrag'} 
  \end{center}
\end{figure}

\begin{table}[h!]
\renewcommand{\arraystretch}{1.5}
  \begin{center}
    \caption{Auswertung des Szenarios 4}  \label{AS4}
    \vspace*{3mm}
    %%%%%%%%%%%%%%% d[0]
    \begin{tabular}{l l l l l l l l l l l l }  \hline 
         $j$ & 0 & 1  & 2 & 3 & 4  & 5 & 6 & 7 & 8 & 9 & 10  \\  \hline
$d_{0}$ &  28073 &   247 &  0 &  1192 &  0 &  2151 &  0 &  3297 &  0 &  0 &  0 \\
$d_{0}$ (\%) &   80.3 &  0.71 &  0 &  3.41 &  0 &  6.15 &  0 &  9.43 &  0 &  0 &  0 \\
\hline
    \end{tabular} \\[3mm]
        \begin{tabular}{ l l l l l l l l l}   \hline    %%%%%%%%%%%%%%% d[0]
    $d_j$ & \multicolumn{2}{c}{Ablehnung (0)} & \multicolumn{2}{c}{Annahme (1)}  & \multicolumn{2}{c}{Lagerentnahme (2)} & \multicolumn{2}{c}{Lagerproduktion (3)}\\
    & Anz. & \% & Anz. & \% & Anz. & \% & Anz. & \% \\ \hline 
$d_{1}$  &  34693 &  99.24 &     20 &   0.06 &   0 &    0 &   247 &   0.7 \\
$d_{2}$  &  33464 &  95.72 &   1463 &   4.17 &   0 &    0 &    33 &  0.09 \\
$d_{3}$  &  33741 &  96.51 &     22 &   0.06 &   0 &    0 &  1197 &  3.41 \\
$d_{4}$  &  29232 &  83.62 &   4007 &  11.43 &  1681 &    4.8 &    40 &  0.11 \\
$d_{5}$  &  32689 &   93.5 &     22 &   0.06 &   0 &    0 &  2249 &  6.42 \\
$d_{6}$  &  23830 &  68.16 &   7100 &  20.27 &  3991 &  11.39 &    39 &  0.11 \\
$d_{7}$  &  31519 &  90.16 &     12 &   0.03 &   0 &    0 &  3429 &  9.78 \\
$d_{8}$  &  17252 &  49.35 &  10408 &  29.73 &  7101 &  20.27 &   199 &  0.57 \\
$d_{9}$  &  34506 &   98.7 &    332 &   0.95 &   122 &   0.35 &   0 &   0 \\
$d_{10}$ &   8501 &  24.32 &  16803 &  48.03 &  9656 &  27.56 &   0 &   0 \\
          \hline
   \end{tabular} \\[3mm]
     \end{center}
\end{table}

\begin{figure}[h!]     
\begin{center}
\includegraphics[width=35mm, trim=0pt 20pt 20pt 0pt, clip]{/Users/Superuser/DP-RM-with-storage/cluster/DP_C_Lang/OP-0.png}
\includegraphics[width=140mm, trim=125pt 0pt 110pt 10pt, clip]{/Users/Superuser/DP-RM-with-storage/cluster/DP_C_Lang/OP-J.png}
    \caption{Veränderung der optimale Politik für das Szenario 4 sofern keine Anfragen eintreffen ($d_0=0$) und je Produktanfrage bei konstanter Ressourcenkapazität ($d_j\text{ entspricht }1\text{ bis }10$)}  \label{SV4}
  \end{center}
\end{figure}

Die Auswertung in der Tabelle \ref{AS4} zeigt für die Cauchyverteilung, dass die Anzahl der optimalen Politik $d_j$ für Produktanfragen $j\in\mathcal{J}$ zur möglichen Lagerproduktion keine besonderen Ausprägungen hat. Die Cauchyverteilung zeichnet sich dadurch aus, dass die Eintrittswahrscheinlichkeiten $p_j(t)$ über dem Wert $0$ liegen und somit über den gesamten Buchungshorizont $T$ eine Anfrage über jedes Produkt $j$ möglich ist, sofern der Leistungserstellungszeitpunkt $\hat t$ nicht überschritten ist. Die optimale Politik für ertragreiche Produktanfragen ist tendenziell eher die Akzeptanz einer Produktanfrage $j$.

Die Veränderung des Verlaufs der optimalen Politik für das Szenario 4 zeigt die Abbildung \ref{SV4}. Bzgl. der optimalen Politik $d_0$ lässt sich entnehmen, dass eine Lagerproduktion für die ertragsarmen Produkte kurz vor Verfall ermittelt ist und somit kurz bevor die Leistung in Produktion gehen kann. Für alle Produktanfragen $j\in\mathcal{J}$ mit hohen Ertrag $r_j$ ist die optimalen Politik $d_j=1$ und damit die Auftragsannahme dieser Produktanfragen. Diese optimale Politik gilt bis zum produktspezifischen Leistungserstellungszeitpunkt $\hat t$. Für die ertragsarmen Anfragen $j\in\mathcal{J}$ ist die optimale Politik $d_j=3$ jeweils kurz vor dem jeweiligen Leistungserstellungszeitpunkt $\hat t$. Damit wird die Kapazität für nachfolgende Anfragen über das Lager gesichert. Diese Politik gilt jedoch nicht für die ertragsarme Produktanfrage $j=9$, da für diese eine Lagerproduktion aufgrund des auslaufenden Buchungshorizonts $T$ nicht möglich ist. Kurz vor Anlauf des Leistungserstellungszeitpunkt $\hat t$ ist für diese Produktanfrage die Annahme die optimale Politik $d_9$.

\textbf{Szenario 5}

Für die nachfolgenden Szenarien wird die Anzahl der Produktanfragen $j\in\mathcal{J}$ auf fünf mögliche Anfragen reduziert. Diese sind jeweils für einen bestimmten Leistungserstellungszeitpunkt $\hat t$ vorgesehen. Die Leistungserstellungszeitpunkte $\hat t$ erstrecken sich weiterhin gleichmäßig über den gesamten Buchungshorizont $T$. Für das Szenario 5 existieren nur diese fünf Produktanfragen. Der Ertrag $r_j$ für die Produktanfragen $j\in\mathcal{J}$ ist dabei gestaffelt. Der Ertrag $r_j$ der Anfrage $j=1$ beträgt $200$ GE und für die weiteren Produktanfragen steigt der Ertrag jeweils um $200$ GE. D. h. für die letzte Produktanfrage $j=5$ wird ein Ertrag in Höhe von $r_5=1000$ erzielt. Bei dem Szenario 5 wird wieder eine linear aufsteigende Wahrscheinlichkeitsfunktion $p_j(t)$ für eine jeden Produktanfrage $j$ verwendet. Tabelle \ref{S5} zeigt die verwendeten Parameter und die Abbildung \ref{SB5} zeigt die Eintrittswahrscheinlichkeiten der Anfragen für das Szenario 5.\\

%%%%%%%%%%% DP2_A_Lang
\begin{table}[h!]
\renewcommand{\arraystretch}{1.5}
  \begin{center}
    \caption{Szenario 5}  \label{S5}
    \vspace*{3mm}
    \begin{tabular}{l l l l l l}   %hier die Spaltenausrichtung, -breite, -begrenzung und -anzahl eintragen
    $T$ & $\hat T$  & $h$ & $c_h^{\hat t}\forall \hat{t}\in{\hat T}$ & $y_h^{\hat t}\forall \hat{t}\in{\hat T}$  & $y_h^{{\hat t},max}\forall \hat{t}\in{\hat T}$  \\  \hline
100 & 5 & 1 & 1 & 0 & 2  \\ \hline
    \end{tabular} \\[3mm]
        \begin{tabular}{p{.5cm} p{.5cm} p{.5cm}  p{.5cm} p{9cm}}   %hier die Spaltenausrichtung, -breite, -begrenzung und -anzahl eintragen
    $j$ & $r_j$  & $a_{1j}$ & $\hat t$ & Verteilung $p_j(t)$ \\  \hline
1 & 200 & 1 & 1 & Linear ansteigend auf $p_j(t)=0,3$ von 100 bis 81   \\
2 & 400 & 1 & 2 & Linear ansteigend auf $p_j(t)=0,3$ von 100 bis 61  \\
3 & 600 & 1 & 3 & Linear ansteigend auf $p_j(t)=0,3$ von 100 bis 41  \\
4 & 800 & 1 & 4 & Linear ansteigend auf $p_j(t)=0,3$ von 80 bis 21  \\
5 & 1000 & 1 & 5 & Linear ansteigend auf $p_j(t)=0,3$ von 40 bis 1 \\
\hline
    \end{tabular} \\[3mm]
     \begin{tabular}{p{7cm}p{5cm}} \hline
     Rechenzeit Systemzustände (h): & \texttt{0:00:00.111966} \\
     Anzahl möglicher Systemzustände: & \texttt{785376} \\
     Anzahl benötigter Systemzustände: & \texttt{17667} \\ 
     Rechenzeit DP (h): & \texttt{1:01:44.564707} \\ 
          Max. Erwartungswert (GE): & \texttt{4606.34} \\ \hline
         \end{tabular} \\[3mm]
  %  {\footnotesize \textbf{In Anlehnung an:} \cite{quante2009management}, S. 44.}\\
        % {\footnotesize \textbf{Quelle:} \url{http://www.rrzn.uni-hannover.de/scientific_computing_doku.html} }   %footnotesize liefert Schrift in Größe 10pt
  \end{center}
\end{table}

\begin{figure}[h!]
  \begin{center}
    \includegraphics[width=80mm, trim=300pt 180pt 300pt 0pt]{/Users/Superuser/DP-RM-with-storage/cluster/DP2_A_Lang/Wverteilung.png}
    \caption{Wahrscheinlichkeitsverteilung Szenario 5}  \label{SB5}
       % {\footnotesize \textbf{Quelle:} ????} 
    %{\footnotesize \textbf{Legende:} Annahme einer Produktauftrag entspricht '$j$', KA='Kein Auftrag'} 
  \end{center}
\end{figure}

\begin{table}[h!]
\renewcommand{\arraystretch}{1.5}
  \begin{center}
    \caption{Auswertung des Szenarios 5}  \label{AS5}
    \vspace*{3mm}
    %%%%%%%%%%%%%%% d[0]
    \begin{tabular}{l l l l l l l l l l l l }  \hline 
         $j$ & 0 & 1  & 2 & 3 & 4  & 5   \\  \hline
$d_{0}$ &  14480 &    24 &   109 &  398 &   2292 &  0 \\
$d_{0}$ (\%) &  83.68 &  0.14 &  0.63 &  2.3 &  13.25 &  0 \\
\hline
    \end{tabular} \\[3mm]
        \begin{tabular}{ l l l l l l l l l}   \hline    %%%%%%%%%%%%%%% d[0]
    $d_j$ & \multicolumn{2}{c}{Ablehnung (0)} & \multicolumn{2}{c}{Annahme (1)}  & \multicolumn{2}{c}{Lagerentnahme (2)} & \multicolumn{2}{c}{Lagerproduktion (3)}\\
    & Anz. & \% & Anz. & \% & Anz. & \% & Anz. & \% \\ \hline 
$d_{1}$ &  17279 &  99.86 &   0 &    0 &   0 &   0 &    24 &   0.14 \\
$d_{2}$ &  17189 &  99.34 &     5 &   0.03 &   0 &   0 &   109 &   0.63 \\
$d_{3}$ &  16870 &   97.5 &    35 &    0.2 &   0 &   0 &   398 &   2.29 \\
$d_{4}$ &  14639 &   84.6 &    78 &   0.45 &   294 &  1.69 &  2292 &  13.18 \\
$d_{5}$ &   6713 &   38.8 &  6922 &  39.92 &  3668 &  21.1 &   0 &    0 \\
          \hline
   \end{tabular} \\[3mm]
     \end{center}
\end{table}

In diesem Szenario zeigt sich, dass die optimale Politik die Annahme der ertragreichen Anfrage $j=5$ ist, sofern diese eintrifft ($d_5=1$).  Für die anderen Anfragen $j<5$ gilt eher die optimale Politik der Lagerproduktion anstelle der Auftragsannahme. Durch eine Lagerproduktion wäre die erneute Annahme der Produktanfrage $j=5$ im Betrachtungszeitraum möglich. Sofern keine Anfragen eintreffen ($d_0$), wird tendenziell zum Ende des Buchungshorizonts $T$ die optimale Politik der Lagerproduktion vom Modell bestimmt.

\begin{figure}[h!]     
\begin{center}
\includegraphics[width=35mm, trim=0pt 20pt 20pt 0pt, clip]{/Users/Superuser/DP-RM-with-storage/cluster/DP2_A_Lang/OP-0.png}
\includegraphics[width=140mm, trim=125pt 0pt 110pt 10pt, clip]{/Users/Superuser/DP-RM-with-storage/cluster/DP2_A_Lang/OP-J.png}
    \caption{Veränderung der optimale Politik für das Szenario 5 sofern keine Anfragen eintreffen ($d_0=0$) und je Produktanfrage bei konstanter Ressourcenkapazität ($d_j\text{ entspricht }1\text{ bis }5$)}  \label{SV5}
  \end{center}
\end{figure}

Abbildung \ref{SV5} zeigt die Veränderung der optimale Politik für das Szenario 5 für den vordefinierten Verlauf der Auftragseingänge. Bei diesem und den nachfolgenden Szenarien muss für eine jeweilige Produktanfrage $j$ beachtet werden, dass die Anzahl der Produkte reduziert ist und die Anfragen zwar um die gleiche Ressource konkurrieren, aber keine gleich verteilte Produktanfrage mit anderem Ertragswert existiert. Für die optimale Politik $d_0$ wird vor dem jeweiligen Leistungserstellungszeitpunkt $\hat t$ die Entscheidung der Lagerproduktion ermittelt. Mit Ausnahme der Produktanfrage $j=5$, für die weiterhin keine Lagerproduktion möglich ist. Für die Produktanfrage $j=4$ wird wiederum für den gesamten Zeitraum $40\ge\hat t\ge 21$ die optimale Politik der Lagerproduktion ermittelt. Bei der optimalen Politik $d_j$ wird für Produktanfragen $j<5$ jeweils die Lagerproduktion vom Modell bestimmt. Diese optimale Politik kommt zustande, da die Produktanfrage $j=5$ den höchsten Ertrags $r_j$ erzielt. Sofern ein Lagerbestand $\textbf{y}^{\hat t}$ verfügbar ist, kann die Akzeptanz dieser ertragreichen Produktanfrage $j=5$ für den Buchungsabschnitt $40\ge\hat t\ge1$ mit einer höheren Wahrscheinlichkeit erfolgen. Damit ist die optimale Politik $d_5$ die Annahme der Produktanfrage über die gesamte Wahrscheinlichkeitsverteilung $p_5(t)$.

\textbf{Szenario 6}

Das Szenario 6 zeichnet sich durch eine normalverteilte Wahrscheinlichkeitsverteilung $p_j(t)$ aus. Dabei weist die Anfrage $j=1$ einen Erwartungswert von $\mu=t=90$ und eine Standardabweichung von $\sigma=1$. Für die weiteren Produktanfragen ist der Erwartungswert $\mu$ jeweils um $t=20$ Perioden verschoben. Die Standardabweichung $\sigma$ variiert jedoch je Produktanfrage $j$ und wird zur Normierung über den gesamten Buchungshorizont $T$ mit einem Skala multipliziert. Damit besitzt eine jeweilige Verteilung $p_j(t)$ höhere Wahrscheinlichkeitswerte über den gesamten Betrachtungszeitraum. Abbildung \ref{SB6} zeigt diese Verteilung. Die fünf möglichen Produktanfragen $j\in\mathcal{J}$ haben für die anderen Parameter die Werte des Szenarios 5 hinterlegt. Tabelle \ref{S6} zeigt die Parameter für das Szenario 6.

%%%%%%%%%%% DP2_N_Lang
\begin{table}[h!]
\renewcommand{\arraystretch}{1.5}
  \begin{center}
    \caption{Szenario 6}  \label{S6}
    \vspace*{3mm}
    \begin{tabular}{l l l l l l}   %hier die Spaltenausrichtung, -breite, -begrenzung und -anzahl eintragen
    $T$ & $\hat T$  & $h$ & $c_h^{\hat t}\forall \hat{t}\in{\hat T}$ & $y_h^{\hat t}\forall \hat{t}\in{\hat T}$  & $y_h^{{\hat t},max}\forall \hat{t}\in{\hat T}$  \\  \hline
100 & 5 & 1 & 1 & 0 & 2  \\ \hline
    \end{tabular} \\[3mm]
        \begin{tabular}{p{1cm} p{1cm} p{1cm}  p{1cm} p{6cm}}   %hier die Spaltenausrichtung, -breite, -begrenzung und -anzahl eintragen
    $j$ & $r_j$  & $a_{1j}$ & $\hat t$ & Verteilung $p_j(t)$ \\  \hline
1 & 200 & 1 & 1 & Normalverteilt $\mathcal{N}(90, 1)$   \\
2 & 400 & 1 & 2 & Normalverteilt $10\cdot\mathcal{N}(70, 10)$  \\
3 & 600 & 1 & 3 & Normalverteilt $10\cdot\mathcal{N}(50, 10)$  \\
4 & 800 & 1 & 4 & Normalverteilt $10\cdot\mathcal{N}(30, 10)$  \\
5 & 1000 & 1 & 5 & Normalverteilt $30\cdot\mathcal{N}(10, 30)$ \\
\hline
    \end{tabular} \\[3mm]
     \begin{tabular}{p{7cm}p{5cm}} \hline
     Rechenzeit Systemzustände (h): & \texttt{0:00:03.354823} \\
     Anzahl möglicher Systemzustände: & \texttt{785376} \\
     Anzahl benötigter Systemzustände: & \texttt{35324} \\ 
     Rechenzeit DP (h): & \texttt{3:49:23.317486} \\ 
          Max. Erwartungswert (GE): & \texttt{4690.53} \\ \hline
         \end{tabular} \\[3mm]
  \end{center}
\end{table}

\begin{figure}[h!]
  \begin{center}
    \includegraphics[width=80mm, trim=300pt 180pt 300pt 0pt]{/Users/Superuser/DP-RM-with-storage/cluster/DP2_N_Lang/Wverteilung.png}
    \caption{Wahrscheinlichkeitsverteilung Szenario 6}  \label{SB6}
  \end{center}
\end{figure}

\begin{table}[h!]
\renewcommand{\arraystretch}{1.5}
  \begin{center}
    \caption{Auswertung des Szenarios 6}  \label{AS6}
    \vspace*{3mm}
    %%%%%%%%%%%%%%% d[0]
    \begin{tabular}{l l l l l l l l l l l l }  \hline 
         $j$ & 0 & 1  & 2 & 3 & 4  & 5   \\  \hline
$d_{0}$ &  29249 &   701 &  1040 &  1476 &  2494 &  0 \\
$d_{0}$ (\%) &  83.66 &  2.01 &  2.97 &  4.22 &  7.13 &  0 \\
\hline
    \end{tabular} \\[3mm]
        \begin{tabular}{ l l l l l l l l l}   \hline    %%%%%%%%%%%%%%% d[0]
    $d_j$ & \multicolumn{2}{c}{Ablehnung (0)} & \multicolumn{2}{c}{Annahme (1)}  & \multicolumn{2}{c}{Lagerentnahme (2)} & \multicolumn{2}{c}{Lagerproduktion (3)}\\
    & Anz. & \% & Anz. & \% & Anz. & \% & Anz. & \% \\ \hline 
$d_{1}$ &  33949 &  97.11 &    310 &   0.88 &    0 &    0 &   701 &     2 \\
$d_{2}$ &  33749 &  96.54 &    159 &   0.45 &    0 &    0 &  1052 &     3 \\ 
$d_{3}$ &  32632 &  93.34 &    211 &    0.6 &    516 &   1.47 &  1601 &  4.57 \\
$d_{4}$ &  30601 &  87.53 &     48 &   0.14 &   1619 &   4.62 &  2692 &  7.68 \\
$d_{5}$ &   9223 &  26.38 &  15587 &  44.55 &  10150 &  28.97 &   0 &   0 \\
          \hline
   \end{tabular} \\[3mm]
     \end{center}
\end{table}

Tabelle \ref{AS6} zeigt die Auswertung für das Szenario 6. Unter Beachtung einer solchen Verteilung der Eintrittswahrscheinlichkeit ist die optimale Politik $d_0$ bzgl. einer Lagerproduktion für über die Buchungsperioden $t\in T$ relativ ähnlich untereinander verteilt. Die Annahme von Produktanfragen $j<5$ scheint für das Modell nicht zur optimalen Politik zu gehören, wobei die Annahme der Produktanfrage $j=5$ relativ häufig zur optimalen Politik gewählt ist. Das Modell sieht eher für die Produktanfragen $j<5$ die Lagerproduktion vor, als die Annahme der Anfragen mit tendenziell niedrigerem Ertrag. % Was ist mit keine Anfragen????

\begin{figure}[h!]     
\begin{center}
\includegraphics[width=35mm, trim=0pt 20pt 20pt 0pt, clip]{/Users/Superuser/DP-RM-with-storage/cluster/DP2_N_Lang/OP-0.png}
\includegraphics[width=140mm, trim=125pt 0pt 110pt 10pt, clip]{/Users/Superuser/DP-RM-with-storage/cluster/DP2_N_Lang/OP-J.png}
    \caption{Veränderung der optimale Politik für das Szenario 6 sofern keine Anfragen eintreffen ($d_0=0$) und je Produktanfrage bei konstanter Ressourcenkapazität ($d_j\text{ entspricht }1\text{ bis }5$)}  \label{SV6}
  \end{center}
\end{figure}

Wird die Abbildung \ref{SV6} betrachtet, dann zeigt sich für das Szenario 6 eine ähnliche Veränderung der optimalen Politik wie im Szenario 3. Beide Szenarien haben eine Normalverteilung unterstellt und weisen bei den Verläuft der optimalen Politik je Anfrage einen punktuellen Wechsel zwischen den Entscheidungsalternativen. Die grundsätzliche Strategie scheint unter Verwendung der ertragsarmen Produktanfragen die \glqq Lagerproduktion{\grqq} zu sein. D. h. die optimale Politik ist es die Kapazitäten vor dem Leistungserstellungszeitpunkt zum Lagerbestand zu transformiert, sofern keine Produktanfragen eintreffen und sofern eine jeweilige Produktanfrage $j\in\mathcal{J}$ eintrifft. Damit steigt die Wahrscheinlichkeit im weiteren Zeitverlauf die ertragreiche Produktanfrage $j=5$ akzeptieren zu können. Dies erhöht den möglichen Gesamtertrag des Unternehmens. Die optimale Politik für Produktanfragen $j<5$ schwankt dabei aufgrund der Normalverteilung zwischen der \glqq Lagerproduktion{\grqq} und der \glqq Ablehnung des Auftrags{\grqq} in gewissen Buchungsabschnitten. Bspw. zeigt dies der Verlauf der optimalen Politik $d_3$ für Anfragen $j=3$. Im Umfeld der Buchungsperiode $t=80$ wechseln die Entscheidungsalternativen untereinander mehrfach und im Bereich der Buchungsperiode $t=60$ ist über einen längeren Buchungsabschnitt die \glqq Ablehnung des Auftrags{\grqq} die optimale Politik. Wird die Verteilung des Szenarios in Abbildung \ref{SB6} betrachtet, dann kann dieser Verlauf der optimalen Politik erklärt werden. Im Buchungsabschnitt um die Periode $t=80$ hat die Produktanfrage $j=2$ ihren höchsten Erwartungswert und im Buchungsabschnitt um die Periode $t=60$ tritt erstmalig die Wahrscheinlichkeit einer Produktanfrage $j=4$ ein. Wird dann zusätzlich noch die ansteigende Wahrscheinlichkeit der ertragreichen Produktanfrage $j=5$ herangezogen, dann wird der Ertragswert, sofern eine Produktanfrage $j=3$ zur Lagerproduktion Verwendung finden soll, jeweils in diesen Buchungsabschnitten durch die Entscheidungsalternative überlagert.\\[.5cm]

\textbf{Szenario 7}

In dem Szenario 7 wird unterstellt, dass am Anfang relativ gleichwertige Anfragen existieren und das Eintreffen dieser Anfragen nochmalverteilt ist. Es gibt aber zusätzlich die Möglichkeit, die auch relativ wahrscheinlich ist, dass eine höherwertige Anfrage $j=5$ eintrifft und einen Ertrag in Höhe von $r_j=5000$ erzielt. Es soll mit dem Modell getestet werden, ob eine solche Anfrage die optimale Politik des Netzwerks beeinflusst. Tabelle \ref{S7} zeigt die verwendeten Parameter und die Abbildung \ref{SB7} zeigt die Wahrscheinlichkeitsverteilung für das Szenario 7.


%%%%%%%%%%%%%% DP3_N_Lang
\begin{table}[h!]
\renewcommand{\arraystretch}{1.5}
  \begin{center}
    \caption{Szenario 7}  \label{S7}
    \vspace*{3mm}
    \begin{tabular}{l l l l l l}   %hier die Spaltenausrichtung, -breite, -begrenzung und -anzahl eintragen
    $T$ & $\hat T$  & $h$ & $c_h^{\hat t}\forall \hat{t}\in{\hat T}$ & $y_h^{\hat t}\forall \hat{t}\in{\hat T}$  & $y_h^{{\hat t},max}\forall \hat{t}\in{\hat T}$  \\  \hline
100 & 5 & 1 & 1 & 0 & 2  \\ \hline
    \end{tabular} \\[3mm]
        \begin{tabular}{p{1cm} p{1cm} p{1cm}  p{1cm} p{6cm}}   %hier die Spaltenausrichtung, -breite, -begrenzung und -anzahl eintragen
    $j$ & $r_j$  & $a_{1j}$ & $\hat t$ & Verteilung $p_j(t)$ \\  \hline
1 & 200 & 1 & 1 & Normalverteilt $10\cdot\mathcal{N}(90, 10)$   \\
2 & 400 & 1 & 2 & Normalverteilt $10\cdot\mathcal{N}(70, 10)$  \\
3 & 200 & 1 & 3 & Normalverteilt $10\cdot\mathcal{N}(50, 10)$  \\
4 & 300 & 1 & 4 & Normalverteilt $10\cdot\mathcal{N}(30, 30)$  \\
5 & 5000 & 1 & 5 & Normalverteilt $30\cdot\mathcal{N}(10, 0.45)$ \\
\hline
    \end{tabular} \\[3mm]
     \begin{tabular}{p{7cm}p{5cm}} \hline
     Rechenzeit Systemzustände (h): & \texttt{0:00:00.097337} \\
     Anzahl möglicher Systemzustände: & \texttt{785376} \\
     Anzahl benötigter Systemzustände: & \texttt{16057} \\ 
     Rechenzeit DP (h): & \texttt{1:02:49.057969} \\ 
          Max. Erwartungswert (GE): & \texttt{6257.86} \\ \hline
         \end{tabular} \\[3mm]
  \end{center}
\end{table}

\begin{figure}[h!]
  \begin{center}
    \includegraphics[width=80mm, trim=300pt 180pt 300pt 0pt]{/Users/Superuser/DP-RM-with-storage/cluster/DP3_N_Lang/Wverteilung.png}
    \caption{Wahrscheinlichkeitsverteilung Szenario 7}  \label{SB7}
  \end{center}
\end{figure}

\begin{table}[h!]
\renewcommand{\arraystretch}{1.5}
  \begin{center}
    \caption{Auswertung des Szenarios 7}  \label{AS7}
    \vspace*{3mm}
    %%%%%%%%%%%%%%% d[0]
    \begin{tabular}{l l l l l l l l l l l l }  \hline 
         $j$ & 0 & 1  & 2 & 3 & 4  & 5   \\  \hline
$d_{0}$ &  14087 &   292 &   259 &   914 &  141 &  0 \\
$d_{0}$ (\%) &  89.77 &  1.86 &  1.65 &  5.82 &  0.9 &  0 \\
\hline
    \end{tabular} \\[3mm]
        \begin{tabular}{ l l l l l l l l l}   \hline    %%%%%%%%%%%%%%% d[0]
    $d_j$ & \multicolumn{2}{c}{Ablehnung (0)} & \multicolumn{2}{c}{Annahme (1)}  & \multicolumn{2}{c}{Lagerentnahme (2)} & \multicolumn{2}{c}{Lagerproduktion (3)}\\
    & Anz. & \% & Anz. & \% & Anz. & \% & Anz. & \% \\ \hline 
$d_{1}$ &  15371 &  97.95 &    30 &   0.19 &   0 &   0 &  292 &  1.85 \\
$d_{2}$ &  13644 &  86.94 &  1403 &   8.89 &   586 &  3.71 &   60 &  0.38 \\
$d_{3}$ &  14690 &  93.61 &    21 &   0.13 &    66 &  0.42 &  916 &   5.8 \\
$d_{4}$ &  12400 &  79.02 &  1695 &  10.75 &  1554 &  9.85 &   44 &  0.28 \\
$d_{5}$ &  12279 &  78.25 &  2298 &  14.57 &  1116 &  7.07 &  0 &   0 \\
          \hline
   \end{tabular} \\[3mm]
     \end{center}
\end{table}

Die Auswertung des Szenarios 7 zeigt in Tabelle \ref{AS7}, dass unter eine solchen Verteilung die optimale Politik der Lagerproduktion von Produktanfragen $j\in\mathcal{J}$ selten bestimmt ist, mit Ausnahme der optimalen Politik der Produktanfrage $j=3$. Für dies Produktanfrage wird häufiger die optimale Politik der Lagerproduktion gewählt ($d_3=3$). Für die anderen Produktanfragen sieht das Modell es eher weniger vor, dass ein Lagerbestand für spätere ertragreiche Produktanfragen aufgebaut wird. Für die optimale Politik $d_0$ wird ebenfalls eher die Produktanfrage $j=3$ für die Lagerproduktion verwendet.\\[.5cm]

\begin{figure}[h!]     
\begin{center}
\includegraphics[width=35mm, trim=0pt 20pt 20pt 0pt, clip]{/Users/Superuser/DP-RM-with-storage/cluster/DP3_N_Lang/OP-0.png}
\includegraphics[width=140mm, trim=125pt 0pt 110pt 10pt, clip]{/Users/Superuser/DP-RM-with-storage/cluster/DP3_N_Lang/OP-J.png}
    \caption{Veränderung der optimale Politik für das Szenario 7 sofern keine Anfragen eintreffen ($d_0=0$) und je Produktanfrage bei konstanter Ressourcenkapazität ($d_j\text{ entspricht }1\text{ bis }5$)}  \label{SV7}
  \end{center}
\end{figure}

Abbildung \ref{SV7} zeigt für das Szenario 7 den Verlauf der optimalen Politik $d_0$ und $d_j\forall j\in\mathcal{J}$. Für die optimale Politik $d_0$ zeigt sich, dass die relativ ertragreichen Produktanfragen $j\in\mathcal{J}$ kurz vor dem Leistungserstellungszeitpunkt $\hat t$ dem Lager ergänzt werden. D. h. eine Entscheidung die Kapazität für diese Produktanfragen $j=2$ und $j=4$ für das Lager zu verwenden und dementsprechend die aus einer Annahme resultierenden Erträge $r_j$ zu verlieren ist erst kurz vor Verderben der Kapazität $\textbf{c}^{\hat t}$ optimal. Bei der optimalen Politik $d_j$ ergibt sich ein differenziertes Bild. Für die ertragsarmen Produktanfragen $j=1$ und $j=3$ ist die Lagerproduktion hauptsächlich die optimale Politik. Bei Anfragen $j=2$ ist die Auftragsannahme die optimale Politik $d_2$ bis zur produktspezifischen Leistungserstellung $\hat t$. Sofern im späteren Verlauf die sehr ertragreiche Produktanfrage $j=5$ eintrifft, ist die optimale Politik die Annahme eine solchen Anfrage. Die optimale Politik $d_j$ für eine Produktanfrage $j=4$ ist hauptsächlich die Annahme der Anfrage. Treffen jedoch keine Anfragen ein, dann ist die optimale Politik kurz vor Verlust der Möglichkeit die Ablehnung einer eintreffenden Anfrage inkl. der Lagerproduktionsentscheidung. Damit steigt die Wahrscheinlichkeit nicht nur einen Ertrag von $r_4=300$ zu erzielen, sondern einen Ertrag in Höhe von $r_5=5000$.

\textbf{Szenario 8}

Beim Szenario 8 wird eine veränderte Normalverteilung bei den Eintrittswahrscheinlichkeiten $p_j(t)$ der Produktanfragen $j\in\mathcal{J}$ angenommen. Die Wahrscheinlichkeit des Eintreffens der Produktanfragen ist am Anfang des Buchungshorizont am größten. D. h. der Erwartungswert $\mu$ einer Produktanfrage $j$ liegt eher in den anfänglichen Buchungsperioden $t$. Abbildung \ref{SB8} zeigt den Kurvenverlauf der Eintrittswahrscheinlichkeiten $p_j(t)$ der Produktanfragen $j\in\mathcal{J}$. Die Erträge $r_j$ der Produktanfragen $j\in\mathcal{J}$ ist bei dem Szenario 5 in der Form verteilt, dass die ersten drei Produktanfragen $j$ einer eher niedrigen Ertrag und die letzten zwei Produktanfragen $j$ einen eher hohen Ertrag erzielen. Tabelle \ref{S8} zeigt die Parameter für das Szenario 8.

%%%%%%%%%%%%%% DP4_N_Lang_Anf
\begin{table}[h!]
\renewcommand{\arraystretch}{1.5}
  \begin{center}
    \caption{Szenario 8}  \label{S8}
    \vspace*{3mm}
    \begin{tabular}{l l l l l l}   %hier die Spaltenausrichtung, -breite, -begrenzung und -anzahl eintragen
    $T$ & $\hat T$  & $h$ & $c_h^{\hat t}\forall \hat{t}\in{\hat T}$ & $y_h^{\hat t}\forall \hat{t}\in{\hat T}$  & $y_h^{{\hat t},max}\forall \hat{t}\in{\hat T}$  \\  \hline
100 & 5 & 1 & 1 & 0 & 2  \\ \hline
    \end{tabular} \\[3mm]
        \begin{tabular}{p{1cm} p{1cm} p{1cm}  p{1cm} p{6cm}}   %hier die Spaltenausrichtung, -breite, -begrenzung und -anzahl eintragen
    $j$ & $r_j$  & $a_{1j}$ & $\hat t$ & Verteilung $p_j(t)$ \\  \hline
1 & 200 & 1 & 1 & Normalverteilt $10\cdot\mathcal{N}(90, 10)$   \\
2 & 300 & 1 & 2 & Normalverteilt $10\cdot\mathcal{N}(80, 10)$  \\
3 & 800 & 1 & 3 & Normalverteilt $10\cdot\mathcal{N}(70, 10)$  \\
4 & 2500 & 1 & 4 & Normalverteilt $10\cdot\mathcal{N}(50, 30)$  \\
5 & 5000 & 1 & 5 & Normalverteilt $10\cdot\mathcal{N}(20, 10)$ \\
\hline
    \end{tabular} \\[3mm]
     \begin{tabular}{p{7cm}p{5cm}} \hline
     Rechenzeit Systemzustände (h): & \texttt{0:00:00.133609} \\
     Anzahl möglicher Systemzustände: & \texttt{785376} \\
     Anzahl benötigter Systemzustände: & \texttt{35324} \\ 
     Rechenzeit DP (h): & \texttt{2:54:15.295931} \\ 
          Max. Erwartungswert (GE): & \texttt{22428.8} \\ \hline
         \end{tabular} \\[3mm]
  \end{center}
\end{table}

\begin{figure}[h!]
  \begin{center}
    \includegraphics[width=80mm, trim=300pt 180pt 300pt 0pt]{/Users/Superuser/DP-RM-with-storage/cluster/DP4_N_Lang_Anf/Wverteilung.png}
    \caption{Wahrscheinlichkeitsverteilung Szenario 8}  \label{SB8}
  \end{center}
\end{figure}

\begin{table}[h!]
\renewcommand{\arraystretch}{1.5}
  \begin{center}
    \caption{Auswertung des Szenarios 8}  \label{AS8}
    \vspace*{3mm}
    %%%%%%%%%%%%%%% d[0]
    \begin{tabular}{l l l l l l l l l l l l }  \hline 
         $j$ & 0 & 1  & 2 & 3 & 4  & 5   \\  \hline
$d_{0}$ &  26733 &  1069 &  2466 &  1712 &  2980 &  0 \\
$d_{0}$ (\%) &  76.47 &  3.06 &  7.05 &   4.9 &  8.52 &  0 \\
\hline
    \end{tabular} \\[3mm]
        \begin{tabular}{ l l l l l l l l l}   \hline    %%%%%%%%%%%%%%% d[0]
    $d_j$ & \multicolumn{2}{c}{Ablehnung (0)} & \multicolumn{2}{c}{Annahme (1)}  & \multicolumn{2}{c}{Lagerentnahme (2)} & \multicolumn{2}{c}{Lagerproduktion (3)}\\
    & Anz. & \% & Anz. & \% & Anz. & \% & Anz. & \% \\ \hline 
$d_{1}$ &  33713 &  96.43 &    176 &    0.5 &    0 &    0 &  1071 &  3.06 \\
$d_{2}$ &  32271 &  92.31 &    147 &   0.42 &    0 &    0 &  2542 &  7.25 \\
$d_{3}$ &  32923 &  94.17 &    171 &   0.49 &    154 &   0.44 &  1712 &  4.88 \\
$d_{4}$ &  30042 &  85.93 &     48 &   0.14 &   1584 &   4.52 &  3286 &  9.38 \\
$d_{5}$ &   8557 &  24.48 &  15990 &  45.71 &  10413 &  29.73 &   0 &   0 \\
          \hline
   \end{tabular} \\[3mm]
     \end{center}
\end{table}

\begin{figure}[h!]     
\begin{center}
\includegraphics[width=35mm, trim=0pt 20pt 20pt 0pt, clip]{/Users/Superuser/DP-RM-with-storage/cluster/DP4_N_Lang_Anf/OP-0.png}
\includegraphics[width=140mm, trim=125pt 0pt 110pt 10pt, clip]{/Users/Superuser/DP-RM-with-storage/cluster/DP4_N_Lang_Anf/OP-J.png}
    \caption{Veränderung der optimale Politik für das Szenario 8 sofern keine Anfragen eintreffen ($d_0=0$) und je Produktanfrage bei konstanter Ressourcenkapazität ($d_j\text{ entspricht }1\text{ bis }5$)}  \label{SV8}
  \end{center}
\end{figure}

Auch die Auswertung des Szenarios 8 in Tabelle \ref{AS8} zeigt, dass relativ selten die Annahme von Produktanfragen $j<5$ als optimale Politik vom Modell gewählt ist. Für die Produktanfrage $j=5$ wird tendenziell häufig die optimale Politik der Auftragsannahme vom Modell bestimmt. Im Vergleich zum Szenario 7 stieg die Anzahl der optimalen Politik bzgl. der Lagerproduktion auf ein Niveau der anderen Szenarien. Sofern keine Anfragen eintreffen, zeigt sich bei der optimalen Politik $d_0$, dass mit ablaufenden Buchungshorizont $T$ die Anzahl der optimalen Politik bzgl. der Lagerproduktion steigt.

Unter Beachtung der Eintrittswahrscheinlichkeit $p_j(t)$ des Szenarios 7 ergibt sich der in der Abbildung \ref{SV8} aufgezeigte Verlauf der optimale Politik sofern keine Anfragen akzeptiert werden. Der Verlauf zeigt, dass für die optimale Politik $d_0$ jeweils der Ausführungsmodus der zur bevorstehende Leistungserstellung $\hat t$ zugehörigen Produktanfrage $j$ für eine mögliche Lagerproduktion verwendet wird. Auch für Produktanfragen $j\in\mathcal{J}$ ergibt sich dieses Bild. Die ertragsarmen Produktanfragen $j<5$ werden für die Lagerproduktion verwendet, sofern diese eintreffen. Selbst für die Produktanfrage $j=4$ wird eine Lagerproduktion als optimale Politik ermittelt, obwohl diese Produktanfrage einen relativ hohen Ertrag $r_4=2500$ erzielen würde. 

\textbf{Szenario 9}

Im Szenario 9 weisen die Produktanfragen $j\in\mathcal{J}$ eine Cauchyverteilung auf, die den höchsten Wahrscheinlichkeitswert $p_j(t)$ in der letztmöglichen Buchungsperiode $t$ besitzt. Damit wird unterstellt, dass die Nachfrage in Annäherung des Leistungserstellungszeitpunkts $\hat t$ einer Produktanfrage $j$ stark ansteigt. Die Erträge $r_j$ sind für dieses Szenario gestaffelt um $200$ GE und verlaufen von $r_1=200$ bis $r_5=1000$. Die Parameter des Szenarios sind in der Tabelle \ref{S9} aufgeführt und die Abbildung \ref{SB9} zeigt die Wahrscheinlichkeitsverteilung für die Produktanfragen $j\in\mathcal{J}$. 

%%%%%%%%%%%%%%%%%%%%% DP2_C_Lang
\begin{table}[h!]
\renewcommand{\arraystretch}{1.5}
  \begin{center}
    \caption{Szenario 9}  \label{S9}
    \vspace*{3mm}
    \begin{tabular}{l l l l l l}   %hier die Spaltenausrichtung, -breite, -begrenzung und -anzahl eintragen
    $T$ & $\hat T$  & $h$ & $c_h^{\hat t}\forall \hat{t}\in{\hat T}$ & $y_h^{\hat t}\forall \hat{t}\in{\hat T}$  & $y_h^{{\hat t},max}\forall \hat{t}\in{\hat T}$  \\  \hline
100 & 5 & 1 & 1 & 0 & 2  \\ \hline
    \end{tabular} \\[3mm]
        \begin{tabular}{p{1cm} p{1cm} p{1cm}  p{1cm} p{6cm}}   %hier die Spaltenausrichtung, -breite, -begrenzung und -anzahl eintragen
    $j$ & $r_j$  & $a_{1j}$ & $\hat t$ & Verteilung $p_j(t)$ \\  \hline
1 & 200 & 1 & 1 & Cauchyverteilt $5\cdot\mathcal{C}(81, 5)$   \\
2 & 400 & 1 & 2 & Cauchyverteilt $5\cdot\mathcal{C}(61, 5)$  \\
3 & 600 & 1 & 3 & Cauchyverteilt $5\cdot\mathcal{C}(41, 5)$  \\
4 & 800 & 1 & 4 & Cauchyverteilt $5\cdot\mathcal{C}(21, 5)$  \\
5 & 1000 & 1 & 5 & Cauchyverteilt $5\cdot\mathcal{C}(1, 5)$ \\ \hline
    \end{tabular} \\[3mm]
     \begin{tabular}{p{7cm}p{5cm}} \hline
     Rechenzeit Systemzustände (h): & \texttt{0:00:00.117737} \\
     Anzahl möglicher Systemzustände: & \texttt{785376} \\
     Anzahl benötigter Systemzustände: & \texttt{35324} \\ 
     Rechenzeit DP (h): & \texttt{3:04:07.716158} \\ 
          Max. Erwartungswert (GE): & \texttt{3809.91} \\ \hline
         \end{tabular} \\[3mm]
  %  {\footnotesize \textbf{In Anlehnung an:} \cite{quante2009management}, S. 44.}\\
        % {\footnotesize \textbf{Quelle:} \url{http://www.rrzn.uni-hannover.de/scientific_computing_doku.html} }   %footnotesize liefert Schrift in Größe 10pt
  \end{center}
\end{table}

\begin{figure}[h!]
  \begin{center}
    \includegraphics[width=80mm, trim=300pt 180pt 300pt 0pt]{/Users/Superuser/DP-RM-with-storage/cluster/DP2_C_Lang/Wverteilung.png}
    \caption{Wahrscheinlichkeitsverteilung Szenario 9}  \label{SB9}
       % {\footnotesize \textbf{Quelle:} ????} 
    %{\footnotesize \textbf{Legende:} Annahme einer Produktauftrag entspricht '$j$', KA='Kein Auftrag'} 
  \end{center}
\end{figure}

\begin{table}[h!]
\renewcommand{\arraystretch}{1.5}
  \begin{center}
    \caption{Auswertung des Szenarios 9}  \label{AS9}
    \vspace*{3mm}
    %%%%%%%%%%%%%%% d[0]
    \begin{tabular}{l l l l l l l l l l l l }  \hline 
         $j$ & 0 & 1  & 2 & 3 & 4  & 5   \\  \hline
$d_{0}$ &  23029 &  1037 &  2370 &   3981 &   4543 &  0 \\
$d_{0}$ (\%) &  65.87 &  2.97 &  6.78 &  11.39 &  12.99 &  0 \\
\hline
    \end{tabular} \\[3mm]
        \begin{tabular}{ l l l l l l l l l}   \hline    %%%%%%%%%%%%%%% d[0]
    $d_j$ & \multicolumn{2}{c}{Ablehnung (0)} & \multicolumn{2}{c}{Annahme (1)}  & \multicolumn{2}{c}{Lagerentnahme (2)} & \multicolumn{2}{c}{Lagerproduktion (3)}\\
    & Anz. & \% & Anz. & \% & Anz. & \% & Anz. & \% \\ \hline 
$d_{1}$ &  33604 &  96.12 &    285 &   0.81 &    0 &    0 &  1071 &   3.06 \\
$d_{2}$ &  31847 &   91.1 &    447 &   1.28 &     14 &   0.04 &  2652 &   7.57 \\
$d_{3}$ &  28318 &     81 &   1438 &    4.1 &    616 &   1.76 &  4588 &  13.09 \\
$d_{4}$ &  21140 &  60.47 &   4906 &  14.01 &   4073 &  11.63 &  4841 &  13.81 \\
$d_{5}$ &   8392 &     24 &  15990 &  45.71 &  10578 &   30.2 &   0 &    0 \\
          \hline
   \end{tabular} \\[3mm]
     \end{center}
\end{table}

Die Auswertung des Szenarios zeigt, dass erneut die Produktanfrage $j=5$ mit einem hohen Ertrag $r_j$ relativ häufig als optimale Politik der Auftragsannahme vom Modell gewählt wird. Eine Lagerhaltungsentscheidung wird anders als bei den vorherigen Szenario häufiger als optimale Politik ermittelt. Unter der Annahme einer solchen Verteilung wird öfters eine solche Politik neben der Politik der Auftragsannahme aufgezeigt. Sofern keine Anfragen eintreffen, wird ebenfalls eine Steigung der optimale Politik $d_o$ für eine mögliche Lagerproduktion festgestellt. Tabelle \ref{AS9} zeigt die komplette Auswertung des Szenarios 9.

\begin{figure}[h!]     
\begin{center}
\includegraphics[width=35mm, trim=0pt 20pt 20pt 0pt, clip]{/Users/Superuser/DP-RM-with-storage/cluster/DP2_C_Lang/OP-0.png}
\includegraphics[width=140mm, trim=125pt 0pt 110pt 10pt, clip]{/Users/Superuser/DP-RM-with-storage/cluster/DP2_C_Lang/OP-J.png}
    \caption{Veränderung der optimale Politik für das Szenario 9 sofern keine Anfragen eintreffen ($d_0=0$) und je Produktanfrage bei konstanter Ressourcenkapazität ($d_j\text{ entspricht }1\text{ bis }5$)}  \label{SV9}
  \end{center}
\end{figure}

Die optimale Politik für dieses Szenario zeigt einen besonderen Verlauf für $d_0$ und $d_j$. Für die optimale Politik $d_0$, bei der keine Anfragen eintreffen und damit die Wahrscheinlichkeitsverteilung $p_0(t)$) gilt, wechselt die für die Lagerproduktion verwendeten Ausführungsmodi für die Produktanfragen $j\in\mathcal{J}$ in den Buchungsperioden $ t \in T$. Bspw. wechselt die optimale Politik zweimal im Buchungsabschnitt $100\ge t\ge81$ zwischen $d_0=1$ und $d_0=2$. Auch im Buchungsabschnitt $80\ge t\ge41$ wechselt in einem längeren Zeitraum die optimale Politik zwischen $d_0=2$ und $d_0=3$. Im Buchungsabschnitt $40\ge t\ge21$ wird eine optimale Politik für die Lagerproduktion durch den Ausführungsmodus der Anfrage $j=4$ nur kurz vor Verfall der Kapazität $\textbf{c}^{\hat t}$ ermittelt. Für die Produktanfragen $j=1$ und $j=2$ gilt ausschließlich die optimale Politik der Lagerproduktion über den gesamten Buchungshorizont $T$, sofern die Anfragen eine Eintrittswahrscheinlichkeit $p_j(t)>0$ besitzen. Für eine Produktanfrage $j=3$ gilt die optimale Politik der Auftragsannahme für den Buchungsabschnitt $100\ge t\ge81$ und nachfolgend die Auftragsablehnung inkl. einer Lagerproduktion. Sofern eine Anfrage $j=4$ eintrifft, ist die optimale Politik die Auftragsannahme über den möglichen Buchungshorizont $T$, wobei diese optimale Politik im Buchungsabschnitt $40\ge t\ge21$ kurz unterbrochen ist. In diesem Zeitraum ist die Ablehnung die optimale Politik $d_4$. Für Produktanfragen $j=5$ gilt über den gesamten Buchungshorizont $T$ die optimale Politik der Auftragsannahme ($d_5=1$). 




