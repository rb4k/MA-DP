% !TEX encoding = UTF-8 Unicode
\chapter{Schlussbemerkung}
\markboth{7 Schlussbemerkung}{}
\setcounter{footnote}{9}

\section*{Zusammenfassung}

Die Arbeit zeigt, dass eine mögliche Lagerhaltung die Inanspruchnahme der Kapazität von endlichen Ressourcen verbessert und den potentiellen Ertrag erhöht. Die verbesserte Kapazitätsallokation kommt zustande, da die Ressourcenkapazitäten zur Erhöhung eines Lagerbestands Verwendung finden. Dies erfolgt, indem Anfragen zur Instandsetzung von defekten Produkten mit niedrigem Ertrag abgelehnt und durch die Entscheidungen einer Lagerproduktion zur Erhöhung eines Lagerbestands von bereits reparierten Produkten ersetzt werden. Bei dem Lagerbestand handelt es sich um Kapazitäten, die mit hoher Wahrscheinlichkeit für spätere Anfragen Beanspruchung finden, damit eine Maximierung des erwarteten Ertragswerts möglich wird. Durch die Verwendung der Kapazität erfolgt kein Verlust der endlichen Ressource, sondern die Sicherung dieser Kapazität für nachfolgende Produktanfragen in Form des Lagerbestands. Zusätzlich ermöglicht das Modell aus dieser Arbeit auch das Betrachten einer möglichen Lagerproduktion sofern keine Anfrage zu einer beliebigen Periode eintrifft. Damit erhält ein Unternehmen für die Planung der Kapazitätsverwendung mehr Flexibilität und erhält dementsprechend eine umfangreichere Politik der möglichen Auftragsannahmen für den Betrachtungszeitraum.

Die Planung der optimalen Politik ist dabei erheblich abhängig von der Wahrscheinlichkeitsverteilung des Eintreffens und der möglichen Erträge der Produktanfragen. Dies resultiert aus den Gleichungen \eqref{GB3} sowie \eqref{GB4}. Für jede mögliche Entscheidungsalternative der Modellgleichung gibt es bestimmte OK, die durch die Parameter beeinflusst werden. Wird beispielsweise ein Netzwerk betrachtet, indem sich im Verlauf des Buchungshorizonts die Wahrscheinlichkeiten des möglichen Eintreffens der Produktanfragen verändern, dann hat diese Veränderung Einfluss auf die OK. Je wahrscheinlicher der Verfall einer Ressource wird, desto höher ist die Wahrscheinlichkeit, dass eine Lagerproduktion zur Sicherung der Kapazitätsbeanspruchung der Ressource den noch möglichen Ertrag maximiert. Der potentielle Ertrag, der durch die Annahme einer solchen ertragsarmen und auslaufenden Anfrage generiert wird, ist nicht mehr ausreichend rentabel und das Modell empfiehlt die Lagerproduktion als Verwendung für die kurz vor Verfall stehende Kapazität. Damit ist im weiteren Verlauf ein höherer Ertrag für das Netzwerk möglich. Anders formuliert bedeutet dies, dass ab einem solchen Punkt bzw. ab einem solchen Systemzustand der Übergang in einen anderen Systemzustand mit einem vorhandenen Lagerbestand nicht gegen die Grenzbedingungen verstößt und dieser Übergang einen höheren Ertragswert ermöglicht. Für ein solches Szenario existiert damit ein besserer Pfad im Graphen, der potentiell mehr Ertrag verspricht.

Die numerische Untersuchung zeigt, dass die Anzahl der Ausprägungen der optimalen Politiken für die Systemzustände mit der Wahrscheinlichkeitsverteilung der Produktanfragen variiert. Es gab in der Untersuchung kein Szenario, welches eine höhere Anzahl an Lagerproduktionen gegenüber der Auftragsannahmen als optimale Politik aufweist. Auch der Fall, dass die optimale Politik der Lagerproduktion die optimale Politik der Auftragsannahme zumindest für eine bestimmte Art von Produktanfragen dominiert, konnte nicht gezeigt werden. Dieser Sachverhalt wird dadurch erklärt, dass für die optimale Politik der Auftragsannahme (und auch für die Lagerentnahme) die jeweiligen Erträge der Produktanfragen bei dem Entscheidungsmodell Berücksichtigung finden. D. h. die Annahme der Produktanfrage verbessert den maximalen Erwartungswert aufgrund der dadurch generierten Erträge. Die Lagerproduktion ist bei dieser Modellformulierung eine Möglichkeit, die Kapazitäten für nachfolgende Produktanfragen zu sichern und tritt kurz vor dem Verfall der Kapazität auf. Sofern eine bestimmte Reihenfolge der Auftragseingänge unterstellt wird, kann die Untersuchung des Verlaufs der optimalen Politik erfolgen. Die Szenarien zeigen unterschiedliche Verläufe für die optimale Politik, sofern keine Anfragen eintreffen und sofern eine jeweilige Produktanfrage eintrifft. In den Szenarien ist damit gezeigt, dass die Wahrscheinlichkeitsverteilung erheblichen Einfluss auf die optimale Politik hat und damit auf den Zeitpunkt, an dem die Sicherung der Ressourcenkapazität vom Modell empfohlen wird. 

\section*{Limitation}
In dieser Arbeit wird die Lagerproduktion eines Produkts betrachtet, bei der die Instandsetzung von defekten Produkten durch die Inanspruchnahme der Kapazität einer einzelnen Ressource erfolgt. Das Modell ermöglicht die Berücksichtigung von unterschiedlichen Ressourcen, wie z. B. Maschinen- und Personaleinsatzstunden. Für die numerische Untersuchung ist die Vereinfachung auf nur eine Ressource jedoch zielführend. Damit war zum einen die schnelle Berechnung der Szenarien möglich und zum anderen ist dadurch die Funktionsweise des Modells im Grundsatz explorierbar. Jedoch sollte für spätere Analysen auch die Berechnung von umfangreicheren Szenarien mit unterschiedlichen Ressourcen zur Erstellung differenzierter Produkte in Betracht kommen.

Des Weiteren ist in der Modellerweiterung \eqref{time} nur der jeweilige Ausführungsmodus der eintreffenden Produktanfrage für eine mögliche Lagerproduktion möglich. Damit steht dem Netzwerk jeweils nur die Produktionskapazität einer abgelehnten Anfrage zur Verfügung. D. h. ein Unternehmen ist immer gezwungen, die Ressourcenkapazität für die Lagererhöhung zu verwenden, die aktuell als Auftrag vorliegt. Damit geht jedoch Flexibilität in der Planung verloren. Daher sollte bei der Modellformulierung \eqref{time} die Möglichkeit bestehen, einen beliebigen Ausführungsmodus über alle möglichen Produktanfragen für die Lagerproduktion zu nutzen, sofern eine Anfrage abgelehnt wird. Damit wäre ebenfalls bei dieser Entscheidung über die Auftragsannahme einer eintreffenden Produktanfrage das Maximum über alle Produktanfragen zu bilden. Bei der in dieser Arbeit umgesetzten Modellerweiterung ist dies nicht berücksichtigt.

Die Implementierung in das Computersystem bietet weiteres Optimierungspotential. Der Algorithmus ist zwar mit einigen Funktionen des Softwarepakets \texttt{Numpy} entwickelt, aber eine komplette Implementierung in die Programmiersprache \texttt{C++} oder \texttt{Java} könnte die Leistung weiter verbessern bzw. die Rechenzeit reduzieren. Sofern weiter an dieser Implementierung festgehalten wird, empfiehlt es sich, die Memofunktion zu überarbeiten. In der aktuellen Fassung der Implementierung basiert die Memofunktion auf der Programmiersprache \texttt{Python}. Auch hier sollte ein Versuch getätigt werden, eine Memofunktion mit dem Softwarepaket \texttt{Numpy} zu entwickeln. Ggf. kann dadurch die vorherige Berechnung aller möglichen Systemzustände des Netzwerks entfallen. Die Berechnung der möglichen Systemzustände ist zwar sehr effizient umgesetzt, aber die Funktion hat einen hohen Hauptspeicherbedarf. Sofern die Berechnung von umfangreicheren Szenarien erfolgen soll, muss diese Funktion parallelisiert werden. Dadurch erfolg die Berechnung der Funktion auf mehreren Knoten (Computern) des Clustersystems unter Nutzung der jeweiligen Speicher. Die Umsetzung dieser Leistungsverbesserungen des Algorithmus konnte in dieser Arbeit nicht mehr erfolgen.

Weiter ist die verwendete Implementierung des DP in rekursiver Form programmiert. Ein iterativer Lösungsweg könnte die Rechenzeit des Algorithmus verringern. Für eine mögliche iterative sowie für die hier betrachtete rekursive Lösung gilt zusätzlich, dass eine mögliche Parallelisierung der Teilprobleme des Auftragsannahmeproblems die Effizienz des Algorithmus weiter verbessern könnte. In dem begrenzten Zeitraum der Erstellung dieser Arbeit erfolgte keine abschließende Umsetzung einer möglichen Parallelisierung der Problemstellung, da die hier verwendete Implementierung keine unabhängige Berechnung der Teilprobleme ermöglicht. Abschließend lässt sich für die Limitation zusammenfassen, dass die hier gezeigte Implementierung zwar viele aktuelle Funktionen des wissenschaftlichen Rechnens nutzt, es sich aber um den ersten Versuch einer Implementierung des Auftragannahmeproblems des Netzwerk RM unter Berücksichtigung einer Lagerhaltung handelt. Es besteht daher Potential zur Verbesserung der Implementierung.

\section*{Ausblick}

Der Schwerpunkt weiterer Forschung bezüglich des Auftragsannahmeproblems im Netzwerk RM mit Lagerhaltungsentscheidungen sollte die Konkretisierung der Problemstellung sein. Bei der hier gezeigten Modellerweiterung handelt es sich um eine einfache Umformung des Grundmodells des Netzwerk RM (siehe Gleichung \eqref{DP} im Vergleich zu \eqref{time}). Es berücksichtig z. B. keine Lagerkosten, keine zeitgebundene Erhöhung von Ressourcen oder die Ermittlung des konkreten Leistungserstellungszeitpunkts einer angenommenen Anfrage. Daher sollte nachfolgend ein Modell entwickelt werden, welches aus einem tatsächlichen Anwendungsfall resultiert und die Problemstellung der auftragsbezogenen Instandhaltungsprozesse umfangreicher abbildet.

Jedoch zeigt die aktuelle Entwicklung im Forschungsbereich des Netzwerk RM, dass ein exaktes Lösungsverfahren keine reellen Probleme in einer angemessenen Berechnungszeit löst. Die Integration eines Verfahrens der Approximation der Problemstellung sollte daher ebenfalls vorangetrieben werden. Durch Integration von Verfahren, wie z. B. dem Verfahren des Bid-Preises, kann die Berechnungszeit reduziert werden. Damit wäre die Übertragung in moderne Anwendungssystemen des RM möglich. Ebenfalls sollte eine Untersuchung auf mögliche Umsetzung der Ansätze des Literaturüberblicks aus Kapitel \ref{Review} erfolgen. Vielversprechend erscheint die Ermittlung des Bid-Preises anhand eines Knappsack-Verfahrens.

%In dieser Arbeit konnte das dynamisch, stochastische Grundmodell des Netzwerk RM um Entscheidungen einer Lagerhaltung für die Auftragsfertigung 



%Sofern eine Lagerhaltung berücksichtigt wird, verstößt es gegen die Anwendungsvoraussetzung der \glqq Nichtlagerfähigkeit{\grqq} Diese 