% !TEX encoding = UTF-8 Unicode
\chapter{Schlussbemerkung}
\markboth{5 Schlussbemerkung}{}
\setcounter{footnote}{9}

\section*{Zusammenfassung}

Die Arbeit zeigt, dass eine mögliche Lagerhaltung die Inanspruchnahme der Kapazität von endlichen Ressourcen verbessert und den erwarteten Ertrag erhöht. Die verbesserte Kapazitätsallokation kommt zustande, da die Ressourcenkapazitäten zur Erhöhung eines Lagerbestands Verwendung finden. Dies erfolgt indem Anfragen zur Instandsetzung von defekten Produkten mit niedrigem Ertrag abgelehnt und durch die Entscheidungen einer Lagerproduktion zur Erhöhung eines Lagerbestands von bereits reparierten Produkten ersetzt werden. Bei dem Lagerbestand handelt es sich um Kapazitäten, die mit hoher Wahrscheinlichkeit für spätere Anfragen mit höherem Ertragswert Beanspruchung finden müssen, damit eine Maximierung des erwarteten Ertragswerts möglich ist. Durch die Verwendung der Kapazität erfolgt kein Verlust der endlichen Ressource, sondern die Sicherung dieser Kapazität für nachfolgende Produktanfragen. Zusätzlich ermöglich das Modell aus dieser Arbeit auch das Betrachten einer möglichen Lagerproduktion von Produkten sofern keine Anfragen eintreffen. Damit erhält ein Unternehmen für die Planung der Kapazitätsverwendung mehr Flexibilität und erhält dementsprechend eine umfangreichere Politik der möglichen Auftragsannahmen für den Betrachtungszeitraum.

Die Planung der optimalen Politik ist dabei erheblich abhängig von der Wahrscheinlichkeitsverteilung des Eintreffens und der möglichen Erträge der Produktanfragen. Dies resultiert aus der Gleichung \eqref{GB4}. Für jede mögliche Entscheidungsalternative der Modellgleichung gibt es bestimmte OK die durch diese Parameter beeinflusst werden. Im Verlauf des Buchungshorizonts eines beispielhaften Netzwerks verändern sich die Wahrscheinlichkeiten des möglichen Eintreffens der Produktanfragen und damit die OK. Je wahrscheinlicher der Verfall einer Ressource wird, desto höher ist die Wahrscheinlichkeit das eine Lagerproduktion der Ressource den noch möglichen Ertrag maximiert. Der mögliche Ertrag durch Annahme einer Anfrage inkl. der weitere Erwartungswert der noch möglichen Erträge ist nicht mehr ausreichend rentabel und das Modell empfiehlt für die Verwendung der Kapazität die Lagerproduktion. Damit ist im weiteren Verlauf ein höherer Ertrag für das Netzwerk möglich. Anders formuliert bedeutet dies, dass ab einem solchen Punkt bzw. ab einem solchen Systemzustand der Übergang in einen anderen Systemzustand mit einem vorhandenen Lagerbestand nicht gegen die Grenzbedingungnen verstößt und im weiteren Verlauf einen höheren Ertragswert verspricht. Es existiert  damit ein besserer Pfad im Netzwerk-Graphen, der potentiell mehr Ertrag verspricht.

Die numerische Untersuchung zeigt, dass die Anzahl der Ausprägungen der optimalen Politik mit der Wahrscheinlichkeitsverteilung der Produktanfragen variiert. Es gab in der Untersuchung kein Szenario, dass eine höhere Anzahl der optimalen Politik der Lagerproduktion gegenüber der Auftragsannahme aufweist. Auch der Fall, dass die optimale Politik der Lagerproduktion die Auftragsannahme zumindest für eine bestimmte Art von Produktanfragen dominiert konnte nicht gezeigt werden. Dieser Sachverhalt wird dadurch erklärt, dass für die optimale Politik der Auftragsannahme (und auch für die Lagerentnahme) die jeweiligen Erträge der Produktanfragen bei dem Entscheidungsmodell Berücksichtigung finden. Sofern eine bestimmte Reihenfolge der Auftragseingänge unterstellt wird, kann die Untersuchung des Verlaufs der optimalen Politik erfolgen. Die Szenarien zeigen unterschiedliche Verläufe für die optimale Politik, sofern keine Anfragen eintreffen und sofern eine jeweilige Produktanfrage eintrifft. Abhängig der Szenarien ist dabe dargelegt, dass der Zeitpunkt des Verderbens einer Ressourcenkapazität erheblichen Einfluss auf die optimale Politik hat. Abhängig der Eintrittswahrscheinlichkeit und der möglichen Erträge erfolgt die Verwendung der Kapazität zur Lagerproduktion eines Produkts in unterschiedlicher Weise in Bezug auf Inanspruchnahme dieser Option und dem Zeitpunkt des Auftretens dieser Alternative.

\section*{Limitation}

In dieser Arbeit wird die Lagerproduktion eines Produkts betrachtet, bei der die Instandsetzung von defekten Produkten aufgrund der Inanspruchnahme der Kapazität einer einzelnen Ressource erfolgt. Das Modell ermöglicht die Berücksichtigung von unterschiedlichen Ressourcen, wie z. B. Maschinen- und Personaleinsatzstunden. Für die numerische Untersuchung ist die Vereinfachung auf nur eine Ressource jedoch zielführend. Damit war zum einen die schnelle Berechnung der Szenarien möglich und zum anderen ist dadurch die Funktionsweise des Modells im Grundsatz explorierbar. Jedoch sollte für spätere Analysen auch die Berechnung von umfangreicheren Szenarien mit unterschiedlichen Ressourcen zur Erstellung differenzierter Produkte in Betracht kommen.

In der Modellerweiterung \eqref{time} ist die Entscheidung der Lagerproduktion bei eintreffenden Anfragen nur für den jeweiligen Ausführungsmodi des Produktanfrage möglich. Damit steht dem Netzwerk jeweils nur die Produktionskapazität einer abgelehnten Anfrage zur Verfügung. D. h. ein Unternehmen ist immer gezwungen die Ressourcenkapazität für die Lagererhöhung zu verwendet, die aktuell als Auftrag vorliegt. Damit geht jedoch viel Flexibilität verloren. Daher sollte bei der Modellformulierung \eqref{time} die Möglichkeit bestehen, dass sofern eine Anfrage abgelehnt wird, ein beliebiger Ausführungsmodi aller möglichen Produktanfragen für die Lagerproduktion genutzt werden können. Damit wäre ebenfalls bei dieser Entscheidung über die Auftragsannahme einer eintreffenden Produktanfrage das Maximum über alle Produktanfragen zu bilden. Bei der in dieser Arbeit umgesetzten Modellerweiterung ist dieses nicht berücksichtigt.

Die Implementierung in das Computersystem bietet weiteres Optimierungspotential. Der Algorithmus ist zwar mit einigen Funktionen des Softwarepakets \texttt{Numpy} entwickelt, aber eine komplette Implementierung in die Programmiersprache \texttt{C++} oder \texttt{Java} könnte die Leistung weiter verbessern bzw. die Rechenzeit reduzieren. Sofern weiter an dieser Implementierung festgehalten wird, empfiehlt es sich die Memofunktion zu überarbeiten. In der aktuellen Fassung der Implementierung basiert die Memofunktion auf der Programmiersprache \texttt{Python}. Auch hier sollte ein Versuch getätigt werden, eine Memofunktion mit dem Softwarepaket \texttt{Numpy} zu entwickeln. Ggf. kann dadurch die vorherige Berechnung aller möglichen Systemzustände des Netzwerks entfallen. Diese Berechnung aller möglichen Systemzustände ist zwar ein sehr effizienter Algorithmus, aber die Funktion zur Berechnung hat einen hohen Hauptspeicherbedarf. Sofern die Berechnung von umfangreichere Szenarien erfolgen soll, muss diese Funktion parallelisiert werden. Dadurch erfolg die Berechnung der Funktion auf mehreren Knoten (Computern) des Clustersystems unter Nutzung des jeweiligen Speichers. Die Umsetzung dieser Leistungsverbesserungen des Algorithmus konnte in dieser Arbeit nicht mehr erfolgen.

Weiter ist die verwendete Implementierung des DP in rekursiver Form programmiert. Ein iterativer Lösungweg könnte die Leistung des Algorithmus verbessern, aber auch eine mögliche Parallelisierung der Teilprobleme des Auftragsannahmeproblems sollte weiter untersucht werden. In dem begrenzten Zeitraum der Erstellung dieser Arbeit erfolgte keine abschließende Umsetzung einer möglichen Parallelisierung der Problemstellung. Die Implementierung nutzt zwar viele aktuelle Funktionen in Bezug des wissenschaftlichen Rechnens, aber es ist der erste Versuch einer Implementierung des Auftragsannahmeproblems des Netzwerk RM unter Berücksichtigung einer Lagerhaltung. Es besteht daher Potential zur Verbesserung der Implementierung.

\section*{Ausblick}

Der Schwerpunkt weiter Forschung in Bezug des Auftragsannahmeproblems im Netzwerk RM mit Lagerhaltungsentscheidung sollte die Konkretisierung der Problemstellung sein. Bei der hier gezeigten Modellerweiterung handelt es sich um eine einfache Umformung des Grundmodells des Netzwerk RM (siehe Gleichung \eqref{DP} im Vergleich zu \eqref{time}). Es berücksichtig z. B. keine Lagerkosten oder eine zeitgebundene Erhöhung von Ressourcen. Daher sollte nachfolgend ein Modell entwickelt werden, welches aus einem tatsächlichen Anwendungsfall resultiert und die Problemstellung umfangreicher abbildet.

Jedoch zeigt die aktuelle Entwicklung im Forschungsbereich des Netzwerk RM in der Auftragsfertigung, dass ein exaktes Lösungsverfahren keine reellen Probleme in einer angemessenen Berechnungszeit löst. Die Integration eines Verfahrens der Approximation der Problemstellung sollte daher ebenfalls vorangetrieben werden. Durch Integration von Verfahren, wie z. B. dem Verfahren des Bid-Preises, reduziert die Berechnungszeit und ermöglicht die Übertragung in moderne Anwendungssystemen für das RM. Ebenfalls können die Ansätze aus dem Literaturüberblick aus Kapitel \ref{Review} auf mögliche Umsetzung untersucht werden. Vielversprechend erscheint die Ermittlung des Bid-Preises anhand eines Knappsack-Verfahrens.

Diese Arbeit zeigt eine Modellformulierung für den Anwendungsfalls des Auftragsannahmeproblems im Netzwerk RM mit Lagerhaltungsentscheidungen. 



%Sofern eine Lagerhaltung berücksichtigt wird, verstößt es gegen die Anwendungsvoraussetzung der \glqq Nichtlagerfähigkeit{\grqq} Diese 