\chapter{Schlussbemerkung}
\markboth{5 Schlussbemerkung}{}
\setcounter{footnote}{9}

\section*{Zusammenfassung}

Die Arbeit zeigt, dass eine mögliche Lagerproduktion und die Lagerentnahme die Inanspruchnahme der Kapazität von endlichen Ressourcen verbessert. Dies erfolgt durch die Übertragung der Ressourcenkapazitäten, indem Anfragen mit niedrigem Ertrag abgelehnt und die dadurch freiwerdende Produktionskapazität Verwendung findet einen Lagerbestand von Ressourcen für spätere Anfragen mit einem höheren Ertrag zu erstellen. Dadurch können zeitlich später eintreffende Anfragen mit höherem Ertrag durch das Lager bedient werden. Zusätzlich ermöglich das Modell auch die Lagerproduktion von Ressourcen sofern keine Anfragen eintreffen. Damit erhält ein Unternehmen für die Planung der Auftragsannahme mehr Flexibilität.

Die Planung der optimalen Politik ist dabei erheblich abhängig von der Wahrscheinlichkeitsverteilung des Eintretens und der möglichen Erträge der Produktanfragen. Dies resultiert aus der Gleichung \eqref{GB4}. Für jede mögliche Entscheidungsalternative der Modellgleichung gibt es bestimmte OK die beeinflusst werden durch die jeweiligen Parameter. Im Verlauf des Buchungshorizont eines beispielhaften Netzwerks verändern diese Parameter die Ausprägung der OK. Je wahrscheinlicher der Verfall von Ressourcen wird, desto höher ist die Wahrscheinlichkeit das eine Lagerproduktion der Ressourcen den noch möglichen Ertrag maximiert. Der Verfall dieser Ressourcen spiegeln die OK wider. An einem solchen Punkt ermöglicht die Ablehnung der Produktanfrage und die Entscheidung zur Lagerproduktion die Überführung des Netzwerks in einen Systemzustand mit einem potentiell höheren Ertrag.

Dabei zeigt die numerische Untersuchung, dass die Anzahl solcher optimalen Politiken mit der Wahrscheinlichkeitsverteilung der Produktanfragen variiert, aber ähnliche Ausmaße annimmt. Es gab in der Untersuchung kein Szenario, dass eine höhere Anzahl an optimaler Politiken der Lagerproduktion gegenüber der Auftragsannahme aufweist. Auch der Fall, dass die optimale Politik der Lagerproduktion die Auftragsannahme zumindest für eine bestimmte Art von Produktanfragen dominiert konnte nicht gezeigt werden. Dieser Sachverhalt wird dadurch erklärt, dass für die optimale Politik der Auftragsannahme (und auch für die Lagerentnahme) die jeweiligen Erträge der Produktanfragen bei dem Entscheidungsmodell berücksichtig werden. D. h. die Lagerproduktion ist eher im Ausnahmefall der optimalen Politik zugehörig.

\section*{Limitation}

In dieser Arbeit wird die Lagerproduktion mit einer möglichen Aufarbeitung von Ressourcen gleichgesetzt, aber auch andere Formulierungen sind denkbar. Bspw. ermöglicht das Modell die Berücksichtigung von unterschiedlichen Ressourcen, wie z. B. Maschinen- und Personaleinsatzstunden. Für die numerische Untersuchung ist die Vereinfachung auf nur eine Ressource jedoch zielführend. Damit war zum einen eine schnelle Berechnung der Szenarien möglich und zum anderen ist dadurch die Funktionsweise des Modells im Grundsatz explorierbar. Jedoch sollte auch die Berechnung von umfangreicheren Szenarien mit unterschiedlichen Ressourcen bei der Betrachtung von zukünftigen Analysen erfolgen.

In der Modellerweiterung \eqref{time} ist die Entscheidung der Lagerproduktion bei eintreffenden Anfragen nur für den jeweiligen Ausführungsmodi möglich. Damit steht dem Netzwerk jeweils nur die Produktionskapazität einer abgelehnten Anfrage zur Verfügung. D. h. ein Unternehmen ist immer gezwungen die Ressourcenkapazität für die Lagererhöhung zu verwendet, die aktuell als Auftrag vorliegt. Damit geht jedoch viel Flexibilität verloren. Daher sollte bei der Modellformulierung \eqref{time} die Möglichkeit bestehen, dass sofern eine Anfrage abgelehnt wird, ein beliebiger Ausführungsmodi der möglichen Produktanfragen des Netzwerks für die Lagerproduktion genutzt wird. Damit wäre ebenfalls bei dieser Entscheidung das Maximum über alle Produktanfragen zu bildet. Bei der Umsetzung der Modellerweiterung in dieser Arbeit konnte dies jedoch nicht umgesetzt werden.

Die Implementierung in das Computersystem bietet weiteres Optimierungspotential. Der Algorithmus ist zwar mit einigen Funktionen des Softwarepakets \texttt{Numpy} entwickelt, aber eine komplette Implementierung in die Programmiersprache \texttt{C++} oder \texttt{Java} könnte die Leistung weiter verbessern bzw. die Rechenzeit reduzieren. Sofern weiter an dieser Implementierung festgehalten wird, empfiehlt es sich die Memofunktion zu überarbeiten. In der aktuellen Fassung der Implementierung basiert die Memofunktion auf der Programmiersprache \texttt{Python}. Auch hier sollte ein Versuch getätigt werden, eine Memofunktion mit dem Softwarepaket \texttt{Numpy} zu entwickeln. Ggf. kann dadurch die vorherige Berechnung aller möglichen Systemzustände des Netzwerks entfallen. Bei der in dieser Arbeit gezeigten Implementierung wurde keine Lösung hierfür gefunden.

Die in dieser Arbeit verwendete Implementierung des DP ist in rekursiver Form programmiert. Ein iterativer Lösungweg könnte die Leistung des Algorithmus verbessern, aber auch eine mögliche Parallelisierung der Teilprobleme des Auftragsannahmeproblems sollte weiter untersucht werden. In dem begrenzten Zeitraum der Erstellung dieser Arbeit erfolgte keine abschließende Umsetzung einer möglichen Parallelisierung der Problemstellung. Die Implementierung nutzt zwar viele aktuelle Funktionen in Bezug des wissenschaftlichen Rechnens, aber es ist der erste Versuch einer Implementierung des Auftragsannahmeproblems des Netzwerk RM unter Berücksichtigung einer Lagerhaltung. Es besteht daher Potential zur Verbesserung der Implementierung.

\section*{Ausblick}

Der Schwerpunkt weiter Forschung in Bezug des Auftragsannahmeproblems im Netzwerk RM mit Lagerhaltungsentscheidung sollte die Konkretisierung der Problemstellung sein. Bei der hier gezeigten Modellerweiterung handelt es sich um eine einfache Umformung des Grundmodells des Netzwerk RM (siehe Gleichung \eqref{DP} im Vergleich zu \eqref{time}). Es berücksichtig z. B. keine Lagerkosten oder eine zeitgebundene Erhöhung von Ressourcen. Daher sollte nachfolgend ein Modell entwickelt werden, welches aus einem tatsächlichen Anwendungsfall resultiert.

Jedoch zeigt die aktuelle Entwicklung im Forschungsbereich des Netzwerk RM in der Auftragsfertigung, dass ein exaktes Lösungsverfahren keine reellen Probleme in einer angemessenen Berechnungszeit löst. Die Integration eines Verfahrens des Bid-Preises sollte daher ebenfalls durch neue Verfahren in Bezug der hier aufgeführten Problemstellung vorangetrieben werden. Damit erfolgt die Reduktion der Berechnungszeit und eine dem Optimum annähernde Lösung kann in möglichen Anwendungssystemen Verwendung finden. Ebenfalls können die Ansätze aus dem Literaturüberblick aus Kapitel \ref{Review} auf mögliche Umsetzung untersucht werden. Vielversprechend erscheint die Ermittlung des Bid-Preises anhand eines Knappsack-Verfahrens. Die numerische Untersuchung zeigt, dass die Anzahl an möglichen Systemzustände mit zunehmender Komplexität stark ansteigt. Daher sollte zwingend der Fokus nachfolgender Forschung auf der Entwicklung einer effizienten Heuristik für das Auftragsannahmeproblem im Netzwerk RM mit Lagerhaltungsentscheidung liegen, damit eine Lösung in angemessener Zeit berechnet werden kann.



%Sofern eine Lagerhaltung berücksichtigt wird, verstößt es gegen die Anwendungsvoraussetzung der \glqq Nichtlagerfähigkeit{\grqq} Diese 
