\chapter{Grundlagen zu auftragsbezogenen Instandhaltungsprozessen}
\markboth{2 Grundlegende Begriffe}{}
\setcounter{footnote}{4}  %um durchgehende Fußnotennummerierung zu haben, hier die Anzahl der bisherigen Fußnoten eintragen

\section{Einordnung in die Produktionswirtschaft}

\section{Charakteristika}

\section{Relevanz für betriebliche Entscheidungen}

\section{Herkunft und Definition des Revenue Managements}
Der Begriff \textit{Revenue Management} wird im deutschsprachigen Raum meist mit \textit{Ertrags}\-\textit{management} oder \textit{Erlösmanagement} übersetzt.\footnote{Vgl. z. B. \cite{zehle1991yield}, S. 486} Yield Management wird als Synonym benutzt.\footnote{Vgl. z. B. \cite{kolisch2006revenue}, S. 319} Dabei greift der Begriff zu kurz, da  Yield im Luftverkehr den Erlös je Passagier und geflogener Meile bezeichnet.\footnote{Vgl. z. B. \cite{weatherford1998tutorial}, S. 69} Daher hat sich der Term \textit{Revenue Management} gegenüber Yield Management durchgesetzt.\footnote{Vgl. \cite{Klein:2008aa}, S. 6} Erste Ansätze des RM wurden in der Praxis entwickelt. Durch die Deregulierung des amerikanischen Luftverkehrsmarktes im Jahr 1978 mussten die traditionellen Fluggesellschaften ihre Wettbewerbsfähigkeit gegenüber Billiganbietern erhöhen und entwickelten das frühe RM.\footnote{Vgl. \cite{Petrick:2009aa}, S. 1-3} In der Literatur ist der Begriff des RM unterschiedlich definiert. \cite{friege1996yield} bezeichnet das RM als \textit{Preis-Mengen-Steuerung}, \cite{daudel1992yield} als \textit{Preis-Kapazitäts-Steuerung} und \cite{talluri2004theory} verstehen es als das gesamtes \textit{Management der Nachfrage}. \citeauthor{klein2001revenue} (2001, S. 248) definiert RM als:

\begin{quote}
\glqq Revenue Management umfasst eine Reihe von quantitativen Methoden zur Entscheidung über Annahme oder Ablehnung unsicherer, zeitlich verteilt eintreffender Nachfrage unterschiedlicher Wertigkeit. Dabei wird das Ziel verfolgt, die in einem begrenzten Zeitraum verfügbare, unflexibel Kapazität möglichst effizient zu nutzen.\grqq
\end{quote}

\cite{Petrick:2009aa} definiert das RM als Ziel einer Unternehmung die Gesamterlöse zu maximieren, die sich aufgrund der speziellen Anwendungsgebiete ergeben. Damit definiert \cite{Petrick:2009aa} das RM als Zusammenfassung aller Interaktionen eines Unternehmens, die mit dem Markt, also der Absatz- oder Nachfrageseite, zusammenhängen. \cite{kimms2005revenue} weisen darauf hin, dass eine differenzierte Betrachtung des Konzepts notwenig ist: Einerseits im Hinblick auf die \textbf{Anwendungsvoraussetzungen} und andererseits im Hinblick auf die \textbf{Instrumente des Revenue Managements}, damit verdeutlicht dargestellt ist, in welchen Branchen das RM Potentiale liefert. Dabei sollten branchenspezifische Besonderheiten, neben den zahlreichen Ähnlichkeiten Berücksichtigung finden, sowie das begrenzte Kapazitätenkontingent, damit die Potentiale des RM zur Maximierung der Gesamterlöse in den Dienstleistungsbranchen erfolgen kann.\footnote{Vgl. z. B. \cite{Martens:2009aa}, S. 11-24}\\

\subsection{Anwendungsvoraussetzungen und Instrumente des Revenue Managements}

Es wurden typische Anwendungsgebiete für das RM aufgezeigt. (???) Jedoch bereitet die Definition weitere Schwierigkeiten. \cite{kimms2005revenue} versuchen durch eine umfangreiche Diskussion einige Erklärungsansätze aufzuzeigen. Zum einen hat das RM vor allem aus dem älteren, englischsprachigen Bereich einen engen Bezug zu konkreten Anwendungsgebieten. Weiter versuchen viele Autoren das komplexe Konzept des Revenue Managements in einer kurzen Erklärung zu verdeutlichen. Dieses läuft letztlich darauf hinaus, dass diese Autoren einige situative Merkmale und Instrumente des Managements vermischen, gleichzeitig aber versuchen, die Zielsetzung festzulegen und das Anwendungsgebiet auf bestimmte Branchen zu beschränken. 

  Die beiden ersteren Definitionen können als Synonym für eines der Instrumente des RM stehen und daher finden diese für das gesamte Konzept keine weitere Verwendung.\footnote{Vgl. z. B. \cite{Petrick:2009aa}}\\

Im Kern lassen sich drei wichtige Perspektiven für eine Definition des Revenue Managements nach \cite{Petrick:2009aa}, \cite{stuhlmann2000kapazitatsgestaltung},  \cite{corsten1999yield} übernehmen:
\begin{itemize}
	\item Ziel ist es die Gesamterlöse unter möglichst optimaler Auslastung der vorhandenen Kapazitäten zu maximieren.
	\item Durch eine aktive Preispolitik wird das reine Kapazitäts- oder Auslastungsmanagement unterstützt.
	\item Für die erfolgreiche Implementierung des Revenue Managements ist eine umfangreiche Informationsbasis notwendig. Es muss u. a. eine möglichst gute Prognose über die zukünftige Nachfrage und Preisbereitschaft der Kunden vorhanden sein.
\end{itemize}
\vspace{0.2cm}
Wie im vorherigen Abschnitt beschrieben, müssen bestimmte Voraussetzungen bestehen, damit die Instrumente des RM zur Anwendung kommen können.

\cite{Petrick:2009aa} weist da\-rauf hin, dass anhand von Anwendungsvoraussetzungen geprüft wird, ob das RM für die jeweilige Situation des Unternehmens (oder die gesamte Branche) zur Maximierung des Gesamterlöses beiträgt. \cite{kimes1989yield} definiert die in der Literatur häufigsten Anwendungsvoraussetzungen:\footnote{Vgl. u. a. \cite{friege1996yield}, S. 616-622, und \cite{weatherford1992taxonomy}, 831-832}
\begin{itemize*}
	\item \glqq weitgehend fixe\grqq\;Kapazitäten
	\item \glqq Verderblichkeit\grqq\;bzw. \glqq Nichtlagerfähigkeit\grqq\;der Kapazitäten und der Leistung
	\item Möglichkeit zur Vorausbuchung von Leistungen
	\item stochastische, schwankende Nachfrage
	\item hohe Fixkosten für die Bereitstellung der gesamten Kapazitäten bei vergleichsweise geringen variablen Kosten für Produktion einer Leistungseinheit
	\item Möglichkeit zur Marktsegmentierung und im Ergebnis dessen zur segmentorientierten Preisdifferenzierung
\end{itemize*}
\vspace{0.2cm}
\cite{Klein:2008aa} setzen sich mit den Anwendungsvoraussetzungen von mehreren Autoren auseinander. Sie konnten Gemeinsamkeiten innerhalb der Definitionen der Autoren finden, aber zeigten auch die Unterschiede und die Kritiken auf. In ihrer Arbeit übernehmen sie die Anwendungsvoraussetzung von \cite{corsten1998yield}: "Marktseitige Anpassungserfordernis steht unternehmesseitigig unzureichendes Flexibilitätspotential hinsichtlich der Kapazität - bezogen auf Mittel- oder Zeitaufwand gegenüber". Zugleich weisen sie jedoch darauf hin, dass zum Verständnis eines komplexen und interdisziplinären Ansatzes auch die Definitionen anderer Autoren im Hinblick auf das Verständnis der Anwendungsvoraussetzungen beitragen.\\

Auf Grundlage der von \cite{friege1996yield} beschriebenen Anwendungsvoraussetzungen hat \cite{Petrick:2009aa} drei Instrumente des RM bestimmt. Die Instrumente benötigen als Grundlage \textit{Daten der Prognose}, damit sie zur Anwendung kommen.\footnote{Die Prognose zählt laut \cite{Petrick:2009aa} nicht als eigenständiges Instrument des RM.} Zu den Instrumenten zählen die \textbf{segmentorientierte Preisdifferenzierung}, die \textbf{Kapazitäten\-steuerung} und die \textbf{Über\-buchungssteuerung}. Es lassen sich unterschiedliche Ab\-hängigkeit\-en der Instrumente untereinander ermitteln.\footnote{Als Beispiel baut die Kapazitätensteu\-erung auf den Ergebnissen der Preisdifferenzierung auf und die Überbuchungssteuerung kann selten ohne Kapazitätensteuerung gelöst werden.}